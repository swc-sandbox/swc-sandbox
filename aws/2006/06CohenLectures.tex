\documentclass[10pt,a4]{seminar} 
\usepackage{amsthm,amsmath,amstext,amsfonts, amssymb} 
\usepackage{times}
\usepackage{semcolor}
\usepackage{graphicx}
\usepackage{multicol}

\input{seminar.bug}
\input{seminar.bg2}

\pagestyle{empty}


\newtheorem*{Theorem}{\bf {\red Theorem}} 
\newtheorem*{Conjecture}{\bf {\red Conjecture}} 
\newtheorem*{Proposition}{\bf {\red Proposition}} 
\newtheorem*{Example}{\bf {\red Example}} 
\newtheorem*{Remark}{\bf {\red Remark}} 
\newtheorem*{Lemma}{\bf {\red Lemma}} 
\newtheorem*{Corollary}{\bf {\red Corollary}} 

\newtheorem*{Definition}{\bf {\red Definition}}
\slideframe{none} 

\input cyracc.def
\newfam\cyrfam
\font\tencyr=wncyr10
\font\sevencyr=wncyr7
\font\fivecyr=wncyr5
\def\cyr{\fam\cyrfam\tencyr\cyracc}
\textfont\cyrfam=\tencyr \scriptfont\cyrfam=\sevencyr% 
\scriptscriptfont\cyrfam=\fivecyr

\newfam\cyifam
\font\tencyi=wncyi10
\font\sevencyi=wncyi7
\font\fivecyi=wncyi5
\def\cyi{\fam\cyifam\tencyi\cyracc}
\textfont\cyifam=\tencyi \scriptfont\cyifam=\sevencyi% 
\scriptscriptfont\cyifam=\fivecyi


\newcommand{\X}{\mbox{{\cyr X}}}


\newcommand{\pphi}{\varphi}
\renewcommand{\a}{{\mathfrak a}}
\renewcommand{\b}{{\mathfrak b}}
\DeclareMathOperator{\Gal}{Gal}
\DeclareMathOperator{\Art}{Art}
\newcommand{\Ima}{\mbox{\rm Im}}
\renewcommand{\O}{{\cal O}}
\renewcommand{\P}{{\mathbb P}}

\renewcommand{\P}{{\mathbb P}}
\newcommand{\leg}[2]{\mbox{$\left(\dfrac{#1}{#2}\right)$}}
\newcommand{\ka}{\kappa} 
\newcommand{\m}{\mu} 
\newcommand{\n}{\nu} 
\newcommand{\la}{\lambda} 
\newcommand{\e}{\epsilon} 
\newcommand{\lep}{\lambda_k^{\epsilon}} 
\newcommand{\ve}{v^{\epsilon}} 
\newcommand{\dk}{d_{2\kappa}} 
\newcommand{\p}{{\mathfrak p}}
\newcommand{\q}{{\mathfrak q}}
\newcommand{\Z}{{\mathbb{Z}}} 
\newcommand{\Q}{{\mathbb{Q}}}
\newcommand{\R}{{\mathbb{R}}} 
\newcommand{\C}{{\mathbb{C}}}
\newcommand{\CC}{{\cal C}}
\newcommand{\F}{{\mathbb{F}}} 
\newcommand{\M}{{\mathcal{M}}}
\newcommand{\G}{{\mathcal{G}}} 
\newcommand{\Sm}{\mathbb{S}_m} 
\newcommand{\Gmn}{{\mathcal{G}_{m,n}}}
\newcommand{\Gmno}{{\mathcal{G}^{\circ}_{m,n}}}
\newcommand{\Gqn}{{\mathcal{G}_{q,n}}}
\newcommand{\Gmmn}{{\mathcal{G}_{m',n}}}
\newcommand{\On}{{\operatorname{O}(n,\R )}}
\newcommand{\Om}{{\operatorname{O}(m,\R )}}
\newcommand{\Gln}{{\operatorname{GL}(n,\R )}}
\newcommand{\Glm}{{\operatorname{GL}(m,\R)}}
\newcommand{\Glk}{{\operatorname{GL}(k,\R)}}
\newcommand{\Glq}{{\operatorname{GL}(q,\R)}}
\newcommand{\Hmu}{\operatorname{H}_{m,n}^{2\mu}}
\newcommand{\Harm}{\operatorname{Harm}}
\newcommand{\HHarm}{\operatorname{Harm}_{m,n}^{\mu}}
\newcommand{\ZZ}{{\mathcal{Z}^{m,q}}} 
\newcommand{\mk}{m_{\kappa}}
\newcommand{\s}{\sigma}
\newcommand{\mmu}{m_{\mu}}
\newcommand{\Pmue}{P_{\mu}^*}
\newcommand{\Pnue}{P_{\nu}^*}
\newcommand{\Pmuie}{P_{\mu_i}^*}
\newcommand{\Pmuiie}{P_{\mu^i}^*}
\newcommand{\Pk}{P_{\kappa}}
\newcommand{\Qk}{Q_{\kappa}}
\newcommand{\Pki}{P_{\kappa_i}}
\newcommand{\Pmu}{P_{\mu}}
\newcommand{\Hka}{\operatorname{H}_{m,n}^{2\kappa}}
\newcommand{\Pnu}{P_{\nu}}
\newcommand{\Ck}{C_{\kappa}}
\newcommand{\Cki}{C_{\kappa_i}}
\newcommand{\Ckii}{C_{\kappa^i}}
\newcommand{\Cke}{C_{\kappa}^*}
\newcommand{\Ckie}{C_{\kappa_i}^*}
\newcommand{\Ckiie}{C_{\kappa^i}^*}
\newcommand{\Cmu}{C_{\mu}}
\newcommand{\Cnu}{C_{\nu}}
\newcommand{\Cmue}{C_{\mu}^*}
\newcommand{\PP}{{\mathbb{P}}}
\newcommand{\D}{{\mathcal{D}}} 
\newcommand{\T}{{\mathbb{T}}}
\newcommand{\Hom}{\operatorname{Hom}}
\newcommand{\diag}{\operatorname{diag}}
\newcommand{\Sym}{\operatorname{Sym}}
\newcommand{\Aut}{\operatorname{Aut}}
\newcommand{\depth}{\operatorname{depth}}
\newcommand{\Stab}{\operatorname{Stab}}
\newcommand{\Id}{\operatorname{Id}}
\newcommand{\Tr}{\operatorname{Tr}}
\newcommand{\uy}{\underline{y}}
\newcommand{\ux}{\underline{x}}
\newcommand{\rk}{\operatorname{rank}}
\newcommand{\im}{\operatorname{im}}
\newcommand{\End}{\operatorname{End}}
\newcommand{\trace}{\operatorname{trace}}
\newcommand{\card}{\operatorname{card}}
\newcommand{\vol}{\operatorname{vol}}
\newcommand{\pr}{\operatorname{pr}} 
\DeclareMathOperator{\N}{{\cal N}}
\newcommand{\al}{\alpha}
\newcommand{\be}{\beta}
\newcommand{\ga}{\gamma}
\renewcommand{\th}{\theta}
\newcommand{\one}{{\bf 1}}
\renewcommand{\atop}[2]{\genfrac{}{}{0pt}{}{#1}{#2}}
\newcommand{\eproof}{\hfill $\fbox{}$ \\} 
\newcommand{\ie}{{\sl i.e.\,}}
\DeclareMathOperator{\lcm}{lcm}

\def\prodp{\operatornamewithlimits{\prod\raise9pt\hbox{\kern-2pt\scriptsize$(p)$\kern-11pt}}}
\def\sumpp{\operatornamewithlimits{\sum\raise9pt\hbox{\kern-2pt\scriptsize$(p)$\kern-11pt}}}
\newcommand{\cote}[1]{\langle #1\rangle}
\newcommand{\z}{\zeta}
\newcommand{\om}{\omega}
\newcommand{\bp}{\backslash p}
\newcommand{\ctl}{\centerline} 
\newcommand{\myblue}{\red}
\newcommand{\myred}{\blue}
\newcommand{\eps}{\varepsilon}
\renewcommand{\a}{{\mathfrak a}}
\renewcommand{\b}{{\mathfrak b}}
\DeclareMathOperator{\LG}{LG}
\newcommand{\isom}{\simeq}

\begin{document} 

\begin{slide}

\title{Explicit Methods for Solving Diophantine Equations}
\author{Henri Cohen} 
\date{} 
\vskip 2.5truecm 
\maketitle 

\vskip 2.5truecm 

\centerline{Laboratoire A2X}
\centerline{Universit\'e Bordeaux 1} 

\vskip 2.5truecm 
\centerline{\myblue{Tucson, Arizona Winter School, 2006}}
 
\end{slide}

\begin{slide}
\ctl{\large{}\myblue{\sc Examples (I)}}

\medskip

Diophantine equation: system of polynomial equations to be solved in
{\red integers}, {\red rational numbers}, or other number rings.

\smallskip

{\red $\bullet$} {\bf Fermat's Last Theorem (FLT)}: in {\blue $\Z$},
{\blue $x^n+y^n=z^n$} with {\blue $n\ge3$} implies {\blue $xyz=0$}.
This gave the impetus for {\red algebraic number theory} by {\red
  Kummer}, {\red Dirichlet}, \dots. Solved by these methods up to
large values of {\blue $n$} (several million). Then {\red Faltings's}
result on rational points on higher genus curves proved that for
{\red fixed} {\blue $n$}, only finite number of (coprime) solutions.
But finally completely solved using {\red elliptic curves}, {\red
  modular forms}, and 
{\red Galois representations} by {\red Ribet}, {\red Wiles}, and 
{\red Taylor--Wiles}. The method of solution is more important than FLT itself.

\end{slide}

\begin{slide}
\ctl{\large{}\myblue{\sc Examples (II)}}

\medskip

{\red $\bullet$} {\bf Catalan's conjecture}: if {\blue $m$} and {\blue
  $n$} are at least {\blue $2$}, nonzero solutions of {\blue $x^m-y^n=1$} come
  from {\blue $3^2-2^3=1$}. Until recently, same status as FLT:
  attacks using algebraic number theory solved many cases. Then
{\red Baker}-type methods were used by {\red Tijdeman} to show that
  the total number of {\blue $(m,n,x,y)$} is finite. Finally
  completely solved by {\red Mih\u{a}ilescu} in 2001, using {\bf only}
  the theory of {\red cyclotomic fields}, but rather deep results
({\red Thaine's theorem}), quite a surprise. Proof later simplified by
  {\red Bilu} and {\red Lenstra}.


\end{slide}
\begin{slide}
\ctl{\large{}\myblue{\sc Examples (III)}}

\medskip

{\red $\bullet$} {\bf The congruent number problem} ({\red
  Diophantus}, 4th century A.D.). Find all integers {\blue $n$} equal
  to the area of a {\red Pythagorean triangle}, i.e. with all sides
  rational (example {\blue $(3,4,5)$} gives {\blue $n=6$}). Easy:
  equivalent to the existence of {\red rational} solutions of
{\blue $y^2=x^3-n^2x$} with {\blue $y\ne0$}. Again, three
  stages. Until the 1970's, several hundred values solved. Then using
the {\red Birch--Swinnerton Dyer conjecture (BSD)}, possible to determine
{\red conjecturally} but {\red analytically} if {\blue $n$} congruent
  or not. Final (but not ultimate) step: a theorem of {\red Tunnell}
  in 1980 giving an immediate criterion for congruent numbers, using
{\red modular forms of half-integral weight}, but still modulo a weak
  form of BSD.

\end{slide}
\begin{slide}
\ctl{\large{}\myblue{\sc Tools (I)}}

\medskip

Almost as many methods to solve Diophantine equations as equations. Attempt at
classification:

{\red $\bullet$} {\bf Local methods:} the use of {\blue $p$}-adic fields, in an 
elementary way (congruences modulo powers of {\blue $p$}), or less elementary 
({\red Strassmann's} or {\red Weierstrass's} theorem, {\blue $p$}-adic power 
series, {\red Herbrand's} and {\red Skolem's} method).

\smallskip

{\red $\bullet$} {\bf Factorization over {\blue $\Z$}}. Not a very powerful 
method, but sometimes gives spectacular results ({\red Wendt's criterion} for 
the first case of Fermat's last theorem, {\red Cassels's} results on 
{\red Catalan's equation}).

\end{slide}

\begin{slide}
\ctl{\large{}\myblue{\sc Tools (II)}}

\medskip

{\red $\bullet$} {\bf Factorization over number fields}, i.e., global methods.
This was in fact the \emph{motivation} for the introduction of number fields 
in order to attack {\red Fermat's last theorem (FLT)}. Even though very 
classical, still one of the most powerful methods, with numerous applications 
and successes.

\smallskip

{\red $\bullet$} {\bf Diophantine approximation methods.} This can come in
many different guises, from the simplest such as {\red Runge's method},
to much more sophisticated ones such as {\red Baker-type methods}.

\smallskip

{\red $\bullet$} {\bf Modular methods,} based on the work of {\red Ribet},
{\red Wiles}, and {\red Taylor--Wiles}, whose first and foremost success is
the complete solution of FLT, but which has had many applications to other
problems.

\end{slide}

\begin{slide}
\ctl{\large{}\myblue{\sc Tools (III)}}

\medskip

In addition, if the set of solutions has a well-understood {\red structure}, 
in many cases one can {\red construct algorithmically} this set of solutions,
and in particular {\red one} solution. Examples are:

{\red $\bullet$} {\bf The Pell--Fermat equation} {\blue $x^2-Dy^2=\pm1$}, and
more generally {\red norm equations} {\blue $\N_{K/\Q}(\al)=m$}, where the
magical algorithm is based on {\red continued fractions} and {\red Shanks's
infrastructure}.

{\red $\bullet$} {\bf Elliptic curves of rank $1$} over {\blue $\Q$}, where
the magical algorithm is based on the construction of {\red Heegner points},
and in particular of the theory of {\red complex multiplication}.

\end{slide}

\begin{slide}
\ctl{\large{}\myblue{\sc Introduction to Local Methods (I)}}

\medskip

Examples of naive use:

{\red $\bullet$} {\bf The equation {\blue $x^2+y^2=3z^2$}.} Dividing by the
square of the GCD, we may assume {\blue $x$} and {\blue $y$} coprime. Then
{\blue $x^2$} and {\blue $y^2$} are congruent to {\blue $0$} or {\blue $1$}
modulo {\blue $3$}, but not both {\blue $0$}, hence 
{\blue $x^2+y^2\equiv1\pmod{3}$}, a contradiction.

\smallskip

{\red $\bullet$} {\bf FLT I for exponent 3}. This is the equation
{\blue $x^3+y^3=z^3$} with {\blue $3\nmid xyz$}. We work {\red modulo}
{\blue $3^2$}: since a cube is congruent to {\blue $0$} or {\blue $\pm1$} 
modulo {\blue $9$}, if {\blue $3\nmid xy$} we have
{\blue $x^3+y^3\equiv-2$}, {\blue $0$}, or {\blue $2$} modulo {\blue $9$},
which is impossible if {\blue $3\nmid z$.}

\end{slide}

\begin{slide}
\vspace*{-26pt}
\ctl{\large{}\myblue{\sc Introduction to Local Methods (II)}}

\medskip

In general need properties of the field {\blue $\Q_p$} of
{\red {\blue $p$}-adic numbers} and ring of integers {\blue $\Z_p$}.
Reminder:

\smallskip

{\red $\bullet$} A {\red homogeneous} equation with integer coefficients 
has a nontrivial solution modulo {\blue $p^n$} for all {\blue $n\ge0$} if and 
only if it has a nontrivial solution in {\blue $\Z_p$} (or in {\blue $\Q_p$}
by homogeneity).

\smallskip

{\red $\bullet$} There is a canonical integer-valued valuation {\blue
  $v_p$} on {\blue $\Q_p^*$}: if {\blue $x\in\Q$} then {\blue
  $v_p(x)$} is the unique integer such that {\blue $x/p^{v_p(x)}$} can
  be written as a rational number with denominator and numerator not
  divisible by {\blue $p$}. It is {\red ultrametric}: 
{\blue $v_p(x+y)\ge\min(v_p(x),v_p(y))$}.

\smallskip

{\red $\bullet$} Elements of {\blue $\Q_p$} such that {\blue $v_p(x)\ge0$} are 
{\red {\blue $p$}-adic integers}, they form a {\red local ring} {\blue $\Z_p$}
with maximal ideal {\blue $p\Z_p$}. Invertible elements of {\blue $\Z_p$}, 
called {\red {\blue $p$}-adic units}, are {\blue $x$} such that
{\blue $v_p(x)=0$}. If {\blue $x\in\Q_p^*$}, canonical decomposition
{\blue $x=p^{v_p(x)}y$} with {\blue $y$} a {\blue $p$}-adic unit.
\end{slide}

\begin{slide}
\ctl{\large{}\myblue{\sc Introduction to Local Methods (III)}}

\medskip

{\red $\bullet$} If {\blue $a\in\Q$} is such that {\blue $v_p(a)\ge0$} and if 
{\blue $v_p(x)\ge1$} then the power series {\blue $(1+x)^a$} converges. 
If {\blue $v_p(a)<0$} the power series converges for 
{\blue $v_p(x)\ge |v_p(a)|+1$} when {\blue $p\ge3$}, and
{\blue $v_p(x)\ge |v_p(a)|+2$} when {\blue $p=2$}. It converges to its
``expected'' value, for instance if {\blue $m\in\Z\setminus\{0\}$} then 
{\blue $y=(1+x)^{1/m}$} satisfies {\blue $y^m=1+x$}.

\smallskip

{\red $\bullet$} {\red Hensel's lemma} (or {\red Newton's method}). Special case:
if {\blue $f(X)\in\Q_p[X]$} and {\blue $\al\in\Q_p$} satisfies
{\blue $v_p(f(\al))\ge1$} and {\blue $v_p(f'(\al))=0$}. There exists 
{\blue $\al^*\in\Q_p$} such that {\blue $f(\al^*)=0$} and 
{\blue $v_p(\al^*-\al)\ge1$}, and {\blue $\al^*$} easily computed by
Newton's iteration.

\smallskip

Testing for {\red local solubility} is usually {\red easy} and
{\red algorithmic}.

\end{slide}
\begin{slide}
\ctl{\large{}\myblue{\sc Local Methods: The Fermat Quartics (I)}}

\medskip

These are the equations {\blue $$x^4+y^4=cz^4\;,$$}
where without loss of generality we may assume that {\blue $c\in\Z$} is not 
divisible by a fourth power. Denote by {\blue $\CC$} the {\red projective 
curve} {\blue $x^4+y^4=c$}.

\smallskip

Note that we will only give the {\red local solubility} results, but that
the {\red global study} involves many methods ({\red factorization} in number 
fields, {\red elliptic curves}), but is far from complete, although it
can solve {\blue $c\le10000$}.

\end{slide}
\begin{slide}
\ctl{\large{}\myblue{\sc Local Methods: The Fermat Quartics (II)}}

\medskip

\begin{Proposition} The curve {\blue $\CC_c$} is everywhere locally soluble 
(i.e., has points in {\blue $\R$} and in every {\blue $\Q_p$}) if and 
only if {\blue $c>0$} and the following conditions are satisfied.
\begin{enumerate}\item {\blue $c\equiv1$} or {\blue $2$} modulo {\blue $16$}.
\item {\blue $p\mid c$}, {\blue $p\neq2$} implies {\blue $p\equiv1\pmod8$}.
\item {\blue $c\not\equiv3$} or {\blue $4$} modulo {\blue $5$}.
\item {\blue $c\not\equiv7$}, {\blue $8$}, or {\blue $11$} modulo {\blue $13$}.
\item {\blue $c\not\equiv4$}, {\blue $5$}, {\blue $6$}, {\blue $9$}, 
{\blue $13$}, {\blue $22$}, or {\blue $28$} modulo {\blue $29$}.
\end{enumerate}\end{Proposition}

\end{slide}
\begin{slide}
\ctl{\large{}\myblue{\sc Local Methods: The Fermat Quartics (III)}}

\medskip

{\bf Ingredients in proof}:

{\red $\bullet$} A {\blue $2$}-adic unit {\blue $x$} is a fourth power in 
{\blue $\Q_2$} if and only if {\blue $x\equiv1\pmod{16\Z_2}$} ({\red power
series} expansion {\blue $(1+u)^{1/4}$}).

\smallskip

{\red $\bullet$} If {\blue $p\nmid 2c$} and {\blue $p\not\equiv1\pmod{8}$}
then {\blue $c$} is a sum of two fourth powers in {\blue $\Q_p$} if and only
if {\blue $\overline{c}$} is a sum of two fourth powers in {\blue $\F_p$}
({\red Hensel's lemma}), and any such {\blue $\overline{c}$} is such a sum if
{\blue $p\equiv3\pmod{4}$} ({\red pigeonhole principle}).

\smallskip

{\red $\bullet$} If {\blue $p\nmid2c$} and {\blue $p\ge37$} then
{\blue $\overline{c}$} is a sum of two fourth powers (the {\red Weil bounds},
here easily provable using {\red Jacobi sums}).

\end{slide}
\begin{slide}
\ctl{\large{}\myblue{\sc Local Methods: Fermat's Last Theorem I (I)}}

\medskip

\begin{Proposition} The following three conditions are equivalent.
\begin{enumerate}\item There exists three {\blue $p$}-adic units 
{\blue $\al$}, {\blue $\be$}, and {\blue $\ga$} such that 
{\blue $\al^p+\be^p=\ga^p$} (in other words FLT I is soluble 
{\blue $p$}-adically).
\item There exists three integers {\blue $a$}, {\blue $b$}, {\blue $c$} in 
{\blue $\Z$} such that {\blue $p\nmid abc$} with 
{\blue $a^p+b^p\equiv c^p\pmod{p^2}$}.
\item There exists {\blue $a\in\Z$} such that {\blue $a$} is not congruent to 
{\blue $0$} or {\blue $-1$} modulo {\blue $p$} with
{\blue $(a+1)^p\equiv a^p+1\pmod{p^2}$}.
\end{enumerate}\end{Proposition}

Proof: Congruences modulo {\blue $p^3$} and Hensel's lemma.
\end{slide}

\begin{slide}
\ctl{\large{}\myblue{\sc Local Methods: Fermat's Last Theorem I (II)}}

\medskip

\begin{Corollary} If for all {\blue $a\in\Z$} such that 
{\blue $1\le a\le (p-1)/2$} we have 
{\blue $(a+1)^p-a^p-1\not\equiv0\pmod{p^2}$}, the 
first case of FLT is true for {\blue $p$}.\end{Corollary}

\smallskip

Note that using {\red Eisenstein reciprocity} (which is a more difficult
{\red global} statement), can prove that {\blue $a=1$} is sufficient in
the above, i.e., {\red Wieferich's criterion}: if 
{\blue $2^{p-1}\not\equiv1\pmod{p^2}$} then FLT I is true for {\blue $p$}
(only known exceptions {\blue $p=1093$} and {\blue $p=3511$}).

\end{slide}
\begin{slide}
\ctl{\large{}\myblue{\sc Local Methods: Strassmann's Theorem (I)}}

\medskip

More sophisticated use of {\blue $p$}-adic numbers: {\blue $p$}-adic
{\red analysis}.

\begin{Theorem} If {\blue $f(X)=\sum_{n\ge0}f_nX^n$} with {\blue
    $f_n\to0$} {\blue $p$}-adically, not identically {\blue $0$},
    exist at most {\blue $N$} elements {\blue $x\in\Z_p$} such that
{\blue $f(x)=0$}, where {\blue $N$} unique integer such that
{\blue $|f_n|\le |f_N|$} for {\blue $n<N$}, and 
{\blue $|f_n|<|f_N|$} for {\blue $n>N$}.\end{Theorem}

Same theorem in {\red extensions} of {\blue $\Q_p$}. Easy proof by
induction on {\blue $N$} using the {\red ultrametric inequality}.

\end{slide}
\begin{slide}
\ctl{\large{}\myblue{\sc Local Methods: Strassmann's Theorem (II)}}

\medskip

{\bf Example}: the equation {\blue $x^3+6y^3=1$} in {\blue $\Z$}.
Set {\blue $\th=6^{1/3}$}, {\blue $K=\Q(\th)$}, {\blue
  $\eps=3\th^2-6\th+1$} {\red fundamental unit} of {\blue $K$} of norm
{\blue $1$}. {\red Dirichlet's unit theorem} implies 
{\blue $x+y\th=\eps^k$} for {\blue $k\in\Z$}. If {\blue
  $\al=\th^2-2\th$} then {\blue $\eps=1+3\al$}, and 
{\blue $$(1+3\al)^k=\exp_3(k\log_3(1+3\al))$$} power series in {\blue
  $k$} (not in {\blue $\al$}) which converges {\blue $3$}-adically.
 Note {\blue $1$}, {\blue $\th$}, {\blue $\th^2$} linearly independent
 over {\blue $\Q_3$} ({\blue $X^3+6$} irreducible in {\blue
   $\Q_3[X]$}). Coefficient of {\blue $\th^2$} in 
{\blue $\eps^k=x+y\th+0\th^2$} equal to {\blue $0$} gives equation in 
{\blue $k$} to which can apply {\red Strassmann}, find {\blue $N=1$},
hence {\blue $k=0$} only solution, so {\blue $(x,y)=(1,0)$}.


\end{slide}
\begin{slide}
\ctl{\large{}\myblue{\sc Factorization over {\blue $\Z$}: Wendt's criterion (I)}}

\medskip

Can give spectacular results. Example: {\red Wendt's criterion} for
{\bf FLT I}.

\begin{Proposition} Let {\blue $p$} be an odd prime, {\blue $k\ge2$} an even
  integer. Assume that {\blue $q=kp+1$} is a {\red prime} such that
{\blue $q\nmid(k^k-1)R(X^k-1,(X+1)^k-1)$} ({\blue $R(P,Q)$} {\red resultant}
of {\blue $P$} and {\blue $Q$}). Then FLT I is true, i.e., 
{\blue $x^p+y^p+z^p=0$} implies {\blue $p\mid xyz$}.
\end{Proposition}

Proof: May assume relatively prime. Write 
{\blue $$-x^p=y^p+z^p=(y+z)(y^{p-1}-y^{p-2}z+\cdots+z^{p-1})\;.$$}
Observe factors {\red relatively prime} (otherwise {\blue $y$} and
{\blue $z$} not relatively prime). Thus exists {\blue $a$} such that
{\blue $y+z=a^p$} and {\blue $y^{p-1}-y^{p-2}z+\cdots+z^{p-1}=s^p$}.
By symmetry {\blue $z+x=b^p$} and {\blue $x+y=c^p$}.


\end{slide}
\begin{slide}
\ctl{\large{}\myblue{\sc Factorization over {\blue $\Z$}: Wendt's criterion (II)}}

\medskip

For {\blue $q=kp+1$}, equation implies
{\blue $$x^{(q-1)/k}+y^{(q-1)/k}+z^{(q-1)/k}\equiv0\pmod{q}\;.$$}
If {\blue $q\nmid xyz$}, implies that {\blue $u=(x/z)^p\bmod{q}$}
satisfies {\blue $u^k-1\equiv0\pmod{q}$} and {\blue
  $(u+1)^k\equiv0\pmod{q}$}, contradicting
 {\blue $q\nmid R(X^k-1,(X+1)^k-1)$}. Thus {\blue $q\mid xyz$}, say
{\blue $q\mid x$}, hence
{\blue \begin{align*}0\equiv 2x&=(x+y)+(z+x)-(y+z)=c^p+b^p+(-a)^p\\
&=c^{(q-1)/k}+b^{(q-1)/k}+(-a)^{(q-1)/k}\pmod{q}\;.\end{align*}}
As above, {\blue $q\mid abc$}, and since {\blue $q\mid x$} and {\blue
  $x$}, {\blue $y$}, and {\blue $z$} pairwise coprime, cannot have 
{\blue $q\mid b^p=z+x$} or {\blue $q\mid c^p=x+y$}, so {\blue $q\mid
  a$}. 

\end{slide}
\begin{slide}
\ctl{\large{}\myblue{\sc Factorization over {\blue $\Z$}: Wendt's criterion (III)}}

\medskip

Thus {\blue $y\equiv -z\pmod{q}$}, hence
{\blue $s^p\equiv py^{p-1}\pmod{q}$}. On the other hand 
{\blue $y=(x+y)-x\equiv c^p\pmod{q}$}, so
{\blue $$s^{(q-1)/k}=s^p\equiv p c^{((q-1)/k)(p-1)}\pmod{q}\;,$$}
and since {\blue $q\nmid c$}, {\blue $p\equiv d^{(q-1)/k}\pmod{q}$} with 
{\blue $d=s/c^{p-1}\bmod{q}$}. Since {\blue $a$}, {\blue $s$} coprime,
we have {\blue $q\nmid s$}, so {\blue $q\nmid d$}, so 
{\blue $p^k\equiv1\pmod{q}$}, and since {\blue $k$} even
{\blue $$1=(-1)^k=(kp-q)^k\equiv k^kp^k\equiv k^k\pmod{q}\;,$$}
contradicting the assumption {\blue $q\nmid k^k-1$}.


\end{slide}

\begin{slide}
\ctl{\large{}\myblue{\sc Factorization over {\blue $\Z$}: Wendt's criterion (IV)}}

\medskip

Wendt's criterion (of course superseded by Wiles et al.) is very 
{\red powerful} since heuristically there will {\bf always} exist a
suitable {\blue $k$}, in fact quite small, and computer searches
confirm this.

\smallskip

Corollary due to Sophie Germain:

\begin{Corollary} If {\blue $p>2$} is prime and {\blue $2p+1$} is
  prime then FLT I is true.\end{Corollary}

Unknown whether there are infinitely many.

\end{slide}
\begin{slide}
\ctl{\large{}\myblue{\sc Factorization over {\blue $\Z$}: {\blue $y^2=x^3+t$}}}

\medskip

By using similar naive methods, can prove the following:

{\red $\bullet$} If {\blue $a$} and {\blue $b$} are odd, if {\blue $3\nmid b$},
and if {\blue $t=8a^3-b^2$} is squarefree, then {\blue $y^2=x^3+t$} has no
{\bf integral} solution.

The example {\blue $t=7=8\cdot1^3-1^2$} was a challenge posed by {\red Fermat}.

\smallskip

{\red $\bullet$} If {\blue $a$} is odd, {\blue $3\nmid b$}, and
{\blue $t=a^3-4b^2$} is squarefree and such that {\blue $t\not\equiv1\pmod8$}.
Then {\blue $y^2=x^3+t$} has no {\bf integral} solution.

\end{slide}
\begin{slide}
\ctl{\large{}\myblue{\sc Cassels's results on Catalan (I)}}

\medskip

Recall that Catalan's equation is {\blue $x^m-y^n=1$}, with 
{\blue $\min(m,n)\ge2$}. Contrary to FLT, {\blue $m=2$} or {\blue $n=2$}
must be included. Proof for {\blue $n=2$} (no nontrivial solution) due to
{\red V.-A.~Lebesgue} in 1850 (not the Lebesgue integral), and involves
{\red factoring over {\blue $\Z[i]$}}, not difficult. Proof for {\blue $m=2$}
considerably more subtle (not really difficult) because there {\bf exist}
the solutions {\blue $(\pm3)^2-2^3=1$}. Done by {\red Ko Chao} in
the 1960's, and involves structure of the unit group of a real quadratic
{\bf order}. 

As for FLT, we are reduced to {\blue $x^p-y^q=1$} with {\blue $p$} and
{\blue $q$} distinct odd primes, with symmetry map
{\blue $(p,q,x,y)\mapsto(q,p,-y,-x)$}. Basic results on this found by Cassels.

\end{slide}
\begin{slide}
\ctl{\large{}\myblue{\sc Cassels's results on Catalan (II)}}

\medskip

Cassels's proof: factoring over {\blue $\Z$}, clever reasoning, and analytic
method called {\red Runge's method}, a form of Diophantine approximation.
As all such, boils down to {\blue $x\in\R$}, {\blue $|x|<1$}, and 
{\blue $x\in\Z$} implies {\blue $x=0$}. 

{\bf Exercise}: use this to find all integral solutions to
{\blue $y^2=x^4+x^3+x^2+x+1$} (hint in the notes).

\smallskip

Need arithmetic lemma and two analytic ones.

{\red $\bullet$} {\bf Arithmetic lemma}: Let {\blue $q$} prime, and
set {\blue $w(j)=j+v_q(j!)$}. Then {\blue $q^{w(j)}\binom{p/q}{j}$} is
an {\bf integer coprime to {\blue $q$}}, and {\blue $w(j)$} is 
{\bf strictly increasing}.

Proof: easy, although there is a slight subtlety (see notes).

\end{slide}
\begin{slide}
\ctl{\large{}\myblue{\sc Cassels's results on Catalan (III)}}

\medskip

{\red $\bullet$} {\bf Analytic lemma I}: If {\blue $q>p>0$} (not necessarily
integral) and {\blue $a\ge1$}, then {\blue $(a^q+1)^p<(a^p+1)^q$}, and if 
{\blue $a>1$} then {\blue $(a^q-1)^p>(a^p-1)^q$}.

Proof: easy undergraduate exercise.

{\red $\bullet$} {\bf Analytic lemma II}: Assume {\blue $p>q$} integers,
{\blue $q\ge3$}, {\blue $p\ge5$} as in Catalan. Set 
{\blue $F(t)=((1+t)^p-t^p)^{1/q}$}, {\blue $m=\lfloor p/q\rfloor+1$}, and let 
{\blue $F_m(t)$} the sum of the terms of degree at most equal to {\blue $m$} 
in the Taylor series expansion of {\blue $F(t)$} around {\blue $t=0$}.
For all {\blue $t\in\R$} such that {\blue $|t|\le1/2$} we have
{\blue $$|F(t)-F_m(t)|\le\dfrac{|t|^{m+1}}{(1-|t|)^2}\;.$$}

Proof: not easy undergraduate exercise (see notes).

\end{slide}
\begin{slide}
\ctl{\large{}\myblue{\sc Cassels's results on Catalan (IV)}}

\medskip

Factorization over {\blue $\Z$}: {\blue $x^p-y^q=1$} gives
{\blue $y^q=x^p-1=(x-1)r_p(x)$} with {\blue $r_p(x)=(x^p-1)/(x-1)$}.
Factors not necessarily coprime but, expanding
{\blue $r_p(x)=((x-1+1)^p-1)/(x-1)$} by the binomial theorem, easy to see that
{\blue $p\mid(x-1)$} is equivalent to {\blue $p\mid r_p(x)$}, that if
{\blue $d=\gcd(x-1,r_p(x))$} then {\blue $d=1$} or {\blue $d=p$},
and that if {\blue $d=p>2$} then {\blue $r_p(x)\equiv p\pmod{p^2}$}, so that
{\blue $v_p(r_p(x))=1$}. Since {\blue $y^q=(x-1)r_p(x)$}, condition
{\blue $d=p$} is equivalent to {\blue $p\mid y$}. Cassels's main theorem
says that this is {\bf always} true, i.e., we never have {\blue $d=1$}.

\end{slide}
\begin{slide}
\ctl{\large{}\myblue{\sc Cassels's results on Catalan (V)}}

\medskip

Proof of Cassels's result split into {\blue $p<q$} and {\blue $p>q$}. The
first case is much easier:

\begin{Proposition} If {\blue $x$} and {\blue $y$} are nonzero integers and 
{\blue $p$} and {\blue $q$} odd primes such that {\blue $x^p-y^q=1$}, then 
when {\blue $p<q$} we have {\blue $p\mid y$}.
\end{Proposition}

Proof: if not, {\blue $x-1$} and {\blue $r_p(x)$} are coprime, so both are
{\blue $q$}th powers since product is. Write {\blue $x-1=a^q$}. Since
{\blue $xy\ne0$}, {\blue $a\ne0$} and {\blue $a\ne-1$}, and
{\blue $(a^q+1)^p-y^q=1$}. Set {\blue $f(z)=(a^q+1)^p-z^q-1$}, decreasing
function of {\blue $z$}. If {\blue $a\ge1$} then
{\blue $f(a^p)=(a^q+1)^p-a^{pq}-1>0$} (binomial theorem), and
{\blue $f(a^p+1)=(a^q+1)^p-(a^p+1)^q-1<0$} by first analytic lemma. Since 
{\blue $f$} strictly decreasing, the {\blue $y$} such that {\blue $f(y)=0$} is
not an integer, absurd. 

\end{slide}
\begin{slide}
\vspace*{-18pt}
\ctl{\large{}\myblue{\sc Cassels's results on Catalan (VI)}}

\medskip

Similarly, if {\blue $a<0$}, we have {\blue $a\le-2$},
and set {\blue $b=-a$}. Since {\blue $p$} and {\blue $q$} are odd 
{\blue $f(a^p)=(a^q+1)^p-a^{pq}-1=-((b^q-1)^p-b^{pq}+1)>0$} (binomial 
theorem), and
{\blue $f(a^p+1)=(a^q+1)^p-(a^p+1)^q-1=-((b^q-1)^p-(b^p-1)^q+1)<0$} again by
the first analytic lemma since $b>1$. Again absurd.

\smallskip

Crucial corollary, due to Hyyr\"o:

\begin{Corollary} Same assumptions, in particular {\blue $p<q$}. Then
{\blue $|y|\ge p^{q-1}+p$}.\end{Corollary}

Proof: since {\blue $p\mid y$} and {\blue $v_p(r_p(x))=1$},
can write {\blue $x-1=p^{q-1}a^p$}, {\blue $(x^p-1)/(x-1)=pv^q$}, 
{\blue $y=pav$}. Set {\blue $P(X)=X^p-1-p(X-1)$}. Clearly 
{\blue $(X-1)^2\mid P(X)$}, so {\blue $(x-1)\mid (x^p-1)/(x-1)-p=p(v^q-1)$},
so {\blue $v^q\equiv1\pmod{p^{q-2}}$}. Since {\blue $q>p$}, 
{\blue $\phi(p^{q-2})=p^{q-3}(p-1)$} coprime to {\blue $q$}, hence
{\blue $v\equiv1\pmod{p^{q-2}}$}. It is easily seen that {\blue $v=1$} is
impossible, so {\blue $v\ge p^{q-2}+1$}, so
{\blue $|y|=pav\ge pv\ge p^{q-1}+p$}.

\end{slide}
\begin{slide}
\ctl{\large{}\myblue{\sc Cassels's results on Catalan (VII)}}

\medskip

Case {\blue $p>q$} more difficult.

\begin{Proposition} If {\blue $x$} and {\blue $y$} are nonzero integers and 
{\blue $p$} and {\blue $q$} odd primes such that {\blue $x^p-y^q=1$}, then 
when {\blue $p>q$} we have {\blue $p\mid y$}.
\end{Proposition}

Proof: as in proof for {\blue $p<q$}, assume by contradiction 
{\blue $p\nmid y$}, so {\blue $x-1=a^q$} hence {\blue $y^q=(a^q+1)^p-1$},
so {\blue $y=a^pF(1/a^q)$} with {\blue $F$} as in analytic lemma II.
Recall {\blue $m=\lfloor p/q\rfloor+1$}, and set 
{\blue $z=a^{mq-p}y-a^{mq}F_m(1/a^q)$}, so that 
{\blue $z=a^{mq}(F(1/a^q)-F_m(1/a^q))$}. By Taylor's theorem
{\blue $t^mF_m(1/t)=\sum_{0\le j\le m}\binom{p/q}{j}t^{m-j}$}, and by 
arithmetic lemma {\blue $D=q^{m+v_q(m!)}$} is a common denominator of
all the {\blue $\binom{p/q}{j}$} for {\blue $0\le j\le m$}. Thus
{\blue $Da^{mq}F_m(1/a^q)\in\Z$}, and since {\blue $mq\ge p$} we have
{\blue $a^{mq-p}y\in\Z$}, so that {\blue $Dz\in\Z$}.

\end{slide}
\begin{slide}
\vspace*{-18pt}
\ctl{\large{}\myblue{\sc Cassels's results on Catalan (VIII)}}

\medskip

We now show {\blue $|Dz|<1$}. Applying analytic lemma II to 
{\blue $t=1/a^q$} (satisfies {\blue $|t|\le1/2$} since {\blue $a\ne\pm1$}):
{\blue $$|z|\le\dfrac{|a|^q}{(|a|^q-1)^2}\le\dfrac{1}{|a|^q-2}\le\dfrac{1}{|x|-3}\;.$$}
By Hyyr\"o's Corollary (with {\blue $(p,q,x,y)$} replaced by 
{\blue $(q,p,-y,-x)$}) we have {\blue $|x|\ge q^{p-1}+q\ge q^{p-1}+3$},
so 
{\blue $$|Dz|\le\dfrac{D}{|x|-3}\le q^{m+v_q(m!)-(p-1)}\;.$$}
Since {\blue $v_q(m!)<m/(q-1)$} for {\blue $m\ge1$} and {\blue $m<p/q+1$},
we have
{\blue $$m+v_q(m!)-(p-1)<m\dfrac{q}{q-1}-(p-1)=\dfrac{3-(p-2)(q-2)}{q-1}\le0$$}
since {\blue $q\ge3$} and {\blue $p\ge5$}, proving {\blue $|Dz|<1$}.

\end{slide}
\begin{slide}
\ctl{\large{}\myblue{\sc Cassels's results on Catalan (IX)}}

\medskip

Since {\blue $Dz\in\Z$}, we have {\blue $Dz=0$}. But
{\blue $$Dz=Da^{mq-p}y-\sum_{0\le j\le m}D\binom{p/q}{j}a^{q(m-j)}\;,$$}
and by the arithmetic lemma
{\blue $$v_q\left(\binom{p/q}{j}\right)<v_q\left(\binom{p/q}{m}\right)=v_q(D)$$}
for {\blue $0\le j\le m-1$}, hence again by the arithmetic lemma
{\blue $$0=Dz\equiv D\binom{p/q}{m}\not\equiv0\pmod{q}\;,$$}
absurd.

\end{slide}
\begin{slide}
\ctl{\large{}\myblue{\sc Cassels's results on Catalan (IX)}}

\medskip

Immediate but crucial corollary of Cassels's theorem:

\begin{Corollary} If {\blue $x$} and {\blue $y$} are nonzero integers and 
{\blue $p$} and {\blue $q$} odd primes such that {\blue $x^p-y^q=1$}, there
exist nonzero integers {\blue $a$} and {\blue $b$}, and positive integers 
{\blue $u$} and {\blue $v$} with {\blue $q\nmid u$} and {\blue $p\nmid v$}
such that
{\blue \begin{align*}x&=qbu,\ x-1=p^{q-1}a^q,\ \dfrac{x^p-1}{x-1}=pv^q,\\
y&=pav,\ y+1=q^{p-1}b^p,\ \dfrac{y^q+1}{y+1}=qu^p\;.\end{align*}}
\end{Corollary}

Proof: easy exercise from the main theorem.
\end{slide}
\begin{slide}
\ctl{\large{}\myblue{\sc Introduction to Number Fields (I)}}

\medskip

Apart from the methods studied above, this is the oldest and most used method
in the subject, and as already mentioned the whole theory of number fields 
arose from the study of Diophantine equations, in particular FLT. Reminder:

\smallskip

{\red $\bullet$} A {\bf number field} {\blue $K$} is a finite extension of 
{\blue $\Q$}, equivalently {\blue $K=\Q(\al)$}, where {\blue $\al$} root of 
a nonzero polynomial {\blue $A\in\Q[X]$}.

\smallskip

{\red $\bullet$} An {\bf algebraic integer} is a root of a {\bf monic} 
polynomial with integer coefficients. The element {\blue $\al$} such that 
{\blue $K=\Q(\al)$} can always be chosen such. The set of algebraic integers
of {\blue $K$} forms a ring, denoted {\blue $\Z_K$}, containing 
{\blue $\Z[\al]$} with finite index, when {\blue $\al$} chosen integral. It 
is a free {\blue $\Z$}-module of rank {\blue $n=[K:\Q]$}, and a 
{\blue $\Z$}-basis of {\blue $\Z_K$} is called an {\bf integral basis}.

\end{slide}
\begin{slide}
\ctl{\large{}\myblue{\sc Introduction to Number Fields (II)}}

\medskip

{\red $\bullet$} The ring {\blue $\Z_K$} is a {\bf Dedekind domain}. The
essential consequence is that any fractional ideal can be decomposed uniquely
into a power product of prime ideals. This is in fact the main motivation.
Crucial fact: if {\blue $\Z[\al]\ne\Z_K$} then it is {\bf never} a Dedekind
domain.

\smallskip

{\red $\bullet$} If {\blue $p$} is a prime, let 
{\blue $p\Z_K=\prod_{1\le i\le g}\p_i^{e_i}$} be the prime power decomposition
of {\blue $p\Z_K$}. The ideals {\blue $\p_i$} are the prime ideals 
{\red ``above''} (in other words containing) {\blue $p$}, the {\blue $e_i$}
are the {\red ramification indexes}, the field {\blue $\Z_K/\p_i$}
is a finite field containing {\blue $\F_p=\Z/p\Z$} with degree denoted by 
{\blue $f_i$}, and we have the important relation
{\blue $\sum_{1\le i\le g}e_if_i=n=[K:\Q]$}.

\end{slide}
\begin{slide}
\ctl{\large{}\myblue{\sc Introduction to Number Fields (III)}}

\medskip

{\red $\bullet$} {\bf Class group} {\blue $Cl(K)$} defined as the quotient of 
fractional ideals by principal ideals, {\bf finite} group with cardinality 
denoted {\blue $h(K)$}.

\smallskip

{\red $\bullet$} {\bf Unit group} {\blue $U(K)$}, group of invertible elements
of {\blue $\Z_K$}, or group of algebraic {\bf integers} of norm {\blue $\pm1$},
is a {\bf finitely generated} abelian group of rank {\blue $r_1+r_2-1$}
({\blue $r_1$} and {\blue $2r_2$} number of real and complex embeddings).
Torsion subgroup finite equal to the group {\blue $\mu(K)$} of roots of unity
in {\blue $K$}.

\end{slide}
\begin{slide}
\ctl{\large{}\myblue{\sc Introduction to Number Fields (IV)}}

\medskip

{\red $\bullet$} A {\bf quadratic field} is {\blue $\Q(\sqrt{t})$}, with
{\blue $t$} squarefree integer different from {\blue $1$}. Its ring of 
integers is either equal to {\blue $\Z[\sqrt{t}]=\{a+b\sqrt{t},\ a,b\in\Z\}$}
when {\blue $t\equiv2$} or {\blue $3$} modulo {\blue $4$}, or 
{\blue $(a+b\sqrt{t})/2$}, with {\blue $a$} and {\blue $b$} integers of same
parity otherwise.

\smallskip

{\red $\bullet$} A {\bf cyclotomic field} is {\blue $K=\Q(\z)$}, with
{\blue $\z$} primitive {\blue $m$}th root of unity. Main result: 
the ring of integers of a cyclotomic field is {\blue $\Z[\z]$}, and no larger.

\end{slide}
\begin{slide}
\ctl{\large{}\myblue{\sc Fermat's last theorem I (FLT I) (I)}}

\medskip

Even though Wendt's criterion is probably always applicable, it is necessary
also to study the {\red algebraic} method, because it also applies to FLT II,
i.e., the case {\blue $p\mid xyz$}.

\smallskip

{\red Notation}: {\blue $\z=\z_p$} primitive {\blue $p$}th root of {\blue $1$},
{\blue $K=\Q(\z)$}, {\blue $\Z_K=\Z[\z]$}, {\blue $\pi=1-\z$}, 
{\blue $\p=\pi\Z_K$} unique prime ideal above {\blue $p$}, and such that
{\blue $\p^{p-1}=p\Z_K$}.

\smallskip

At first, people thought that {\blue $\Z_K$} is always a unique factorization
domain. Unfortunately, completely false: on the contrary, only a (known)
finite list of {\blue $p$} are such.

\end{slide}
\begin{slide}
\ctl{\large{}\myblue{\sc Fermat's last theorem I (FLT I) (II)}}

\medskip

Anyway, let's assume first that {\blue $\Z_K=\Z[\z]$} is a UFD. We prove:

\begin{Lemma} Assume that {\blue $\Z[\z]$} is a UFD. If {\blue $x^p+y^p=z^p$}
with {\blue $p\nmid xyz$}, there exist {\blue $\al\in\Z[\z]$} and a unit 
{\blue $u$} of {\blue $\Z[\z]$} such that {\blue $x+y\z=u\al^p$}.
\end{Lemma}

Proof: may assume {\blue $x$}, {\blue $y$}, {\blue $z$} pairwise coprime.
Our equation can be {\red factored} over {\blue $\Z[\z]$} as
{\blue $$(x+y)(x+y\z)\cdots(x+y\z^{p-1})=z^p\;.$$}
Claim: the factors are pairwise coprime. If some {\blue $\om$} divides 
{\blue $x+y\z^i$} and {\blue $x+y\z^j$} for {\blue $i\neq j$}, it divides also
{\blue $y(\z^i-\z^j)$} and {\blue $x(\z^j-\z^i)$}, hence {\blue $\z^i-\z^j$}
since {\blue $x$} and {\blue $y$} are coprime (in {\blue $\Z$} hence in
{\blue $\Z[\z]$}). Since {\blue $(\z^i-\z^j)\mid p$} in {\blue $\Z[\z]$}, we 
have {\blue $\om\mid p$}, and on the other hand 
{\blue $\om\mid (x+y\z^i)\mid z$}. Since {\blue $p$} and {\blue $z$} are 
coprime, {\blue $\om\mid1$}, in other words is a unit, proving the claim.

\end{slide}
\begin{slide}
\ctl{\large{}\myblue{\sc Fermat's last theorem I (FLT I) (III)}}

\medskip

Thus product of pairwise coprime elements in {\blue $\Z[\z]$} equal to a 
{\blue $p$}th power, so up to multiplication by a unit, each one is, since 
{\blue $\Z[\z]$} is a PID by assumption, proving the lemma.

\smallskip

Unfortunately, as mentioned, not very useful since condition too strict.
This is where {\red ideals} play their magic. Denote by {\blue $h_p$} the
{\red class number} of the cyclotomic field {\blue $K=\Q(\z_p)$}.
Then:

\end{slide}
\begin{slide}
\ctl{\large{}\myblue{\sc Fermat's last theorem I (FLT I) (IV)}}

\medskip

{\bf The above lemma is still valid if we only assume {\blue $p\nmid h_p$}.}

\smallskip

To see why, note that the proof of the lemma is valid verbatim if we replace
``elements'' by ``ideals'': there is unique factorization in ideals, the
coprimeness of the factors remain, and we deduce that each ideal
{\blue $\a_i=(x+y\z^i)\Z_K$} is a {\blue $p$}th power of an ideal, say
{\blue $\a_i=\b_i^p$}. Crucial ingredient: since the class number is finite,
{\blue $\b_i^{h_p}$} is a {\red principal ideal}. Since {\blue $\b_i^p$} also
is, and {\blue $pu+h_pv=1$} for some {\blue $u$}, {\blue $v$}, it follows that
{\blue $\b_i$} itself is principal. If {\blue $\b_1=\al\Z_K$} then
{\blue $\a_1=(x+y\z)\Z_K=\al^p\Z_K=\b_1^p$}, so {\blue $x+y\z=u\al^p$}
for some unit {\blue $u$}, proving the lemma.

\end{slide}
\begin{slide}
\ctl{\large{}\myblue{\sc Fermat's last theorem I (FLT I) (V)}}

\medskip

{\red $\bullet$} The rest of the proof in case {\blue $p\nmid h_p$} is 
specific and easy (see notes). It uses however an additional crucial
ingredient, {\red Kronecker's theorem}: if {\blue $\al$} is an algebraic
{\bf integer} such that all the conjugates of {\blue $\al$} in {\blue $\C$}
have norm equal to {\blue $1$}, then it is a {\red root of unity}. 
In particular, if {\blue $u$} is a unit of {\blue $\Z[\z]$}, then
{\blue $\overline{u}/u$} is a root of unity.

\smallskip

{\red $\bullet$} A prime such that {\blue $p\nmid h_p$} is called a
{\red regular prime}. Known that there are infinitely many {\bf irregular}
primes, conjectured infinitely many regular with density 
{\blue $e^{-1/2}=0.607\dots$}.

\end{slide}
\begin{slide}
\ctl{\large{}\myblue{\sc Fermat's last theorem I (FLT I) (VI)}}

\medskip

{\red $\bullet$} One of the crucial ingredients in the proof was
{\blue $\b^{h_p}$} principal for all ideals {\blue $\b$}. We say that
{\blue $h_p$} {\red annihilates} the class group {\blue $Cl(K)$}. However, 
{\blue $K/\Q$} is a {\red Galois} extension (even {\red abelian}) with Galois 
group {\blue $G\isom(\Z/p\Z)^*$}. The {\red group ring} {\blue $\Z[G]$} acts
on {\blue $Cl(K)$}, and we can look for other elements of {\blue $\Z[G]$}
which annihilate {\blue $Cl(K)$}. One such is given by the {\red Stickelberger
element}, and more generally elements of the {\red Stickelberger ideal}.

\end{slide}
\begin{slide}
\ctl{\large{}\myblue{\sc Fermat's last theorem I (FLT I) (VII)}}

\medskip

This is used in a {\bf simple way} by Mih\u{a}ilescu for Catalan: if
{\blue $x^p-y^q=1$} with {\blue $p$}, {\blue $q$} odd primes and 
{\blue $xy\ne0$}, then {\red double Wieferich condition}:
{\blue $$p^{q-1}\equiv1\pmod{q^2}\text{\quad and\quad}q^{p-1}\equiv1\pmod{p^2}\;.$$}
Note no class number condition. Only seven such pairs known, but expect
infinitely many.

More sophisticated annihilator of {\blue $Cl(K)$} given by {\red Thaine's
theorem}, used in an essential way by Mih\u{a}ilescu for the complete proof
of Catalan, but also in different contexts such as the {\red Birch and
Swinnerton-Dyer conjecture}.

\end{slide}
\begin{slide}
\ctl{\large{}\myblue{\sc {\blue $y^2=x^3+t$} revisited}}

\medskip

Have already seen in special cases. Can factor in {\blue $\Q(\sqrt{t})$} or
in {\blue $\Q(\root3\of{t})$}. Problem with {\bf units} which are not roots
of unity. Sometimes can take care of that, but not always.
{\blue $\Q(\root3\of{t})$} always has such units, so we do not use.
{\blue $\Q(\sqrt{t})$} also does if {\blue $t>0$}, so we assume {\blue $t<0$}.
Thus write {\blue $(y-\sqrt{t})(y+\sqrt{t})=x^3$} in {\bf imaginary}
quadratic field {\blue $K=\Q(\sqrt{t})$}. If factors not coprime, much more
messy. To have factors coprime need both {\bf {\blue $t$} squarefree} and
{\blue $t\not\equiv1\pmod8$}. The first condition is not really essential,
the second is. We then deduce that the ideal {\blue $(y-\sqrt{t})\Z_K$} is
the cube of an ideal, and to conclude as in FLT I we absolutely need the
condition {\blue $3\nmid |Cl(K)|$}. Under all these restrictions, easy to
give complete solution (see notes). For {\bf specific} {\blue $t$}, must
solve {\red Thue equations}.

\end{slide}
\begin{slide}
\vspace*{-10pt}
\ctl{\large{}\myblue{\sc The Super-Fermat equation {\blue $x^p+y^q=z^r$} (I)}}

\medskip

A whole course to itself! Most salient points:

{\red $\bullet$} Must {\bf add} the condition {\blue $x$}, {\blue $y$},
{\blue $z$} coprime because not homogeneous, otherwise usually easy to 
construct infinitely many ``stupid'' (nontrivial) solutions. Example:
{\blue $$55268479930183339474944^3+50779978334208^5=6530347008^7\;.$$}
Set {\blue $\chi=1/p+1/q+1/r-1$}. Different behavior according to {\bf sign} 
of {\blue $\chi$}:

\smallskip

{\red $\bullet$} If {\blue $\chi>0$} ({\red elliptic case}) {\bf complete and 
disjoint parametrizations} of the (infinite) set of solutions ({\red
Beukers}). 

\smallskip

{\red $\bullet$} If {\blue $\chi<0$}, only a {\bf finite} number of solutions
({\red Darmon--Granville}, using {\red Faltings}). Only a few cases solved
(rational points on curves of genus {\blue $g\ge1$}), and only ten cases
known. Would need {\bf effective form} of Faltings.

\end{slide}
\begin{slide}
\ctl{\large{}\myblue{\sc The Super-Fermat equation {\blue $x^p+y^q=z^r$} (II)}}

\medskip

Elliptic case corresponds up to permutation to {\blue $(p,q,r)=(2,2,r)$}
({\red dihedral} case), {\blue $(p,q,r)=(2,3,3)$} ({\red tetrahedral} case),
{\blue $(p,q,r)=(2,3,4)$} ({\red octahedral} case), and
{\blue $(p,q,r)=(2,3,5)$} ({\red icosahedral} case), because they correspond
to the finite subgroups of {\blue $PSL_2(\C)$}.

\smallskip

{\bf Two totally different methods} to treat the elliptic case. The
first is again factoring over suitable number fields (never very large).
The proofs are tedious and in the notes for some dihedral cases (very easy),
and the octahedral case {\blue $(p,q,r)=(2,4,3)$}. In the latter we only
use {\blue $\Z[i]$} by writing {\blue $x^2+y^4=z^3$} as 
{\blue $(x+y^2i)(x-y^2i)=z^3$}. One obtains exactly {\bf four} disjoint
homogeneous $2$-variable parametrizations of the coprime solutions. Existence 
not surprising, their disjointness (i.e., any coprime solution is represented 
by a single parametrization) more surprising.
\end{slide}
\begin{slide}
\ctl{\large{}\myblue{\sc The Super-Fermat equation {\blue $x^p+y^q=z^r$} (III)}}

\medskip

For completeness, they are the following:
{\blue $$\begin{cases}
x=4ts(s^2-3t^2)(s^4+6t^2s^2+81t^4)(3s^4+2t^2s^2+3t^4)&\\
y=\pm(s^2+3t^2)(s^4-18t^2s^2+9t^4)&\\
z=(s^4-2t^2s^2+9t^4)(s^4+30t^2s^2+9t^4)\;,&
\end{cases}$$}
with {\blue $s\not\equiv t\pmod2$} and {\blue $3\nmid s$}.
{\blue $$\begin{cases}
x=\pm(4s^4+3t^4)(16s^8-408t^4s^4+9t^8)&\\
y=6ts(4s^4-3t^4)&\\
z=16s^8+168t^4s^4+9t^8\;,&
\end{cases}$$}
with {\blue $t$} odd and {\blue $3\nmid s$}.
\end{slide}
\begin{slide}
\vspace*{-14pt}
\ctl{\large{}\myblue{\sc The Super-Fermat equation {\blue $x^p+y^q=z^r$} (IV)}}

\medskip

{\blue $$\begin{cases}
x=\pm(s^4+12t^4)(s^8-408t^4s^4+144t^8)&\\
y=6ts(s^4-12t^4)&\\
z=s^8+168t^4s^4+144t^8\;,&
\end{cases}$$}
with {\blue $s$} odd and {\blue $3\nmid s$}.
{\blue $$\begin{cases}
x=\pm2(s^4+2ts^3+6t^2s^2+2t^3s+t^4)(23s^8-16ts^7-172t^2s^6-112t^3s^5&\\
\phantom{=\pm2(}\kern90pt-22t^4s^4-112t^5s^3-172t^6s^2-16t^7s+23t^8)&\\
y=3(s-t)(s+t)(s^4+8ts^3+6t^2s^2+8t^3s+t^4)&\\
z=13s^8+16ts^7+28t^2s^6+112t^3s^5+238t^4s^4&\\
\phantom{=\pm2(}\kern90pt+112t^5s^3+28t^6s^2+16t^7s+13t^8\;,&
\end{cases}$$}
with {\blue $s\not\equiv t\pmod2$} and {\blue $s\not\equiv t\pmod3$}.

\end{slide}
\begin{slide}
\ctl{\large{}\myblue{\sc The Super-Fermat equation {\blue $x^p+y^q=z^r$} (V)}}

\medskip
The factoring method works in all elliptic cases {\bf except} in the
icosahedral case {\blue $(p,q,r)=(2,3,5)$}. Here we must use a completely
different tool (applicable also in the other elliptic cases), invented in
this context by {\red F.~Klein}, but reinterpreted in modern terms by 
{\red Grothendieck} and {\red Belyi}, the theory of {\red dessins d'enfants}.
This is a term coined by Grothendieck to describe 
{\red coverings} of {\blue $\P^1(\C)$} ramified in at most {\blue $3$}
points. Using this, obtain in an {\bf algorithmic manner} complex polynomials
{\blue $P$}, {\blue $Q$}, and {\blue $R$} (homogeneous in two variables) such 
that (for instance) {\blue $P^2+Q^3=R^5$}. These polynomials can in fact be 
chosen with coefficients in a {\bf number field}, and by making
{\blue $PSL_2(\C)$} act on them and introducing a suitable {\red reduction
theory}, can find all parametrizations. Program initiated by {\red F.~Beukers}
and finished by his student {\red J.~Edwards} ({\blue $27$} disjoint 
parametrizations).

\end{slide}

\begin{slide}
\ctl{\large{}\myblue{\sc Introduction to Elliptic Curves (I)}}

\medskip

A very important set of Diophantine equations is the search for {\red rational
points} (or sometimes {\red integral points}) on {\red curves}. Curves are
best classified by their {\red genus}, related to the degree. Example:
a {\red nonsingular plane curve} of degree {\blue $d$} has genus
{\blue $g=(d-1)(d-2)/2$} (so {\blue $g=0$} for lines and conics, {\blue $g=1$}
for plane cubics, {\blue $g=3$} for plane quartics). A {\red hyperelliptic
curve} {\blue $y^2=f(x)$} where {\blue $f$} has degree {\blue $d$} and no
multiple roots has genus {\blue $g=\lfloor (d-1)/2\rfloor$} (so
{\blue $g=0$} for {\blue $d=1$} or {\blue $2$}, {\blue $g=1$} for
{\blue $d=3$} or {\blue $4$}, {\blue $g=2$} for {\blue $d=5$} or {\blue $6$}.

\smallskip

An {\red elliptic curve} {\blue $E$} over some field {\blue $K$} is a curve
of {\red genus {\blue $1$}}, together with a {\blue $K$}-rational point.
\end{slide}
\begin{slide}
\vspace*{-23pt}
\ctl{\large{}\myblue{\sc Introduction to Elliptic Curves (II)}}

\medskip

Study of elliptic curves important for many reasons: curves of genus zero
very well understood (everything algorithmic). curves of genus {\blue $g\ge2$}
very difficult to handle; in addition, elliptic curves have a
{\bf very rich structure}, coming in particular from the fact that they
have a natural {\bf group law}. Reminder on elliptic curves:

\smallskip

{\red $\bullet$} In practice, an elliptic curve can be given by {\red
equations}. The simplest is as a simple {\red Weierstrass equation} 
{\blue $y^2=x^3+ax^2+bx+c$}, or a generalized Weierstrass equation 
{\blue $y^2+a_1xy+a_3y=x^3+a_2x^2+a_4x+a_6$} (canonical numbering), together 
with the condition that the curve be nonsingular. More
generally {\red nonsingular plane cubic} with rational point, 
{\red hyperelliptic quartic} with square leading coefficient 
{\blue $y^2=a^2x^4+bx^3+cx^2+dx+e$}, intersection of two quadrics,
and so on. All these other realizations can algorithmically be transformed
into Weierstrass form, so we will assume from now on that this is the case.

\end{slide}
\begin{slide}
\ctl{\large{}\myblue{\sc Introduction to Elliptic Curves (III)}}

\medskip

{\red $\bullet$} Set of projective points of an elliptic curve (if
{\blue $y^2=x^3+ax^2+bx+c$}, affine points plus the point at infinity
{\blue $\O=(0:1:0)$}) form an {\bf abelian group} under the secant and tangent
method of Fermat (brief explanation: if {\blue $P$} and {\blue $Q$} are 
distinct points on the curve, draw the line joining {\blue $P$} and 
{\blue $Q$}; it meets the curve in a third point {\blue $R$}, and 
{\blue $P+Q$} is the symmetrical point of {\blue $R$} with
respect to the {\blue $x$}-axis. If {\blue $P=Q$}, do the same with the 
tangent). {\bf Warning}: if equation of elliptic curve is not a plane cubic,
the geometric construction of the group law must be modified.

\smallskip

{\red $\bullet$} If {\blue $K=\C$}, {\blue $E(\C)$} in canonical bijection 
with a quotient {\blue $\C/\Lambda$}, where {\blue $\Lambda$} is a 
{\bf lattice} of {\blue $\C$}, thanks to the {\red Weierstrass
{\blue $\wp$} function} and its derivative.

\end{slide}
\begin{slide}
\ctl{\large{}\myblue{\sc Introduction to Elliptic Curves (IV)}}

\medskip

{\red $\bullet$} If {\blue $K=\F_q$}, important {\red Hasse bound}
{\blue $|E(\F_q)-(q+1)|\le 2\sqrt{q}$}. Essential in particular in
cryptography.

\smallskip

{\red $\bullet$} If {\blue $K=\Q_p$} (or a finite extension), we have a good 
understanding of {\blue $E(\Q_p)$} thanks in particular to {\red Kodaira}, 
{\red N\'eron}, and {\red Tate}.

\smallskip

{\red $\bullet$} And what if {\blue $K=\Q$} (or a number field)? Most
interesting, and most difficult case. Deserves a theorem to itself.
\end{slide}
\begin{slide}
\ctl{\large{}\myblue{\sc Introduction to Elliptic Curves (V)}}

\medskip

This is the theorem of {\red Mordell}, generalized by {\red Weil} to
number fields and to Abelian varieties.

\begin{Theorem} If {\blue $E$} is an elliptic curve over a number field
{\blue $K$}, the group {\blue $E(K)$} is a {\bf finitely generated} abelian
group (the Mordell--Weil group of {\blue $E$} over {\blue $K$}).\end{Theorem}

Thus {\blue $E(K)\isom E(K)_{\text{tors}}\oplus\Z^r$}, with
{\blue $E(K)_{\text{tors}}$} a finite group, and {\blue $r$} is called the 
{\red rank} of {\blue $E(K)$}. {\blue $E(K)_{\text{tors}}$} is easily and
algorithmically computable, and only a finite number of possibilities for it,
known for instance for {\blue $K=\Q$} by a difficult theorem of {\red Mazur}. 

{\bf One of the major unsolved problems on elliptic curves is to compute 
algorithmically the rank {\blue $r$}, together with a system of generators}. 

\end{slide}
\begin{slide}
\ctl{\large{}\myblue{\sc Introduction to Elliptic Curves (VI)}}

\medskip

Goal of the rest of the course: explain methods to compute {\blue $E(\Q)$},
either rigorously, or heuristically. There is no general algorithm, but only 
partial ones, which luckily work in ``most'' cases. Sample techniques:

\smallskip

{\red $\bullet$} {\bf {\blue $2$}-descent}, with or without a rational
{\blue $2$}-torsion point

\smallskip

{\red $\bullet$} {\bf {\blue $3$}-descent} with rational torsion subgroup
(more general descents possible, but for the moment not very practical, work
in progress of a lot of people, some present here).

\end{slide}
\begin{slide}
\ctl{\large{}\myblue{\sc Introduction to Elliptic Curves (VII)}}

\medskip

{\red $\bullet$} Use of {\bf {\blue $L$}-functions} to compute the rank,
but not the generators.

\smallskip

{\red $\bullet$} The {\red Heegner point method}, one of the most beautiful
and amazing aspects of this subject, important both in theory and in practice,
applicable to {\bf rank {\blue $1$}} curves, which should form the vast 
majority of curves of nonzero rank, the only ones where we must work.
This is also the subject of the student project.

\end{slide}
\begin{slide}
\ctl{\large{}\myblue{\sc {\blue $2$}-Descent without {\blue $2$}-torsion point (I)}}

\medskip

The general idea of {\red descent}, initiated by Fermat, is to map a
(possibly large) point on a given curve (or more general variety) to 
smaller points on other curves. Here {\red smaller} means that the number of
{\red digits} is {\red divided} by some {\blue $k>1$}, so it is very efficient
when applicable.

The simplest is {\blue $2$}-descent on an elliptic curve when there exists a
rational {\blue $2$}-torsion point (see text). We study the slightly more
complicated case where such a point does not exist. In other words, let
{\blue $y^2=x^3+ax+b$}, where we assume {\blue $a$} and {\blue $b$} in 
{\blue $\Z$} and {\blue $x^3+ax+b=0$} without rational roots, hence irreducible
over {\blue $\Q$}. Denote by {\blue $\th$} a root, and set {\blue $K=\Q(\th)$}.

\end{slide}
\begin{slide}
\ctl{\large{}\myblue{\sc {\blue $2$}-Descent without {\blue $2$}-torsion point (II)}}

\medskip

Define the map {\blue $\al$} from {\blue $E(\Q)$} to {\blue $K^*/{K^*}^2$}
by {\blue $\al(\O)=1$} and {\blue $\al((x,y))=x-\th$} modulo {\blue ${K^*}^2$}.
Fundamental result, easy to prove using definition of group law by
secant and tangent:

\begin{Proposition} {\blue $\al$} is a {\red group homomorphism} whose
kernel is equal to {\blue $2E(\Q)$}. In particular, it induces an
{\red injective} homomorphism from {\blue $E(\Q)/2E(\Q)$} to 
{\blue $K^*/{K^*}^2$}, and the rank {\blue $r$} of {\blue $E(\Q)$} is
equal to {\blue $\dim_{\F_2}(\Ima(\al))$}.\end{Proposition}

(Note that we assume no {\blue $2$}-torsion.)

\end{slide}
\begin{slide}
\ctl{\large{}\myblue{\sc {\blue $2$}-Descent without {\blue $2$}-torsion point (III)}}

\medskip

Thus describe {\blue $\Ima(\al)$}. For this need {\blue $T$}-{\red Selmer 
group} of a {\bf number field} (not of the elliptic curve).

{\red $\bullet$} {\blue $T$} finite set of prime ideals of {\blue $K$}.

\smallskip

{\red $\bullet$} {\blue $U_T(K)$} group of {\blue $T$}-units {\blue $u$}
of {\blue $K$} ({\blue $v_{\p}(u)=0$} for {\blue $p\notin T$}).

\smallskip

{\red $\bullet$} {\blue $Cl_T(K)$} {\blue $T$}-class group, equal to
{\blue $Cl(K)/<T>$} with evident notation.

\smallskip

{\red $\bullet$} A {\blue $T$}-{\red virtual square} {\blue $u\in K^*$} is 
such that {\blue $2\mid v_{\p}(u)$} for all {\blue $\p\notin T$}.

\smallskip

{\red $\bullet$} The {\blue $T$}-{\red Selmer group} {\blue $S_T(K)$} is the
quotient of the group of {\blue $T$}-virtual squares by the group 
{\blue ${K^*}^2$} of nonzero squares.

\end{slide}
\begin{slide}
\ctl{\large{}\myblue{\sc {\blue $2$}-Descent without {\blue $2$}-torsion point (IV)}}

\medskip

We have {\blue $S_T(K)\isom (U_T(K)/U_T(K)^2)\times Cl_T(K)[2]$}, so easily
computable using a computer algebra system.

Basic result linking {\blue $\Ima(\al)$} with {\blue $S_T(K)$} (not difficult):

\begin{Proposition} Let {\blue $T$} be the set of prime ideals {\blue $\q$}
such that {\blue $\q\mid 3\th^2+a$} and {\blue $q\mid [\Z_K:\Z[\th]]$},
where {\blue $q$} prime number below {\blue $\q$}. Then {\blue $\Ima(\al)$}
is equal to the group of {\blue $\overline{u}\in S_T(K)$} such that
{\blue $\N_{K/\Q}(u)$} (for any lift {\blue $u$}) is a square in {\blue $\Q$} 
and such that there {\red exists} a lift {\blue $u$} of the form 
{\blue $x-\th$}.
\end{Proposition}

{\red $\bullet$} Sometimes {\blue $[\Z_K:\Z[\th]]=1$} so {\blue $T=\emptyset$}
(but {\blue $S_T(K)$} may still be nontrivial).

\smallskip

{\red $\bullet$} The condition on the norm is algorithmic. Unfortunately 
the existence of a lift {\blue $u$} of the form {\blue $x-\th$} is not
(but luckily feasible in many cases).

\end{slide}
\begin{slide}
\ctl{\large{}\myblue{\sc {\blue $2$}-Descent without {\blue $2$}-torsion point (V)}}

\medskip

So let {\blue $G$} the group of {\blue $\overline{u}\in S_T(K)$} whose lifts
have square norm. To determine if {\blue $\overline{u}$} has a lift
{\blue $x-\th$}: write {\blue $u=u_2\th^2+u_1\th+u_0$} for any lift,
{\blue $u_i\in\Q$}. All lifts are of the form {\blue $u\ga^2$} for
{\blue $\ga=c_2\th^2+c_1\th+c_0$}, and
{\blue $$u\ga^2=q_2(c_0,c_1,c_2)\th^2-q_1(c_0,c_1,c_2)\th+q_0(c_0,c_1,c_2)\;,$$}
with {\blue $q_i$} {\red explicit quadratic forms}. Condition reads
{\blue $q_2(c_0,c_1,c_2)=0$} and {\blue $q_1(c_0,c_1,c_2)=1$}. The first
equation can be checked for solubility by {\red Hasse--Minkowski} 
(local-global principle for quadratic form), and then {\red parametrized by 
quadratic forms} in two variables (as super-Fermat equation). Replacing in
second equation gives {\red quartic}, and dehomogenenizing gives
a {\red hyperelliptic quartic equation} {\blue $y^2=Q(x)$}.

\end{slide}
\begin{slide}
\vspace*{-11pt}
\ctl{\large{}\myblue{\sc {\blue $2$}-Descent without {\blue $2$}-torsion point (VI)}}

\medskip

If not everywhere locally soluble, can again exclude {\blue $\overline{u}$}. 
Otherwise search for solutions. If found, {\blue $\overline{u}\in\Ima(\al)$}, 
if not we are stuck.

\smallskip

The group of {\blue $\overline{u}\in S_T(K)$} for which the corresponding 
quartic is everywhere locally soluble is called the {\blue $2$}-Selmer group
of the {\bf elliptic curve}, and is the smallest group containing
{\blue $E(\Q)/2E(\Q)$} which can be determined algorithmically using
{\blue $2$}-descent. It is denoted {\blue $S_2(E)$}. The quotient of
{\blue $S_2(E)$} by the (a priori unknown) subgroup {\blue $E(\Q)/2E(\Q)$}
is the {\bf obstruction} to {\blue $2$}-descent, and is equal to
{\blue $\X(E)[2]$}, the part of the {\red Tate--Shafarevitch group} of 
{\blue $E$} killed by {\blue $2$}.

\smallskip

{\blue $2$}-descent quite powerful, basis of {\red Cremona's}
{\tt mwrank} program. If fails, can try a {\bf second descent} (solve
the quartics), or a {\blue $3$}-descent or higher.

\smallskip

{\bf Exercise}: try your {\blue $2$}-descent skills for {\blue $y^2=x^3\pm16$}
(both have rank {\blue $0$}).

\end{slide}
\begin{slide}
\ctl{\large{}\myblue{\sc Example of {\blue $3$}-Descent (I)}}

\medskip

Not difficult but too long to explain, so we give an interesting example.

\smallskip

{\red Goal}: given nonzero integers {\blue $a$}, {\blue $b$}, and {\blue $c$},
determine if there exists a nontrivial solution to {\blue $ax^3+by^3+cz^3=0$}.

\smallskip

As usual, easy to give condition for everywhere local solubility (see text).

To go further, for any {\blue $n\in\Z_{\ge1}$} let {\blue $E_n$} be the
elliptic curve {\blue $y^2=x^3+n^2$}. The point {\blue $T=(0,n)$} is torsion
of order {\blue $3$}. We define a {\blue $3$}-descent map {\blue $\al$} from
{\blue $E(\Q)$} to {\blue $\Q^*/{\Q^*}^3$} by setting {\blue $\al(\O)=1$}, 
{\blue $\al(T)=4n^2$}, and otherwise {\blue $\al((x,y))=y-n$}, all modulo
cubes. Easy direct check that {\blue $\al$} is a {\bf group homomorphism},
kernel easy to compute (not needed here).

Projective curve {\blue $\CC=\CC_{a,b,c}$} with equation 
{\blue $ax^3+by^3+cz^3=0$} closely linked to curve {\blue $E=E_{4abc}$}
as follows.

\end{slide}
\begin{slide}
\ctl{\large{}\myblue{\sc Example of {\blue $3$}-Descent (II)}}

\medskip

\begin{Proposition} Define {\blue $\phi(x,y,z)=(-4abcxyz,\ -4abc(by^3-cz^3),\ ax^3)$}.
\begin{enumerate}\item The map {\blue $\phi$} sends {\blue $\CC(\Q)$} into 
{\blue $E(\Q)$} (in projective coordinates).
\item Let {\blue $G=\{(X,Y,Z)\}\in E(\Q)$} such that
{\blue $c(Y-4abcZ)=bZ\la^3$} for some {\blue $\la\in\Q^*$}. Then
{\blue $\Ima(\phi)=\phi(\CC(\Q))$} is equal to {\blue $G$} together with
{\blue $\O$} if {\blue $c/b\in{\Q^*}^3$}, and {\blue $T$} if
{\blue $b/a\in{\Q^*}^3$} (and immediate to give preimages).
\item The set {\blue $\CC(\Q)$} is nonempty if and only if {\blue $b/c$}
modulo cubes belongs to {\blue $\Ima(\al)\subset \Q^*/{\Q^*}^3$}.
\end{enumerate}\end{Proposition}

Proof: (1) and (2) are simple verifications. For (3), 
{\blue $\CC(\Q)\ne\emptyset$} iff {\blue $\Ima(\phi)\ne\emptyset$}, hence
iff either there exists {\blue $(X,Y,Z)\in E(\Q)$} and {\blue $\la\in\Q^*$}
with {\blue $c(Y-4abcZ)=bZ\la^3$}, or if {\blue $c/b$} or {\blue $b/a$}
are cubes. Note {\blue $\la=2cz/x$}. This easily implies that 
{\blue $b/c\in\Ima(\al)$}.

\end{slide}
\begin{slide}
\ctl{\large{}\myblue{\sc Example of {\blue $3$}-Descent (III)}}

\medskip

Thus, to test solubility of {\blue $ax^3+by^3+cz^3=0$}, proceed as follows.
First test everywhere local solubility (easy). Then compute 
the Mordell--Weil group {\blue $E(\Q)$} using {\blue $2$-descent} and/or
a software package like Cremona's {\tt mwrank} (of course may be difficult),
also torsion subgroup (easy). If {\blue $(P_i)_{1\le i\le r}$} basis of
free part, then classes modulo {\blue $3E(\Q)$} of {\blue $P_0=T$} and the
{\blue $P_i$} form an {\blue $\F_3$}-basis of {\blue $E(\Q)/3E(\Q)$}.
Then check if {\blue $b/c$} modulo cubes belongs to the group generated by
the {\blue $(\al(P_i))_{0\le i\le r}$} in {\blue $\Q^*/{\Q^*}^3$}, simple
linear algebra over {\blue $\F_3$}. Completely algorithmic, {\bf apart from}
the MW computations.

\end{slide}
\begin{slide}
\ctl{\large{}\myblue{\sc Example of {\blue $3$}-Descent (IV)}}

\medskip

{\bf Examples}: 

{\red $\bullet$} {\blue $x^3+55y^3+66z^3=0$}. Everywhere locally soluble, 
cannot solve algebraically as far as I know. Use above method. Find torsion 
subgroup of order {\blue $3$} generated by {\blue $P_0=T=(0,14520)$}. In a 
fraction of a second, {\tt mwrank} (or {\blue $2$}-descent) says rank
{\blue $1$} and a generator {\blue $P_1=(504,18408)$}. Then modulo cubes
{\blue $\al(P_0)=2^2\cdot 3^2\cdot 5^2\cdot11$} and 
{\blue $\al(P_1)=2\cdot3^2$}, while {\blue $b/c=2^2\cdot3^2\cdot5$}. Linear
algebra immediately shows {\blue $b/c$} {\bf not} in group generated by
{\blue $\al(P_0)$} and {\blue $\al(P_1)$}, so no solution.

\end{slide}
\begin{slide}
\ctl{\large{}\myblue{\sc Example of {\blue $3$}-Descent (V)}}

\medskip

{\red $\bullet$} Descent not always negative: {\blue $x^3+17y^3+41z^3=0$}.
Torsion subgroup of order {\blue $3$} generated by {\blue $P_0=T=(0,2788)$}.
Rank {\blue $1$} and generator 
{\blue $P_1=(355278000385/2600388036, -426054577925356417/132604187507784)$}.
Modulo cubes {\blue $\al(P_0)=17^2.41^2$}, {\blue $\al(P_1)=17^2$}, 
and {\blue $b/c=17\cdot 41^2$}, so since {\blue $\al$} group homomorphism,
{\blue $\al(P_0+P_1)=b/c$} modulo cubes.
Find in projective coordinates
{\blue $P_0+P_1=(X,Y,Z)=(5942391203335522320, 251765584367435734052, 3314947244332625)$}, compute {\blue $\la$} such that {\blue $c(Y-4abcZ)=bZ\la^3$}, find
{\blue $\la=8363016/149105$}. Since {\blue $\la=2cz/x$}, find (up to
projective scaling) {\blue $z=101988$}, {\blue $x=149105$}, hence
{\blue $y=140161$}.

\end{slide}
\begin{slide}
\ctl{\large{}\myblue{\sc The use of {\blue $L(E,s)$} (I)}}

\medskip

Extremely important method which at least determines whether or not
{\blue $E$} has nontorsion points, without giving them.

Definition of {\blue $L(E,s)$}. Assume {\bf minimal Weierstrass equation}
over {\blue $\Q$}. If {\blue $E$} has {\red good reduction}
at {\blue $p$}, define {\blue $a_p=p+1-|E(\F_p)|$} and {\blue $\chi(p)=1$}. 
Otherwise define {\blue $\chi(p)=0$}, and {\blue $a_p=0$} if 
{\bf triple point} ({\red additive reduction}), {\blue $a_p=1$} or 
{\blue $a_p=-1$} if {\bf double point} with or without rational tangents 
({\red split or nonsplit multiplicative reduction}). In fact
{\blue $a_p=p+1-|E(\F_p)|$} still true. Then
{\blue $$L(E,s)=\prod_p\dfrac{1}{1-a_pp^{-s}+\chi(p)p^{1-2s}}\;,$$}
converges for {\blue $\Re(s)>3/2$} because of Hasse bound
{\blue $|a_p|\le 2p^{1/2}$}.

\end{slide}
\begin{slide}
\vspace*{-25pt}
\ctl{\large{}\myblue{\sc The use of {\blue $L(E,s)$} (II)}}

\medskip

Most important theorem, due to {\red Wiles}, {\red Taylor--Wiles}, et al.:
{\blue $L(E,s)$} extends to a holomorphic function to the whole complex
plane, satisfying a functional equation 
{\blue $$\Lambda(E,2-s)=\eps(E)\Lambda(E,s)\;,$$}
with {\blue $\eps(E)=\pm1$} (the {\red root number}), and
{\blue $\Lambda(E,s)=N^{s/2}(2\pi)^{-s}\Gamma(s)L(E,s)$}.
Here {\blue $N$} is the {\red conductor}, divisible by all bad primes and
easily computable by {\red Tate's algorithm}.

Because of BSD, are interested in the value {\blue $L(E,1)$}. If
{\blue $\eps(E)=-1$}, trivially {\blue $L(E,1)=0$}. Otherwise, {\bf automatic 
consequence}, we have the {\bf exponentially convergent} (so
{\bf easy to compute}) series
{\blue $$L(E,1)=2\sum_{n\ge1}\dfrac{a_n}{n}e^{-2\pi n/\sqrt{N}}\;.$$}

\end{slide}
\begin{slide}
\ctl{\large{}\myblue{\sc The use of {\blue $L(E,s)$} (III)}}

\medskip

{\red \bf The Birch and Swinnerton-Dyer conjecture}: precise statement, but
tells us in particular that there exist nontorsion points in {\blue $E(\Q)$}
(i.e., {\blue $E(\Q)$} infinite) if and only if {\blue $L(E,1)=0$}.
Unfortunately, except in the rank {\blue $1$} case ({\red Heegner point 
method}) does not help us much in {\bf finding} the points. 

What is known ({\red Rubin, Kolyvagin, etc...}): 

{\red $\bullet$} If {\blue $L(E,1)\ne 0$} then no nontorsion points. 

\smallskip

{\red $\bullet$} If {\blue $L(E,1)=0$} but {\blue $L'(E,1)\ne0$} then 
{\bf rank {\blue $1$}}, in particular exists nontorsion points, and can be
found using Heegner points.

\smallskip

On the other hand, if {\blue $L(E,1)=L'(E,1)=0$}, {\bf nothing} known,
although BSD conjecture says rank at least {\blue $2$}.

\end{slide}
\begin{slide}
\ctl{\large{}\myblue{\sc The Heegner point method (I)}}

\medskip

This is a remarkable way to use {\blue $L(E,s)$} to {\bf find} a nontorsion
rational point, works only when the rank is equal to {\blue $1$} (otherwise
always gives a torsion point, even in rank {\blue $r\ge2$}).

Tools: {\red complex multiplication} and the {\red modular parametrization}.
For the algorithm, need to understand the theorems, but not the proofs. In 
fact, {\bf conjectures} are sufficient since checking rational points is
trivial.

\smallskip

{\red Setup}: {\blue $E$} elliptic curve over {\blue $\Q$} and
{\blue $L(E,s)=\sum_{n\ge1}a_nn^{-s}$}.

\end{slide}
\begin{slide}
\vspace*{-18pt}
\ctl{\large{}\myblue{\sc The Heegner point method (II)}}

\medskip

{\bf Modular parametrization}: Wiles's theorem is equivalent to
{\blue $f_E(\tau)=\sum_{n\ge1}a_nq^n$} ({\blue $q=\exp(2i\pi\tau)$}) is
a {\red modular form} of weight {\blue $2$} on {\blue $\Gamma_0(N)$}.
Equivalently still, {\blue $2i\pi f_E(\tau)d\tau$} is a {\bf holomorphic 
differential}, {\red invariant} under {\blue $\Gamma_0(N)$} up to the
{\red period lattice} of {\blue $f_E$}, i.e.,
{\blue $$\phi(\tau)=2i\pi\int_{i\infty}^\tau f_E(z)\,dz=\sum_{n\ge1}\dfrac{a_n}{n}q^n$$}
does not depend on chosen path, and defines map from 
{\blue ${\cal H}/\Gamma_0(N)$} to {\blue $\C/\Lambda$}, easily 
extended to map from closure {\blue $X_0(N)$} to {\blue $\C/\Lambda$}, where
{\blue $\Lambda$} lattice generated by 
{\blue $2i\pi\int_{i\infty}^\ga f_E(z)\,dz$}, with {\blue $\ga\in\Q$} a 
{\red cusp}. {\bf Usually} (always happens in practice, if not can easily
be dealt with) have {\blue $\Lambda\subset\Lambda_E$}, with
{\blue $E(\C)=\C/\Lambda_E$}, so get a map from {\blue $X_0(N)$} to 
{\blue $\C/\Lambda_E$}, and composing with the {\red Weierstrass 
{\blue $\wp$} function}, get map {\blue $\pphi$} from {\blue $X_0(N)$} to 
{\blue $E(\C)$}, the {\red modular parametrization}. Wiles: exists and unique
up to sign.

\end{slide}
\begin{slide}
\vspace*{-8pt}
\ctl{\large{}\myblue{\sc The Heegner point method (III)}}

\medskip

{\bf Complex multiplication (CM)}: say {\blue $\tau$} is a {\red CM point}
if {\blue $\tau\in{\cal H}$} is a root of quadratic equation 
{\blue $AX^2+BX+C=0$} with {\blue $A$}, {\blue $B$}, {\blue $C$} integral
with {\blue $B^2-4AC<0$}. Make this unique by requiring {\blue $\gcd(A,B,C)=1$}
and {\blue $A>0$}, then set {\blue $\Delta(\tau)=B^2-4AC$}.

\smallskip

{\bf Basic result of CM} (in our context): if {\blue $\tau$} is a 
{\bf suitable} CM point, then {\blue $\pphi(\tau)\in E(H)$} and not only 
{\blue $\pphi(\tau)\in E(\C)$}, where {\blue $H$} is the {\red Hilbert class 
field} of {\blue $K=\Q(\sqrt{D})$}. This is the {\bf magic} of CM: create 
{\bf algebraic} numbers using {\bf analytic} functions ({\red Kronecker's 
dream of youth}: do this for other number fields).

\end{slide}
\begin{slide}
\ctl{\large{}\myblue{\sc The Heegner point method (IV)}}

\medskip

Assume for simplicity {\blue $D=\Delta(\tau)$} {\red discriminant} of a
quadratic field ({\red fundamental discriminant}).

{\bf Definition}: Given {\blue $N$}, {\blue $\tau$} is a {\red Heegner point 
of level {\blue $N$}} if it satisfies the equivalent conditions:

{\red $\bullet$} {\blue $\Delta(N\tau)=\Delta(\tau)$}

\smallskip

{\red $\bullet$} {\blue $N\mid A$} and {\blue $\gcd(A/N,B,CN)=1$}

\smallskip

{\red $\bullet$} {\blue $N\mid A$} and {\blue $D\equiv B^2\pmod{4N}$}.

\end{slide}
\begin{slide}
\ctl{\large{}\myblue{\sc The Heegner point method (V)}}

\medskip

{\bf Basic properties}: 

{\red $\bullet$} Let {\blue $\tau$} Heegner point of level {\blue $N$}.
If {\blue $\ga\in\Gamma_0(N)$} then {\blue $\ga(\tau)$},
{\blue $W(\tau)=-1/(N\tau)$}, and more generally {\blue $W_Q(\tau)$} ({\red
Atkin--Lehner operators}) are again Heegner points of level {\blue $N$}.

\smallskip

{\red $\bullet$} Recall natural correspondence between {\blue $SL_2(\Z)$}
classes of {\red binary quadratic forms} and the {\red ideal class group}
of corresponding quadratic field. This easily generalizes to
{\blue $\Gamma_0(N)$}-equivalence as follows: natural correspondence between
{\blue $\Gamma_0(N)$}-equivalence classes of Heegner points of discriminant 
{\blue $D$} and level {\blue $N$} and the set of {\bf pairs}
{\blue $(\beta,[\a])$}, with {\blue $[\a]$} ideal class, and 
{\blue $\beta\in\Z/2N\Z$} such that {\blue $\beta^2\equiv D\pmod{4N}$}.

\end{slide}
\begin{slide}
\ctl{\large{}\myblue{\sc The Heegner point method (VI)}}

\medskip

{\bf Main theorem of CM}:  Let {\blue $\tau=(\beta,[\al])$} Heegner point
of discriminant {\blue $D$} (fundamental) and level {\blue $N$},
{\blue $K=\Q(\sqrt{D})$}, {\blue $H$} Hilbert class field of {\blue $K$}
(maximal unramified Abelian extension of {\blue $K$}, 
{\blue $\Gal(H/K)\isom Cl(K)$} through the {\red Artin map} {\red $\Art$}). 
Recall {\blue $\pphi$} modular parametrization from {\blue $X_0(N)$} to
{\blue $E$}. Then:

{\red $\bullet$} {\blue $\pphi(\tau)\in E(H)$} ({\red algebraicity})

\smallskip

{\red $\bullet$} If {\blue $[\b]\in Cl(K)$} then ({\red Shimura reciprocity}):
{\blue $$\pphi((\be,[\a]))^{\Art([\b])}=\pphi((\be,[\a\b^{-1}]))$$}
Also formula for {\blue $\pphi(W((\be,[\a])))$} and
{\blue $\pphi(W_Q((\be,[\a])))$}.

\smallskip

{\red $\bullet$} {\blue $\pphi((-\be,[\a]^{-1}))=\overline{\pphi((\be,[\a]))}$}.


\end{slide}
\begin{slide}
\ctl{\large{}\myblue{\sc The Heegner point method (VII)}}

\medskip

{\bf Consequence}:  Can compute the trace of {\blue $\pphi(\tau)$} on the
elliptic curve by
{\blue $$P=\sum_{\sigma\in\Gal(H/K)}\pphi((\be,[\a]))^{\s}=\sum_{[\b]\in Cl(K)}\pphi((\be,[\a\b^{-1}]))=\sum_{[\b]\in Cl(K)}\pphi((\be,[\b]))\;,$$}
the sum being computed with the {\bf group law} of {\blue $E$}. By Galois 
theory we will have {\blue $P\in E(K)$}, so we have considerably reduced the 
field of definition of the algebraic point on {\blue $E$}. In addition,
easy result: 

If {\blue $\eps(E)=-1$} (which is our case since rank {\blue $1$}), then in
fact {\blue $P\in E(\Q)$}, which is what we want.

\end{slide}
\begin{slide}
\ctl{\large{}\myblue{\sc The Heegner point method (VII)}}

\medskip

Thanks in particular to {\red Gross--Zagier} and {\red Kolyvagin}, know
that {\blue $P$} is {\bf nontorsion} if and only if {\blue $r=1$} (already
known) {\bf and} {\blue $L(E_D,1)\ne0$}, where {\blue $E_D$} is the
{\red quadratic twist} of {\blue $E$} by {\blue $D$} (equation
{\blue $Dy^2=x^3+ax+b$}).

Point {\blue $P$} often {\bf large} multiple of generator, can reduce it
considerably again by using Gross--Zagier. Get very nice algorithm.

\smallskip

{\bf Example}: congruent number problem for {\blue $n=157$}, curve
{\blue $y^2=x^3-157^2x$}. Rank {\blue $1$}. Already reasonably large
example. In a couple of minutes, find {\blue $P=(x,y)$} with
numerator and denominator of {\blue $x$} having up to {\blue $36$}
decimal digits.

\smallskip

For details on all of this, see student presentation.

\end{slide}
\begin{slide}
\ctl{\large{}\myblue{\sc Computation of integral points (I)}}

\medskip

Assume now that Mordell--Weil group {\blue $E(\Q)$} computed, say
{\blue $(P_i)_{1\le i\le r}$} generators.

\smallskip

{\red Goal}: compute {\blue $E(\Z)$}, i.e., {\bf integral points}.
Immediate warning: {\red depends on the chosen model}, contrary to
{\blue $E(\Q)$}.

If {\blue $P\in E(\Z)\subset E(\Q)$}, can write
{\blue $P=T+\sum_{1\le i\le r}x_iP_i$} with {\blue $x_i\in\Z$} and
{\blue $T$} a torsion point. Easy result is
{\blue $|x|\ge c_1e^{c_2H^2}$}, with {\blue $H=\max_i|x_i|$} and
{\blue $c_1$}, {\blue $c_2$} easily computable explicit constants.

\smallskip

Now use {\red elliptic logarithm} {\blue $\psi$} 
({\blue $E(\C)\isom\C/\Lambda$}, and {\blue $\psi$} maps {\blue $P\in E(\C)$}
to {\blue $z\in\C$} modulo {\blue $\Lambda$} such that 
{\blue $(\wp(z),\wp'(z))=P$}).

\end{slide}
\begin{slide}
\ctl{\large{}\myblue{\sc Computation of integral points (II)}}

\medskip

Consequence of above: easy to show that if {\blue $|x|\ge c_3$} explicit,
then {\blue $|\psi(P)|\le c_5e^{-c_2H^2/2}$} (if we choose 
{\blue $\psi(P)$} as small as possible.

\smallskip

On the other hand, thanks to a very important theorem of {\red S.~David}
on {\red linear forms in elliptic logarithms}, generalizing Baker-type
results to the elliptic case, can prove that we have an inequality
for {\blue $\psi(P)$} in the other direction, which {\bf contradicts} the
above for {\blue $H$} sufficiently large. Every constant {\bf explicit}.
Thus get {\bf upper bound} for {\blue $H$}, and as usual in Baker-type
estimates, very large. Typically find {\blue $H\le 10^{100}$}
(recall {\blue $P=T+\sum_{1\le i\le r}x_iP_i$} and {\blue $H=\max_i|x_i|$}).

\end{slide}
\begin{slide}
\ctl{\large{}\myblue{\sc Computation of integral points (II)}}

\medskip

In the above, essential that Baker bounds be {\bf explicit}, but {\bf not}
essential that they be {\bf sharp} (e.g., {\blue $10^{80}$} or {\blue
$10^{100}$} is just as good), because now we use the {\red magic} of the
LLL algorithm: find small vectors in lattices, and this allows you either
to find linear dependence relations between complex numbers, {\bf or} to
show that if an approximate relation exists then coefficients are {\bf bounded}
very effectively. Roughly obtain a {\bf logarithmic decrease} in the
size of the upper bound.

\end{slide}
\begin{slide}
\ctl{\large{}\myblue{\sc Computation of integral points (III)}}

\medskip

{\bf Example}: {\blue $y^2+y=x^3-7x+6$}, famous curve because elliptic curve
with smallest conductor of rank {\blue $3$}, important in obtaining 
effective lower bounds for the class number of imaginary quadratic fields
({\red Goldfeld, Gross--Zagier}). Using David's bounds, find
{\blue $H\le 10^{60}$}. Using LLL once, reduce this spectacularly to
{\blue $H\le 51$}. Using LLL a second time, reduce this to
{\blue $H\le 11$} (diminishing returns: another LLL gives {\blue $10$}, then
no improvement). Now a direct search very easy (less than {\blue $10000$}
trials), and find exactly {\blue $36$} integral points, a very large number.

Phenomenon not completely understood: elliptic curve of high rank with respect
to conductor have {\bf many} integral points.

\end{slide}

\end{document} 





\end{document} 



