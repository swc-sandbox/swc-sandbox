\documentclass{article}
\begin{document}
\begin{center}
{\Large\bfseries Project for ``$p$-adic modular forms'' lectures.}
\end{center}
\begin{flushright}
KMB, 6 Feb 2001
\end{flushright}
\bigskip
In this project I will sketch a method which I believe will give an
explicit formula for all the entries of a matrix representing
$U$ on overconvergent 2-adic modular forms of level~1 and
any even weight~$k$. This explicit formula will make it much easier
to say things about the overconvergent slopes in these cases---for
example it should lead very easily to explicit lower bounds for
Newton polygons. Even so, there are still plenty of conjectures, which will
give us lots of things to think about, even assuming we can get the
explicit formulae to come out.

\medskip

The genesis of the project is Lawren Smithline's PhD thesis. Before I read
this thesis, I had no idea that it was possible to compute explicit
matrices representing $U_p$ on the space of weight~$k$ overconvergent
modular forms. Now I've seen the idea, I realise that in fact it is not
too difficult. Smithline was concerned with the case~$p=3$ and tame level~1,
and weight~0. Using an extension and minor simplification of his ideas,
this project will attempt to deal with the case~$p=2$, tame level~1, and any
even integer weight.

\section*{Introduction and weight~0}

There is a unique Eisenstein series $E_2(q)=1+24q+\ldots$ of weight~2 and
level~2. It is explicitly given as $1+24\sum_{n\geq1}a_nq^n$,
where $a_n$ is the sum of the odd divisors of~$n$.
Unfortunately (at least for those of you who do not like computers)
it may occasionally be necessary to have to compute the $q$-expansions of
some other forms too. But there are programs to do this sort of thing,
and assuming I've brought my laptop to the conference, I will have
access to them, so I could do any computations required.

Using standard facts about $p$-adic modular forms, check that
$E_2(q)$ and $E_2(q^2)$ are both overconvergent $2$-adic modular
forms of level~1 and weight~2. Check furthermore that $E_2(q^2)$
has no zeros on the ordinary locus. Hence the ratio $t=E_2(q)/E_2(q^2)$
is a $p$-adic modular form of weight~0. It will be an~$r$-overconvergent
form for some $r<1$---can you say anything about~$r$? This is not
necessary, but it would be nice. I think it might be possible
to compute some explicit bounds but I'm not sure. It might involve
some tricks with the $j$-invariant.

Now $t-1$ can be thought of as an overconvergent function on some
strict neighbourhood $X$ of the ordinary locus at level 1. Why will $t$
have a unique zero on~$X$ (assuming~$X$ is a small enough strict
neighbourhood)?
The zero is clearly at the cusp. The region $|t-1|\leq B$ will define
a disk~$D$ in~$X$, for appropriate~$B$. Let $d$ be an appropriate
multiple of~$t-1$, so that
the disk~$|d|\leq1$ does indeed define a strict neighbourhood of
the ordinary locus, which is preserved by the~$U$ operator. You probably
have a range of choices for the normalisation actually, each of which
will give slightly different discs, but the theory should be independent
of these choices. Can you explicitly say, given your choices, that
functions on~$D$ are $r$-overconvergent functions for some explicit~$r<1$?

Now for the fun part. An explicit basis for modular forms on~$D$
is just $\{1,d,d^2,d^3,\ldots\}$. What is $U(d^n)$ in terms of
this basis? A computer might be helpful, but in fact the actual amount
of computer calculations you have to do is minimal if you see all the
tricks. Now one can see the matrix for~$U$ in weight~0.

Warm-up question: can you get sufficiently good bounds on the elements
of this matrix to deduce a quadratic lower bound for the Newton polygon
of the characteristic power series of $U$ in weight~0? Smithline indicates
how to do this in his thesis, although he uses a different choice of~$d$.

\section*{Interlude: a conjecture about weight~0 slopes.}

I conjecture that the valuations of the eigenvalues of~$U$ on 2-adic
cuspidal overconvergent forms of weight~0 are 3,7,13,15,17,\ldots. I will
explain how to generate this sequence. First of all start with
the sequence $2,?,?,2,2,?,?,2,2,?,?,2,\ldots$, where the ?s are
unknown entries that we will work out in due course. This sequence
could be described as ``2, then alternate double-?s and double-2s''.

Now fill in the question marks, using the sequence ``4, and then
alternate double-?s and double 4s'', and simply skipping over
the terms we know already. We get the sequence
$2,4,?,2,2,?,4,2,2,4,?,2,2,?,4,2,\ldots$.

Now do the same thing with 6: fill in the remaining question marks
with the sequence ``6, and then alternate double-?s and double-6s''.

Eventually the sequence goes $2,4,6,2,2,8,4,2,2,4,10,2,\ldots$.
Now form the sequence $1,3,7,13,15,17,25,\ldots$ whose successive
differences are the terms in our $2,4,6,2\ldots$ sequence. Throw away
the first 1 and you get what I firmly believe should be the cuspidal slopes
at weight~0. Can you prove that this is the case?

\section*{Other weights.}

Now we have done all this work in weight~0, it is surprisingly easy
to work out a matrix for~$U$ at weight~$k=2m$ for all integers~$m$.
Firstly you have to check that you can use $E_2(q)^md^n$ for $n=0,1,2,\ldots$
as a basis. Then you have to work out $U(E_2(q)^md^n)$ in terms of this basis.
Hence you can work out a matrix representing $U$ in weight~$2m$.
Again the warm-up question: can you work out an explicit quadratic
lower bound for the Newton polygon of the characteristic power series?

I have conjectures for the valuations of the eigenvalues of~$U$ for
arbitrary even weights. They are elementary, but messy, to explain and I won't
do it here (bug me for a preprint). But here are some consequences of these
conjectures,
all of which are, I believe, open. I will ignore the ``Eisenstein
slope'', the 0 which occurs at every weight, here, so ``slope'' below
refers to the slopes of the overconvergent cusp forms, a ``formula''
for which you have computed above. So here are the conjectures.

\section*{Lots of conjectures.}

I have no idea how to do any of these. Do the calculations you have
so far help?

\begin{itemize}
\item All slopes at all even weights are integers.

\item At weight $-2$ all slopes occur with multiplicity precisely~2. Similarly
at weight~4---except for the first slope, which is~3, all other
slopes occur with multiplicity~2. These weights are related by the theta
operator, so the two observations are basically equivalent.

\item At weight~0 and~4 all slopes are odd. At weights 2 and -2 all slopes
are even (again some of these things imply others).

\item The $n$th slope is between~$3n$ and~$6n$, and furthermore these
bounds are obtained infinitely often. In particular there is a
quadratic upper bound for the Newton polygon.

\item At weight~$k>0$ there are no forms of slope~$s$ for~$s$
in the range $(k/3,2k/3)$, other than forms of slope $(k-2)/2$
(these bounds might be slightly wrong, but I think they're OK.
The idea that there might be a hole here is basically due to Gouvea.)

\item The Gouvea-Mazur conjectures: if $k$ and $k'$ are congruent mod~$2^n$
then the number of times a slope $s<n$ shows up at weights~$k$ and~$k'$
should be the same. This is basically known if one replaces $2^n$
by something like $2^{n^2}$, but is still open in the form above.

\end{itemize}
\end{document}


