\documentclass[12pt]{article}

\usepackage{amsmath}

\newcommand{\nc}{\newcommand}
\nc{\abs}[1]{\left| #1 \right|}
\nc{\en}{\enspace}
\nc{\qbold}[1]{\boldsymbol{\mathsf{#1}}}
\nc{\ra}{\rightarrow}
\nc{\x}{\times}

\nc{\cO}{\mathcal{O}}
\nc{\C}{\qbold{C}}
\nc{\F}{\qbold{F}}
\nc{\Q}{\qbold{Q}}
\nc{\Z}{\qbold{Z}}

\begin{document}

\begin{center}
\Large \textbf{$P$-adic Modular Forms by Kevin Buzzard\\
Lecture 2}
\end{center}

\paragraph{$\bullet$}
We know modular forms exist because over $\C$ there exists an explicit
example (according to the classical definition), namely the Eisenstein
series: $E_k : \tau \mapsto \sum_{\substack{m,n\in\Z\\ \text{not both }0}}
\frac{1}{(m\tau + n)^k}, k \geq 4$.
This is a level $1$ modular form of weight $k$.
A known fact is that $E_{p-1} \equiv 1 \pmod{p}$, where $p$ is prime and $p
\geq 5$.
So in the ring $\Z_p[[q]]$, $E_{p-1}$ ``looks'' invertible becuase it has a
power series where the constant term is a unit.
However, no classical forms are of negative weight so $1/E_{p-1}$ is not a
classical form.
In addition, for example, $E_4(\pi i/3) = E_4(i) = 0$ so $1/E_4$ is not a
holomorphic function on the upper half plane.

\paragraph{$\bullet$}
The following is the Deligne/Katz approach to defining a $p$-adic modular
form.
\\

We have $E/R$, where $E$ is an elliptic curve over $R$, an $\F_p$-algebra.
Now let $\omega \in H^0(E, \Omega^1_{E/R})$ and $\eta \in H^1(E,\cO_E)$ be
its dual.
Consider the absolute Frobenius map, $F_{\text{abs}} : \cO_E \ra \cO_E,\en f
\mapsto f^p$, an additive homomorphism of sheaves of abelian groups.
Now define $A(E/R,\omega) \in R$, (which is actually the Hasse invariant) by
setting $F_{\text{abs}}^*(\eta) = A(E/R,\omega) \cdot \eta$ which gives us
that $A(E/R, \lambda \omega) = \lambda^{1-p} A(E/R, \omega), \en \lambda \in
R^{\x}$.

So $A$ is a modular form of weight $p-1$.
Note that the boundedness condition for a modular form is satisfied by
looking at $A(\text{Tate}(q),\omega_{\text{can}})$ since the restriction of
a plane curve over $\F_p[[q]]$ is the Tate curve over $\F_p((q))$.
Now $A(\text{Tate}(q),\omega_{\text{can}}) = 1$ so if $p \geq 5$, then
$E_{p-1} \equiv A \pmod{p}$.
Therefore $A = E_{p-1} \pmod{p}$ by the $q$-expansion principle, which says
that two modular forms of level $1$ and the same weight are equal if they
have the same $q$-expansion.

We want a $p$-adic theory of modular forms that strongly identifies a
modular form with its $q$-expansion so that what ``looks'' invertible, as in
$E_{p-1} \pmod{p}$, is invertible.
So because the Hasse invariant $A(E/R,\omega) = 0$ if and only if $E$ is
supersingular, we want to somehow throw away elliptic curves which are
supersingular or have supersingular reduction.

\paragraph{$\bullet$}
Katz's definition of a $p$-adic modular form:\\
Let $p \geq 5$, $R_0$ be the ring of integers in a finite extension of
$\Q_p$, and $R$ an $R_0$-algebra in which $p$ is nilpotent.
\\
A test object is

\begin{minipage}[c]{6in}
1.\en an elliptic curve $E/R$\\
2.\en a nowhere-vanishing differential $\omega \in H^0(E,\Omega^1_{E/R})$\\
3.\en an element $Y \in R$ such that $Y \cdot E_{p-1}(E/R,\omega) = 1 \in
R$.
\end{minipage}
\\
\\
A \emph{$p$-adic modular form} of level $1$ of weight $k$ defined over $R_0$
is a rule $f$ sending $(E/R,\omega,Y)$ to an element of $R$ such that

\begin{minipage}[c]{6in}
a)\en $f(E/R,\lambda \omega, \lambda^{p-1} Y) = \lambda^{-k}
f(E/R,\omega,Y)$\\
b)\en $f(\text{Tate curve over }R_0/p^n R_0((q)),\omega_{\text{can}},1)$ is
in $R_0/p^n R_0[[q]]$ for all $n \geq 1$.
\end{minipage}
\\
and $f$ depends only the isomorphism class of data and behaves well under
pullback.
\\
\\
Note that classical modular forms over $R_0$ are already $p$-adic modular
forms.

\paragraph{$\bullet$}
The class of $p$-adic modular forms is too large to work with, but we can
create subtlety through the following definition:
\\
Let $R_0$ be the ring of integers in a finite extension of $\Q_p$ and choose
$\rho\in R_0\setminus 0$, and $R$ an $R_0$-algebra in which $p$ is
nilpotent.
\\
A $\rho$-overconvergent test object is

\begin{minipage}[c]{6in}
1.\en an elliptic curve $E/R$\\
2.\en a nowhere-vanishing differential $\omega \in H^0(E,\Omega^1_{E/R})$\\
3.\en an element $Y \in R$ such that $Y \cdot E_{p-1}(E/R,\omega) = \rho$.
\end{minipage}
\\
\\
A \emph{$\rho$-overconvergent modular form} is a rule on these test objects
satisfying the conditions (a) and (b) in the definition of a $p$-adic
modular form.
\\
\\
Note that if $\abs{\rho} < 1$ then some of these test objects might have
supersingular geometric fibers.

\paragraph{$\bullet$}
We don't want to throw away too much so the idea is to define $E$, an
elliptic curve, as having ``very supersingular reduction'' if
$A(E\!\!\!\pmod{p}\;/\; R/pR ) = 0 \in R/pR$.
Now if $R$ is the ring of integers in a highly ramified extension of $\Q_p$
then $R/pR$ could be huge so maybe there are a lot of elliptic curve whose
reductions are supersingular but not very supersingular.


\end{document}


