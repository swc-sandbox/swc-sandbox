\documentstyle[12pt]{article}
\title{Outline of lectures on Model Theory and
Diophantine Geometry, by Anand Pillay and
Thomas Scanlon}
\date{ }
\newtheorem{Theorem}{Theorem}[section]
\newtheorem{Proposition}[Theorem]{Proposition}
\newtheorem{Definition}[Theorem]{Definition}
\newtheorem{Remark}[Theorem]{Remark}
\newtheorem{Lemma}[Theorem]{Lemma}
\newtheorem{Corollary}[Theorem]{Corollary}
\newtheorem{Fact}[Theorem]{Fact}
\newtheorem{Conjecture}[Theorem]{Conjecture}
\begin{document}
\maketitle

\section{Course description.}
The general theme will be the use of
model-theoretic methods in proofs of the
Manin-Mumford conjecture and variants.
\newline
If $K$ is an algebraically closed field, $G$ a
commutative algebraic group over $K$ and
$\Gamma$ an abstract subgroup of $G(K)$, we
will say that $\Gamma$ is of {\em Lang type},
if for any $n$ and subvariety $X$ of $G^{n}$,
$X(K)\cap \Gamma^{n}$ is a finite union of
cosets of subgroups of $\Gamma^{n}$. The
Manin-Mumford conjecture says that if the
characteristic is $0$ and $G$ is a semiabelian
variety, then the group $Tor(G)$ of torsion
points of $G(K)$ is of Lang type.
\newline
Hrushovski \cite{Hrushovski-MM} gave a proof of
the Manin-Mumford conjecture, using the model
theory of difference fields \cite{C-H}.
Scanlon
\cite{Scanlon} proved ``Manin-Mumford for
Drinfeld modules" in positive characteristic
(see below), also using the model theory of
difference fields \cite{C-H-P}.

\vspace{5mm}
\noindent
Pillay's lectures will contain a simplified
and self-contained account of
Hrushovski's proof of Manin-Mumford in the
spirit of \cite{Pillay}.  A key step will be an
elementary proof of the following algebraic
result, valid in all characteristics:
\begin{Proposition} Let $A$ be a semiabelian
variety and $X$ be an irreducible subvariety of
$A$ containing $0$ which generates $A$. Assume
$Stab_{A}(X)$ is trivial. Let
$\phi$ be a separable isogeny of $A$ such that
$\phi(X) = X$. Then for some $n$, $\phi^{n}$
is the identity.
\end{Proposition}
The methods, coming from
\cite{Pillay-Ziegler}, are closely related to
those in recent preprints of Pink and Roessler
(\cite{Pink-Roessler1},
\cite{Pink-Roessler2}), but definability in
difference fields plays a simplifying role.
\newline
Pillay will also discuss how and why the
statements of Manin-Mumford and
Mordell-Lang, together with uniform
definability of types in algebraically closed
fields, automatically yield uniformities as
the subvariety $X$ of $A$ varies in an
algebraic family.

\vspace{5mm}
\noindent
Scanlon's talks will be around the
Drinfeld modules version of
Manin-Mumford. Let
$K$ be a field of characteristic $p>0$. The
ring of endomorphism of the additive group
${\bf G}_{a}$ defined over $K$ can be
identified with the ``twisted" polynomial ring
$K\{\tau\} = \{a_{0} + a_{1}\tau + .. +
a_{n}\tau^{n}: n\in {\bf N}, a_{i}\in K\}$,
where
$\tau$ acts as
$x\rightarrow x^{p}$.
$\pi:K\{\tau\}\rightarrow K$ takes a polynomial
$f$ as above to its constant term $a_{0}$. By
a {\em Drinfeld module} we will mean a
homomorphism
$\phi:{\bf F}_{p}[t]\rightarrow K\{\tau\}$
such that $\phi(t)\notin K$. The Drinfeld
module $\phi$ equips the additive group of
any $K$-algebra with an ${\bf
F}_{p}[t]$-module structure. The Drinfeld
module
$\phi$ is said to have {\em generic
characteristic} if the kernel of
$\pi\circ\phi$ is $(0)$.

\vspace{2mm}
\noindent
Denis \cite{Denis} raised a series of
conjectures for Drinfeld modules, analogous to
the Manin-Mumford, Mordell-Lang conjectures.
Scanlon will explain the proof of the Drinfeld
module version of Manin-Mumford:
\begin{Proposition} Let $K$ be an algebraically
closed field of characteristic
$p>0$, and
$\phi:{\bf F}_{p}[t]\rightarrow K\{\tau\}$ a
Drinfeld module of generic characteristic. Then
The ${\bf F}_{p}[t]$-torsion submodule of
$(K,+)$ is of Lang type.
\end{Proposition}
Scanlon will also explain how the Drinfeld
module version of Mordell-Lang with division
points follows from the Drinfeld module
version of Mordell-Lang without division
points.


\vspace{5mm}
\noindent
{\bf PREREQUISITES.} Familiarity with the
language of algebraic geometry, as in say
Shafarevich \cite{Shafarevich}, as well as
with first order model theory. An acquaintance
with the first five  chapters of
\cite{Bouscaren}, by Bouscaren, Ziegler,
Lascar, Pillay, and Hindry, respectively,
would be helpful. A good reference on Drinfeld
modules is Goss' book \cite{Goss},
specifically pages 63-92 in Chapter 4.


\section{Project}
Theorem 3.1 of \cite{Pink-Roessler2} gives
a generalization of Proposition 1.1 above to
the case where $\phi$ is not necessarily
separable. One case of this generalization is:
\begin{Proposition} (char = $p > 0$.) Let $A$
be a semiabelian variety, $\phi$ an isogeny of
$A$, and $X$ a
$\phi$-invariant subvariety of $A$ containing
$0$. Assume that $X$ generates $A$. Let
$Frob$ the the Frobenius map, and suppose that
for positive integers
$r,s$,
$\phi^{s}$ is the composition of $Frob^{r}$
with a separable isogeny from $Frob^{r}(A)$ to
$A$. Then $A$ can be defined over ${\bf
F}_{p^{r}}$ and $\phi^{s} = Frob^{r}$ on $A$.
\end{Proposition}
Project 1 will to be generalize the proof of
Proposition 1.1 in the lectures to a proof of
Proposition 2.1

\vspace{5mm}
\noindent
Project 2 will be to prove a ``function field"
version of Denis' Mordell-Lang conjecture. We
call Drinfeld modules $\phi$ and $\psi$
equivalent if there is a scalar $\lambda$ such
that $\psi(t) = \lambda^{-1}\phi(t)\lambda$.
We define the modular transcendence degree of
$\phi$ to be the least transcendence degree of
a field $L$ such that $\phi$ is equivalent to
a Drinfeld module $\psi$ defined over $L$. The
project is to prove the following conjecture.
\begin{Conjecture} Let $K$ be an algebraically
closed field of characteristic $p>0$, $\phi$ a
Drinfeld module over $K$ of positive
modular transcendence degree with
$\pi\circ\phi(t) = 0$, and $\Gamma \leq (K,+)$
a finitely generated ${\bf F}_{p}[t]$-module.
Then $\Gamma$ is of Lang type.
\end{Conjecture}
It should be possible to prove Conjecture 2.2
along the lines of Hrushovski's proof
\cite{Hrushovski-ML} of the function field
Mordell-Lang. That is, working in a
suitable separably closed field
$L$ of finite Ersov invariant over which the
data are defined, let
$\phi^{\sharp}(L)$ the type-definable group
$\cap_{n}\phi(t)^{n}(L)$ and one has to prove
the modularity of $\phi^{\sharp}(L)$. By
\cite{Bouscaren-Delon}, this is equivalent to
the nonexistence of a definable isogeny
between $\phi^{\sharp}(L)$ and
$L^{p^{\infty}}$.

\begin{thebibliography}{99}
\bibitem{Bouscaren} E. Bouscaren (ed.), Model
Theory and Algebraic Geometry, Lecture Notes
Mathematics, 1696, Springer, 1998.

\bibitem{Bouscaren-Delon} E. Bouscaren and F.
Delon, Minimal groups in separably closed
fields, Jpurnal of Symbolic Logic, 67 (2002),
239-259.

\bibitem{C-H} Z. Chatzidakis
and E. Hrushovski, Model theory of difference
fields, Transactions AMS, 351 (1999),
2997-3071.

\bibitem{C-H-P} Z. Chatzidakis, E. Hrushovski,
and Y. Peterzil, Model Theory of difference
fields II: periodic ideals and the
trichomotomy theorem, Proceedings of London
Math. Soc. 85(2002), 257-311.

\bibitem{Denis} L. Denis, Geometrie
diophantine sur les modules de Drinfeld, in
The Arithmetic of Function Fields (ed. D.
Gross et al) 1992.

\bibitem{Goss} D. Goss, Basic Structures of
Function Field Arithmetic, Springer, 1996.

\bibitem{Hrushovski-MM} E. Hrushovski, The
Manin-Mumford conjecture and the model theory
of difference fields, Annals of Pure and
Applied Logic, 112 (20001), 43-115.

\bibitem{Hrushovski-ML} E. Hrushovski, The
Mordell-Lang conjecture for function fields,
Journal AMS 9 (1996), 667-690.

\bibitem{Pillay} A. Pillay, Mordell-Lang for
function fields in characteristic zero,
revisited, to appear in Compositio Math. (See
``recent preprints" at
http://www.math.uiuc.edu/People/pillay.html)

\bibitem{Pillay-Ziegler} A. Pillay and M.
Ziegler, Jet spaces of varieties over
differential and difference fields. (See
``recent preprints" at
http://www.math.uiuc.edu/People/pillay.html)

\bibitem{Pink-Roessler1} R. Pink and D.
Roessler, On Hrushovski's proof of the
Manin-Mumford conjecture. (See ``recent
preprints" at
http://www.math.ethz.ch/$\tilde{\ }$pink/preprints.html)

\bibitem{Pink-Roessler2} R. Pink and D.
Roessler, On $\psi$-invariant subvarieties of
semiabelian varieties and the Manin-Mumford
conjecture (See ``recent preprints" at
http://www.math.ethz.ch/$\tilde{\ }$pink/preprints.html)

\bibitem{Scanlon} T. Scanlon, Diophantine
geometry of the torsion of a Drinfeld module,
Journal of Number Theory, vol. 97, Number 1,
(2002), 10-25.

\bibitem{Shafarevich} I. Shafarevich,
Algebraic Geometry I, II, Springer-Verlag,
1994.
\end{thebibliography}


\end{document}