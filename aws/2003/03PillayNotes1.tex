\documentstyle[12pt]{article}
\title{Lectures 1 and 2 of ``Model Theory and
Diophantine Geometry", Arizona Winter School,
2003}
\author{Anand Pillay\\University of Illinois
at Urbana-Champaign}
\newtheorem{Theorem}{Theorem}[section]
\newtheorem{Proposition}[Theorem]{Proposition}
\newtheorem{Definition}[Theorem]{Definition}
\newtheorem{Remark}[Theorem]{Remark}
\newtheorem{Lemma}[Theorem]{Lemma}
\newtheorem{Corollary}[Theorem]{Corollary}
\newtheorem{Fact}[Theorem]{Fact}
\newtheorem{Conjecture}[Theorem]{Conjecture}
\begin{document}
\maketitle

\section{Introduction}
These notes are for the first two lectures of
the lecture series "Model Theory and
Diophantine Geometry" by Thomas Scanlon and
myself. The full series of lectures will be on
the Manin-Mumford conjecture and variants.
As is well-known, model-theoretic ideas,
specifically definability theory in difference
fields of characteristic zero, were used by
Hrushovski \cite{Hrushovski-MM} to give another
proof of the Manin-Mumford conjecture
concerning the intersection of  subvarieties
of a (semi)-abelian variety $A$ with the group
of torsion points of $A$. Subsequently Scanlon
\cite{Scanlon} used the positive characteristic
difference field theory to prove Denis'
conjecture, a version of Manin-Mumford for
Drinfeld modules (and this is the only known
proof). Scanlon will present his proof (in
lectures 3-5), which will use the whole
model-theoretic machinery. I will give a
self-contained ``new" proof (in lecture 2) of
Manin-Mumford, avoiding model theory other
than some elementary properties of
existentially closed difference fields.


I would like to thank Piotr Kowalski for many
comments and suggestions which helped me
considerably in preparing these notes.


\section{Difference fields and differential
fields}
The main purpose of this section is to
introduce existentially closed difference
fields which will play a central role in most
of the lectures in our series.

I will use freely basic notions of first
order logic: theories, complete
theories, formulas, definable sets, the notion
of being definable {\em over} a given set of
parameters, quantifier-elimination, and
sometimes saturated models.

The
main first order theories which we shall
consider here are
$ACFA$, the theory of existentially closed
fields with an automorphism, and its
completions.

As a motivating and somewhat ``easier"
example, we will also discuss
$DCF_{0}$, the theory of differentially
closed fields of characteristic zero.

\vspace{5mm}
\noindent
In the background is the theory $ACF$ of
algebraically closed fields. The language of
$ACF$ is that of rings $(+,-,\cdot,0,1)$, and
$ACF$ has quantifier elimination in this
language. The completions are obtained by
fixing the characteristic. It is often
convenient to regard the objects of algebraic
geometry (varieties, morphisms,..) as definable
sets and functions in an algebraically closed
field. This is not so much due to
ignorance of the scheme-theoretic point of
view, but is rather because this presentation
is directly amenable to current model-theoretic
methods. The point of view essentially
coincides with that of Weil's Foundations. That
is, we work in an algebraically closed field
$K$ of uncountable transcendence degree. An
affine variety $V$ is a subset of $K^{n}$
defined by finitely many polyomial equations.
If $k$ is a countable subfield over which $V$
is defined, then a generic point of $V$ over
$k$ is a point $a\in V$ such that
$tr.deg(k(a)/k) = dim(V)$. The assumptions on
$K$ ensure that such generic points exist.
For irreducible $V$ defined over $k$, any given
$k$-constructible property will hold on a
Zariski open subset of $V$ if and only if it
holds at a generic point of $V$ over $k$.
This formalism passes over to abstract
varieties in the sense of Weil.

\vspace{5mm}
\noindent
The notion of an existentially closed
structure in a given class of structures is
rather central: Let $T$ be a $\forall\exists$
theory in a language $L$. An ec structure for
$T$ is an $L$-structure $M$ which is a
substructure of a model of $T$ and such that
whenever $M\subseteq N\models T$, and
$\phi(x_{1},..,x_{n})$ is a quantifier-free
firmula with parameters from $M$ which has a
solution in
$N$, then $\phi(x_{1},..,x_{n})$ has a solution
in
$M$. An ec structure for $T$ will actually be
itself a model of $T$. In many cases, the
class of ec structures for $T$ is
axiomatizable, namely is the class of models
of a first order
$L$-theory $T'$ containing $T$. In that case
we say that $T'$ is the model companion of
$T$. $T'$ has the feature of being
model-complete (any $L$-formula is equivalent
mod
$T$ to both an existential and universal
formula). Again in many situations $T'$ has
outright quantifier-elimination. In any case,
saturated models of $T'$ will be ``universal
domains" for studying models of the original
theory $T$.

Let us start by briefly describing $DCF_{0}$,
as here the relationship between definability
and geometry is clearest. A derivation
$\partial$ on a field $K$ is an additive
homomorphism such that $\partial(x\cdot y) =
\partial(x)\cdot y + x\cdot\partial(y)$. A
differential field is a field equipped with a
derivation. If $(K,\partial)$ is such, then a
differential polynomial over $K$ in
indeterminates $x_{1},..,x_{n}$ is a
polynomial over $K$ in indeterminates
$x_{1},..,x_{n},\partial(x_{1}),..,
\partial(x_{n}),\partial^{2}(x_{1}),..,
\partial^{2}(x_{n}),...$. If $P(x)$ is a
differential polynomial over $K$ in the single
indeterminate $x$ then the order of $P$ is the
greatest $n$ such that $\partial^{n}(x)$
appears nontrivially in $P$. An (affine)
differential algebraic variety is the zero set
of a finite number of differential polynomials.
There are associated notions of differential
polynomial map and differential rational map.


$DF_{0}$ denotes the
theory of differential fields (fields equipped
with a derivation) of characteristic $0$ in
the language of rings together with a symbol
$\partial$ for the derivation. $DF_{0}$ {\em
does} have a model companion, and this is the
(complete) theory $DCF_{0}$, the theory of
differentially closed fields of characteristic
zero. The easiest to state axioms for
$DCF_{0}$ are those found by L. Blum, and
concern differential polynomial equations in a
single differential indeterminate $x$: if
$f,g$ are differential polynomials over $K$ and
the order of
$f$ is strictly greater than the order of $g$
then the system
$f(x) = 0, g(x)\neq 0$ has a solution in $K$.
An important fact is that $DCF_{0}$ has
quantifier-elimination in the given language.
So definable subsets of $K^{n}$ are Boolean
combinations of ``differential
algebraic varieties", and definable maps are
piecewise differential rational
functions. Although this will not play
a role in these first lectures, it is
worth mentioning another important
model-theoretic property of $DCF_{0}$: it is
$\omega$-stable. This means that if $K$ is a
countable differentially closed field, then
the number of complete $n$-types over $K$ is
countable. The $\omega$-stability of a theory
gives rise to an intrinsic ordinal valued
dimension, called Morley rank, which can be
assigned to formulas or definable sets in
models of the theory. Roughly speaking, a
definable set $X$ has Morley rank $0$ if it
is finite, and Morley rank $\geq\alpha + 1$
if there are pairwise disjoint definable
subsets $X_{i}$ of $X$ for $i=1,2,...$ such
that each $X_{i}$ has Morley rank
$\geq\alpha$.

Let us fix $K =
(K,+,\cdot,\partial)$ a saturated
differentially closed field, which note is
also an algebraically closed field. Let
$X\subset K^{n}$ be definable over a
differential subfield $k$. We call $X$
finite-dimensional if there is a finite bound
on
$tr.deg.(k(a,\partial(a),\partial^{2}(a)...)/k)$.
for $a\in X$. The ``category" of
finite-dimensional definable sets in $K$ and
definable maps between them, is what was
relevant and useful for issues such as the
geometric version of Mordell-Lang. $X$ is
finite-dimensional if and only if the Morley
rank of $X$, $RM(X)$ is finite. Among such
sets is the field ${\cal C}$ of constants of
$K$, another algebraically closed field. The
structure induced on
${\cal C}$ from living
in $(K,+,\cdot,\partial)$ is just the field
structure. So our category of finite Morley
rank sets includes ``algebraic geometry", but
is much richer.

It is worth mentioning a rather more geometric
way of interpreting finite Morley rank sets in
$DCF_{0}$. Let $V\subseteq K^{n}$ be an
irreducible algebraic variety. The first
prolongation $\tau(V)$ of $V$ will be the
subvariety of $K^{2n}$ given by the
defining equations for $V$ (in the first $n$
coordinates), as well as the equations
\newline
$\sum_{i=1,..,n}(\partial
P/\partial x_{i})(x_{1},..,x_{n})v_{i} +
P^{\partial}(x_{1},..,x_{n})$.
\newline
as $P$ ranges over generators of the ideal of
$V$.
\newline
Here $P^{\partial}$ is the result of hitting
the coefficients of $P$ with the derivation
$\partial$.
\newline
If $V$ is defined over the constants, then
$\tau(V)$ is precisely the tangent bundle of
$V$. In any case $\tau(V)$ comes with a
canonical projection $\pi:\tau(V)\rightarrow
V$. Let $s:V\rightarrow \tau(V)$ be a map (in
the algebraic-geometric sense) which is also a
section of $\pi$. Pairs of the form $(V,s)$
are precisely the ``algebraic $D$-varieties" of
Buium, and belong to algebraic geometry.

An important fact about differentially closed
fields, is that in the above situation,
$(V,s)^{\sharp} =_{def} \{a\in V(K): s(a) =
(a,\partial(a))\}$ is {\em Zariski-dense} in
$V$. Moreover, up to finite Boolean
combination, every definable set of finite
Morley rank in $K$ is of the form
$(V,s)^{\sharp}$.

\vspace{5mm}
\noindent
Let us now turn to difference fields. Fix a
language
$L_{\sigma}$ consisting of the language of
rings $L = \{+,\cdot,-,0,1\}$ together with a
unary function symbol $\sigma$.
$ACF_{\sigma}$ is the
$\forall\exists$
$L_{\sigma}$-theory expressing that $K$ is an
algebraically closed field and $\sigma$ is an
automorphism. ($K$ being a ring and $\sigma$
being an endomorphism would also be enough.)
We call a model $(K,\sigma)$ of $ACF_{\sigma}$
a difference field.

It turns out that $ACF{_\sigma}$ does have a
model companion, namely the class of
existentially closed difference fields is
axiomatizable. In fact the additional axioms
are precisely:
\newline
(*) for any irreducible variety $V$ over $K$,
and any irreducible variety $W\subset
V\times\sigma(V)$ defined over $K$ which
projects dominantly onto both $V$ and
$\sigma(V)$, there is $a\in V(K)$ such that
$(a, \sigma(a))\in W$.

\vspace{2mm}
\noindent
To actually write down the axioms requires
quantifying over the coefficients in the
defining polyomials of $V$ and $W$ as well as
knowing that things such as "irreducibility"
are constructible conditions.

In any case the resulting theory is usually
called $ACFA$. $ACFA$ is not complete, even
after fixing the characteristic. The
completions are determined by the isomorphism
type of the algebraic closure of the prime
field considered as a difference field with
the retriction of $\sigma$.

Again we have the notion of a difference
polynomial over a difference field
$(K,\sigma)$: a polynomial over $K$ in
indeterminates $x_{1},..,x_{n},\sigma(x_{1}),..
.,\sigma(x_{n}), \sigma^{2}(x_{1}),..$.
A difference-algebraic variety is a subset of
$K^{n}$ defined by finitely many difference
polynomial equations.
So the quantifier-free definable sets
in difference fields are Boolean combinations
of difference-algebraic varieties.
$ACFA$ does {\em not} have
quantifier-elimination, but as for
pseudofinite fields, it is rather close. Any
$L_{\sigma}$-formula $\phi(x_{1},..,x_{n})$ is
equivalent, modulo $ACFA$ to a formula of the
form
$\exists
y(\theta({\bar\sigma}(x_{1}..,x_{n}),
{\bar\sigma}(y))$ where ${\bar \sigma}({\bar
x})$ denotes $({\bar x},\sigma({\bar
x}),..,\sigma^{m}({\bar x}))$ for some $m$,
$\theta$ is a quantifier-free $L$-formula (that
is a formula in the language of rings) and
$\theta({\bar z},{\bar w})$ implies that
${\bar w}$ is (field-theoretically) algebraic
over ${\bar z}$.

If $(K,\sigma)$ is a model of $ACFA$, the fixed
field
$Fix(\sigma)$ of
$\sigma$ in
$K$ is a pseudofinite field (namely an
infinite model of the theory of finite fields).
In fact the structure
$(Fix(\sigma),+,\cdot)$ is {\em strongly stably
embedded} in the structure
$(K,+,\cdot,\sigma)$. Namely,
if
$X\subset Fix(\sigma)^{n}$ is definable with
parameters in the structure
$(K,+,\cdot,\sigma)$, then $X$ is definable in
the structure $(Fix(\sigma),+,\cdot)$ with
parameters.

$ACFA$ (or rather its completions) are not
$\omega$-stable. They have a somewhat weaker
model-theoretic property, {\em
supersimplicity}, which will appear in later
lectures. On the other hand $ACFA$ {\em is}
quantifier-free $\omega$-stable, meaning that
over any countable model there are only
countably many complete quantifier-free types.
This  yields an ordinal-valued Morley rank for
quantifier-free formulas, defined as above,
but restricted to quantifier-free formulas.

Let us now fix a (saturated) model
$(K,+,\cdot,\sigma)$ of $ACFA$, and
definability will mean definablity in this
structure (with parameters). Let
$X$ be definable over a difference subfield
$k$. We say that $X$ is finite-dimensional if
there is a finite bound on
$tr.deg(k(a,\sigma(a),\sigma^{2}(a),..)/k)$
for $a\in X$. If $X$ is quantifier-free
definable (for example a difference-algebraic
variety), then $X$ will be finite-dimensional
if and only $X$ has finite
quantifier-free Morley rank.

The interesting category for us will be
that of finite rank difference-algebraic
varieties. Among the sets here are the fixed
fields $Fix(\sigma^{n})$ in characteristic
zero, and more generally
$Fix(\sigma^{r}\circ Fr^{m})$ in positive
characteristic where
$Fr$ is the Frobenius automorphism.

As in the
$DCF_{0}$ case, quantifier-free definable sets
of finite rank have canonical representations
up to Boolean combination.  Let $V$ be an
irreducible variety, and let $W$ be a
correspondence between $V$ and the variety
$\sigma(V)$. This means that $W$ is an
irreducible algebraic subvariety of
$V\times\sigma(V)$ inducing a generic
finite-to-finite correspondence between $V$
and $\sigma(V)$. Given such data $V$ and
$W$, let $(V,W)^{\sharp}$ denote
$\{a\in V(K): (a,\sigma(a))\in W\}$. Then the
finite rank difference-algebraic varieties are
essentially sets of the form $(V,W)^{\sharp}$.
The axioms for $ACFA$ imply that
$(V,W)^{\sharp}$ is Zariski-dense in $V$. Note
the following special case of $(V,W)$: $V$ is
defined over $Fix(\sigma)$ (so $\sigma(V) =
V$), and $W$ is the graph of a dominant
morphism $\phi:V\rightarrow V$. So the
classification of finite rank difference
varieties in a sense subsumes the
algebraic geometrical classification of such
pairs
$(V,\phi)$.

Let us elaborate on this latter category
somewhat, and state and prove an easy result
which will be used later. $(K,\sigma)$ is as
before a (saturated) model of $ACFA$, in
particular a universal domain for algebraic
geometry.  By a ``algebraic
$\sigma$-variety" we will mean, for now, an
irreducible algebraic variety $V$ equipped
with a dominant morphism
$\phi:V\rightarrow \sigma(V)$. (Note this is
an object of algebraic geometry, and is not
the same thing formally as a finite rank
difference-algebraic variety.) By a
$\sigma$-morphism between
$(V,\phi)$ and $(W,\psi)$ we mean a morphism
$f:V\rightarrow W$ (in the sense of algebraic
geometry) such that
$\sigma(f)\circ\phi = \psi\circ f$ on $V$. By a
$\sigma$-rational map between
$(V,\phi)$ and $(W,\psi)$ we mean a
rational (not everywhere defined) map from $V$
to $W$ such that for generic $a\in V$,
$\sigma(f)(\phi(a)) = \psi(f(a))$. By an
algebraic $\sigma$-group we mean a connected
algebraic group $G$, equipped with an isogeny
$\phi:G\rightarrow \sigma(G)$. (Note that the
group operation will then by a
$\sigma$-morphism.) Finally we will call an
algebraic $\sigma$-variety $(V,\phi)$ {\em
trivial} if $V$ is defined over $Fix(\sigma)$
and
$\phi$ is the identity. So note that our
objects are algebraic varieties with
additional structure and the morphisms are
just algebraic morphisms respecting this
additional structure. On the other hand, note
that if $f$ is a $\sigma$-(iso)-morphism
between $(V,\phi)$ and $(W,\psi)$, then $f$
induces a map (bijection) between
$(V,\phi)^{\sharp}$ and $(W,\psi)^{\sharp}$.
Moreover, for such $f$ and $g$, if
$f|(V,\phi)^{\sharp} = g|(V,\phi)^{\sharp}$
then $f=g$ (by Zariski-denseness).

\begin{Fact} Suppose that $(G,\phi)$ is an
algebraic $\sigma$-group, $X$ is an
irreducible $\sigma$-subvariety of $G$
containing the identity, $X$ generates $G$,
and $(X,\phi|X)$ is $\sigma$-birationally
isomorphic to a trivial $\sigma$-variety. Then
$(G,\phi)$ is $\sigma$-isomorphic (as an
algebraic $\sigma$-group) to a trivial
algebraic $\sigma$-group.
\end{Fact}
{\em Proof.} Let $f:X\rightarrow Y$ be the
$\sigma$-birational map between $(X,\phi|X)$
and $(Y,id)$ where $Y$ is defined over
$Fix(\sigma)$. As $X$ generates $G$,
multiplication induces a surjective morphism
$\pi:X^{d} \rightarrow G$ (for some $d$) and
this is clearly a $\sigma$-morphism. On the
other hand, $f^{-1}$ induces a
$\sigma$-birational isomorphism between
$Y^{d}$ and $X^{d}$. Composing, we obtain a
dominant $\sigma$-rational map $h$ from
$(Y^{d},id)$ to $(G,\phi)$. We would like to
obtain from this a trivial
$\sigma$-variety $Z$, and a
$\sigma$-birational isomorphism between
$(Z,id)$ and $G(\phi)$. Consider the
equivalence relation $E$ on generic points of
$Y^{d}$: $E(a,b)$ iff $h(a) = h(b)$. As $h$ is
a $\sigma$-rational map, we see that $h(a) =
h(b)$ implies $\sigma(h)(a) = \sigma(h)(b)$,
so $E$ is $\sigma$-invariant, as is its
Zariski-closure, which is therefore defined
over $Fix(\sigma)$. This yields a rational
dominant map $h'$ defined over $Fix(\sigma)$
from $Y^{d}$ to some variety $Z$ defined over
$Fix(\sigma)$ such that for generic points
$a,b$ of $Y^{d}$, $h'(a) = h'(b)$ iff $E(a,b)$
iff $h(a) = h(b)$. Composing $h'^{-1}$ with
$h$ yields a $\sigma$-birational isomorphism
$h''$ between $(Z,id)$ and $(G,\phi)$.
The group operation on $G$
induces, via $h''^{-1}$, a generically
associative operation $Z\times Z\rightarrow
Z$, which is $\sigma$-invariant, hence defined
over
$Fix(\sigma)$. Weil's theorem
then yields an algebraic group $H$ defined
over $k$, and $h''$ extends to a
($\sigma$)-isomorphism between
$(H,id)$ and $(G,\phi)$.


\section{The Manin-Mumford conjecture:
statement and background}
The setting of the
theorem will be characteristic zero. We will
prove:
\begin{Theorem} Let $k$ be a number field, $A$
a semiabelian variety defined over $k$, and
$X$ an irreducible subvariety of $A$ also
defined over $k$. Let $Tor(A)$ denote the
group of torsion elements of $A$, a subgroup
of $A({\bar k})$. Then the Zariski closure of
$X\cap Tor(A)$ is a finite union of translates
of abelian subvarieties of $A$.
\end{Theorem}

Some words of explanation: An abelian variety
is a connected algebraic group whose underlying
variety is projective (or complete). An
algebraic torus is an algebraic group
isomorphic to some finite power of ${\bf
G}_{m}$ the multiplicative group. A
semiabelian variety is a commutative algebraic
group which, as an algebraic group,
is an extension of an abelian variety by an
algebraic torus. A semiabelian variety is
divisible. The torsion elements of a
semiabelian variety form a Zariski-dense
subgroup. Also there are only finitely many
elements of order
$r$ for any given $r$. In fact if $A$ is an
abelian variety then $T_{p}(A)$ the group of
elements of $A$ with order a power of $p$ is
${\bf Z}_{p^{\infty}}^{2dim(A)}$, and if $A$
is an algebraic torus then this is ${\bf
Z}_{p^{\infty}}^{dim(A)}$.

We will refer to the
statement of the theorem as the Manin-Mumford
conjecture although this was first stated in
the case where $X$ is a curve of genus $\geq
2$ embedded in its Jacobian $A$. In
this case the conclusion can be restated as
$X\cap Tor(A)$ is finite. (For otherwise some
$X$ will already be the Zariski closure of its
intersection with $Tor(A)$ and as $X$ is
irreducible, the conclusion of the theorem
forces $X$ to be an abelian subvariety of $A$,
up to translation, hence an elliptic curve,
contradicting genus being $\geq 2$.)

Various versions of this conjecture were
proved by Raynaud, Hindry, McQuillan,
Bogomolv, Ullmo-Zhang, Buium (some with
explicit bounds). Hrushovski
\cite{Hrushovski-MM} gave a proof of
Theorem 3.1, using the model theory of
difference fields of characteristic zero. The
first step involved essentially capturing
$Tor(A)$ by a finite rank difference equation
of a special kind. The second step involved
showing that the resulting finite-dimensional
difference-algebraic group is ``modular".
This second step proceeded via an analysis of
orthogonality and definable groups in
$ACFA_{0}$, using results from
\cite{C-H}. In our proof
below, the first step is identical, but we
give what amounts to a different and direct
proof of the second step, in the language of
algebraic $\sigma$-groups, following ideas
in
\cite{Pillay-Ziegler}. Closely related things
were done in the papers
\cite{Pink-Roessler1} and
\cite{Pink-Roessler2} of Pink and Roessler. In
fact it was after seeing these papers that I
realized that various easy reductions allow a
direct application of the jet-map methods
from \cite{Pillay-Ziegler}. Jet maps,
following Abramovich, also appear in
Pink-Roessler's second paper
\cite{Pink-Roessler2}, but our formalism,
working in existentially closed difference
fields and considering linear difference
equations on jets, seems to have some
advantages.
\section{The proof.}
Let $A,X,k$ be as in the assumptions of the
theorem 3.1. First some notation. We write the
group operation on
$A$ as $+$ and so also $0$ for the identity. If
$K$ is an extension field of $k$ (such as
${\bar k}$ for example),
$\sigma$ is an automorphism of $K$ over
$k$, and $P(T)\in {\bf Z}(T)$, say $P(T) =
a_{n}T^{n} + ... a_{1}T + a_{0}$, then
$P(\sigma)$ denotes the following (non
algebraic) endomorphism of $A(K)$:
$P(\sigma)(x) = a_{n}\sigma^{n}(x) + ... +
a_{1}\sigma(x) + a_{0}x$.

\vspace{5mm}
\noindent
First note that by replacing $X$ by an
irreducible component of the Zariski closure
of $X\cap Tor(A)$, we may assume that $X\cap
Tor(A)$ is Zariski-dense in $X$. (The new $X$
will be defined over a finite extension of
$k$, still a number field.) Replacing $X$ by a
$X-a$ for some $a\in X\cap Tor(A)$, we may
assume in addition that $0\in X$. We now have
to prove that $X$ is a semiabelian subvariety
(that is, a connected algebraic subgroup) of
$A$.

\begin{Lemma} After possibly replacing $k$
by a finite extension, there is an automorphism
$\sigma$ of ${\bar k}$ over $k$, and a monic
polynomial $P(T)\in {\bf Z}(T)$ which has no
complex roots of unity among its roots, such
that $Tor(A) \subseteq Ker(P(\sigma))$.
\end{Lemma}
{\em Proof.} The argument, following
Hrushovski, is purely algebraic. Pink-Roessler
reproduce the argument too. I have nothing new
to say here, but will give a sketch for the
sake of completeness. Let ${\bf p}$ be a prime
of good reduction for $A$. This means that
${\bf p}$ is a prime ideal of the ring of
integers of the number field $k$, and that
after reducing the equations defining $A$ mod
${\bf p}$ one obtains a semiabelian variety
$A_{\bf p}$ over ${\bf F}_{q}$ (for a suitable
prime power $q = p^{r}$) whose abelian and
linear parts have the same dimensions as those
of $A$. In particular if $L, {\bar A}$ are the
linear, abelian parts of $A$ then
$L_{\bf p}$, ${\bar A}_{\bf p}$ are the linear
and abelian parts of $A_{\bf p}$. Let
$T_{p}'(-)$ denote prime-to-$p$ torsion
points. So we have an exact sequence:
\newline
$0 \rightarrow T_{p}'(L_{\bf p}) \rightarrow
T_{p}'(A_{\bf p}) \rightarrow T_{p}'({\bar
A}_{\bf p}) \rightarrow 0$. Let $\sigma$ be
the automorphism $x\rightarrow x^{q}$ of the
algebraic closure of ${\bf F}_{q}$. A result of
Weil
\cite{Weil} yields a monic integral polynomial
$F_{1}(T)\in{\bf Z}(T)$ without complex roots
of unity among its roots such that $F(\sigma)
= 0$ on $T_{p}'({\bar A}_{\bf p})$. On the
other hand, if $L_{\bf p}$ is isomorphic to a
power  of the multiplicative group over ${\bf
F}_{q^{l}}$ then taking $F_{2}(T)$ to be
$T^{l}-q^{l}$, $F_{2}(\sigma)$ will vanish on
$T_{p}'(L_{\bf p})$. Thus, if $F(T)$ is the
product of $F_{2}$ and $F_{1}$ then
$F(\sigma)$ vanishes on $T_{p}'(A_{\bf p})$.

Now, if $L$ is the maximal unramified extension
of the completion $k_{\bf p}$ of $k$ at ${\bf
p}$, then $\sigma$ lifts to an automorphism
$\sigma'$ of
$L$, $T_{p}'(L)\subseteq A(L)$, and the
reduction map yields a bijection between
$T_{p}'(A)$ and $T_{p}'(A_{\bf p})$, and thus
an abstract isomorphism between
$(T_{p}'(A),+,\sigma')$ and $T_{p}'(A_{\bf
p}),+,\sigma)$. Hence $F(\sigma')$ vanishes on
$T_{p}'(A)$.

We then easily obtain an automorphism
$\sigma'$ of ${\bar k}$ which vanishes on
$T_{p}'(A)$. We can do exactly the same thing
with any other prime of good reduction of $A$.
Now a result of Serre implies, that after
passing to a finite extension $k_{1}$ of $k$,
and if ${\bf p}$ is a prime of $k_{1}$ of good
reduction for $A$, then $k_{1}(T_{p}(A))$ and
$k_{1}(T_{p}'(A))$ are linearly disjoint over
$k_{1}$. We assume that $k_{1} = k$, and pick
two primes ${\bf p}$, ${\bf l}$ of good
reduction for $A$. By the previous paragraph,
we find automorphisms $\sigma$, $\tau$ of
${\bar k}$ over $k$ and monic integer
polynomials
$P_{p}(T)$, $P_{l}(T)$ without complex roots
of unity among their roots such that
$P_{p}(\sigma)$ vanishes on $T_{p}'(A)$ and
$P_{l}(\tau)$ vanishes on $T_{l}'(A)$. In
particular $F_{l}(\tau)$ vanishes on
$T_{p}(A)$. By linear disjointness,
$\sigma|k(T_{p}'(A))$ and $\tau|k(T_{p}(A))$
extend to a common automorphism $\sigma'$ of
${\bar k}$ over $k$. As $Tor(A) = T_{p}(A) +
T_{p}'(A)$, taking $P = P_{p}P_{l}$,
$P(\sigma')$ vanishes on $Tor(A)$, and the
lemma is proved.

\vspace{5mm}
\noindent
We now perform an elementary reduction to get
to the context of ``algebraic groups equipped
with an isogeny". Let
$H = Ker(P(\sigma)) \subset A({\bar k})$. Note
that
$H$ is Zariski-dense in $A$, as $Tor(A)$ is.
Assume that $P(T) = T^{n} + a_{n-1}T^{n-1} +
.. + a_{1}T + a_{0}$ (with $a_{i}\in {\bf
Z}$).  Let $A^{n} = A\times A \times..\times
A$ ($n$ times), and likewise for $X^{n}$. Let
$H_{1} =
\{(x,\sigma(x),..,\sigma^{n-1}(x)):x\in H\}$.
Let $\phi$ be the (algebraic) endomorphism of
$A^{n}$ defined by: $\phi(x_{0},..,x_{n-1}) =
(x_{1},x_{2},..,x_{n-2},-a_{0}x_{0}-a_{1}x_{1}
-..-a_{n-1}x_{n-1})$. Let
$\pi:A^{n}\rightarrow A$ be the projection
onto the first coordinate.

\begin{Remark} For each $r\geq 1$ $\phi^{r} -
id$ is an isogeny of $A^{n}$.
\end{Remark}
{\em Proof.}  Note first that $P(\phi) = 0$.
We leave it as an exercise to show that for
each $r$ there is an integral polynomial
$P_{r}(T)$ with no complex roots of unity amng
its roots such that $P_{r}(\phi^{r}) =0$. Now
if $\phi^{r} - id$ is not an isogeny, then
there is a connected positive-dimensional
semiabelian subvariety $B$ of $A^{n}$ on which
$\phi^{r}$ acts as the identity. As there are
only finitely many elements of $B$ of any given
order it follows that $P_{r}(1) = 0$, a
contradiction.

\begin{Lemma} There are a semiabelian variety
$B$ of $A^{n}$ (defined over $k$) and an
irreducible subvariety $X'$ of $B$ (defined
over a finite extension of $k$) such that
\newline
(i) $\pi|B:B\rightarrow A$ is surjective,
\newline
(ii) $\phi|B$ is an isogeny of $B$ with itself,
\newline
(iii) for some $m\geq 1$,
$\phi^{m}(X')\subseteq X'$.
\newline
(iv)  $\pi(X')$ is a Zariski-dense subset of
$X$.
\end{Lemma}
{\em Proof.} Note first that
\newline
(*) $H_{1}$ is
precisely $\{x\in A^{n}({\bar k}): \sigma(x)
= \phi(x)\}$.
\newline
Let
$B_{1}$ be the
Zariski closure of
$H_{1}$ in
$A^{n}$, and $B$ the connected component of
$B_{1}$. $B_{1}$ and $B$ are defined over
$k$. Because
$\pi(H_{1}) = H$ and $H$ is Zariski-dense in
$A$, it follows that $\pi$ maps $B_{1}$ onto
$A$ and thus $\pi$ maps $B$ onto $A$ yielding
(i).
\newline
As $B_{1}$ is defined over $k$, and $\sigma$
fixes $k$ pointwise, for any $b\in H_{1}$,
$\phi(b)\in B_{1}$, hence
$\phi(B_{1})\subseteq B_{1}$. For any Zariski
open subset $U$ of $B_{1}$ defined over $k$,
$U$ meets $H_{1}$, hence $\phi(B_{1})$ meets
$U$. So $\phi(B_{1}) = B_{1}$. Hence also
$\phi(B) = B$, giving (ii).
\newline
Let us now write $\pi$ for $\pi|B$. Let
$B^{\sharp}$ denote $B\cap H_{1}$. Then
$\pi(B^{\sharp})$ has finite index in $H$
hence, as $0\in X$, $\pi(B^{\sharp}\cap X)$ is
Zariski-dense in $X$. Let $Y$ be the Zariski
closure of $\pi^{-1}(X)\cap B^{\sharp}$ in
$B$. As in the proof of (ii), $Y$ is defined
over $k$ and $\phi(Y)\subseteq Y$. Also
$\pi(Y)$ is Zariski-dense in $X$. So clearly
there is an irreducible component $X'$ of $Y$
say, which contains $0$ and such that
$\pi(X')$ is Zariski-dense in $X$. $X'$ is
defined over some finite extension of $k$.
Moreover, as $\phi$ will permute the
irreducible commponents of $Y$,
$\phi^{m}(X')\subseteq X'$ for some $m\geq 1$.
This proves (iii) and (iv).

\vspace{5mm}
\noindent
With the data given by the lemma, we have two
cases.
\newline
{\em CASE I.} $X'$ is a
semiabelian subvariety of $B$.
\newline
But then $\pi(X') = X$ will be a semiabelian
subvariety of $A$, and the theorem is proved.

\vspace{2mm}
\noindent
{\em CASE II.} Otherwise.
\newline
We seek a contradiction. Let $S$ be the
stabilizer of $X'$ in $B$, namely $\{b\in B:
b+X = X\}$. Then $S$ is $\phi^{m}$-invariant,
and $X'/S \subseteq B/S$ is
positive-dimensional, with trivial stabilizer.
Moreover $\phi^{m}$ is an isogeny of $B/S$
with itself, $X'/S$ is $\phi^{m}$-invariant,
and, by Remark 4.2, $\phi^{rm}-1$ is an
isogeny of $B/S$ with itself, for all $r\geq
1$. So, changing notation, the contradiction
will follow from the next  general
proposition. This is essentially 7.1 in
\cite{Pink-Roessler2}, and can be considered as
as an endomorphism analogue of the result 2.1
in \cite{Pillay} on algebraic $D$-groups. Our
proof follows closely the latter.
\begin{Proposition} Let $A$ be a semiabelian
variety, $\phi:A\rightarrow
A$ a separable isogeny, and
$X\subseteq A$ a subvariety of $X$
containing $0$. Suppose that $\phi(X)\subseteq
X$ and that $Stab_{A}(X) = \{0\}$. Then for
some $r\geq 1$, $\phi^{r}|A_{X}$ = identity,
where $A_{X}$ is the semiabelian subvariety of
$A$ generated by $X$.
\end{Proposition}
{\em Proof.} It is convenient to work in a
saturated existentially closed difference field
$(K,\sigma)$ such that all the data
is defined over $k = Fix(\sigma)$. We will
identify
$A$ and $X$ with their sets $A(K), X(K)$ of
$K$-points. Let
$A^{\sharp} = \{a\in A:\sigma(a)=\phi(a)\}$
and likewise for $X^{\sharp}$. As
$(K,\sigma)\models ACFA$, $A^{\sharp}$ is
Zariski-dense in $A$ and $X^{\sharp}$ is
Zariski-dense in $X$.

Let ${\cal M}$ be the maximal ideal of the
local ring of $A$ at $0$. For $p\geq 1$, let
$j_{p}(A)_{0}$ be the $p$-jet of $A$ at $0$,
namely the dual space to ${\cal M}/{\cal
M}^{p+1}$. $j_{p}(A)_{0}$ is a
finite-dimensional $K$-vector space (defined
over $k$). For any subvariety $Y$ of $A$
passing through
$0$ we obtain likewise $j_{p}(Y)_{0}$ as a
subspace of
$j_{p}(A)_{0}$. If $Z$ varies in an algebraic
family of subvarieties of $A$ all passing
through $0$, then there is sufficiently large
$p$ such that $Z$ is determined within this
family by $J_{p}(Z)_{0}\subseteq
j_{p}(A)_{0}$. We will apply this to the
family $\{X-t:t\in X\}$ of translates of $X$
by elements of $X$. For suitably large $p$ and
for some $r$, we have a rational map
$f:X\rightarrow Gr_{r}(L)$, defined over
$k$, where
$L = j_{p}(A)_{0}$, $Gr_{r}(L)$ is the variety
of $r$-dimensional subspaces of $L$, and $f(t)
= j_{p}(X-t)_{0}$. Moreover, as $X$ has
trivial stabilizer in $A$, $f$ is a birational
isomorphism of $X$ with the Zariski closure
of its image, $Y$ say. For $t\in X$ write
$f(t) = L_{t} < L$.

By separability, $\phi$ induces a linear
automorphism $\phi'$ of $L$, defined over $k$.
Moreover as $L$ is defined over $k$,
$\sigma(L) = L$, hence
$L^{\sharp} =
\{v\in L(K):\sigma(v) =
\phi'(v)\}$ is Zariski-dense in
$L$.

Now suppose $t\in X^{\sharp}$. Then
$\sigma(X-t) = \phi(X-t)$. Hence
$\sigma(L_{t}) = \phi'(L_{t})$, whereby
$L_{t}^{\sharp} = \{v\in L_{t}:\sigma(t) =
\phi'(t)\}$ is again Zariski-dense in $L_{t}$.
Note that $L_{t}^{\sharp}$ is precisely
$L_{t}\cap L^{\sharp}$.

$L$ is a $K$-vector space of dimension $m$
say, and $L^{\sharp}$ is a $k$-vector subspace
of the same dimension. Moreover a basis $b$
can be found for $L^{\sharp}$ over $k$ which is
simultaneously a basis for $L$ over $K$. With
respect to the basis $b$, $L$ identifies with
$K^{n}$ (hence $Gr_{r}(L)$ with
$Gr_{r}(K^{n})$), and
$L^{\sharp}$ identifies with
$k^{n}$. For $t\in X^{\sharp}$,
$L_{t}^{\sharp}$ is then a subspace of
$k^{n}$. As this is Zariski-dense in $L_{t}$
it implies that $L_{t}$ is defined over $k$,
namely $L(t)\in Gr(K^{n})(k)$. As $X^{\sharp}$
is Zariski-dense in $X$, it follows that a
Zariski-dense set of points of $Y = f(X)$ are
defined over $k$, hence so is $Y$.

So we have so far proved:
\newline
{\em Claim 1.} There is a
birational map $f$ between $X$ and a variety
$Y$, such that $Y$ is defined over $k$ and for
$t\in X^{\sharp}$, $f(t)\in Y(k)$.

\vspace{2mm}
\noindent
Now we consider the semiabelian subvariety
$A_{X}$ of $A$ generated by $X$. Note that
$\phi(A_{X}) = A_{X}$. So by Fact 2.1, we have:

\vspace{2mm}
\noindent
{\em Claim 2.} There is an algebraic group $B$
defined over $k$ and an (algebraic) isomorphism
$h$ between $A_{X}$ and $B$ such that
$h(A_{X}^{\sharp}) = B(k)$.

\vspace{5mm}
\noindent
The graph of $f$ is a semiabelian subvariety
of $A_{X}\times B$. As $A_{X}\times B$ is
defined over $k$, $h$ (and so also $h^{-1}$) is
defined over a finite extension $k_{1}$ of
$k$. It follows that $A_{X}^{\sharp}$ is
contained in $A_{X}(k_{1})$. For some $n$,
$k_{1}$ is contained in $Fix(\sigma^{n})$. So
$\phi^{n}$ is the identity on $A_{X}^{\sharp}$.
By Zariski-denseness, $\phi^{n}$ is the
identity on $A_{X}$. This completes the proof
of Proposition 4.4, and thus also the
proof of Theorem 3.1.

\begin{Remark} From the proof
above, one can deduce Corollary 4.1.13 of
\cite{Hrushovski-MM}, at least in its
quantifier-free version: Work in
$(K,\sigma)\models ACFA_{0}$ as above. Let $A$
be a semiabelian variety defined over
$Fix(\sigma)$. Let $P(T)$ be a a polynomial
over the integers with no complex roots of
unity among its roots. Then $B =
Ker(P(\sigma))< A(K)$ is "quantifier-free
modular", namely every quantifier-free
definable subset of $B^{n}$ is a Boolean
combination of translates of quantifier-free
definable subgroups.
\end{Remark}

\section{Project/ Exercise}
In Pink and Roessler's paper
\cite{Pink-Roessler2}, the following is proved
(Proposition 7.3 in that paper).
\begin{Proposition} (char. $p > 0$.)
Let $A$ be a semiabelian variety, and
$\phi:A\rightarrow A$ an isogeny. Assume that
$r,s$ are postive integers, and that there is
a separable isogeny $\lambda$ from $Fr^{r}(A)$
to $A$ such that $\phi^{s} =
\lambda\circ Fr^{r}$. Let $X$ be an
irreducible subvariety of $A$ containing $0$
such that $\phi(X)\subseteq X$, $Stab_{A}(X)$
is trivial, and
$X$ generates $A$. Show that there is an
isomorphism $f$ of $A$ with a semiabelian
variety $A_{0}$ defined over ${\bf F}_{p^{r}}$
such that $(f(\phi))^{s} = Fr^{r}$ on $A_{0}$.
\end{Proposition}

Some words of explanation. $Fr$ denotes the
Frobenius automorphism $x\rightarrow x^{p}$.
This, as well as its powers, act on
varieties, as well on points of varieties. So
$Fr^{r}$ is a bijective morphism between $X$
and $Fr^{r}(X)$. $f(\phi)$ in the last line of
the proposition denotes the isogeny
$f\circ\phi\circ f^{-1}$ of
$A_{0}$. ${\bf F}_{p^{r}}$ is the finite field
with $p^{r}$ elements. The proof of the
proposition in \cite{Pink-Roessler2} seems
somewhat involved.

The project is to find a simple proof of
Proposition 6.1 along the lines of our proof of
4.4, and using a suitable modification of
Fact 2.1.

\vspace{5mm}
\noindent
Another general question relating these
lectures to those of Scanlon, is whether
the methods above apply to the Drinfeld modules
version of Manin-Mumford.







\begin{thebibliography}{99}

\bibitem{C-H} Z. Chatzidakis
and E. Hrushovski, Model theory of difference
fields, Transactions AMS, 351 (1999),
2997-3071.


\bibitem{Hrushovski-MM} E. Hrushovski, The
Manin-Mumford conjecture and the model theory
of difference fields, Annals of Pure and
Applied Logic, 112 (20001), 43-115.

\bibitem{Pillay} A. Pillay, Mordell-Lang for
function fields in characteristic zero,
revisited, to appear in Compositio Math. (See
``recent preprints" at
http://www.math.uiuc.edu/People/pillay.html)

\bibitem{Pillay-Ziegler} A. Pillay and M.
Ziegler, Jet spaces of varieties over
differential and difference fields. (See
``recent preprints" at
http://www.math.uiuc.edu/People/pillay.html)

\bibitem{Pink-Roessler1} R. Pink and D.
Roessler, On Hrushovski's proof of the
Manin-Mumford conjecture. (See ``recent
preprints" at
http://www.math.ethz.ch/~pink/preprints.html)

\bibitem{Pink-Roessler2} R. Pink and D.
Roessler, On $\psi$-invariant subvarieties of
semiabelian varieties and the Manin-Mumford
conjecture (See ``recent preprints" at
http://www.math.ethz.ch/~pink/preprints.html)

\bibitem{Scanlon} T. Scanlon, Diophantine
geometry of the torsion of a Drinfeld module,
Journal of Number Theory, vol. 97, Number 1,
(2002), 10-25.

\bibitem{Weil} Andre Weil, {\em
Varieties abeliennes et courbes algebrique},
Hermann, Paris 1948.
\end{thebibliography}


\end{document}