\magnification=1200


Outline of the lectures `` Motivic and $p$-adic integration"
by Fran\c cois Loeser

\bigskip

My goal is to explain the basics of $p$-adic integration
and motivic integration
and to discuss some connections with  Model Theory.

\bigskip
\noindent{\bf Lecture 1: $p$-adic integration.}
We will explain the basics of  $p$-adic integration on smooth varieties,
its relation with number of points of reductions modulo $p^n$
(Oesterl\' e's Theorem)
and applications to rationality of Poincar\'e series (work of Igusa and Denef).
We shall conclude by presenting Denef's results on the measure
of definable sets.

\bigskip
\noindent{\bf Lecture 2 : Motivic integration. }
Arc spaces. Grothendieck
rings of varieties. Construction of
motivic measures and basic properties. Change of variable formula.
Applications to rationality results.


\bigskip
\noindent{\bf Lecture 3 : Assigning virtual motives to definable sets.}
We shall explain first Chow motives, Galois stratifications
and quantifier elimination for pseudo finite fields.
Then we will be able to assign a virtual motive
to definable sets. We shall explain how it relates to counting points.

\bigskip
\noindent{\bf Lecture 4: Arithmetic motivic integration.}
Using results from the previous lecture, we shall round the loop by
explaining how one can construct
motivic integrals that specialize to $p$-adic ones. If time allows
we shall show how this fits in a much more general framework.


\bigskip


\noindent Prerequisites : Familiarity with the language of Algebraic Geometry (as developed in
Hartshorne's book) and with the most elementary
part of Model Theory.



\bigskip

\noindent References: 

\noindent Lecture 1


\noindent [1] J.Igusa,
An introduction to the theory of local zeta functions.
AMS/IP Studies in Advanced Mathematics, 14.
American Mathematical Society, Providence, RI; International Press, Cambridge, MA, 2000.



\noindent [2] J.Denef,
Arithmetic and geometric applications of quantifier elimination for valued fields.
Model theory, algebra, and geometry, 173--198,
Math. Sci. Res. Inst. Publ., 39, Cambridge Univ. Press, Cambridge, 2000

\noindent Lecture 2 

\noindent [3] J.Denef, F. Loeser,
Geometry on arc spaces of algebraic varieties.
European Congress of Mathematics, Vol. I
(Barcelona, 2000), 327--348, Progr. Math., 201, Birkh\"user, Basel, 2001 







\noindent [4] J.Denef, F. Loeser,
Germs of arcs on singular algebraic varieties and motivic integration. 
Invent. Math.
135 (1999), no. 1, 201--232.

\noindent Lecture 3, 4 


\noindent [5] M.Fried, M.Jarden,
Field arithmetic. Ergebnisse der Mathematik und ihrer Grenzgebiete (3) 11. Springer-Verlag, Berlin, 1986. 

\noindent [6] J.Denef, F.Loeser,
Definable sets, motives and $p$-adic integrals.
J. Amer. Math. Soc. 14 (2001), no. 2,
429--469.

\noindent [7] T.Hales,
Can $p$-adic integrals be computed?
math.RT/0209001.


\noindent [8] J.Denef, F.Loeser,
Motivic integration and the Grothendieck group of pseudo-finite fields
AG/0207163.


\noindent [9] J.Denef, F.Loeser,
On some rational generating series occuring in arithmetic geometry,
available at 
http://www.dma.ens.fr/~loeser/.



\bigskip




\noindent Project 1)  The quantifier  elimination Theorem of
Fried and Sacerdote plays a basic role in Lecture 3.
The proof presented in the book [5] (Proposition 25.9 of [5]) 
is given in a very algebraic language
and could be translated in more geometric terms using basic
knowledge of Algebraic Geometry
such as Galois Theory for coverings of schemes, 
The project has 2 steps:

- the first is to present a neat self-contained geometric proof of 
  Proposition 25.9 of [5].

- the second is to find (and to prove)
  a generalization of that result over a more general base than the spectrum
 of a field.

Suggested readings for the project: Familiarity with the relevant chapters
of [5] and learning about geometric aspects Galois covers in 
[13].


\medskip


Project 2) The complete proof of Theorem 6.4.1 in [9]
(the main result in lecture 3) is scattered
between 3 places ([6], [8] and [9]).
The project would be to rearrange the arguments given or sketched in these
papers
to be able to write down  a self contained direct proof of the Theorem.

Suggested readings for the project:



Learn basics about Chow motives in:

[10] A. Scholl,
Classical motives. Motives (Seattle, WA, 1991), 163--187, Proc. Sympos. Pure Math., 55, Part 1, Amer.
  Math. Soc., Providence, RI, 1994. 


Get a look to the paper (without going through proofs)

[11] S.del Ba�o Rollin, V.Navarro Aznar, 
On the motive of a quotient variety. 
Collect. Math. 49 (1998), no. 2-3, 203--226.



The standard modern reference
for coverings of varieties and 
schemes is SGA1, available on the arxiv

 
[12] A.Grothendieck,  M.Raynaud, SGA 1,
Rev\^ etements \'etales et groupe fondamental  
math.AG/0206203

but this may seem somewhat arid for most. A more leisurely introduction
can be found
in

[13] J.-P.Serre, 
Topics in Galois theory, Research Notes in
Mathematics, 1, 
Jones and Bartlett Publishers, Boston, MA, 1992.


\bye




