\magnification=\magstephalf
\vsize=22.5truecm
\voffset=-.1truecm
\hsize=16truecm
\mathsurround=1.75pt
\baselineskip=13.5pt
\nopagenumbers
%=============================================================================
\font\maimicrm=cmr10 at 10truept
\font\maimicsl=cmsl10 at 10truept
\font\maimicit=cmti10 at 10truept
\font\marebf=cmbx10 at 12truept
\font\maimarebf=cmbx10 at 14truept
%=============================================================================
\def\hhb#1{\hbox to#1pt{}}  \def\hhh{\hhb{1}}
\def\pn{\par\noindent} \def\ssn{\smallskip\noindent}
\def\msn{\medskip\noindent} \def\bsn{\bigskip\noindent}
%=============================================================================

\ssn
{\maimarebf Aim/Description of the Course}

\msn
The aim of this course is to show how arithmetic and
birational geometry is encoded in the elementary theory
of a function field $K$ over some base field $k$, where
the base $k$ is either a number field, or a finite field,
of an algebraically closed field (such as the complex
numbers). One of the main questions here is to give 
{\it sentences\/} in the language of fields by which 
one can ``detect'' the transcendence degree ${\rm td}(K|k)$. 
Clearly, the usual way we do this, namely: 
``$\exists\>(t_1,\dots,t_d)\in K$ which do not satisfy
a non-trivial polynomial relation over $k$, and all $x\in K$ 
are algebraic over $k(t_1,\dots,t_d)$'' is not a sentence 
in the language of fields. Second, supposing that we have 
found sentences as above, give further sentences in the 
language of fields by which one can describe the isomorphism 
type of $K$ (as a field). 

\ssn
Concerning answers: Detecting the transcendence degree 
is ``difficult'' in the case $k$ is a number field. (The 
proof uses the Milnor Conjecture, as proved by Voevodsky 
et al, but just as a ``black box''.) To the contrary, the 
case where $k$ is finite or algebraically closed is easy... 
The problem of detecting the isomorphy type of $K$ from 
its elementary theory is not completely solved yet. We 
will nevertheless show that this {\it is possible\/} in 
the case $K$ is a function field of ``general type''. 
Here we should remark that the ``conservation of 
difficulties'' applies: The geometric situation is much 
more difficult to tackle than the arithmetic one...

\def\blt{$\bullet$}
\msn
{\bf Literature:}
\pn 
\item{\blt} F.~Pop: {\it Elementary equivalence 
versus Isomorphism,\/} Invent.\ Math.\ 150 (2002), 385--408. 
\ssn
\item{} In order to be able to understand the arguments 
of the proof, thus what we plan to do, the following 
basic knowledge is necessary/expected:
\ssn
\item{\blt} The usual model theoretic rudiments (like 
in any introductory book on model theoretic algebra). 
See e.g., Chang--Keisler: {\it Model Theory\/} (3rd ed.), 
North Holland, Amsterdam~1990.  
\ssn
\item{\blt} Basic facts about Galois cohomology, in particular 
the relation between cohom.dim.\ and the transcendence degree. 
Much of these facts mostly used as ``black boxes''. See for 
instance J.-P. Serre: {\it Galois Cohomology,\/} Springer 
Verlag 1997.
\ssn
\item{\blt} Basic facts about quadratic forms and the Witt 
ring (of anisotropic quadratic forms), Pfister forms. In 
particular, their relation to algebraic K-theory (and 
the Milnor Conjecture), and the Galois cohomology. Much 
of these facts mostly used as ``black boxes''. See e.g., 
A.~Pfister: {\it On the Milnor Conjecture: History, 
Influence, applications,\/} Jahresbericht DMV 102 (2002), 
no.1, 15--41. Here lots of further literature on the 
subject can be found.
\ssn
\item{\blt} Basic facts about varieties (over number fields,
finite fields, or algebraically closed fields). See
e.g., D. Mumford: {\it The red book on Varieties and 
Schemes.\/} Mostly (but not only) used as a ``black box''.
\ssn
\item{\blt} Also, basic facts about: First, schemes of finite 
type, e.g., the Chebotarev Density Theorem, like in 
J.-P. Serre: {\it Zeta and L Functions,\/} Arithm.\ Alg.\ 
Geometry, pp.~82--92, Harper\&Row, New York 1965; and 
second, varieties of general type, like in Sh.~Iitaka: 
{\it Algebraic geometry,\/} Springer Verlag. Facts mostly 
used as ``black boxes''.
\ssn
\msn
{\bf A possible project:} Give the ``axiomatic'' of the 
finitely generated fields and/or function fields over 
algebraically closed fields. (This should follow from a
close analysis of the proof of the result under discussion:
``\hhb1elem.\hhb1equiv. $\Rightarrow$ isom.'') 

\bye

