%% Lecture notes on Toric Calabi-Yau , Morningside
%% Lectured by Daqing Wan

\documentclass[a4paper,oneside,11pt]{article}
\usepackage{amsmath,amsthm,amsfonts,amssymb,amscd}
\theoremstyle{plain} \theoremstyle{definition}
\newtheorem{Thm}{Theorem}[section]
\newtheorem{Lem}[Thm]{Lemma}
\newtheorem{Prop}[Thm]{Proposition}
\newtheorem{Cor}[Thm]{Corollary}
\newtheorem{Defn}[Thm]{Definition}

\theoremstyle{remark}
\newtheorem{Rk}{Remark}
\renewcommand{\theRk}{}

\newenvironment{pf}{ \begin{proof} }{ \end{proof} }

\def\bb #1{ {\mathbb #1} }
\newcommand{\Z}{\mathbb{Z}}
\newcommand{\Q}{\mathbb{Q}}
\newcommand{\R}{\mathbb{R}}
\newcommand{\C}{\mathbb{C}}
\newcommand{\f}{\mathbb{F}}
\newcommand{\h}{{\rm H}}
\newcommand{\vol}{\mathop{\mathrm{Vol}}}

\newcommand{\Mod}{\text{ mod }}

%% Defined for these lecture notes.
\def\sd { S_{\Delta, p} }
\def\sdt { \tilde S_{\Delta, p} }
\def\hap1 {\sum_{u \in L(\Delta)}}

\begin{document}

\title{ \Large \bf Zeta Functions of Toric Calabi-Yau Hypersurfaces}
\author{\large \bf Daqing Wan\footnote{MCM workshop lectures (Beijing, Dec. 26-27, 2003), notes taken by Guohua Peng. Expanded by Doug. Haessig for the Arizona Winter School (March 14-17, 2004).}}
%\date{Dec 26-27, 2003}
\maketitle

\tableofcontents

\section{Toric Geometry}
\subsection{$n$-Torus}
Denote by $\mathbb{G}_m^n$ the algebraic $n$-torus over $\f_q$.
Notice that its $\bb F_{q^k}$-rational points are $\bb
G(\f_{q^k})=(\f_{q^k}^*)^n$ and so $\# \bb G_m^n(\f_{q^k})
=(q^k-1)^n$. It follows that its zeta function is rational:
\begin{align*}
Z(\mathbb{G}_m^n/\f_q, T) :&= \exp( \sum_{k=1}^\infty
\frac{\# \mathbb{G}_m^n(\f_{q^k})}{k} T^k ) \\
=&\prod_{i=0}^n (1-q^iT)^{(-1)^{n-i-1}{n\choose i}}.
\end{align*}

\subsection{Basic problem}
Given a Laurant polynomial $f(x_1,\cdots,x_n) \in
\f_q[x_1^\pm,\cdots,x_n^\pm]$, we may define an affine toric
hypersurface
\[
U_f := \{ x \in \bb G_m^n | f(x) =0 \} \hookrightarrow
\mathbb{G}_m^n.
\]
Wanting to understand the sequence of integers obtained by
counting the $\bb F_{q^k}$-rational points of $U_f$ leads to its
zeta function:
\[
Z(U_f / \f_q, T) = \exp( \sum_{k=1}^\infty
\#U_f(\f_{q^k})\frac{T^k}{k})\in 1+T\Z[[T]].
\]
As with the $n$-torus, we wonder whether this too will be a
rational function. Indeed, Dwork has shown this to be true.

\begin{Thm}[Dwork]
$Z(U_f/\f_q, T) \in \Q (T)$.
\end{Thm}

A consequence of this theorem is the existence of a formula for
the numbers $\#U_f(\f_{q^k})$ in terms of the zeros and poles of
the zeta function. However, how well do we know these zeros and
poles? Knowing their precise values seems to be too difficult in
general, so, we may ask for weaker results concerning their
absolute values ($p$-adic and over $\bb C$). Also, we
may wonder how these zeros and poles vary in a family.

\subsection{Projective toric hypersurfaces}\label{Ss: toric}
Let $K$ be a field. With our Laurant polynomial $f$, write $f(x) =
\sum_{j=1}^J a_j x^{V_j}$ where $a_j \in K$ and
$V_j=(v_{1j},\cdots,v_{nj}) \in \Z^n$. Associated to $f$ is its
Newton polytope:
\[
\Delta(f) := \Delta := \text{ the closed convex hull of the
$\{V_j\}$'s in $\R^n$}.
\]
We will assume $\dim \Delta=n$. The Newton polytope will be used
to define a graded algebra $S_\Delta$ as follows. First, define
the the polytope $\bar \Delta \subset \bb R^{n+1}$, which is one
dimension higher than $\Delta$, as the closed convex hull of the
origin in $\bb R^{n+1}$ and the points $(1,V_j) \in \R^{n+1}$.
Next, define the cone $C(\overline{\Delta})$ as the cone generated
by $\overline{\Delta}$. Observe that $C(\bar \Delta) =
\bigcup_{k=1}^\infty k\overline{\Delta}$. Next, define the monoid
$L(\overline{\Delta})$ as the lattice points in the cone
$C(\overline{\Delta})$. It may be shown that $L(\bar \Delta)$ is a
finitely generated monoid. Finally, define the $K$-algebra
\[
S_\Delta := K[L(\overline{\Delta})].
\]
This means $S_\Delta$ consists of all finite sums of $a_u x^u$
where $a_u \in K$ and $u \in L(\bar \Delta)$. Since $L(\bar
\Delta)$ is a finitely generated monoid, $S_\Delta$ is a finitely
generated $K$-algebra. Further, we may define a grading on
$S_\Delta$ by $deg(x^u) := u_0$ where $u = (u_0, \ldots, u_n)$.
Therefore,
\[
S_{\Delta} = \bigoplus_{d=0}^\infty (S_\Delta)_d
\]
where $(S_\Delta)_d$ is the $K$-submodule of $S_\Delta$ consisting
of all elements of $S_\Delta$ of degree $d$.

Since $S_\Delta$ is a finitely generated graded $K$-algebra, we
may define a $K$-scheme $\bb P_\Delta := \rm{Proj\ } S_\Delta$.
This is the toric variety associated to $\Delta$. Observe that
this toric variety only depends on those terms of $f$ that lie on
the vertices of $\Delta$. So, we may think of $\bb P_\Delta$ as an
analogue of projective space. That is, since $x_0 f \in
(S_\Delta)_1$, we may define $\overline{U_f} := \rm{Proj\ }
S_\Delta/(x_0f)$. Notice that $\bar U_f$ embeds in $\bb P_\Delta$
by construction. Thus, we call $\bar U_f$ a toric hypersurface in
$\bb P_\Delta$. It follows that we have the diagram:
$$ \begin{CD}
\hskip 28mm \overline{U}_f @> >> \mathbb{P}_{\Delta}\\
\text{projective closure}@AA  A @A  AA \text{compactification w.r.t. $\Delta$}\\
\hskip 28mm U_f @> >> \mathbb{G}_m^n.
\end{CD} $$

This raises the new questions: what is $Z(\overline{U_f}, T)$ and
how is it related to $Z(U_f, T)$?

\subsection{$\Delta$-regularity}
In this section, we define the notion of a $\Delta$-regular
polynomial $f$. 

Let $\tau \subset \Delta$ be a face of the polytope of any
dimension ranging between zero and $n$. Define the restriction of
$f$ to $\tau$ as $f^\tau = \sum_{V_j\in \tau} a_j x^{V_j}$. Using
the operator $E_i := x_i \frac{\partial}{\partial x_i}$, define
$f_i := E_i f$ for each $i=1, \cdots, n$.

\begin{Defn}
$f$ is called \emph{$\Delta$-regular} if for each face $\tau \in
\Delta$ of any dimension, the system
\[
f^\tau = f_1^\tau = \cdots = f_n^\tau = 0
\]
has no common solutions in $\mathbb{G}_m^n(K^{\text{alg.
clos.}})$.
\end{Defn}

We may reformulate the definition of $\Delta$-regularity as follows.
Define $F := x_0 f - 1 \in S_\Delta$. Notice that
\[
F_i := E_i F = x_i \frac{\partial F}{\partial x_i} =
\begin{cases}
x_0f, & i=0\\
x_0f_i & i=1,\cdots,n
\end{cases}
\]
and $F_i \in (S_\Delta)_1$. For each $i$, define $U_{F_i}=
\text{Proj} S_\Delta /(F_i)$.

\begin{Prop}
$f$ is $\Delta$-regular if and only if $\bigcap_{i=0}^n
U_{F_i}=\emptyset.$
\end{Prop}

\subsection{Homological formulation of $\Delta$-regularity}
\vskip 3mm Each $F_i\in S_\Delta$ acts on $S_\Delta$ by
multiplication:
$$\aligned F_i:\, S_\Delta&\rightarrow S_\Delta\\ g&\mapsto
F_ig\endaligned$$

$F_iF_j=F_jF_i$.

Let $K.(S_\Delta, F_0,\cdots, F_n)$ be the Koszul complex
$$0\longrightarrow S_\Delta\, e_0\wedge\cdots\wedge e_n {\stackrel{\partial}
  {\longrightarrow}}\cdots {\stackrel{\partial}
{\longrightarrow}}\bigoplus_{i=0}^nS_\Delta\, e_i{\stackrel{\partial}
{\longrightarrow}}S_\Delta\longrightarrow 0 $$
$$\partial (a\,e_{i_1}\wedge \cdots \wedge e_{i_j})=\sum_{k=1}^j (-1)^k
F_{i_k}(a)\,e_{i_1}\wedge \cdots \wedge
\widehat{e_{i_k}}\wedge\cdots \wedge e_{i_j}$$

\qquad $\h_0(K.)=S_\Delta/(F_0,F_1,\cdots F_n)=R_f$, the Jacobian ring
of $f$.

\begin{Prop} \quad  TFAE (the following are equivalent):
\begin{enumerate}\item[1)]  $f$ is $\Delta$-regular.
\item[2)] $\{F_0,F_1,\cdots, F_n\}$ forms a regular sequence of
$S_\Delta$. \item[3)] $\h_i(K.)=0,\forall\, i\ge 1$.
\item[4)]$\dim_{K}\h_0(K.)<\infty $. \item[5)]$\dim_K
\h_0(K.)=d(\Delta)=n!\vol(\Delta)\in
\Z_{>0}$.\end{enumerate}\end{Prop}

\subsection{Hodge numbers}\vskip 3mm
\begin{Defn} Let $\Delta\subseteq\R^n$ be
$n$-dimensional integral convex in $\R^n$. Let
$$W_\Delta(k)=\#(\Z^n\cap k\Delta)=\dim_K(S_\Delta)_k.$$
and $$\sum_{k=0}^\infty W_\Delta(k)T^k,\quad \text{the Poincare
series of } S_\Delta.$$\end{Defn}

\begin{Defn}
Define$$h_\Delta(k)=\dim_K(R_f)_k$$ and $$\sum_{k\ge
0}h_\Delta(k)T^k,\quad \text{the Poincare series of $R_f$},$$
where $f$ is $\Delta$-regular and
$$R_f=S_\Delta\big/(F_0,F_1,\cdots,F_n), \dim
R_f=\text{d}(\Delta)=n!\vol(\Delta).$$\end{Defn}

$$\aligned\Rightarrow\quad &(1-T)^{n+1}\sum_{k\ge 0} W_\Delta(k)T^k=\sum_{k\ge
0}h_\Delta(k)T^k,\quad \text{ of degree }\le n.\\ &
h_\Delta(k)=W_\Delta(k)-{ n+1 \choose 1}W_\Delta(k-1)+{n+1\choose
2}W_\Delta(k-2)+\cdots\endaligned$$

\begin{Thm}[Ehrhart] There exists a polynomial $\Lambda(t)$ of
degree $n$ such that \begin{enumerate}\item[1)] for $k\in \Z_{\ge
0}$, $W_\Delta(k)=\Lambda(k)$;  \item[2)] for $k\in \Z_{>0}$,
$W_\Delta(k)^*:=\#\{ {\rm interior\ lattice\ points\ in\ }
k\Delta\}=(-1)^n\Lambda(k)$ $$(\Rightarrow
(1-T)^{n+1}\sum_{k=0}^\infty W_\Delta^*(k)T^k=\sum_{k\ge
0}h_\Delta^*(k)T^k,\quad \text{a polynomial of degree} \leq n+1);$$
\item[3)]duality: $h_\Delta^*(k)=h_\Delta(n+1-k),
k=0,1,\cdots,n+1$.\end{enumerate}
\end{Thm}
\begin{Prop} $$f, \Delta\text{-regular over }
\C\Rightarrow h^k(P\h_c^{n-1}(U_f))=h_\Delta(k+1).$$
\end{Prop}
\begin{Defn} Let $HP(\Delta)$ denote the Hodge polygon in $\R^2$ with
vertices $(0,0)$ and $
(\sum_{k=0}^mh_\Delta(k),\sum_{k=0}^mkh_\Delta(k)),m=0,1,\cdots,n$.
\end{Defn}

\subsection{Reflexive $\Delta$ and Calabi-Yau hypersurface}

\begin{Defn}Let $\Delta\subseteq\R^n$, convex, integral, $n$-dimensional.
Assume $O$ is in the interior of $\Delta$. Define
$$\Delta^*=\left\{(x_1,\cdots,x_n)\in \R^n \,\Big|\, \sum_{i=1}^n x_iy_i\ge -1,
\forall\, (y_1,\cdots, y_n)\in \Delta\right\}.$$
\end{Defn}

$\Delta^*$ is also an $n$-dimensional convex polytope, not necessarily
integral. Clearly, $(\Delta^*)^*=\Delta$.
\begin{Defn} $\Delta$ is called  reflexive if $\Delta^*$ is also
integral.
\end{Defn}

$\underline{\bf Example}$\quad  $\Delta_{a,b}$:\quad
\setlength{\unitlength}{1cm}
\begin{picture}(5,1)
\put(0,0){\line(1,0){4}} \put(0.5,0){\circle*{0.06}}
\put(0.35,-0.5){$-b$} \put(2.3,0){\circle*{0.06}}
\put(2.25,-0.5){0} \put(3.2,0){\circle*{0.06}}
\put(3.15,-0.5){$a$}
\end{picture}

$\hskip 1.3cm\Longrightarrow\, \Delta_{a,b}^*$:\quad
\setlength{\unitlength}{1cm}\begin{picture}(5,1)
\put(0,0){\line(1,0){4}} \put(0.5,0){\circle*{0.06}}
\put(0.35,-0.5){$-\frac{1}{a}$} \put(2.3,0){\circle*{0.06}}
\put(2.25,-0.5){0} \put(3.2,0){\circle*{0.06}}
\put(3.15,-0.5){$\frac{1}{b}$}
\end{picture}\vskip 8mm

$\Delta_{a,b}$ is reflexive iff $a,b=1$.
\begin{Defn}
Let $W$ be an irreducible normal $n$-dimensional projective
variety with Gorenstein canonical singularities. Then $W$ is
called a \emph{Calabi-Yau variety} if
\begin{enumerate}\item[1)] the dualizing sheaf 
$\hat{\Omega}_W^n=O_W$ is trivial; \item[2)] $\h^i(W,O_W)=0, \forall
\ 0<i<n$.\end{enumerate}\end{Defn}

Elliptic curves and $K3$-surfaces are CY.
\begin{Thm}[Hibi, Batyrev] TFAE:
\begin{enumerate}\item[1)] $\Delta$ is reflexive. \item[2)] For
any hyperplane $H=\{(x_1,\cdots,x_n)\in\R^n\mid
\sum_{i=1}^na_ix_i=1\}$ such that $H\cap\Delta$ is a codimension 1
face of $\Delta$, we have $a_i\in\Z$. \item[3)] Hodge numbers are
symmetric: $h_\Delta(k)=h_\Delta(n-k), 0\le k\le n$. \item[4)] The
closure $\overline{U_f}$ of $U_f$ ($f$ is $\Delta$-regular) in
$\mathbb{P}_\Delta$ is a CY variety with canonical singularities.
\end{enumerate}
\end{Thm}
\begin{Defn} For $\Delta$ reflexive; $f$ $\Delta$-regular, $U_f$
is called an affine toric CY hypersurface.\end{Defn}
\begin{Defn}
Denote
$$M_p(\Delta)=\{f/\overline{\f}_p\mid\Delta(f)=\Delta,\,
f\text{ is $\Delta$-regular}\}.$$ Let $\Delta$ be reflexive.
The family $\{f\in M_p(\Delta)\}$ is called the mirror family of $\{g\in
M_p(\Delta^*)\}$ over $\f_p$.
\end{Defn}

\underline{Question}: If $g$ is the ``mirror'' of $f$,
\quad $Z(f/\f_q,T)\,\leftrightsquigarrow\, Z(g/\f_q,T)?$

\begin{Defn} A reflexive $\Delta$ in $\R^n$ is called \emph{Fano},
if \begin{enumerate}\item[1)] $\Delta$ is simplicial, i.e., each
codimension 1 face of $\Delta$ is a simplex. And \item[2)] The
vertices of each codimension 1 face of $\Delta$ form a $\Z$-basis
of $\Z^n$ in $\R^n$.\end{enumerate}
\end{Defn} \begin{Prop}Reflexive $\Delta$ is Fano $ \Longleftrightarrow $
$\mathbb{P}_{\Delta^*}$ is smooth.
\end{Prop}

\subsection{A basic example}
Take $$\Delta=\langle e_1,e_2,\cdots,e_n,-(e_1+\cdots+e_n)\rangle,
$$
where $e_i$'s are the standard unit vectors in $\R^n$. Then
$$\Delta^*=\langle (n,-1,\cdots,-1),(-1,n,\cdots,-1),\cdots,
(-1,\cdots,-1,n),(-1,-1,\cdots,-1)\rangle.$$
$\Delta$ is
reflexive, $\Delta$ Fano, but $\Delta^*$ NOT Fano if $n>1$. For $n=2$,

\begin{center} \setlength{\unitlength}{1cm}
\begin{picture}(4,2)
\put(0,0){\vector(1,0){4}}\put(2,-2){\vector(0,1){4}}
\put(2,1){\line(1,-1){1}}
\put(1,-1){\line(1,2){1}}\put(1,-1){\line(2,1){2}}\put(0.1,-1.3){${-(e_1+e_2)}$}
\put(2.15,1){${e_2}$}\put(2.95,-0.3){${e_1}$}\put(3.6,0.5){$\Delta$}
\end{picture}$\quad\Longrightarrow\quad$\setlength{\unitlength}{1cm}
\begin{picture}(4,2.5)
\put(0,0){\vector(1,0){4}}\put(2,-2){\vector(0,1){4}}%%axic
\put(1,1.9){\line(1,-1){2.91}}\put(1,-1){\line(0,1){2.9}}
\put(1,-1){\line(1,0){2.9}}
\put(0.5,0.1){$-1$}\put(-0.2,0.5){$\Delta^*$}
\put(2.05,-1.35){$-1$}\put(2.5,0.5){$x_1+x_2=1$}
\end{picture}\end{center}
\vskip 3cm Let
$$f(\lambda,x)=x_1+x_2+\cdots+x_n+\frac{1}{x_1x_2\cdots
x_n}-\lambda.$$ It's clear that $\Delta(f)=\Delta$.

$f(\lambda,x)$ is $\Delta $-regular $ \Longleftrightarrow $
$\lambda\not\not= (n+1)\alpha, \alpha^{n+1}=1$.

Mirror family:
$$\aligned g(\lambda,x)=&\frac{x_1^{n+1}}{x_1x_2\cdots x_n}+\cdots+
\frac{x_n^{n+1}}{x_1x_2\cdots x_n}+\frac{1}{x_1x_2\cdots
x_n}-\lambda\\ =&\frac{1}{x_1x_2\cdots
x_n}(x_1^{n+1}+\cdots+x_n^{n+1}+1-\lambda x_1x_2\cdots x_n)\\
\stackrel{x_i\ne 0}{\leftrightsquigarrow}& 1+
x_1^{n+1}+\cdots+x_n^{n+1}-\lambda x_1x_2\cdots x_n\endaligned$$

Projective closure in $\mathbb{P}^n$:
$$x_0^{n+1}+ x_1^{n+1}+\cdots+x_n^{n+1}-\lambda x_0x_1x_2\cdots x_n=0$$
(the well known family of CY hypersurfaces in $\mathbb{P}^n$.)

\vskip 1cm Let
$$G=\left(\Z/(n+1)\Z\right)^{n-1}=\left\{(\zeta^{(1)},\cdots,\zeta^{(n)})
\,\Big|\,(\zeta^{(i)})^{n+1}=1,\, \prod_{i=1}^n \zeta^{(i)}=1
\right\}.$$ Then $G$ acts on $U_{g(\lambda,x)}$:
$$(\zeta^{(1)},\cdots,\zeta^{(n)})(x_1,\cdots,x_n)=
(\zeta^{(1)}x_1,\cdots,\zeta^{(n)}x_n).$$

\begin{Prop} $U_{f(\lambda,x)}=U_{g(\lambda,x)}/G$.
\end{Prop}
\begin{pf} If $g(\lambda,x)=0$ for some $x, x_i\ne 0$, let
$$\aligned \begin{cases} y_1=&x_1^{n+1}/x_1\cdots x_n\\ &\vdots\\
y_n=&x_n^{n+1}/x_1\cdots
x_n\end{cases}\Rightarrow&\begin{cases}  x_1\cdots x_n=y_1\cdots y_n\\
 \hskip 6mm x_i^{n+1}=y_iy_1\cdots y_n\end{cases}\\ \Rightarrow&\
 y_1+\cdots +y_n+\frac{1}{y_1\cdots y_n}-\lambda=0.\endaligned$$\end{pf}

 \underline{\bf Exercise}:

 $\Delta=\Delta(x_1+\cdots+x_n+\frac{1}{x_1\cdots x_n}-\lambda)$
$\Rightarrow h_\Delta(0)=h_\Delta(1)=\cdots= h_\Delta(n)=1$.
 (Betti number $d(\Delta)=n+1$.)

 \section{Zeta Functions}

\subsection{L-functions of exponential sums}

 For $f\in \f_q[x_1^{\pm},\cdots,x_n^{\pm}]$,
 $U_f=\{f=0\}\hookrightarrow \mathbb{G}_m^n$, we have
 $$Z(U_f,T)=\exp(\sum_{k=1}^\infty \#U_f(\f_{q^k})
 \frac{T^k}{k}).$$

 Let $$\aligned \Psi: \f_p&\rightarrow\C^*\\ x&\mapsto\psi(x)=
 \exp(\frac{2\pi ix}{p})\endaligned$$
 be a nontrivial character of $\f_p$. Then $$\Psi\circ
 \text{Tr}_{\f_{q^k}/{\f_p}}:\, \f_{q^k}\rightarrow\C^*$$ induces a
 nontrivial character of $\f_{q^k}$.

 The exponential sum
 $$S_k(x_0f)=\sum_{x_i\in \f_q^*} \Psi\circ
 \text{Tr}_{\f_{q^k}/{\f_p}}(x_0f).$$

 It's easy to compute $$\aligned q^k\#U_f(\f_{q^k})=
 &\sum_{{x_i\in \f_{q^k}^*}\atop {1\le i\le n}}\sum_{x_0\in\f_q}\Psi\circ
 \text{Tr}_{\f_{q^k}/{\f_p}}(x_0f)\\
 =&(q^k-1)^n+S_k(x_0f)\\ =&
 \#\mathbb{G}_m^n(\f_{q^k})+S_k(x_0f).\endaligned$$
 Then $$Z(U_f,qT)=Z(\mathbb{G}_m^n,T)L(x_0f,T),$$
 where
 $$L(x_0f,T)=\exp(\sum_{k=1}^\infty\#S_k(x_0f)\frac{T^k}{k}).$$

 $\Rightarrow $It is enough to study $L(x_0f,T)$.

\subsection{Dwork's $p$-adic analytic character}
Consider the Artin-Hasse series
\[
t + \frac{t^p}{p} + \frac{t^{p^2}}{p^2} + \cdots.
\]
The Newton polygon of this tells us that there are exactly $p-1$
roots of this series of slope $\frac{1}{p-1}$. Let $\pi$ be one of
these roots, and so $ord_p(\pi) = \frac{1}{p-1}$. Using this, we
may define a \emph{splitting function}
\[
\theta(t):=\exp\left((\pi t) + \frac{(\pi t)^p}{p} + \cdots
\right) \in \bb Q_p(\pi)[[T]].
\]
Since
\[
\exp\left(t + \frac{t^p}{p} + \cdots\right) =
\prod_{(k,p)=1}(1-t^k)^{-\frac{\mu(k)}{k}},
\]
it follows that $\theta(t)$ converges on $|t|_p <
p^{\frac{1}{p-1}}$. In particular, $\theta$ is defined at the
Teichm\"uller points in $\bb C_p$. Splitting functions have the
following remarkable properties:

{\bf Property 1.} $\theta(1)$ is a primitive $p$-th root of unity.

{\bf Property 2.} We may define a nontrivial additive character
\[
\psi_k: \bb F_{p^k} \rightarrow \bb C_p^* \quad \text{by} \quad
\psi_k(\bar x) := \theta(x) \theta(x^p) \cdots
\theta(x^{p^{k-1}}) = \psi_1 ({\rm Tr}_{\bb F_{p^k}/\bb F_p}(\bar x)).
\]
where $x$ is the Teichm\"uller representative of $\bar x$.

\subsection{Analytic representation of $S_k(x_0 f)$} 
Write $x_0 \bar f(x) = \sum_{j=1}^J \bar a_j x_0 x^{v_j} \in \bb F_q[x_0^{\pm
1}, \ldots, x_n^{\pm 1}]$. Then, with $q = p^a$, we have
\begin{align}
S_k(x_0f) &= \sum_{\bar x_i \in \bb F_{q^k}^*} \psi_k (x_0 \bar f(\bar
x)) \notag \\ &= \sum_{\bar x_i \in \bb F_{q^k}^*} \prod_{j=1}^J
\psi_k(\bar a_j x_0 x^{v_j}) \notag \\
&= \sum_{x_i^{q^k - 1} = 1, x_i \in \bar{\bb Q_p}} \prod_{j=1}^J
\prod_{i = 0}^{ak-1} \theta( (a_j x_0 x^{v_j} )^{p^i} ) \label{E:
theta} \\
&= \sum_{x_i^{q^k - 1} = 1} F_a(f, x) F_a(f, x^q) \cdots F_a(f,
x^{q^{k-1}})
\end{align}
where we have lifted the coefficients of $f$ to $\bb C_p$, that
is, $a_j = Teich(\bar a_j)$, and,
\[
F_a(f, x) := \prod_{i = 0}^{a-1} \prod_{j=1}^J \theta(a_j x_0
x^{v_j})^{p^i}.
\]
This is the $p$-adic analytic representation of $S_k(x_0 f)$
that we will use.

\subsection{Frobenius endmorphism}
Recall $S_\Delta$ from section \ref{Ss: toric}, with $K$ replaced by ${\bb Z}_p$. Now, define
\[
\sd := \{ \sum_{u \in L(\bar \Delta)} A_u \pi^{u_0} x^u | A_u \in \bb
Z_p, A_u \rightarrow 0 \}.
\]
Note, $\sd$ is isomorphic to the $p$-adic completion of $S_\Delta$
at $p$. Now, with a norm defined by $\| \sum A_u \pi^{u_0} x^u \|
:= \sup_u | A_u |_{p}$, we see that $\sd$ is a Banach $\bb
Z_p$-module. By construction, we see that $\Gamma := \{ \pi^{u_0}
x^{u} | u \in L(\bar \Delta) \}$ is an orthonormal basis for
$\sd$, that is, the coefficients tend to zero.

Consider the field $\bb Q_q(\pi)$ and its Galois group over $\bb
Q_p(\pi)$, which is cyclic of order $a$ generated by $\tau$. By
definition, $\tau$ sends Teichm\"uller representatives to their
$p$-th power.

Using notation from the last section, define
\[
F(f,x) := \prod_{j=1}^J \theta( a_j x_0 x^{v_j})
\]
and
\[
G(x) := F(f,x) F^\tau(f, x^p) F^{\tau^2}(f, x^{p^2}) \cdots.
\]

On the space $\sd \otimes \bb Z_q[\pi]$, define the compact
operators
\[
\phi_1 := \psi_p \circ F(f,x)
\]
and
\[
\phi_a := \psi_q \circ F_a(f, x)
\]
where, $q = p^a$, and
\[
\psi_p( \sum A_u x^u ) := \sum A_{pu}^{\tau^{-1}} x^u.
\]
Note, we may formally write
\[
\phi_1 = G(x)^{-1} \circ \psi_p \circ G(x) \qquad \text{and}
\qquad \phi_a = G(x)^{-1} \circ \psi_q \circ G(x),
\]
where 
\[
\psi_q( \sum A_u x^u ) := \sum A_{qu} x^u.
\]

\subsection{Rationality of $L(x_0f,T)$ and $Z(U_f/\bb F_q,
T)$} \label{Ss: rationality} Now $\phi_a$ has the following
amazing property called the \emph{Dwork trace tormula}:
\[
S_k(x_0 f) = (q^k - 1)^{n+1} Tr(\phi_a^k)
\]
where $Tr$ denotes the trace of the operator. Recall the relation
\[
\frac{1}{det(I - \phi_a T)} = \exp \sum_{k \geq 1} \frac{
Tr(\phi_a^k)}{k}{T^k}.
\]
Combining these with the binomial theorem, we see that
\begin{align*}
L(x_0 f, T) &= \exp \sum_{k \geq 1} \frac{S_k(x_0 f)}{k} T^k
\\
&= \prod_{i=0}^{n+1} \left[ det(I - q^i \phi_a T)
\right]^{(-1)^{n-i} \binom{n+1}{i}}.
\end{align*}
This looks like rationality, however, remember that the operator
$\phi_a$ acts on $\sd \otimes \bb Z_q[\pi]$, an infinite
dimensional space and so the characteristic polynomials are
actually power series. However, since this operator is compact,
$\det(I - q^i \phi_a T)$ is a $p$-adic entire function. Therefore,
the $L$-function is $p$-adic meromorphic.

To prove rationality, we need to use an extension of a theorem of
Borel proven by Dwork.
\begin{Thm}[Borel]
Let $g(T) \in \bb Z[[T]]$. Then $g(T) \in \bb Q(T)$ if and only if
$g(T)$ satisfies both
\begin{enumerate}
\item $g(T)$ converges in some neighborhood of the origin in
${\bb C}$. \item $g(T)$ is $p$-adic meromorphic.
\end{enumerate}
\end{Thm}

We obtain
\begin{Thm}[Dwork]
$L(x_0 f, T) \in \bb Q(T)$ and so $Z(U_f / \bb F_q, T) \in \bb
Q(T)$.
\end{Thm}
To prove this, we need only show that $L(x_0 f)$ converges in
some neighbourhood of the origin in ${\bb C}$. Now,
\[
|S_k(x_0 f)|_{\bb C} \leq (q^k - 1)^{n+1} \leq q^{k(n+1)}
\]
and since
\[
\sum_{k \geq 1} \frac{q^{k(n+1)}}{k} T^k
\]
converges for $|T|_{\bb C} < 1/q^{n+1}$, we see that $L(x_0 f,
T)$ converges for any $|T|_{\bb C} < 1/q^{n+1}$. This proves the
theorem.


\subsection{$p$-adic Cohomological formula for
$L(x_0f,T)$}\label{Ss: Cohom Form} As mentioned in section
\ref{Ss: rationality}, we may define a compact operator $\phi_a$
on a $p$-adic Banach module $B := \sd \otimes \bb Z_q[\pi]$. We
may also define differential operators
\[
D_i := G(x)^{-1} \circ x_i \frac{\partial}{\partial x_i} \circ
G(x)
\]
for each $i=0, 1, \ldots, n$ acting on $B$. Since these commute,
we may create a Koszul complex $K_\bullet(B, D_0, \ldots, D_n)$,
the top line of the commutative diagram below. Also, since $\phi_a
\circ D_i = q D_i \circ \phi_a$, we may define a chain map between
complexes:
\[
\begin{CD}
0 @>>> B^{\binom{n+1}{n+1}} @>d>> B^{\binom{n+1}{n}} @>d>> \cdots
@>d>> B^{\binom{n+1}{0}} @>>> 0 \\
@VVV @Vq^{n+1}\phi_aVV @Vq^n \phi_aVV @VVV @VV\phi_aV @VVV \\
0 @>>> B^{\binom{n+1}{n+1}} @>d>> B^{\binom{n+1}{n}} @>d>> \cdots
@>d>> B^{\binom{n+1}{0}} @>>> 0
\end{CD}
\]
where
\[
B^{\binom{n+1}{i}} := B \otimes \Lambda^i (\oplus_{j=0}^n \bb Z e_j)
\]
and $d: B^{\binom{n+1}{i}} \rightarrow B^{\binom{n+1}{i-1}}$ is
defined by
\[
d(a e_{j_1} \wedge \cdots \wedge e_{j_i}) := \sum_{k=0}^i (-1)^k
D_{j_k}(a) e_{j_1} \wedge \cdots \wedge \hat e_{j_k} \wedge \cdots
\wedge e_{j_i}.
\]
We may rewrite the $L$-function as follows.
\begin{align*}
L(x_0 f, T)^{(-1)^n} &= \prod_{i=0}^{n+1} det(I - T q^i \phi_a |
B)^{(-1)^i \binom{n+1}{i}} \\
&= \prod_{i=0}^{n+1} det(I - T q^i \phi_a |
B^{\binom{n+1}{i}})^{(-1)^i} \\
&= \prod_{i=0}^{n+1} det(I - T q^i \phi_a | H_i( K_\bullet(B, D_0,
\ldots, D_n)))^{(-1)^i}.
\end{align*}

Now, if $f$ is $\Delta$-regular, then all the
homology spaces are trivial except for $i=0$, in which case 
$$H_0( K_\bullet(B, D_0, \ldots, D_n)) = B/\sum_{i=0}^n D_i(B)$$
is a free $\bb Z_q[\pi]$-module of rank $d(\Delta)$. 
That is the essence of the next two theorems.

\begin{Thm}[Adolphson-Sperber]
If $f$ is $\Delta$-regular, then $L(x_0f,T)^{(-1)^n}$ is a
polynomial of degree $d(\Delta)=n!\vol(\Delta)$.
\end{Thm}

 \begin{Thm}[Denef-Loeser] If $f$ is $\Delta$-regular, then
 $L(x_0f,T)^{(-1)^n}$ is mixed of weight $\le n+1$. That is, if
 $$L(x_0f,T)^{(-1)^n}=\prod_{i=1}^{d(\Delta)}(1-\alpha_iT),$$ then
 $$|\alpha_i|=\sqrt{q}^{w_i},\, w_i\in \Z\cap[0, n+1].$$ \end{Thm}
\vskip 3mm
 Let $$e_j=\#\{1\le i\le d(\Delta)\mid w_i=j\},\quad 0\le j\le
 n+1.$$There exists a very complicated combinatorial formula for
 $e_j$.

 \underline{\bf Example}:\quad Let $\Delta$ be a simplex and
 $$c_0=1, c_i=\sum_{\tau\subset\Delta, \text{face}\atop {\dim\tau=i-1}}
 \vol(\tau),\quad i\ge 1.$$ Then
 $$e_0=1, e_j=\sum_{i=0}^j (-1)^{j-i}i!{n+1-i\choose n+1-j}c_i,\quad
 j\ge 1.$$
 \underline{\bf Exercise}:\quad $f(\lambda,x)=x_1+\cdots+x_n+
 \frac{1}{x_1\cdots x_n}-\lambda$, $\Delta$-regular. Compute
 $e_j$.

 \subsection{Newton polygon for $L(x_0f,T)^{(-1)^n}$}
 Let $f$ be $\Delta$-regular over $\f_q$. Write
 $$L(x_0f,T)^{(-1)^n}=\sum_{m=0}^{d(\Delta)}A_mT^m,\quad A_0=1, A_m\in\Z.$$
 Define the $q$-adic Newton polygon of $L(x_0f,T)^{(-1)^n}$ to be the lower
 convex closure in $\R^2$ of the points
 $(m,\text{ord}_q(A_m)),\, m=0,1,\cdots,d(\Delta)$. Denote this
 polygon by $NP(f)$.

\begin{center}
 \setlength{\unitlength}{1cm}
\begin{picture}(10,4)
\put(0,-0.5){\circle*{0.1}}\put(0,-0.5){\line(5,1){2.5}}
\put(2.5,0){\circle*{0.1}}\put(2.5,0){\line(4,1){1}}
\put(3.5,0.25){\circle*{0.1}}\put(3.5,0.25){\line(3,1){2}}
\put(5.5,0.92){\circle*{0.1}} \put(5.5,0.92){\line(2,1){2}}
\put(7.5,1.92){\circle*{0.1}}\put(7.5,1.92){\line(1,1){0.5}}%%other points
\put(1,2){\circle*{0.1}}\put(2,3){\circle*{0.1}}
\put(8.5,1.5){$NP(f)$} \put(7,3){\circle*{0.1}}
\put(4.35,0.7){$s$}  \put(3.5,0){$\underbrace{\hskip
2cm}$}\put(4.4,-0.6){\text{$h_s=$\small{horizontal length of the
slope $s$ side}}} \put(3,-1.5){\text{Newton polygon of $f$}}
\end{picture}\end{center}\vskip 2cm
\begin{Thm} $NP(f)$ has a side of slope $s$ with horizontal length
$h_s$ iff there are exactly $h_s$ reciprocal zeros $\alpha_i$'s
such that
$$\text{ord}_q(\alpha_i)=s,\quad\text{i.e.}, |\alpha_i|=q^{-s}.$$

\underline{\bf Question}. For $s\in \Q\cap [0, n+1]$, $h_s=?$
\end{Thm}

\begin{Thm}[Adolphson-Sperber]
$f$ is $\Delta$-regular $\Rightarrow$ $NP(f)\ge HP(\Delta)$, with
endpoints coincide, where $HP(\Delta)$ is the Hodge polygon of
$\Delta$.
\end{Thm}

An outline of the proof is as follows. See section \ref{Ss: Cohom
Form} for some relevant notions. We define an operator $\phi_1$ on
our Banach module $B := \sd \otimes \bb Z_q[\pi]$. This induces an
operator on the finite dimensional homology space 
$$H_0 := H_0(K_\bullet(B, D_0, \ldots, D_n)) = B/\sum_{i=0}^n D_i(B)$$ 
and so may be represented by a
matrix if we provide a basis. Choosing a monomial basis $\Gamma_I
:= \{ \pi^{u_0} x^u | u \in I \}$, we may explicitly estimate the
$p$-adic order of the entries of the matrix $A_1$ representing
$\phi_1$ to get a (Hodge) filtration: $\phi_1(\Gamma_I) = \Gamma_I A_1$, where 
\[
A_1 = \left(\begin{array}{rrrr}
M_{00} &  M_{01} & M_{02} & \cdots \\
pM_{10} & pM_{11} & pM_{12} & \cdots \\
p^2M_{20} & p^2 M_{21} & p^2 M_{22} & \cdots \\
\vdots & \vdots & \vdots & \ddots
\end{array}
\right)
\]
where $M_{ij}$ is a matrix with $h_\Delta(i)$ rows and
$h_\Delta(j)$ columns. Further, the entries of $M_{ij}$ have
$ord_p \geq 0$. Relating this to our operator $\phi_a$ via the
relation $\phi_1^a = \phi_a$ and using an argument of Dwork's, we
may show the $q$-adic Newton polygon of $det(I - T \phi_a | H_0)$
lies above the Hodge polygon, which is defined as the lower convex
hull of the points
\[
\left( \sum_{i=0}^m h_\Delta(i), \sum_{i=0}^m i \cdot h_\Delta(i)
\right)_{m=0, 1, \ldots, n}.
\]

\underline{\bf Definition}. If $NP(f)=HP(\Delta)$, then $f$ is called ordinary.
In this case, $L(x_0f, T)^{(-1)^n}$ has exactly $h_{\Delta}(k)$ reciprocal zeros
$\alpha_i$'s such that ord$_q(\alpha_i)=k$ for all $0\leq k\leq n$.

\subsection{Variation of $NP(f)$ with $p$}

{\bf Conjecture}: Let $f\in\Z[x_1^{\pm},\cdots,x_n^{\pm}]$
be $\Delta$-regular. Then there exist infinitely many primes $p$
such that $NP(f \Mod p )=HP(\Delta)$ ($p$ is then called ordinary).
One further conjectures that the density $\delta(f)$ of ordinary primes
exists and is positive.

\underline{\bf Example}. If $f=x_1+x_2+\frac{1}{x_1x_2}-\lambda$ is $\Delta$-regular
and hence defines an elliptic curve over $\Q$, the density $\delta(f)$ is
either $1/2$ if $f$ has CM (Deuring) or $1$ if $f$ has no CM (Serre).




 \subsection{Variation of $NP(f)$ with $f$ ($p$ fixed)}

 Let $$M_p(\Delta)(\overline{\f}_p)=\{f\in
 \overline{\f}_p[x_1^{\pm},\cdots,x_n^{\pm}]
 \mid \Delta(f)=\Delta,\, f \text{ is $\Delta$-regular}\}.$$
This set is non-empty if $p>d(\Delta)$.

 $f\in
 M_p(\Delta)(\overline{\f}_p)\Rightarrow f\in M_p(\Delta)(\f_q)$ for some $q$.

 \hskip 2.7cm $\Rightarrow$ $q$-adic $NP(f)$ is defined, independent of

 \hskip 2.7cm the choice of the defining field $\f_q$.

The relatively cohomology is locally free and thus forms an overconvergent 
$\sigma$-module and in fact an overconvergent $F$-crystal on $M_p(\Delta)$. 
We obtain

 \begin{Thm}[Grothendieck-Katz]
The global minimun
$$GNP(\Delta, p)=\inf_{f\in M_p(\Delta)}NP(f)$$
exists and is precisely attained for all $f$
in a Zariski open dense subset $U_p(\Delta)\hookrightarrow M_p(\Delta)$.
This minimun polygon
$GNP(\Delta,p)$ is called the generic Newton polygon of the family $M_p(\Delta)$.
\end{Thm}

Thus, Newton polygon goes up under specialization, that is, for $f\in M_p(\Delta)$, 
$$NP(f)\ge GNP(\Delta, p)\ge HP(\Delta).$$
The first equality holds for all $f\in U_p(\Delta)$.


\underline{\bf Definition}: \quad If $GNP(\Delta,p)=HP(\Delta)$, $\Delta$ is called ordinary at $p$ or
 generically ordinary at $p$.

\underline{\bf Question}: Which primes $p$ are ordinary for $\Delta$?


\subsection{Generically ordinary primes}

{\bf Conjecture} (Adolphson-Sperber):\quad $\Delta$ is ordinary for $p\gg0$.



\begin{Prop} Let $\Delta$ be minimal (i.e., no lattice points on 
$\Delta$ other than vertices). If $p \equiv 1 (~{\rm mod} d(\Delta))$, 
then $\Delta$ is ordinary at $p$
\end{Prop}

For minimal $\Delta$, $x_0f$ becomes a diagonal, 
the L-function can be computed directly using Gauss sums 
and the slopes can be found using the Stickelberger theorem. 
This is the local case. Note also for minimal $\Delta$, 
one has $d(\Delta)=1$ if $n\leq 2$. 


\begin{Thm}[Wan] \begin{enumerate}\item[1)] If $n\le 3$,
$\Delta$ is ordinary for $p>d(\Delta)$. \item[2)]If $n\ge
4$, there exists $n$-dimensional $\Delta$ which is NOT ordinary for all primes
$p$ in a certain congruence class.
\item[3)]There exists $D^*(\Delta)>0$ such that $\Delta$ is
ordinary for $p\equiv 1(\Mod D^*(\Delta))$.
\end{enumerate}
\end{Thm}


Part 1) and part 3) follow from the 
collapsing decomposition (to be explained in the lectures) and a finer form of 
the above local proposition. 


{\bf Conjecture}. There is a positive integer $\mu(\Delta)$ such that the set
of almost all (except for finitely many) ordinary primes for $\Delta$
consists of the primes in certain congruence classes modulo $\mu(\Delta)$.



\subsection{Generically ordinary Calabi-Yau hypersurfaces}

\begin{Thm}[Wan] Let $\Delta$ be reflexive.
\begin{enumerate}\item[1)]If $n=\dim(\Delta)\le 4$, then $\Delta$ is
ordinary for $p>d(\Delta)$.\item[2)]If $\Delta$ is Fano, then
$\Delta$ is always ordinary for every $p$.\end{enumerate}
\end{Thm}

Part 2) follows from the star decomposition theorem. The case $n=4$ of 
Part 1) follows from a combination of the star decomposition and the 
collapsing decomposition (to be explained in the lectures). 

\underline{\bf Questions}:\begin{enumerate}\item[1)]For reflexive
$\Delta$ with $n=\dim(\Delta)\ge 5$, is $\Delta$ ordinary for
$p>d(\Delta)$?\item[2)]If $\Delta$ is reflexive and ordinary at
$p>d(\Delta)$, is $\Delta^*$ ordinary at $p$?\end{enumerate}
(already yes if $n\leq 4$ or $\Delta$ is Fano)

\subsection{Basic example}
Take$$f(\lambda,x)=x_1+\cdots+x_n+\frac{1}{x_1\cdots
x_n}-\lambda.$$ Then
$$\aligned
\Delta=&\Delta(f)=\langle e_1,e_2,\cdots,e_n,-(e_1+e_2+\cdots+e_n)\rangle,\\
\Delta^*=&\langle(n,-1,\cdots,-1),\cdots,(-1,-1,\cdots,n),(-1,-1,\cdots,-1)\rangle.\endaligned$$

\begin{Thm} Both $\Delta$ and $\Delta^*$ are ordinary for all primes $p$.
\end{Thm}


\begin{Thm}\begin{enumerate}\item[1)] Let $f(\lambda,x)$ be
$\Delta$-regular over $\f_q$. Then
$$L(x_0f,T)^{(-1)^n}=\prod_{i=0}^n(1-\alpha_i(\lambda)T),\quad
d(\Delta)=n+1.$$ \item[2)] $\alpha_0(\lambda)=1,\,
|\alpha_i(\lambda)|=\sqrt{q}^{n+1}$.\item[3)] $\Delta$ is ordinary
at $p$. That is, except for finitely many $\lambda$, we have
ord$_q(\alpha_i(\lambda))=i, 1\le i\le n$.
\end{enumerate}
\end{Thm}
\noindent\emph{Proof of 2)}. \quad Since endpoints for $NP(f)$ and
$HP(\Delta)$ coincide, we have
$$\text{ord}(\alpha_0\alpha_1\cdots \alpha_n)=
\text{ord}(\alpha_1\cdots
\alpha_n)=\sum_{k=0}^nkh_\Delta(k)=\frac{n(n+1)}{2}.$$ By
Denef-Loeser (Theorem 2.2), $|\alpha_i|\le q^{\frac{n+1}{2}}$. Now
$\alpha_1\cdots \alpha_n\in\Z$ implies that $(\alpha_1\cdots
\alpha_n)^2=\alpha_1\cdots \alpha_n\overline{\alpha}_1\cdots
\overline{\alpha}_n\le q^{n(n+1)}$.

But ord$(\alpha_1\cdots \alpha_n)^2=n(n+1)$, then $(\alpha_1\cdots
\alpha_n)^2=q^{n(n+1)}$. Hence $|\alpha_i|=q^{\frac{n+1}{2}}$ for
all $1\le i\le n$.\vskip 5mm

Let
$g(\lambda,x)=x_0^{n+1}+\cdots+x_n^{n+1}-\lambda x_0x_1\cdots
x_n$. For almost all $\lambda$,  it defines a smooth projective hypersurface in ${\mathbb{P}}^n$.
Then
$$Z(g(\lambda,x),T)=\frac{P(T)^{(-1)^n}}{(1-T)(1-qT)\cdots(1-q^{n-1}T)},$$
where $P(T)$ is a polynomial of degree ${\frac{n(n^n-(-1)^n)}{n+1}}$,
pure of weight $n-1$. The Newton polygon of $P(T)\ge
HP$ (Dwork).

\underline{\bf Question}:\quad Is this family $g(\lambda, x)$ of projective hypersurfaces
in ${\mathbb{P}}^n$
generically ordinary for all $p>n+1$? (yes for $n\le 3$.)

\subsection{Zeta functions of affine toric hypersurfaces}

Let $f\in \f_q[x_1^{\pm},\cdots,x_n^{\pm}], \Delta=\Delta(f), f$ is
$\Delta$-regular. Then trivially $T=1$ is a root of
$L(x_0f,T)^{(-1)^n}$. And
$$\frac{L(x_0f,T)^{(-1)^n}}{1-T}=P(f,qT)$$ is a polynomial in
$1+T\Z[T]$ of degree $d(\Delta)-1$ (with slope $\ge 1$), where
$P(f,T)$ is a polynomial in $\Z[T]$. We have
$$\aligned Z(U_f,qT)=&\prod_{i=0}^n(1-q^iT)^{(-1)^{n-i-1} {n\choose
i}}L(x_0f,T)\\ =&\prod_{i=1}^n(1-q^iT)^{(-1)^{n-i-1} {n\choose
i}}(\frac{L(x_0f,T)^{(-1)^n}}{1-T})^{(-1)^n}\\
=&\prod_{i=1}^n(1-q^iT)^{(-1)^{n-i-1} {n\choose
i}}P(f,qT)^{(-1)^n},\\
Z(U_f,T)=&\prod_{i=0}^{n-1}(1-q^iT)^{(-1)^{n-i} {n\choose
i+1}}P(f,T)^{(-1)^n},\endaligned$$
where$$P(f,T)=\prod_{i=0}^{d(\Delta)-2}(1-\beta_i T)$$ and
the $\beta_i$'s are algebraic numbers.

\begin{Defn}The primitive Hodge
polygon $PHP(\Delta)$ is the polygon in $\R^2$ with vertices
$(0,0)$ and $(\sum_{k=0}^m h_\Delta(k),\sum_{k=0}^m
(k-1)h_\Delta(k)), 1\le m\le n$.
\end{Defn}
\begin{center}
 \setlength{\unitlength}{1cm}
\begin{picture}(5,4)
\put(0,0){\circle*{0.1}}\put(0,0){\line(1,0){2}}
\put(2,0){\circle*{0.1}}\put(2,0){\line(1,1){1}}
\put(3,1){\circle*{0.1}}\put(3,1){\line(1,2){0.5}}
\put(3.5,2){\circle*{0.1}} \put(3.5,2){\line(1,3){0.7}}
%%other points
\put(0.9,0.1){0}\put(2.3,0.5){1}\put(4,1.5){$PHP(\Delta)$}
\put(0,-0.1){$\underbrace{\hskip
2cm}$}\put(2,-0.1){$\underbrace{\hskip 1cm}$}
\put(0.7,-0.5){$_{h_\Delta(1)}$}\put(2.2,-0.5){$_{h_\Delta(2)}$}
 \put(0.3,-1.3){\text{Primitive Hodge polygon of $\Delta$}}
\end{picture}\end{center}\vskip 1.3cm

\hskip 7mm$P(f,T)=\det(I-Frob_q T\mid \text{PH}_c^{n-1}(U_f\otimes \overline{\f}_q,\Q_\ell))$
\vskip 1mm $\Longrightarrow$ All results on $L(x_0f,T)^{(-1)^n}$
carry over to $P(f,T)$.
\begin{Cor}If $f$ is $\Delta$-regular over $\f_q$, then $$
\#U_f(\f_{q^k})=\frac{(q^k-1)^n+(-1)^{n+1}}{q^k}+(-1)^{n+1}(\beta_0^k
+\beta_1^k+\cdots+\beta_{d(\Delta)-2}^k),$$where $|\beta_i|\le
q^{\frac{n-1}{2}}$.\end{Cor}

\section{$p$-adic Variation}

\subsection{$p$-adic Analytic formula for the Frobenius matrix}

Let $f(\overline{\lambda},x)$ be the universal family of $f\in
M_p(\Delta)$. For $f(\overline{\lambda},x)\in M_p(\Delta)(\f_q)$,
$$P({f(\overline{\lambda},x)}, T)=\det (I-Frob_qT\mid
\text{PH}_c^{n-1})=\det(I-F(\lambda)T).$$ Here $F(\lambda$ is a matrix of size
$(d(\Delta)-1)\times (d(\Delta)-1)$. Is there any $p$-adic analytic formula for
$F(\lambda)$? Since the relative cohomology forms a locally free overconvergent 
$\sigma$-module, one obtains 

\begin{Thm} Zariski locally on $M_p(\Delta)$, there
exists an overconvergent 
matrix $A(\lambda)$ of size $(d(\Delta)-1)\times (d(\Delta)-1)$ of the form
$$A(\lambda)=\begin{pmatrix}A_{00}(\lambda)&A_{00}(\lambda)&\cdots\\
pA_{10}(\lambda)&pA_{11}(\lambda)&\cdots\\
\vdots&\vdots& \\
p^{n-1}A_{n-1,0}(\lambda)&p^{n-1}A_{n-1,1}(\lambda)&\cdots\end{pmatrix},$$
where  $A_{ij}(\lambda)$ has $h_\Delta(i+1)$ rows and
$h_\Delta(j+1)$ columns, whose entries are overconvergent functions
on the lifting of $M_p(\Delta)$ with norm $\le 1$, satisfies the
following property:

\begin{enumerate}\item[]if $f(\overline{\lambda},x)\in
M_p(\Delta)(\f_{p^a})$ and
$\lambda=\text{Teich}(\overline{\lambda})$, then one can take
$$F(\lambda)=A(\lambda^{p^{a-1}})\cdots A(\lambda^p)A(\lambda).$$
\end{enumerate}
That is,
$P(f(\overline{\lambda},x),T)=\det(I-A(\lambda^{p^{a-1}})\cdots
A(\lambda^p)A(\lambda)T)$.\end{Thm}

\subsection{Deformation theory and Picard-Fuch equation}

Let $p>2$. Since the relatively cohomology forms an overconvergent $F$-crystal 
whose underlying differential equation is the Picard-Fuch equation, we deduce that 
the matrix $A(\lambda)$ as above can be expressed in terms of a fundamental solution
matrix $C(\lambda)$ of the Picard-Fuch equation:

$$A(\lambda)=C(\lambda^p)^{-1}A(\lambda_0)C(\lambda),$$
where $\lambda_0$ is a regular point.


\underline{\bf Remark}:\quad $C(\lambda)$ is NOT analytic on the
closed unit disk near $\lambda_0$.


\underline{\bf Example}:\quad Let $$\pi^{p-1}=-p,
\lambda^{p^a}=\lambda, \theta(\lambda)=\exp(-\pi
\lambda^p)\cdot\exp(\pi\lambda) \rightsquigarrow A(\lambda),$$
$$\Psi\circ
\text{Tr}_{\f_{q^k}/{\f_p}}(\overline{\lambda})=\theta(\lambda^{p^{a-1}})
\cdots\theta(\lambda^p)\theta(\lambda).\hskip 2.6cm$$ If
$P(f(\overline{\lambda},x),T)=\prod_{i=0}^{d(\Delta)-2}(1-\alpha_i(\lambda) T)\in \Z[T]$ has
$h_\Delta(k+1)$ reciprocal roots with slope $k$, $k=0,1,\cdots, n-1$,
then $$\aligned P(f(\overline{\lambda},x),T)=&\prod_{i=0}^{n-1}P_i(\lambda, T),\quad P_i(\lambda, T)\in \Z_p[T],\\
\deg P_i(\lambda, T)=&h_\Delta(i+1)\\ i=&\text{ slope of } P_i(\lambda, T).\endaligned$$

\underline{\bf Question}: Any $p$-adic analytic formula for $P_i(\lambda, T)$?

\subsection{Hodge-Newton decomposition and unit root formula}

Let $\Delta$ be ordinary at $p>2$ and $H_p(\Delta)$ be the ordinary
locus of $M_p(\Delta)$. One wishes to find a new basis such that 
the new matrix 
$$\begin{pmatrix}I_{00} && -E_{01}(\lambda^p) \\
0 && I_1\end{pmatrix} 
A(\lambda)
\begin{pmatrix}I_{00} && E_{01}(\lambda) \\
0 && I_1\end{pmatrix}
=\begin{pmatrix}
B_{00}(\lambda)&&*\\ 0&&pA'(\lambda) \end{pmatrix}.$$ 
This defines a $p$-adic contraction map and thus 
it has a unique solution matrix $E_{01}(\lambda)$ which is a 
convergent matrix of $h_{\Delta}(0)$ rows and $\sum_{i=1}^n h_{\Delta}(i)$ columns. 
By induction, one then shows that there exists a convergent matrix
$D(\lambda)$ on the lifting of $H_p(\Delta)$ such that
$$D(\lambda^p)^{-1}A(\lambda)D(\lambda)=\begin{pmatrix}
B_{00}(\lambda)&&*&*\\ 0&&pB_{11}(\lambda)&*\\
0&&0&\ddots\end{pmatrix}.$$ 
Then
$$P(f(\overline{\lambda},x),T)
=\prod_{i=0}^{n-1}\det\left(I-p^{ai}B_{ii}(\lambda^{p^{a-1}})\cdots
B_{ii}(\lambda^p)B_{ii}(\lambda)T\right),$$ where
$f(\overline{\lambda}, x)\in H_p(\Delta)(\f_{p^a})$,
$B_{ii}(\lambda)$ is convergent (not overconvergent) on the closed unit disk,
$$B_{ii}(\lambda)=C_{ii}(\lambda^p)^{-1}B_{ii}(0)C_{ii}(\lambda),$$
where $C_{ii}(\lambda)$ is a fundamental solution matrix of a
piece of the Picard-Fuch equation. The $p$-adic analytic formula for $P_i(\lambda, T)$ is then
$$P_i(\lambda,T)= \det\left(I-p^{ai}B_{ii}(\lambda^{p^{a-1}})\cdots
B_{ii}(\lambda^p)B_{ii}(\lambda)T\right).$$


\subsection{Unit root $L$-function and $p$-adic Galois representation}

Let $\Delta$ be ordinary at $p$. As above,
$$A(\lambda)\sim\begin{pmatrix}
B_{00}(\lambda)&&*&*&*\\
0&&pB_{11}(\lambda)&*&*\\ 0&&0&\ddots&*\\
0&&0&0&p^{n-1}B_{n-1, n-1}(\lambda)\end{pmatrix}.$$ Each
$B_{ii}(\lambda)$ is invertible on the lifting of $H_p(\Delta)$ and hence 
it defined a unit root F-crystal on $H_p(\Delta)$. Alternatively, we have 

\begin{Thm}[Katz] Each $B_{ii}$ defines a continuous $p$-adic representation
$$\rho_i:\, \pi_1^{\text{arith}}(H_p(\Delta)/\f_p)\longrightarrow
GL_{h_{\Delta}(i+1)}(\Z_p),$$
such that 
$$\rho_i(\text{Frob}_\lambda)=B_{ii}(\lambda^{p^{a-1}})\cdots
B_{ii}(\lambda^p)B_{ii}(\lambda).$$ 
\end{Thm}
It is clear that the L-function 
$$L(\rho_i, T)=\prod_{\overline{\lambda}\in
H_p(\Delta)\atop{\text{closed
point}}}\frac{1}{\det\left(I-T^{\deg(\lambda)}
\rho_i(\text{Frob}_\lambda)\right)}\in
1+T(\overline{\Q}\cap\Z_p)[[T]]$$ is analytic in $|T|_p<1$.

\subsection{Dwork's unit root conjecture}

\begin{Thm}[Wan] $L(\rho_i, T)$ is $p$-adic meromorphic everywhere.
\end{Thm}

Let $\rho=\rho_i$. Write the $p$-adic Weierstrass factorization
$$ L(\rho_i,
T)=\frac{\prod_{j=1}^\infty(1-z_j^{(1)}T)}
{\prod_{j=1}^\infty(1-z_j^{(2)}T)},\quad z_j^{(1)}\rightarrow
0,z_j^{(2)}\rightarrow 0.$$

\underline{\bf Question}:\quad Let $K_p=\Q_p(z_j^{(1)},
z_j^{(2)} | 1\leq j <\infty)$.  Is $[K_p: \Q_p]<\infty$? ($p$-adic RH for $L(\rho_i,
T)$).

\begin{Defn} Let $$r_\rho^+=\limsup_{x\rightarrow\infty}
\frac{\log(1+ \#\{i\mid\text{ord}_q(z_i^{(1)})\le
x\}+\#\{j\mid\text{ord}_q(z_j^{(2)})\le x\})}{\log x}$$
\end{Defn}

This is called the order of the $p$-adic meromorphic function $L(\rho, T)$.
It measures the size of $L(\rho, T)$. Clearly, we have $0\le r_\rho^+\le +\infty$.

\medskip
\underline{\bf Question}:\quad $r_\rho^+< +\infty$?


\begin{Thm}[Wan] If rank$(\rho)=1\Rightarrow r_{\rho}^+< +\infty$.
\end{Thm}
(true for the family $x_1+\cdots +x_n +\frac{1}{x_1\cdots x_n} -\lambda$)

\begin{Defn} Let $$r_\rho^-=\limsup_{x\rightarrow\infty}
\frac{ \log(1+ |\#\{i\mid\text{ord}_q(z_i^{(1)})\le
x\}-\#\{j\mid\text{ord}_q(z_j^{(2)})\le x\}|)}{\log x}$$
\end{Defn}

Clearly, $0\leq r_\rho^- \leq r_\rho^+ \leq +\infty$.

\medskip
\underline{\bf Question}:\quad $r_\rho^-< +\infty$? (yes if rank$(\rho)=1$)



\subsection{$p$-adic Monodromy group}

Let $\Delta$ be ordinary at $p$. Let
$$\rho=\rho_i:\, \pi_1^{\text{arith}}(H_p(\Delta)/\f_p)\longrightarrow
GL_{h_{\Delta}(i+1)}(\Z_p).$$
Then $G_p(\Delta, i)=\rho_i(\pi_1^{\text{arith}})$ is a $p$-adic
Lie-group.

\bigskip
\underline{\bf Question}:\quad $G_p(\Delta, i)=?$

\bigskip
\underline{\bf Example} (Igusa). For the elliptic family $x_1+x_2+\frac{1}{x_1x_2}-\lambda$,
one has
$$G_p(\Delta, 0)=G_p(\Delta, 1) =\Z_p^* =GL_1(\Z_p).$$


\end{document}
