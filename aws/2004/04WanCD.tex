%%  Daqin Wan's course description

\documentclass[10XSpt]{article}
%\usepackage{amsfonts}
%\usepackage{amsmath,amssymb}

%\renewcommand{\theequation}{\thesection.\arabic{equation}}
%\setcounter{equations}{section}
\newfont{\Bb}{msbm10 scaled\magstep1}
\newfont{\Bbs}{msbm10 scaled\magstep0}

\newcommand{\fq}{\mbox{\Bb F}_{q}}
\newcommand{\fqk}{\mbox{\Bb F}_{q^{k}}}
\newcommand{\fpk}{\mbox{\Bb F}_{p^{k}}}
\newcommand{\fqr}{\mbox{\Bb F}_{q^{k-1}}}
\newcommand{\sfqk}{\mbox{\Bbs F}_{q^{k}}}
\newcommand{\fp}{\mbox{\Bb F}_{p}}
\newcommand{\fqbar}{\bar{\mbox{\Bb F}}_{q}}
\newcommand{\ftwoa}{\mbox{\Bb F}_{2^{a}}}
\newcommand{\Tr}{\mbox{\rm Tr}}
\newcommand{\sTr}{\mbox{\rm \scriptsize Tr}}
\newcommand{\Teich}{\mbox{\rm Teich}}
\newcommand{\unram}{\mbox{\rm \scriptsize unram}}

\newcommand{\Qbar}{\bar{\mbox{\Bb Q}}}
\newcommand{\Qp}{\mbox{\Bb Q}_{p}}
\newcommand{\K}{K} %% ramified extension of degree m
\newcommand{\Zp}{\mbox{\Bb Z}_{p}}
\newcommand{\Z}{\mbox{\Bb Z}}
\newcommand{\C}{\mbox{\Bb C}}
\newcommand{\Zl}{\mbox{\Bb Z}_{\ell}}
\newcommand{\ord}{\mbox{\rm ord}}
\newcommand{\sord}{\mbox{\rm \scriptsize ord}} %% small version for
\newcommand{\degp}{\mbox{\rm deg}_{+}}
\newcommand{\degm}{\mbox{\rm deg}_{-}}
\newcommand{\NP}{\mbox{\rm NP}}
\newcommand{\HP}{\mbox{\rm HP}}
\newcommand{\GNP}{\mbox{\rm GNP}}                          %% exponent

\newcommand{\ring}{R} %% ring of integers of Omega_{0}
\newcommand{\Oh}{{\cal O}}
\newcommand{\SoftOh}{\tilde{{\cal O}}} %% Change this

%% Miscellaneous

\newcommand{\wt}{\mbox{\rm wt}}
\newcommand{\swt}{\mbox{\rm \scriptsize wt}}
\newcommand{\wttil}{\tilde{\mbox{\rm wt}}}
\newcommand{\swttil}{\tilde{\mbox{\rm \scriptsize wt}}}
\newcommand{\Gal}{\mbox{\rm Gal}} %% ditto
\newcommand{\Norm}{\mbox{\rm Norm}}
\newcommand{\Ker}{\mbox{\rm Ker}}

%% Other fields etc

\newcommand{\N}{\mbox{\Bb N}}
\newcommand{\sN}{\mbox{\Bbs N}}
\newcommand{\Q}{\mbox{\Bb Q}}
%\newcommand{\Z}{\mbox{\Bb Z}}
\newcommand{\Znn}{\mbox{\Bb Z}_{\geq 0}}
\newcommand{\sZ}{\mbox{\Bbs Z}}
\newcommand{\sZnn}{\mbox{\Bbs Z}_{\geq 0}}
\newcommand{\R}{\mbox{\Bb R}}
\newcommand{\LA}{L} %% the banach space
\newcommand{\LAtil}{\tilde{L}}
\newcommand{\A}{A} %% the ring zp[pi,epsilon]
\newcommand{\Atil}{\tilde{A}} %% the ring zp[pi,epsilon,gamma]
\newcommand{\M}{M} %% the matrix M

\newtheorem{theorem}{Theorem}[section]
\newtheorem{corollary}[theorem]{Corollary}
\newtheorem{lemma}[theorem]{Lemma}
\newtheorem{maintheorem}{Main Theorem}
\newtheorem{proposition}[theorem]{Proposition}
\newtheorem{definition}[theorem]{Definition}
\newtheorem{conjecture}[theorem]{Conjecture}
\newtheorem{question}[theorem]{Question}


%% \newtheorem{algorithm}[theorem]{Algorithm}

%% Environments

%\newenvironment{proof}{{\it Proof:}}{\hfill $\Box$ \newline}

\newenvironment{exafont}{\begin{bf}}{\end{bf}}
\newenvironment{example}{\vspace{0.3cm}\par\noindent\refstepcounter{theorem}\begin{exafont}Example \thetheorem\end{exafont}\hspace{\labelsep}}{\vspace{0.3cm}\par}
\newenvironment{algorithm}{\vspace{0.3cm}\par\noindent\refstepcounter{theorem}\begin{exafont}Algorithm \thetheorem\end{exafont}\hspace{\labelsep}}{\vspace{0.3cm}\par}
\newenvironment{note}{\vspace{0.3cm}\par\noindent\refstepcounter{theorem}\begin{exafont}Note
\thetheorem\end{exafont}\hspace{\labelsep}}{\vspace{0.3cm}\par}
\newenvironment{observation}{\vspace{0.3cm}\par\noindent\refstepcounter{theorem}\begin{exafont}Observation \thetheorem\end{exafont}\hspace{\labelsep}}{\vspace{0.3cm}\par}

%\setcounter{equation}{section}

\def\V{{\mathcal V}}
\def\U{{\mathcal U}}
\def\F{{\bf F}_q}


\title{{Zeta Functions of Toric Calabi-Yau Hypersurfaces}\\ 
(Course and Project Description)}

\begin{document}

\author{Daqing Wan}


\maketitle

The aim of this course is to apply certain recent developments in 
Dwork's $p$-adic theory to study the $p$-adic variation of the 
zeta function attached to a family of affine toric Calabi-Yau hypersurfaces 
over finite fields, leading up to Dwork's unit root conjecture. 

\section{The basic example}

An important example is the following family of 
$(n-1)$-dimensional affine toric Calabi-Yau hypersurfaces 
$$ X_{\lambda}: X_1 +\cdots +X_n + {1\over X_1\cdots X_n} + \lambda =0, \ X_i \not=0,$$
parametrized by $\lambda$. This is the mirror family (actually an affine open piece 
of it) of the standard Calabi-Yau family in the projective $n$-space defined by 
$$Y_{\lambda}: X_0^{n+1}+X_1^{n+1} +\cdots + X_n^{n+1} + \lambda X_0X_1\cdots X_n=0.$$
The zeta function of this mirror pair of families over finite fields have 
been studied in great detail by Dwork \cite{D1} for $n=2, 3$ and more recently 
by Candelas, de la Ossa and Rodriguez Villegas \cite{COR} for $n=4$. 
The zeta function 
and its roots are closely related to the solutions of the Picard-Fuch 
differential equations. 
In this course, we study the zeta function of the family $X_{\lambda}$ 
for all $n$ and its $p$-adic variation when the parameter $\lambda$ 
varies. 



Let $\F$ be the finite field of $q$ elements, where $q=p^a$ and $p$ is a 
prime number. Removing a finite number of the values for $\lambda$, we 
assume that the toric hypersurface $X_{\lambda}$ is $\Delta$-regular. 
Let $\lambda \in \F$. For a positive integer $k$, let $N_k(\lambda)$ denote the 
number of ${\bf F}_{q^k}$-rational points on $X_{\lambda}$. 
We shall explain how to prove the following results. 

1. There are $n$ algebraic integers $\alpha_0(\lambda), \cdots, 
\alpha_{n-1}(\lambda)$ 
such that for each positive integer $k$, one has the formula 
$$N_k(\lambda) = {{(q^k-1)^n +(-1)^{n+1}}\over q^k} + 
(-1)^{n+1}\sum_{i=0}^{n-1} \alpha_i(\lambda)^k,$$
where the $\alpha_i(\lambda)$ are the Frobenius eigenvalues acting 
on the primitive part of the cohomology. 
Furthermore, each $\alpha_i(\lambda)$ has complex absolute value $\sqrt{q}^{n-1}$. 

2. The family $X_{\lambda}$ is generically ordinary for every $p$, that is, 
the generic Newton polygon coincides with the Hodge polygon.  
This means that there is a non-zero Hasse polynomial $H_p(\lambda)$ over ${\bf F}_p$ 
for each prime number $p$, 
such that for all $a$ and all $\lambda \in {\bf F}_{p^a}$  
with $H_p(\lambda)\not=0$, we have 
$$\alpha_i(\lambda) = p^{ai}u_i(\lambda), (0\leq i\leq n-1),$$
where each $u_i(\lambda)$ is a $p$-adic unit.  

3. For each $p$, there are $p$-adic ``analytic" 
functions $f_{i,p}(x)$ ($0\leq i \leq n-1)$  
such that for each ordinary $\bar{\lambda} \in {\bf F}_{p^a}$, one has the 
$p$-adic analytic formula for each unit $u_i(\bar{\lambda})$:  
$$u_i(\bar{\lambda}) = f_{i,p}(\lambda) f_{i,p}(\lambda^p)\cdots 
f_{i,p}(\lambda^{p^{a-1}}),$$
where $\lambda$ denotes the Teichmuller lifting of $\bar{\lambda}$. 
Each $f_{i,p}(x)$ is formally of the form $g_{i,p}(x)/g_{i,p}(x^p)$, 
where $g_{i,p}(x)$ is a solution of the Picard-Fuch equation. 

4. Each $f_{i,p}(x)$ defines the Frobenius matrix of 
a rank one unit root F-crystal, or equivalently, 
a rank one $p$-adic Galois representation for the arithmetic fundamental 
group of the ordinary parameter space over ${\bf F}_p$. For each integer $k$, 
the $k$-th power unit root L-function 
is defined to be 
$$L(u_i^k, T) =\prod_{\bar{\lambda}}{1\over 1-u_i(\lambda)^k T^{{\rm deg}(\lambda)}},
$$
where $\bar{\lambda}$ runs over the closed points of the ordinary parameter space 
over ${\bf F}_p$. Each of these unit root L-functions is a $p$-adic 
meromorphic function as conjectured by Dwork. Furthermore, for this example, 
they are $p$-adic meromorphic function of finite order!    

\section{Course content}


We hope to summarize the essence of some of the following materials 
in the setting of toric hypersurfaces and explain how to apply them 
to study the above basic example. 

1. Dwork's trace formula, $p$-adic meromorphic continuation 
of the zeta function via $p$-adic Fredholm determinant of 
a compact operator, see \cite{W2} for an elementary exposition.  
Cohomological formulas.  The results of 
Adolphson-Sperber \cite{AS} and Denef-Loeser \cite{DL} as applied 
to toric hypersurfaces.  


2. The Katz type conjecture. Newton polygon and its relation to Hodge polygon 
\cite{Ma}\cite{AS2}.  
Generic Newton polygon, generic ordinary primes, 
star decomposition theorem \cite{W1} 
and collapsing decomposition theorem \cite{W5}.   

3. Deformation theory \cite{D0}, 
Newton-Hodge decomposition and unit root formula \cite{Ka}\cite{D2}. 

4. Dwork's unit root conjecture \cite{D2}, 
the speaker's work \cite{W3}\cite{W4} and Grosse-Kl\"onne's 
rank one generalization \cite{GK}.  


\section{Student project} 


Take your favorite universal family (other than the above basic example) 
of toric Calabi-Yau hypersurfaces of dimension at least $4$, say, 
arising from other applications. 
That is, take your favorite reflexive 
integral polyhedron $\Delta$ in ${\bf R}^n$ containing the origin as an 
interior point, where $n\geq 5$. 
Determine if this family is generically ordinary 
for every prime number $p$, using possibly a combination of the star 
decomposition theorem and the collapsing decomposition 
theorem \cite{W5} (the first part of this article explains how to 
use the decomposition theorems to determine if a family is generically 
ordinary at $p$).  If you cannot find such a family, 
ask someone in mirror symmetry 
(or at the winter school) to give you a nice family of toric CY hypersurfaces 
to play with. Another possibility is to explore the link between the 
unit root formula and mirror symmetry.  

\section{Prerequisites}

The minimal background is Koblitz's book \cite{Ko}, which 
contains basic properties of $p$-adic numbers, $p$-adic 
analytic functions, Newton polygon, and Dwork's rationality proof.  
The last part will be reviewed in the course. 
A further reading in this direction is Monsky's book \cite{Mo}, which 
explains Dwork's $p$-adic cohomological formula and how to 
calculate it in the setting of smooth projective hypersurface case. 
For a classical motivation on the relation between Newton polygon 
and Hodge polygon, see Mazur's expository article \cite{Ma}.  
For Hodge numbers and Picard-Fuch equation of affine 
toric Calabi-Yau hypersurfaces, we refer to Batyrev's paper \cite{Ba}. 
For a nice and self-contained exposition on the link between 
zeta functions and periods for the quintic Calabi-Yau hypersurfaces, 
see paper \cite{COR}.  
Attendee interested in 
algorithms for computing the zeta function is referred to the 
expository article \cite{W6}. 





\begin{thebibliography}{[MT1]}

%
\bibitem{AS} A. Adolphson and S. Sperber, Exponential sums and Newton polyhedra: 
cohomology and estimates, Ann. Math., 130(1989), 367-406. 

%
\bibitem{AS2}A. Adolphson and S. Sperber, 
On the zeta function of a complete intersection,  Ann. Sci. \'Ecole Norm. Sup. 
(4) 29 (1996), no. 3, 287-328.

%
\bibitem{Ba} V.V. Batyrev, Variation of the mixed Hodge structure of affine hypersurfaces 
in algebraic tori, Duke Math. J., 69(1993), 349-409.  

%
\bibitem{DL}J. Denef and F. Loeser, Weights of exponential sums, intersection cohomology, 
and Newton polyhedra, Invent. Math., 106(1991), 275-294. 


%
\bibitem{D0}D. Dwork, On the zeta functions of a hypersurface II, 
Ann. Math., 80(1964), 227-299. 

%
\bibitem{D1}
B. Dwork, $p$-adic Cycles, Publ Math., IHES., 37(1969), 27-115. 

%
\bibitem{D2} B. Dwork,
Normalized period matrices II, Ann. Math., 98(1973), 1-57.


%
\bibitem{COR} P. Candelas, X. de la Ossa and F. Rodriguez Villegas, 
Calabi-Yau manifolds over finite fields, I, preprint. 
``http://xxx.lanl.gov/abs/hep-th/0012233".  


%
\bibitem{GK} E. Grosse-Kl\"onne, On families of pure slope L-functions, Documenta Math., 
8(2003), 1-42. 
%
\bibitem{Ka1}N. Katz, On the differential equation satisfied by period matrices, 
Publ. Math., IHES., 35(1968), 71-106. 

%
\bibitem{Ka}N. Katz, Travaux de Dwork,
S\'eminaire Bourbaki, expos\'e 409 (1971/72),
Lecture Notes in Math., No. 317, 1973, 167-2000.

%
\bibitem{Ko}N. Koblitz, $p$-adic Numbers, $p$-adic Analysis and Zeta Functions, 
GTM, Vol 58, Springer-Verlag, 1996. 

%
\bibitem{Ma}B. Mazur, Frobenius and the Hodge filtration, Bull. AMS., 78(1972), 
653-667. 

%
\bibitem{Mo}P. Monsky, $p$-adic Analysis and Zeta Functions, Lectures in Math., 
Kyoto Univ., Kinokuniya Bookstore, Tokyo. 

%
\bibitem{W1} D. Wan, Newton polygons of zeta functions and L-functions, 
Ann. Math., 137(1993), 247-293.


%
\bibitem{W2} D. Wan, Meromorphic continuation of L-functions of $p$-adic
representations, Ann. Math., 143(1996), 469-498.


%
\bibitem{W3} D. Wan, Higher rank case of Dwork's conjecture, J. Amer. Math. Soc.,
13(2000), 807-852.

%
\bibitem{W4} D. Wan, Rank one case of Dwork's conjecture,
J. Amer. Math. Soc., 13(2000), 853-908.


%
\bibitem{W5} D. Wan, Variation of $p$-adic Newton polygon for L-functions 
of exponential sums, ``http://www.math.uci.edu/{$\sim$}dwan/preprint.html". 

%
\bibitem{W6} D. Wan, Algorithmic theory of zeta functions over finite fields, 
to appear in MSRI Algorithmic Number Theory Proceedings, 
``http://www.math.uci.edu/{$\sim$}dwan/preprint.html". 




\end{thebibliography}





\end{document}


