\documentclass{article}
\setlength{\textwidth}{6truein}
\setlength{\oddsidemargin}{0truein}\setlength{\evensidemargin}{0truein}
\setlength{\parindent}{0pt} \setlength{\parskip}{\bigskipamount}
\def\Qp{\mathbf{Q}_{p}}
\renewcommand\refname{Suggested Reading}

\def\germ{\frak}

\def\cprime{\/{\mathsurround=0pt$'$}}
\def\GL{\mathrm{GL}}
\def\SL{\mathrm{SL}}
\def\PSL{\mathrm{PSL}}
\def\Cp{{\bf C}_p}
\def\Zp{{\bf Z}_{p}}
\def\Z{{\bf Z}}
\def\Q{{\bf Q}}
\def\PGL{\mathrm{PGL}}
\def\dim{\mathop{dim}}
\def\Hom{\mathop{\rm Hom}\nolimits}
\def\X{\mathcal{X}}
\def\Ends{\mathop{\rm Ends}}
\def\Edges{\mathop{\rm Edges}\nolimits}
\def\Ver{\mathop{\rm Ver}\nolimits}
\def\Res{\mathop{\rm Res}\nolimits}
\def\R{{\bf R}}
\def\C{\mathbf{C}}
\def\O{\mathcal{O}}
\def\ord{\mathop{\rm ord}\nolimits}
\def\into{\hookrightarrow}
\def\OX{\mathcal{O}_{\mathcal{X}}}
\def\ind{\mathop{\rm ind}\nolimits}
\def\Symm{\mathop{\rm Symm}\nolimits}
\begin{document}
\begin{center}
\Large The $p$-adic upper half plane \\
Course and Project Description \\
Arizona Winter School, March, 2007 \\
 Samit Dasgupta and Jeremy
Teitelbaum
\end{center}
The $p$-adic upper half plane $\X$ over a $p$-adic field $K$ is a
one-dimensional rigid analytic space whose points in any complete
extension $L/K$ are given by
$$
\X(L)={\bf P}^{1}(L)\backslash{\bf P}^{1}(K).
$$
This space was first introduced by Mumford, where it plays a key
role in the generalization to higher genus of Tate's theory of
$p$-adic uniformization of elliptic curves with semistable
reduction.  Slightly later, Drinfeld and Cerednik showed that
appropriate quotients of this space by discrete arithmetic subgroups
of $\PGL_{2}(K)$ coming from quaternion algebras yield Shimura
curves.  Since that time, through work of Morita, Schneider,
Bertolini, Darmon, Iovita, the authors, and many others, this space
and its relationships to arithmetic have been the subject of
intensive study.  In this course, we will study some aspects of this
recent work.

The first part of the course will be a construction of the $p$-adic
upper half plane $\X$ as a rigid space and a study of its
relationship to the Bruhat-Tits tree $T$ of $\PGL_{2}$. This tree
$T$ classifies norms on a fixed two dimensional vector space $V$ up
to scaling, and there is a map
$$ r:\X\to T$$ that commutes with the action of $\PGL_{2}$ on both
spaces.  We will also study the compactification of $T$ obtained by
adding its set of ends, a set homeomorphic to ${\bf P}^{1}(K)$.

In the second part of the course, we will explore the relationship,
first described by Schneider, between rigid analytic one-forms on
$\X$, ``harmonic" functions on the edges of the tree $T$, and
distributions on the (common) boundary ${\bf P}^{1}(K)$ of $\X$ and
$T$.  The key ideas in this part of the course are the residue map,
which carries rigid one-forms to harmonic functions on $T$, and the
integral transform, or ''Poisson Kernel," which carries measures on
${\bf P}^{1}(K)$ back to rigid one-forms.

In the third part, we will discuss a second, entirely different,
construction of measures on ${\bf P}^{1}(K)$ due to Darmon
 that also uses the tree, but that blends classical
harmonic forms and rigid geometry in a remarkable way.

In the last part of the course, we will examine some of the
arithmetic applications of the theory developed so far.  In
particular, we will describe the construction of two different
$\mathcal{L}$ invariants. Such invariants, first observed
experimentally in \cite{MTT}
,  relate the special values of $p$-adic
$L$-functions and classical $L$-functions of a modular form $f$ in
cases where the two functions have functional equations of different
signs; in that case, the $p$-adic $L$-function has an extra order of
vanishing, and its critical value differs from that of the classical
one by a $p$-adic number $\mathcal{L}(f)$.  We will describe the
quaternionic construction of an $\mathcal{L}(f)$-invariant due to
the second author, and the ``double-integral" construction due to
Darmon and Orton.  As time permits, we will discuss Orton's proof
that her $\mathcal{L}$-invariant depends only on the Galois
representation of $f$ locally at $p$, as originally conjectured in
\cite{MTT}.  Also time permitting, we will describe Breuil's definition of the $\mathcal{L}$-invariant \cite{b}.
\newpage

\begin{thebibliography}{ABC}

\bibitem{b} C. Breuil, S\'erie Sp\'eciale $p$-adique et Cohomolologie \'Etale Compl\'et\'ee. IHES preprint, 2003.

\bibitem{darmonHpH} H. Darmon, Integration on $\mathcal{H}\sb p\times\mathcal{H}$ and
arithmetic applications.  Ann. of Math. (2)  154  (2001),  no. 3,
589--639. 

\bibitem{GvDP} L. Gerritzen\ and\ M. van der Put,
{\it Schottky groups and Mumford curves}, Lecture Notes in Math.,
817, Springer , Berlin, 1980;  (see esp.
Chapter IX).

\bibitem{MTT}B. Mazur, J. Tate\ and\ J. Teitelbaum,
On $p$-adic analogues of the conjectures of Birch and
Swinnerton-Dyer , Invent. Math. {\bf 84} (1986), no.~1, 1--48.

\bibitem{orton} L. Orton, An elementary proof of a weak exceptional zero
conjecture.  Canad. J. Math.  56  (2004),  no. 2, 373--405.

\bibitem{JTT} J. T. Teitelbaum, Values of $p$-adic $L$-functions and a $p$-adic
Poisson kernel , Invent. Math. {\bf 101} (1990), no.~2, 395--410.

\end{thebibliography}

\newpage
\begin{center}
Project Description
\end{center}

Let ${\bf H}$ be the Hamilton quaternion algebra over ${\bf Q}$, and
let $A$ be the Hurwitz maximal order, with $5$ inverted:
$$
A={\bf Z}[\frac{1}{5},i,j,k,\epsilon]
$$
where $\epsilon=(1+i+j+k)/2$. Let $\Gamma=A^{*}/(\Z[1/5])^*$. Since
${\bf H}$ is split at $5$, we may choose an algebra embedding ${\bf
H}\to M_{2}(\Q_5)$, so that $\Gamma$ becomes a subgroup of
$\PGL_{2}(\Q_5)$.

We can identify (at least)  the following set of interesting
subgroups of $\Gamma$:
\begin{enumerate}
\item $\Gamma_{+}=\Gamma\cap\PSL_{2}(\Q_5)$
\item $\Gamma(P)=\{\gamma\in\Gamma : \gamma\equiv 1\pmod P\}$ where
$P$ is the unique prime ideal of $A$ above $2$.
\item $\Gamma_{+}(P)=\Gamma_{+}\cap\Gamma(P).$
\item $\Gamma(2)=\{\gamma\in\Gamma : \gamma\equiv 1\pmod 2\}$.
\item $\Gamma_{+}(2)=\Gamma_{+}\cap\Gamma(2)$
\end{enumerate}


The goal of the project is to explicitly compute as many as possible
of the following objects as $\mathbf{\Gamma}$ runs through the
groups given above:
\begin{enumerate}
\item The genus of the quotient curve
$X_{\mathbf{\Gamma}}=X/\mathbf{\Gamma}$. By Cerednik-Drinfeld, these
curves are Shimura curves classifying abelian varieties with
quaternionic multiplication by the indefinite algebra of
discriminant $10$ over ${\bf Q}$, with some level structure.
\item The structure of the stable reduction of
$X_{\mathbf{\Gamma}}$.
\item The Hecke action on the Jacobian of $X_{\mathbf{\Gamma}}$.
\item A basis for the space of harmonic cocycles of weights $2$ and
$4$.
\item The relationship of these Shimura curves to the elliptic
curves of conductor $20$ and $40$.
\item The $p$-adic period matrix of the Jacobians of
$X_{\mathbf{\Gamma}}$.
\item The quaternionic $\mathcal{L}$ invariant for the unique new
form of weight $4$ and level $10$.
\item The special values of the classical and $p$-adic $L$-function
for the elliptic curve of conductor $40$, compared with the previous
computation.
\end{enumerate}
\end{document}
