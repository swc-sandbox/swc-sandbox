\documentclass{article}

\title{Guide to the Arizona Winter School}
\date{}

\begin{document}
\maketitle


Welcome to the Arizona Winter School! The next five days are going to
be intense, so before you plunge in, look over these suggestions on
how to get the most out of them. We've divided the suggestions into
three sections, one for students, one for speakers, and one for postdoctoral
assistants (feel free to cheat and look at the other sections if you want).


\section{For students}
\label{sec:students}


The most important thing to remember is that the Arizona Winter School
is for you, not for the speakers, postdocs, or other people attending
the workshop. The courses, notes, projects, problems, and study groups have
been designed for you. Take part in them!


\subsection{How it all works}
\label{sec:how-it-all}

There are four (sometimes five) related topics of each school. 
For every topic, we have invited an expert (or two) to give a series of talks, for which 
lecture notes are posted in advance.  We have also asked the speakers  to
propose in advance a related project on which some of you will be
working.  Each speaker has requested and been assigned a postdoctoral
(sometimes tenure-track, even) assistant to help with the project.  In addition,
we have asked one or two postdoctoral fellows to create problems that can be
worked on during the school.  

You get to rank preferences for whether you 
wish to work in a particular project group, problem session, or study group.
You will be assigned in groups to one of these, though you should feel free to move around
if it suits you better when the time comes (at least unofficially... you will usually
not be part of the presentation if you weren't that project group in the first place, though).  
If you're in a project group, you'll be working with the speaker (your team leader) and
 postdoctoral assistant for that lecture series both before and during the school, and you will be part of 
a presentation with your team members at the end.  If you're in a problem session, then
you'll work with the postdoctoral fellow and your fellow students on the fun and
challenging problem sets during the evenings.  If you're in a study group, then you'll
work with you fellow students to understand the material of one of the courses, and yes,
the speaker and postdoctoral assistant for that course should be happy to help if you
don't understand something: just ask! 

\subsection{Before the school (i.e., now!)}
\label{sec:readings}

Read the project descriptions, lecture notes, and problems that have been
posted on the AWS web site.  If you're in a project group or study group, 
try to at least read those for your group, all of them if at all possible. 
Follow up on the references cited, and discuss them with your team members. S
tart to think about the projects that have been proposed. If you don't understand 
something, email your team leader and your team members with your
questions.

You aren't going to have a lot of time to do all of this during the
school itself.  If you come as prepared as you possibly can, you will get a
lot more out of it.  If you feel overwhelmed by the project
description, show it and the lecture notes to your Ph.D. advisor and ask for
suggested background reading. If you have time, organize a local
seminar at your institution on background material for one or more of
the topics. 


\subsection{Lectures}
\label{sec:lectures}
Read the notes in advance.
Don't be afraid to ask questions during the lectures.  Grab the front
rows; they are for you. If you don't want to ask a question during the
lecture, go up to the speaker afterwards. If there are points you
don't understand, ask the speaker to clarify them.


\subsection{Working groups}
\label{sec:working-groups}
There's a good chance that some of you will get stuck or very confused at some point
during your work on your assigned task.  Don't think that you are the only
one!  There are plenty of others who are and have been in the same boat.  
There are many people you can ask for help: your team
leader, your fellow team members, a friendly postdoc or senior
graduate student who happens to be floating around, or one of the
organizers of the school. Go back and look at the lecture notes and
papers that you read in preparation for the school, and see if they
shed any new light on the problem.


\subsection{Evening sessions}
\label{sec:even-quest-sess}
These are where most of the work and learning happens. Take advantage!  
Ask one of the speakers or postdocs to expand on that day's topic, or to
give a preview of what is coming.  Winter School alums will probably come down to
watch you suffer; make them work by helping you (you'll be able to
pick them by the fond smile of reminiscence on their faces).

``Ombudspeople" (typically a Winter School alum and a Southwestern Center member) 
will be available at the evening sessions and will try to resolve any issues that come up.  
If you are having trouble, but not quite sure who to ask or how to ask about it, then they 
are the people to go to.

%\subsection{Professional development component}
%\label{sec:prof-devel-comp}
%These activities are helpful and there is no exam on them. Take
%advantage of them.

\subsection{Presentations}
\label{sec:presentations}
If you're in a project group, you won't have a lot of time to give your part of the presentation, 
so make it as brief and efficient as possible. Practice your presentation with other members
of the group. Most novice speakers make the mistake of preparing too
much material; don't try to fill up all the time available. You will
have questions from the audience, and it generally takes longer to
explain something than it does to think it through in your head. If
you have messy details to report, don't report them. Summarize the
key points or put them on an overhead slide. Coordinate with your
team members so that you all use the same notation and don't have to
repeat it. 

\subsection{What to do in your spare time}
\label{sec:what-do-your}
You don't need to worry about this, you won't have any (except perhaps for part of the wonderful
free afternoon).

\section{For speakers}
\label{sec:speakers}


\subsection{Lectures}
\label{sec:lectures-1}
The most important thing to remember is that the Arizona Winter School
is not for the big shots sitting in the front row (they should be sitting further back anyway),  
it is for the students. There is a wide range of levels, and we
want to serve them all. Many AWS speakers have made the mistake of
preparing too much material for the time available. It is better to
make the talks clear and understandable and use the evening question
sessions for filling in extra details.  If at all possible, coordinate your presentations 
with the other speakers, especially if you think there might be some overlap in 
or relationship between your talks.  A bit of coherence to the school goes a 
long way.

\subsection{Working groups}
\label{sec:working-groups-1}
A team of graduate students will be assigned the project you
proposed. You are their team leader. You are responsible for getting
them through the project, and preparing them to make a coherent
presentation on their work at the end of the school. 
You will also have an assistant: please
discuss the project with him or her beforehand. 
Take the time to
get to know your students by email before the conference. If there are some who
seem less prepared, suggest reading to them. Meet with your students
early during the school, and set up a regular system of work sessions
with them.  In addition to
mathematical help, they may well need help on how to prepare a
presentation. Serious attention to the team projects can pay off well;
in the past, some of the projects have produced publishable work.

\subsection{Evening question sessions}
\label{sec:even-work-sess}
The evening question sessions are a crucial part of your job; that's
where students who didn't understand a point in your lecture can ask
you about it, and that's where your team members will get guidance
from you. This is where your assistant can be a great help: it's going
to take a coordinated effort from both of you to get them through the task.

There is likely to be a separate study group working on understanding
the material from your lectures as well.  Please be willing to occasionally
check on them and/or answer their questions so they can get the
most possible out of the school as well.

\subsection{What to do in your spare time}
\label{sec:what-do-your-1}
Alas, you won't have any of this either (again, except for the wonderful free afternoon). 

\section{For assistants}
\label{sec:assistants}

This brief section is for those who agreed to assist one of the speakers with their project group
or run an evening problem session.

\subsection{Project groups}
\label{sec:project-groups}
Get to know your group early on!  Why not introduce yourself before the school?
If you're assisting a project, then while the speaker holds the ultimate responsibility
for it, you play a crucial role in ensuring its success.  Ask the speaker beforehand what
is expected from you and how you're going to coordinate duties in the evenings.  Often,
you may want to arrange to meet with your group between lectures to chart their progress
as well.  Sometimes, you'll be able to give them that extra one-on-one attention that they need.

\subsection{Problem sessions}
\label{sec:problem-session}
Again, get to know your group early on.  You're in charge of making the problem session run 
smoothly.   Have your problems prepared well in advance.  Have an idea in mind for how you
would like the sessions to go, and be flexible when it comes to the actual sessions.
Be ready to spend long hours with the students at the working session!  And have lots of patience: 
they don't know the material as well as you do yet.
  
\subsection{What to do in your spare time}
\label{sec:what-do-your-1}
Again, this does not exist, except (if you're lucky) you might get to enjoy the 
wonderful free afternoon. 

\end{document}




%%% Local Variables: 
%%% mode: latex
%%% TeX-master: t
%%% End: 

% LocalWords:  postdocs AWS postdoc alums devel comp sess
