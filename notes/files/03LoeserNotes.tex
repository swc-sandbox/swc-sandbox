
\documentclass[english,12pt]{amsart}
\usepackage{amssymb}
\usepackage{amsfonts}
\usepackage{amscd}
%\usepackage[T1]{fontenc}
\usepackage[all]{xypic}
%\usepackage{aspmproc}




\textwidth=15cm
\oddsidemargin=5mm
\evensidemargin=5mm
\textheight=21.5cm
\parindent=0cm



\swapnumbers
\let\Bbb\mathbb
\def\limproj{\mathop{\oalign{lim\cr
\hidewidth$\longleftarrow$\hidewidth\cr}}}
\def\deg{{\rm deg}}


\def\aa{{\mathbf a}}\def\bb{{\mathbf b}}
\def\limind{\mathop{\oalign{lim\cr
\hidewidth$\longrightarrow$\hidewidth\cr}}}
\def\Var{{\rm Var}}
\def\Sing{{\rm Sing}}
\def\ac{{\rm ac}}
\def\Def{{\rm Def}}
\def\DefR{{\rm Def}_{R}}
\def\kvar{{K_0 ({\rm Var}_k)}}
\def\VarR{\Var_{R}}
\def\GL{{\rm GL}}
\def\Spec{{\rm Spec}}
\def\Gr{{\rm Gr}}
\def\SL{{\rm SL}}
\def\eff{{\rm eff}}
\def\longhookrightarrow{\mathrel\lhook\joinrel\longrightarrow}
\let\cal\mathcal
\let\got\mathfrak
\def\gP{{\got P}}
\def\pp{{\mathbf p}}
\def\AA{{\mathbf A}}
\def\BB{{\mathbf B}}
\def\CC{{\mathbf C}}
\def\DD{{\mathbf D}}
\def\EE{{\mathbf E}}
\def\FF{{\mathbf F}}
\def\GG{{\mathbf G}}
\def\HH{{\mathbf H}}
\def\II{{\mathbf I}}
\def\JJ{{\mathbf J}}
\def\KK{{\mathbf K}}
\def\LL{{\mathbf L}}
\def\MM{{\mathbf M}}
\def\NN{{\mathbf N}}
\def\OO{{\mathbf O}}
\def\PP{{\mathbf P}}
\def\QQ{{\mathbf Q}}
\def\RR{{\mathbf R}}
\def\SS{{\mathbf S}}
\def\TT{{\mathbf T}}
\def\UU{{\mathbf U}}
\def\VV{{\mathbf V}}
\def\WW{{\mathbf W}}
\def\XX{{\mathbf X}}
\def\YY{{\mathbf Y}}
\def\ZZ{{\mathbf Z}}


\def\cA{{\mathcal A}}
\def\cB{{\mathcal B}}
\def\cC{{\mathcal C}}
\def\cD{{\mathcal D}}
\def\cE{{\mathcal E}}
\def\cF{{\mathcal F}}
\def\cG{{\mathcal G}}
\def\cH{{\mathcal H}}
\def\cI{{\mathcal I}}
\def\cJ{{\mathcal J}}
\def\cK{{\mathcal K}}
\def\cL{{\mathcal L}}
\def\cM{{\mathcal M}}
\def\cN{{\mathcal N}}
\def\cO{{\mathcal O}}
\def\cP{{\mathcal P}}
\def\cQ{{\mathcal Q}}
\def\cR{{\mathcal R}}
\def\cS{{\mathcal S}}
\def\cT{{\mathcal T}}
\def\cU{{\mathcal U}}
\def\cV{{\mathcal V}}
\def\cW{{\mathcal W}}
\def\cX{{\mathcal X}}
\def\cY{{\mathcal Y}}
\def\cZ{{\mathcal Z}}
\mathchardef\alphag="7C0B
\mathchardef\betag="7C0C
\mathchardef\gammag="7C0D
\mathchardef\deltag="7C0E
\mathchardef\varepsilong="7C22
\mathchardef\varphig="7C27
\mathchardef\psig="7C20
\mathchardef\zetag="7C10
\mathchardef\epsilong="7C0F
\mathchardef\rhog="7C1A
\mathchardef\taug="7C1C
\mathchardef\upsilong="7C1D
\mathchardef\iotag="7C13
\mathchardef\thetag="7C12
\mathchardef\pig="7C19
\mathchardef\sigmag="7C1B
\mathchardef\etag="7C11
\mathchardef\omegag="7C21
\mathchardef\kappag="7C14
\mathchardef\lambdag="7C15
\mathchardef\mug="7C16
\mathchardef\xig="7C18
\mathchardef\chig="7C1F
\mathchardef\nug="7C17
\mathchardef\varthetag="7C23
\mathchardef\varpig="7C24
\mathchardef\varrhog="7C25
\mathchardef\varsigmag="7C26
\mathchardef\Omegag="7C0A
\mathchardef\Thetag="7C02
\mathchardef\Sigmag="7C06
\mathchardef\Deltag="7C01
\mathchardef\Phig="7C08
\mathchardef\Gammag="7C00
\mathchardef\Psig="7C09
\mathchardef\Lambdag="7C03
\mathchardef\Xig="7C04
\mathchardef\Pig="7C05
\mathchardef\Upsilong="7C07


\newtheorem{theorem}[subsubsection]{Theorem}
\newtheorem{lem}[subsubsection]{Lemma}
\newtheorem{klem}[subsubsection]{Key-Lemma}
\newtheorem{cor}[subsubsection]{Corollary}
\newtheorem{prop}[subsubsection]{Proposition}
\newtheorem{problem}[subsubsection]{Problem}
\newtheorem{kp}[subsubsection]{Key-Point}
\newtheorem{sconjecture}[subsubsection]{Rationality conjecture (strong form)}
\newtheorem{wconjecture}[subsubsection]{Rationality conjecture (weak form)}
\newtheorem{monoconjecture}[subsubsection]{Monodromy conjecture}
\newtheorem{gmonoconjecture}[subsubsection]{Generalised monodromy conjecture}
\newtheorem{holoconjecture}[subsubsection]{Holomorphy conjecture}
\newtheorem{gholoconjecture}[subsubsection]{Generalised holomorphy conjecture}

\theoremstyle{definition}
\newtheorem{definition}[subsubsection]{Definition}
\newtheorem{example}[subsubsection]{Example}
\newtheorem{xca}[subsubsection]{Exercise}
\newtheorem{def-prop}[subsubsection]{Proposition-Definition}
\newtheorem{def-theorem}[subsubsection]{Theorem-Definition}

\theoremstyle{remark}
\newtheorem{remark}[subsubsection]{Remark}
\newtheorem{remarks}[subsubsection]{Remarks}



\theoremstyle{plain}



\numberwithin{equation}{subsection}
\def\limind{\mathop{\oalign{lim\cr
\hidewidth$\longrightarrow$\hidewidth\cr}}}
\def\boxit#1#2{\setbox1=\hbox{\kern#1{#2}\kern#1}%
\dimen1=\ht1 \advance\dimen1 by #1
\dimen2=\dp1 \advance\dimen2 by #1
\setbox1=\hbox{\vrule height\dimen1 depth\dimen2\box1\vrule}%
\setbox1=\vbox{\hrule\box1\hrule}%
\advance\dimen1 by .4pt \ht1=\dimen1
\advance\dimen2 by .4pt \dp1=\dimen2 \box1\relax}
\def \pext{\, \hbox{\boxit{0pt}{$\times$}}\,}



\newcommand{\sur}[2]{\genfrac{}{}{0pt}{}{#1}{#2}}
\renewcommand{\theequation}{\thesubsection.\arabic{equation}}


\def\longhookrightarrow{\mathrel\lhook\joinrel\longrightarrow}
\let\cal\mathcal
\let\got\mathfrak
\def\AA{{\mathbf A}}
\def\BB{{\mathbf B}}
\def\CC{{\mathbf C}}
\def\DD{{\mathbf D}}
\def\EE{{\mathbf E}}
\def\FF{{\mathbf F}}
\def\GG{{\mathbf G}}
\def\HH{{\mathbf H}}
\def\II{{\mathbf I}}
\def\JJ{{\mathbf J}}
\def\KK{{\mathbf K}}
\def\LL{{\mathbf L}}
\def\MM{{\mathbf M}}
\def\NN{{\mathbf N}}
\def\OO{{\mathbf O}}
\def\PP{{\mathbf P}}
\def\QQ{{\mathbf Q}}
\def\RR{{\mathbf R}}
\def\SS{{\mathbf S}}
\def\TT{{\mathbf T}}
\def\UU{{\mathbf U}}
\def\VV{{\mathbf V}}
\def\WW{{\mathbf W}}
\def\XX{{\mathbf X}}
\def\YY{{\mathbf Y}}
\def\ZZ{{\mathbf Z}}


\def\cA{{\mathcal A}}
\def\cB{{\mathcal B}}
\def\cC{{\mathcal C}}
\def\cD{{\mathcal D}}
\def\cE{{\mathcal E}}
\def\cF{{\mathcal F}}
\def\cG{{\mathcal G}}
\def\cH{{\mathcal H}}
\def\cI{{\mathcal I}}
\def\cJ{{\mathcal J}}
\def\cK{{\mathcal K}}
\def\cL{{\mathcal L}}
\def\cM{{\mathcal M}}
\def\cN{{\mathcal N}}
\def\cO{{\mathcal O}}
\def\cP{{\mathcal P}}
\def\cQ{{\mathcal Q}}
\def\cR{{\mathcal R}}
\def\cS{{\mathcal S}}
\def\cT{{\mathcal T}}
\def\cU{{\mathcal U}}
\def\cV{{\mathcal V}}
\def\cW{{\mathcal W}}
\def\cX{{\mathcal X}}
\def\cY{{\mathcal Y}}
\def\cZ{{\mathcal Z}}


\def\ord{{\rm ord}}
\mathchardef\alphag="7C0B
\mathchardef\betag="7C0C
\mathchardef\gammag="7C0D
\mathchardef\deltag="7C0E
\mathchardef\varepsilong="7C22
\mathchardef\varphig="7C27
\mathchardef\psig="7C20
\mathchardef\zetag="7C10
\mathchardef\epsilong="7C0F
\mathchardef\rhog="7C1A
\mathchardef\taug="7C1C
\mathchardef\upsilong="7C1D
\mathchardef\iotag="7C13
\mathchardef\thetag="7C12
\mathchardef\pig="7C19
\mathchardef\sigmag="7C1B
\mathchardef\etag="7C11
\mathchardef\omegag="7C21
\mathchardef\kappag="7C14
\mathchardef\lambdag="7C15
\mathchardef\mug="7C16
\mathchardef\xig="7C18
\mathchardef\chig="7C1F
\mathchardef\nug="7C17
\mathchardef\varthetag="7C23
\mathchardef\varpig="7C24
\mathchardef\varrhog="7C25
\mathchardef\varsigmag="7C26
\mathchardef\Omegag="7C0A
\mathchardef\Thetag="7C02
\mathchardef\Sigmag="7C06
\mathchardef\Deltag="7C01
\mathchardef\Phig="7C08
\mathchardef\Gammag="7C00
\mathchardef\Psig="7C09
\mathchardef\Lambdag="7C03
\mathchardef\Xig="7C04
\mathchardef\Pig="7C05
\mathchardef\Upsilong="7C07



\author{Fran\c cois Loeser}
\title[$p$-adic and motivic integration]{Arizona Winter School Lecture Notes on 
$p$-adic and motivic integration}
\address{{\'E}cole Normale Sup{\'e}rieure,
D{\'e}partement de math{\'e}matiques et applications,
45 rue d'Ulm, 75230 Paris Cedex 05, France (UMR 8553 du CNRS)}
\email{Francois.Loeser@ens.fr}
\urladdr{http://www.dma.ens.fr/~loeser/}
\dedicatory{Revised, September 2003}
\begin{document}


\maketitle
\noindent
\setcounter{tocdepth}{1}
%\tableofcontents

\section{$p$-adic integration}

\subsection{The $p$-adic measure}Let $p$ be a prime number.
We consider a field $K$ 
with a valuation $\ord : K^{\times}\rightarrow \ZZ$,
extended to $K$ by $\ord (0) = \infty$.
We denote by $\cO_K$ the valuation ring 
$\cO_K = \{x \in K | \ord (x) \geq 0\}$ and we fix an uniformizing parameter
$\varpi$, that is,  an element of valuation $1$ in $\cO_K$.
The ring $\cO_K$
is a local ring with maximal 
ideal $\cM_K$ of $\cO_K$ generated by $\varpi$.
We shall assume the residue field
$k := \cO_K / \cM_K$ is finite with $q = p^e$ elements.
We endow $K$ with a norm by setting
$|x| := q^{- \ord (x)}$ for $x$ in $K$.
We shall furthermore assume $K$
is complete for $| \_ |$.



It follows in particular that the abelian groups
$(K^n, +)$ are locally compact, hence they have a canonical
Haar measure $\mu_n$, unique up to multiplication by a non zero
constant, so we may  assume $\mu_n (\cO_K^n) = 1$.
The measure  $\mu_n$ is the unique $\RR$-valued Borel measure on
$K^n$
which is invariant by translation and such that $\mu_n (\cO_K^n) = 1$.
For instance the measure of $a + \varpi^m \cO_K^n$ is $q^{- mn}$.
For any measurable subset $A$ of $K^n$ and any $\lambda$ in $K$,
$\mu_n (\lambda A) = \vert \lambda \vert^n \mu_n (A)$. More generally,
for every $g$ in ${\rm GL}_n (K)$, 
\begin{equation}\label{gl}
\mu_n (gA) = \vert {\rm det} g\vert \mu_n (A). 
\end{equation}


If $f$ is, say, a $K$-analytic function on $A$,
we
set 
$$
\int_A \vert f \vert \mu_n := \int_A \vert f \vert \vert dx \vert
:=
\sum_{m \in \ZZ}  \mu_n (\ord (f) = m) q^{-m},
$$
assuming the series $\sum_{m \in \ZZ}  \mu_n (\ord(f) = m) q^{-m}$
is convergent in $\RR$. More generally, we define similarly
$\int_A \vert f \vert^s \vert dx \vert$
by
$\sum_{m \in \ZZ}  \mu_n (\ord (f) = m) q^{-ms}$ whenever it makes sense.
For instance, when $n = 1$,
we have, for $s > 0$ in $\RR$,
\begin{equation}\label{cal}
\begin{split}
\int_{x \in \cO_K, \ord (x) \geq m} \vert x \vert^s &= \sum_{j \geq m} q^{-sj} \int_{\ord (x) = j} \vert dx \vert
= \sum_{j \geq m} q^{-sj} (q^{-j}- q^{-j -1})\\
&=(1 -q^{-1})q^{- (s +1) m} / (1- q^{-( s+ 1)}).\\
\end{split}
\end{equation}



\subsection{Integration on analytic varieties}Formula (\ref{gl}) is a very special form of the following fundamental change of variables
formula (see \cite{igusabook} p. 111):


\begin{prop}[The $p$-adic change of variables formula]\label{pcvf}
Let $U$ be an open subset of $K^n$
and consider  $K$-analytic functions $f_1, \dots, f_n$
on $U$. Assume
$f = (f_1, \dots, f_n) : U \rightarrow K^n$
is a  $K$-analytic isomorphism between
$U$ and an open subset $V$ of  $K^n$. 
Then,
for every integrable function $\varphi$ on $V$,
$$
\int_V \varphi \mu_{n \vert V}
=
\int_U (\varphi \circ f)
\vert \partial (f_1, \dots, f_n) / \partial (x_1, \dots, x_n) \vert
\mu_{n \vert U},
$$

where $\partial (f_1, \dots, f_n) / \partial (x_1, \dots, x_n)$
is the determinant of the jacobian matrix of $f$.
\end{prop}

Let $X$ be an $n$-dimensional smooth $K$-analytic manifold.
One assigns to any $K$-analytic $n$-differential form $\omega$
on $X$ a measure $\mu_{\omega}:= \vert \omega \vert$ as follows.
Take an atlas $\{(U, \phi_U)\}$ of $X$. Write
$(\phi_U^{-1})^* \omega_{\vert U} = f_U dx_1 \wedge \dots \wedge dx_n$.
If $A$ is small enough to be contained in some $U$,
we set $\mu_{\omega} (A) := \int_{\phi_U (A)} \vert f_U \vert \vert dx \vert$.
It follows from
the change of variables formula that the measure may be extended uniquely
by additivity to any  $A$ in a way which is
independent of the choice 
of the atlas.

\subsection{Rationality of a Poincar\'e series}\label{ig}Let $f$
be a polynomial in $\cO_K [x_1, \dots, x_n]$.
Denote by
$N_m$
the number of elements $x$
in
$(\cO_K / \varpi^{m + 1} \cO_K)^n$ such that $f (x) \equiv 0 \mod \varpi^{m + 1}$
and set $$Q (T):= \sum_{m \geq 0}N_m T^m.$$
When
$K = \QQ_p$, Borevich and Shafarevich conjectured
that $Q (T)$ 
is always a rational function of $T$.


\begin{theorem}[Igusa]Assume $K$ is of characteristic zero (i.e. 
$K$ is a finite extension of $\QQ_p$). Then the series
$Q (T)$ is rational. More precisely it is of the form
$$\frac{R (T)}{\prod_{j \in F}(1 - q^{-a_j}T^{b_j})}$$
with $R (T)$ in $\ZZ [p^{-1}][T]$, $F$ finite,
$a_j$ in $\NN$ and $b_j$ in $\NN \setminus \{0\}$.
\end{theorem}


The idea of Igusa's proof is the following. We  refer to \cite{igusa} or
\cite{igusabook} for more details.
One first observe
that $$N_m = q^{(m + 1)n}\mu_n (\{ x \in \cO_K^n \vert \ord f (x) \geq m + 1\}).$$
By an easy  calculation similar to (\ref{cal})
one deduces
the relation
$$
Q (q^{-n -s}) = \frac{q^n}{1 - q^{-s}}(1 - I (s)),
$$
with
$$I (s) :=
\int_{\cO_K^n} \vert f \vert^s \vert dx \vert.$$
Hence it is sufficient to prove the rationality
of $I (s)$ as function of $q^{-s}$.
This is achieved in the following way. By Hironaka's resolution (this is the place
where the hypothesis that $K$ is of characteristic zero is crucial),
there exists a smooth compact manifold $Y$
and an analytic
morphism $h : Y \rightarrow \cO_K^n$,  obtained by composition of
blowing up smooth centers, which is an isomorphism away from
the locus of $f = 0$.
By the change of variables formula
$I(s)$ may be expressed
as 
$$
I (s)
=
\int_Y \vert f  \circ h \vert^s \vert h^* dx \vert.
$$
On $Y$,  $f  \circ h$ and $h^* dx$ are both locally monomial,
i.e. of the form $f  \circ h = u \prod y_i^{N_i}$
and
$h^* dx = v \prod y_i^{n_i} dy$,
with $u$ and $v$ units, and $y = (y_1, \dots, y_n)$
local coordinates, 
in which case the explicit calculation of the integral
becomes very easy, since it is a product of integrals of type
(\ref{cal}).

\subsection{The Serre series}Instead of
considering the number
$N_m$ of approximate solutions modulo $ \varpi^{m + 1}$
of $f = 0$ in $(\cO_K / \varpi^{m + 1} \cO_K)^n$, one may want to consider
approximate solutions that can be lifted to actual solutions of
$f = 0$ in $\cO_K^n$.
More precisely, we denote by $\tilde N_m$ the number of elements $y$
in
$(\cO_K / \varpi^{m + 1} \cO_K)^n$ such that $y \equiv  x \mod \varpi^{m + 1}$,
with $x$ in $\cO_K^n$ such that $f (x) = 0$.
The corresponding generating series 
$$P(T):= \sum_{m \geq 0}\tilde N_m T^m$$
was first considered by Serre who raised the question of
its  rationality, that was solved (in characteristic zero)
by Denef in \cite{denef}.


\begin{theorem}[Denef]\label{prat}Assume $K$ is of characteristic zero. Then the series
$P (T)$ is rational. More precisely it is of the form
$$\frac{R (T)}{\prod_{j \in F}(1 - q^{-a_j}T^{b_j})}$$
with $R (T)$ in $\ZZ [p^{-1}][T]$, $F$ finite,
$a_j$ in $\ZZ$ and $b_j$ in $\NN \setminus \{0\}$.
\end{theorem}
 Since
$$\tilde N_m =
q^{(m + 1)n}\mu_n (\{ y \in \cO_K^n \vert \exists
x, f (x) = 0, x \equiv y \mod \varpi^{m +1}\}),$$
one can reduce the rationality of $P (T)$ to the rationality
of the integral $$J (s) := \int_{\cO_K^n} d (x, V)^s \vert dx \vert,$$
where $d (x, V)$ is the distance function to
the hypersurface $V$ defined by $f = 0$.
One then sees that a major difference with the Igusa case
\ref{ig} occurs: the function 
$d (x, V)$ is in general not analytic, due to the presence of quantifiers
in its definition. So, in the proof of his Theorem, Denef had to use
Macintyre's Theorem on quantifier elimination, which we shall explain now.



Let us mention that, in the positive characteristic case, the rationality
of $Q (T)$ would follow at once,  as soon as  Hironaka resolution
will be known. For the rationality 
of $P (T)$, the situation is much more open, since,
in this setting, one does not know, even conjecturally, what
could be a sensible 
analogue of Macintyre's Theorem.

\subsection{Definable subsets of $\QQ_p$}For simplicity we shall
assume $K = \QQ_p$, the case of finite extensions of $\QQ_p$
being quite similar.


Let $\cL_{\rm Mac}$ denote the first order language
whose variables run over $\QQ_p$ and with symbols
to denote $+, - \times, 0, 1$ and, for every $d = 2, 3, 4$ \dots,
a symbol $P_d$ to denote the predicate ``$x$ is a $d$-th power in
$\QQ_p$''. 
Moreover, for every element in $\ZZ_p$,
there is a symbol to denote that element.
As for any first order language, formulas of
$\cL_{\rm Mac}$ are built up from the above specified symbols and variables,
together with 
the logical connectives
$\wedge$ (and), $\vee$ (or), $\neg$ (not), the quantifiers $\exists$,
$\forall$ and the equality symbol $=$.
Macintyre's Theorem \cite{angus} states that $\QQ_p$
has quantifier elimination in the language $\cL_{\rm Mac}$,
meaning that every formula in that language is equivalent in $\QQ_p$
to a formula without quantifiers.
A subset of $\QQ_p^n$ is called semi-algebraic if is definable by a 
(quantifier-free) formula in $\cL_{\rm Mac}$.

We shall also consider the first order language $\cL_{\rm Pres}$
of Presburger arithmetic. In this language variables run over $\ZZ$
and symbols are $+, \leq, 0, 1$ and, for  every $d = 2, 3, 4$ \dots,
a symbol  to denote the binary relation $x \equiv y \mod d$.
One should note there is no symbol in $\cL_{\rm Pres}$ for multiplication.
It is an old result of Presburger that $\ZZ$
has quantifier elimination in the language $\cL_{\rm Pres}$.

It is also useful to consider  the first order language $\cL$
with two sorts of variables: a  first sort of variable running
over $\QQ_p$ and a second sort running over $\ZZ$. The symbols
of
$\cL$ consist of the  symbols of $\cL_{\rm Mac}$ for the first sort,
the  symbols of $\cL_{\rm Pres}$ for the second sort,
and a symbol to denote the valuation function
$\ord : \QQ_p \setminus \{0\} \rightarrow \ZZ$.
As remarked in \cite{denef}, it follows
from Macintyre's Theorem that $\QQ_p$ has elimination 
of quantifiers in the language
$\cL$ and every subset of
$\QQ_p^n$ which is definable in $\cL$ is semi-algebraic.
A function is called $\cL$-definable if its graph is $\cL$-definable.



\subsection{Denef's Cell Decomposition Theorem}


In \cite{denef} Denef gave two proofs of Theorem \ref{prat}. We already
mentioned
the first one, which uses
Hironaka resolution and Macintyre's Theorem. The second one 
was based on the following cell decomposition Theorem \ref{cdt}
which Denef originally deduced
from  Macintyre's Theorem (in fact, Macintyre's Theorem also
easily follows from Theorem \ref{cdt}). 
Then, in \cite{Dcell} Denef gave a direct 
proof of Theorem \ref{cdt}



\begin{theorem}[Denef's $p$-adic cell decomposition]\label{cdt}
Let $f_i (x, t)$, $1 \leq i \leq m$, be polynomials in $\QQ_p [x, t]$,
with $x = (x_1, \dots, x_{n - 1})$ and $t$ another variable.
Fix an integer $d \geq 2$. There exists a finite partition of
$\QQ_p^n$ into subsets (called cells) of the form
$$
A =  
\Bigl\{
(x, t) \in \QQ_p^n \Bigm \vert x \in C \, \text{and} \, 
\vert a_1 (x) \vert
\square_1 \vert t - c (x) \vert \square_2 \vert a_2 (x) \vert
\Bigr\},
$$
where $C$ is an $\cL$-definable subset of $\QQ_p^{n - 1}$, $ \square_i$
denotes either $\leq$, $<$, or no condition, and
$a_i (x)$ and $c (x)$ are $\cL$-definable functions from
$\QQ_p^{n - 1}$ to $\QQ_p$
such that, for every $(x, t)$ in $A$,
$$
f_i (x, t) = u_i (x, t)^d h_i (x) (t - c (x))^{\nu_i},
$$
for $1 \leq i \leq m$, where $u_i  (x, t)$ is a unit on $A$, $h_i (x)$ 
is an $\cL$-definable function, and $\nu_i$ is in $\NN$.
\end{theorem}


\subsection{Basic Theorem on $p$-adic
integration and applications}


The following general result is proved by successive
application of the cell decomposition Theorem and integration with respect to the $t$-variable,
cf. \cite{D85} and \cite{D2000}.


\begin{theorem}[Denef]\label{btp}Let
$(A_{\lambda, \ell})_{\lambda \in \QQ_p^k, \ell \in \ZZ^r}$
be an $\cL$-definable family of bounded subsets of $\QQ_p^n$,
meaning that the relation $x \in A_{\lambda, \ell}$ may be expressed by a formula 
in the language $\cL$, with variables $x, \lambda$ and $\ell$.
Let $\alpha (x, \lambda, \ell)$  be a $\ZZ$-valued $\cL$-definable function
on $\QQ_p^n \times \QQ_p^k \times \ZZ^r$.
Assume that all values of $\alpha$ are $\geq 0$.
Then the integral
$$
I_{\lambda, \ell} :=  \int_{A_{\lambda, \ell}} p^{- \alpha (x, \lambda, \ell)} \vert dx \vert
$$
is a $\QQ$-valued function of $\lambda, \ell$
belonging to the $\QQ$-algebra generated by functions of
the form $\theta (\lambda, \ell)$ and
$p^{\theta (\lambda, \ell)}$ with $\theta (\lambda, \ell)$ a $\ZZ$-valued
$\cL$-definable function.
\end{theorem}




In the special case where there is no variable $\lambda$,
the function $I (\lambda)$ in Theorem \ref{btp} is particularly simple:
it is built from Presburger functions (i.e. $\cL_{\rm Pres}$-definable functions)
using multiplication, exponentiation and $\QQ$-linear combinations.
From this observation Denef could deduce 
in an elementary way the following general rationality
statement.


\begin{theorem}[Denef]\label{grat}Assume the notation of Theorem \ref{btp} with no
$\lambda$ involved. Then
the series
$$
\sum_{\ell \in \NN^r}I (\ell) T^{\ell}
$$
in $\QQ [[T_1, \dots, T_r]]$ is a rational function of $T$.
\end{theorem}

Remark that the rationality of
$Q (T)$ and $P (T)$ is a direct consequence of Theorem \ref{grat}.


We now give a striking application to the problem of counting subgroups.
For a finitely generated group $G$ and an integer $n \geq 1$, 
let us denote by $a_n (G)$ the number 
number of subgroups of order $n$ in $G$. This is always
a finite number (cf. \cite{GSS}).

The following Theorem is due to 
Grunewald, Segal and Smith \cite{GSS}:
\begin{theorem}If $G$ is a torsion-free
finitely generated nilpotent group, then the series
$\sum_m a_{p^m} (G) T^m$
is rational, for every prime $p$.
\end{theorem}

The Theorem is proved by expressing $a_{p^m} (G) $ in terms of
a $p$-adic integral
$$\int_{A_m} p^{- \theta (x)} \vert dx \vert,$$
with $(A_m)_{m \in \NN}$ and $\theta$ definable in $\cL$ and applying 
Theorem \ref{grat}.




%\subsection{The Serre invariant}

\section{Prehistory of motivic integration:
Proving results over $\CC$ by computing $p$-adic integrals}\label{prehistory}


\subsection{The topological zeta function}\label{tzf}Let $X$ be a
smooth complex algebraic variety of dimension $n$
and $D$ an effective divisor on $X$. By a log-resolution of $(X, D)$
we mean a proper birational morphism
$h : Y \rightarrow X$ with $Y$ smooth, which is an isomorphism away from
the preimage of the support of $D$
and such that $h^{-1}( D)$ a divisor with normal crossings.
We denote by  $E_i$, $i \in J$ the irreducible components
of $h^{-1}( D)$ (in particular each  $E_i$ is smooth).
We set $E_I = \cap_{i \in I}E_i$, for $I \subset J$ and
$E_I^{\circ} = E_I \setminus \cup_{j \notin I}E_j$. Finally we denote by
$N_i$ the multiplicity of $E_i$ in $h^{-1}(D)$, i.e.
$h^{-1}(D) = \sum_{i \in J} N_i E_i$
and also we write
$\Omega^n_Y = h^{*}\Omega^n_X + \sum_{i \in J}(\nu_i - 1) E_i$,
where $\Omega^n$
stands for the sheaf of algebraic differential forms of degree $n$.
(We  use here the same notation for invertible sheaves and the
corresponding divisors.)
Let $W$  be a complex algebraic variety.
We denote ${\rm Eu} (W)$
its topological  Euler-Poincar\'e characteristic with compact supports:
${\rm Eu}  (W) := \sum_i (-1)^i {\rm rk}_{\CC} H^i_c (W (\CC))$. 
In fact, it can be shown, but we shall not use it, that 
${\rm Eu}  (W) $ is also equal to the
topological  Euler-Poincar\'e characteristic with compact supports
$ \sum_i (-1)^i {\rm rk}_{\CC} H^i (W (\CC))$.


Now, we can state the following result, which was first proved
using $p$-adic integration in \cite{jams92}:


\begin{theorem}[Denef-Loeser]\label{ttt}Let $X$ be a smooth complex algebraic variety of dimension $n$
and $D$ a divisor on $X$. Then
the rational function
\begin{equation}\label{tft}
Z_{\rm top} (X, D)(s) :=
\sum_{I \subset J}
\frac{{\rm Eu}(E_I^{\circ})}{\prod_{i \in I}(N_i s + \nu_i)}
\end{equation}
does not depend of the choice of a log-resolution $h : Y \rightarrow X$,
but only of the pair $(X, D)$.
\end{theorem}




Let us explain the idea of the proof.
We shall  assume, as in \cite{jams92}, that $X = \AA^n$
and $D = f^{-1} (0)$, with $f$ a polynomial in $\CC [x_1, \dots, x_n]$
but
the proof in general works just the same.
We shall write $Z_{{\rm top}, f} (s)$ for $Z_{\rm top} (X, D)(s)$.
Now, we shall make the assumption that the coefficients of $f$
all lie in the same number field
$K$, i.e. $f $ is in $K [x_1, \dots, x_n] $
(in general, we can only assume
they lie in a field of finite type over
$\QQ$, but the basic idea of the proof still remains the
same, see \cite{jams92}).

Now for every prime ideal $\gP$
in the ring of integers $\cO_K$, we denote by $K_{\gP}$ the corresponding local
field, with ring of integers $\cO_{\gP}$ and residue field
$k_{\gP}$.
We consider the local zeta function
$$Z_{f, K_{\gP}} (s) := \int_{\cO_{\gP}^n} \vert f\vert_{\gP}^s \vert dx \vert_{\gP},$$
where $\vert \_ \vert_{\gP}$ stands for the $\gP$-adic norm on $K_{\gP}$.
Consider now a log-resolution
$h : Y \rightarrow X$ defined over $K$.
It follows from a formula of Denef \cite{D87}
that, for almost all $\gP$,
\begin{equation}\label{red}
Z_{f, K_{\gP}} (s)
=
q^{-n} 
\sum_{I \subset J} {\rm card} (E_I^{\circ} (k_{\gP}))
\prod_{i \in I}\frac{(q - 1)q^{- (N_i s + \nu_i)}}{1 - q^{- (N_i s + \nu_i)}},
\end{equation}
with $q = {\rm card} k_{\gP}$.
Here we should explain what we mean
by $ {\rm card} (E_I^{\circ} (k_{\gP}))$. For $Z$ a variety over
$K$ we choose a model $\cZ$ over $\cO_K$, i.e. a variety over 
$\cO_K$ wich is isomorphic to $Z$ over $K$, and we set
${\rm card} (X (k_{\gP})) = {\rm card} ((\cX \otimes k_{\gP}) (k_{\gP}))$.
Of course, this may depend on the choice of the model $\cX$, but 
this will be the case
only for a finite number of prime ideals $\gP$, 
so it makes sense to consider
${\rm card} (X (k_{\gP}))$ for almost all $\gP$.
For $e \geq 1$, let us write $ K_{\gP}^{(e)}$ for the unramified extension of
$K_{\gP}$ of degree $e$. Its residue field $k_{\gP}^{(e)}$ has $q^e$ elements.
Also, for almost all $\gP$,  equation (\ref{red}) still holds when
replacing $K_{\gP}$ by $ K_{\gP}^{(e)}$, 
yielding
\begin{equation}\label{ered}
Z_{f, K_{\gP}^{(e)}} (s)
=
q^{-en} 
\sum_{I \subset J} {\rm card} (E_I^{\circ} (k_{\gP}^{(e)}))
\prod_{i \in I}\frac{(q^e - 1)q^{- e(N_i s + \nu_i)}}{1 - q^{- e(N_i s + \nu_i)}}.
\end{equation}
Now, taking formally the limit as $e \mapsto 0$ in
(\ref{ered}) gives us (\ref{tft}). This is quite clear, once we
know
that $\lim_{e \mapsto 0} {\rm card}W (k_{\gP}^{(e)}) = {\rm Eu}  W (\CC)$,
for almost all $\gP$, when $W$
is a variety over $K$.
This last fact follows from Grothendieck's trace formula for the Frobenius acting
on $\ell$-adic cohomology and standard comparison results between
$\ell$-adic and classical Betti cohomology.
Indeed, we have ${\rm card}W (k_{\gP}^{(e)}) = \sum \alpha_i^e
- \sum \beta_j^e$, with $ \alpha_i$ and $ \beta_j$ the eigenvalues,
respectively in
even and odd degree,
of the Frobenius acting on  $\ell$-adic cohomology groups with compact supports
with compact supports, and taking $e = 0$ just gives the trace of the identity,
i.e. the alternating sum of the ranks of $\ell$-adic cohomology groups with compact supports.
Of course, this is just a rough sketch of the proof and further work is
required in order to show this process of
taking limits as $e \mapsto 0$ really makes sense.



\subsection{Birational Calabi-Yau varieties have the same
Betti numbers}\label{cy}

Let $X$ be a smooth complex projective variety of dimension $n$. We say $X$
is Calabi-Yau if $X$ admits
a nowhere vanishing degree $n$ algebraic
differential form $\omega$. This is equivalent to
the sheaf $\Omega^n_X$ being trivial.
Recall the Betti numbers $b_i (X)$ are the ranks of the cohomology
groups $H^i (X (\CC), \CC)$.
Considerations from theoretical physics (string theory)
led to the guess that 
birational Calabi-Yau varieties should have the same
Betti numbers (and even the same Hodge numbers, cf. \ref{hdp}).


This was proved by Batyrev \cite{batyrev} using
$p$-adic integration and the Weil conjectures.


\begin{theorem}[Batyrev]\label{bat}Let $X$ and $X'$ be complex
Calabi-Yau varieties of dimension $n$.
Assume 
$X$ and $X'$  are birationally equivalent.
Then they have the same Betti numbers.
\end{theorem}



Let us  sketch the proof. For simplicity, we assume,
as in the proof of Theorem \ref{ttt} that
$X$, $X'$ and all the
data are defined over some number field $K$
(in general they are defined only over some field of finite type,
but the basic idea of the proof is the same).
We keep the notation of \ref{tzf}.
By Hironaka there exists a smooth projective $Y$ defined over $K$,
and birational proper morphisms (also defined over $K$)
$h : Y \rightarrow X$ and $h' : Y \rightarrow X'$. 
Furthermore we may assume
there exists a divisor with normal crossings $E = \cup_{i \in J} E_i$
such that the exceptional locus of $h$  and $h'$
respectively, is a finite union of $E_i$'s. We may write
$\Omega^n_Y = h^{*}\Omega^n_X + \sum_{i \in J}(\nu_i - 1) E_i$
and
$\Omega^n_Y = h'{}^{*}\Omega^n_{X'} + \sum_{i \in J}(\nu'_i - 1) E_i$.
Since $h^{*}\Omega^n_X$ and $h'{}^{*}\Omega^n_{X'}$ are both trivial, 
it follows\footnote{This is not completely evident, but can
be proved quite easily using
elementary algebraic geometry. Check it as  an exercise.}
that $\nu_i = \nu'_i$ for every $i$ in $J$.
One then deduces
follows from the change of variables formula, that for almost all $\gP$,
with a slight abuse of notation, with have
$$
\int_{X (K_{\gP})} \vert \omega  \vert_{\gP}
=
\int_{X' (K_{\gP})} \vert \omega' \vert_{\gP}
$$
and the same holds  for all unramified extensions $K_{\gP}^{(e)}$.
Indeed, we may express by the change of variables formula
both integrals as  the same integral over the rational points of $Y$.
Since,
for almost all $\gP$ and every $e$,
$\int_{X (K_{\gP}^{(e)})} \vert \omega  \vert_{\gP}$ is
equal to 
$q^{-en}{\rm card}(X (k^{(e)}_{\gP}))$
(this is a special case of Denef's result above mentionned
that goes back at least to A. Weil), it follows that for almost
all $\gP$, the reductions of (some model of) $X$ and $X'$ mod $\cM_{\gP}$
have same the zeta function. 
On the other side, for proper smooth varieties over a finite field,
the zeta function determines the
$\ell$-adic Betti numbers by Deligne's proof of the Weil 
conjectures, hence the result follows from standard comparison results
between $\ell$-adic and usual Betti numbers.

\begin{remark}The above proof gives in fact the following stronger
result (see \cite{batyrev}): if $X$ and $X'$ are two $n$-dimensional
smooth proper
complex varieties that are $K$-equivalent, meaning that there
exists
birational proper morphisms 
$h : Y \rightarrow X$ and $h' : Y \rightarrow X'$ with $Y$ 
smooth proper such that the invertible sheaves
$h^* (\Omega^n_X)$ and $h^* (\Omega^n_{X'})$ are isomorphic,
then $X$ and $X'$ have the same Betti numbers.
\end{remark}

\subsection{Towards motivic integration}The fact that one
can use $p$-adic integration to prove results over $\CC$  may look appealing
to model theorists - after all
using finite fields to prove surjectivity of injective 
polynomial complex morphisms goes back to Ax \cite{inj} - 
but it was  challenging  for geometers to find a more direct approach.
Of course, one obvious try would like to perform some kind of integration
over the field $\CC ((t))$, but since it is not  locally compact  it
is hopeless to  construct any reasonable
real valued measureon it.
The real breakthrough happened at the end of 1995, when 
Maxim Kontsevich got the idea of motivic integration: one should replace the
real numbers by (the completion of) the Grothendieck ring of algebraic varieties
as he explained in his seminal Orsay talk \cite{maxim}.



\section{Additive invariants and Grothendieck rings}



\subsection{Additive invariants of algebraic varieties}\label{3.1}Let $R$ be a ring. We denote by ${\rm Var}_R$
the category of algebraic varieties over $R$, i.e. reduced and separated schemes of 
finite type over
$R$.
If $X$ and $X'$ are varieties over $R$,
we
denote by $X \times X'$ their cartesian product
in  ${\rm Var}_R$. By definition $X \times X' $
is equal to
$(X \otimes_{\Spec R} X')_{\rm red}$,
that is the scheme $X \otimes_{\Spec R} X'$
endowed with its reduced structure.
An additive invariant
$$
\lambda : 
{\rm Var}_R
\longrightarrow
S,
$$
with $S$ a ring, assigns to any $X$
in ${\rm Var}_R$ an element $\lambda (X)$ of $S$
such that
$$
\lambda (X) = \lambda (X')
$$
for $X \simeq X'$,
$$
\lambda (X) = \lambda (X') + \lambda (X \setminus X'),
$$
for $X'$ closed in $X$,
and
$$
\lambda (X \times X')
=
\lambda (X) \lambda (X')
$$
for every $X$ and $X'$.




Let us remark that additive invariants
$\lambda$ naturally extend to take their values
on
constructible subsets  of algebraic varieties.
Indeed a constructible subset $W$
may be written as a finite
disjoint union of locally closed subvarieties $Z_i$, $i \in I$. One may define
$\lambda (W)$ to be $\sum_{i \in I} \lambda (Z_i)$. By the very axioms, this is independent
of the decomposition into locally closed subvarieties.


\subsection{Examples}


\subsubsection{}There exists a universal
additive invariant
$[\_] : {\rm Var}_R \rightarrow K_0 ({\rm Var}_R)$ 
in the sense that 
composition with $[\_] $ gives a bijection between
ring morphisms $ K_0 ({\rm Var}_R) \rightarrow S$
and 
additive invariants ${\rm Var}_R \rightarrow S$.
The construction of $K_0 ({\rm Var}_R)$ is quite easy:
take the
free abelian group on isomorphism classes $[X]$
of objects of ${\rm Var}_R$ and mod out by the relation
$[X] = [X'] + [X \setminus X']$ for $X'$ closed in $X$.
The product is now defined by $[X] [X'] = [X \times X']$.




We shall denote by $\LL$ the class of the affine line $\AA^1_R$ in
$K_0 ({\rm Var}_R)$. An important role will be played by
the ring 
$\cM_R := K_0 ({\rm Var}_R)[\LL^{- 1}]$ obtained by localization with respect to the 
multiplicative set generated by $\LL$.
This construction is analogue to the construction
of the category of Chow motives from the category of effective
Chow motives by localization with respect to the Lefschetz motive.
(Remark that the morphism $\chi_c$
of \ref{vm} sends $\LL$ to the class of the Lefschetz motive.)






One should stress that very little is known about the structure
of the rings $K_0 ({\rm Var}_R)$ and $\cM_R$ even when $R$ is a field.
Let us just
quote a result by Poonen \cite{poonen}
saying that 
when $k$ is a field of characteristic zero the ring
$K_0 ({\rm Var}_k)$ is not a domain. For instance, even for a field $k$, it is not known whether
the localization morphism
 $K_0 ({\rm Var}_k) \rightarrow \cM_k$ is injective or not.

\remark In fact, the ring 
$K_0 ({\rm Var}_k)$
as well as the canonical morphism
$\chi_c : K_0 ({\rm Var}_k) \rightarrow
K_0 ({\rm CHMot}_k)$,
were already considered by Grothendieck in a letter to Serre dated
August 16, 1964, cf. p. 174 of \cite{gs}.



\subsubsection{Euler characteristic}Here $R = k$ is a field. When $k$ is a subfield
of $\CC$, the Euler characteristic
${\rm Eu} (X) := \sum_i (-1)^i {\rm rk} H^i_c (X (\CC), \CC)$
give rise to an additive invariant
${\rm Eu} : {\rm Var}_k \rightarrow \ZZ$. For general $k$, replacing
Betti cohomology with compact support
by $\ell$-adic cohomology with compact support,
$\ell \not= {\rm char} k$, one gets  an additive invariant
${\rm Eu}_{\ell} : {\rm Var}_k \rightarrow \ZZ$, which does not
depend on $\ell$.
Since the Euler number of the affine line is 1, 
the Euler characteristic extends to a morphism $\cM_k \rightarrow \ZZ$.

\subsubsection{Hodge polynomial}\label{hdp}Let us assume 
$R = k$ is a field of characteristic zero.
Then it follows from Deligne's Mixed Hodge Theory that
there is a unique 
additive invariant
$H : {\rm Var}_k \rightarrow \ZZ [u, v]$,
which assigns to a smooth projective variety $X$ over $k$
its usual Hodge polynomial
$$
H (u, v) := \sum_{p, q} (-1)^{p + q} h^{p,q} (X) u^p v^q ,
$$
with $h^{p,q} (X) = \dim H^q (X, \Omega^p_X)$ the $(p,q)$-Hodge number of $X$.
This is is also a consequence of Bittner's Theorem that we shall
explain in \ref{bitt}.
Since $H (\AA^1_k) = uv$,
$h$ extends to a ring morphism
$\cM_k \rightarrow \ZZ [u, v, (uv)^{-1}]$.


\subsubsection{Virtual motives}\label{vm}More generally, 
when $R = k$ is a field of characteristic zero, there exists by Gillet and Soul\'e \cite{GS},
Guillen and Navarro-Aznar \cite{GN},
a unique additive invariant $\chi_c : {\rm Var}_k \rightarrow K_0 ({\rm CHMot}_k)$,
which 
assigns to a smooth projective variety $X$ over $k$
the class of its Chow motive, where $ K_0 ({\rm CHMot}_k)$
denotes the Grothendieck ring of the category of Chow
motives over $k$ (with rational coefficients).



Let us explain what  Chow motives and the category
$K_0 ({\rm CHMot}_k)$ are.
Let 
$\cV$ denote the category of smooth and projective $\CC$-schemes.
For  an object $X$ in $\cV$ and an integer $d$, $\cZ^{d} (X)$
denotes the free abelian group generated by irreducible subvarieties of
$X$ of codimension $d$. We define the rational Chow group
$A^{d} (X)$ as the quotient of $\cZ^{d} (X) \otimes \QQ$
modulo rational equivalence.
For $X$ and $Y$ in $\cV$,  we denote by ${\rm Corr}^{r} (X, Y)$ the
group of correspondences of degree $r$ from $X$ to $Y$. If $X$ is purely
$d$-dimensional, ${\rm Corr}^{r} (X, Y) = A^{d + r} (X \times Y)$,
and if $X = \coprod X_{i}$, 
${\rm Corr}^{r} (X, Y) = \oplus \, {\rm Corr}^{r} (X_{i}, Y)$.
The category $\textrm{Mot}$ of $\CC$-motives may be defined as follows
(cf. \cite{Scholl}).
Objects of $\textrm{Mot}$ are triples $(X, p, n)$ where $X$ is in $\cV$,
$p$ is an idempotent (i.e. $p^{2} = p$) in ${\rm Corr}^{0} (X, X)$, and
$n$ is an integer in $\ZZ$. If $(X, p, n)$
and $(Y, q, m)$ are motives, then
$$
{\rm Hom}_{\textrm{Mot}} ((X, p, n), (Y, q, m))
=
q \, {\rm Corr}^{m - n} (X, Y) \, p.
$$
Composition of morphisms is given by composition of correspondences.
The category $\textrm{Mot}$ is  additive, $\QQ$-linear, and pseudo-abelian.
There is a natural tensor product on $\textrm{Mot}$, 
defined on objects by
$$
(X, p, n) \otimes (Y, q, m) = (X \times Y, p \otimes q, n + m).
$$
We denote by $h$ the functor $h : \cV^{\circ} \rightarrow
\textrm{Mot}$ which sends an object $X$ to $h (X) = (X, {\rm id}, 0)$
and a morphism $f : Y \rightarrow X$ to its graph in
${\rm Corr}^{0} (X, Y)$. This functor is compatible with the
tensor product and the unit motive $1 = h ({\rm Spec} \, \CC)$ is the identity
for the product. We denote by ${\LL}$ the Lefschetz motive
$\LL = ({\rm Spec} \, \CC, {\rm id}, -1)$. One can prove
there is a canonical isomorphism
$h (\PP^{1}) \simeq 1 \oplus \LL$,
so, in some sense, $\LL$ corresponds to $H^{2} (\PP^{1})$.
%For any field $E$ containing $\QQ$
%one defines similarly
%the category $\textrm{Mot} \otimes E$ of motives with coefficients
%in $E$, by
%replacing
%the Chow groups
%$A^{\cdot}$ by $A^{\cdot} \otimes_{\QQ} E$.

Since algebraic correspondences naturally
act  on cohomology, any cohomology theory on the category $\cV$ factors
through $\textrm{CHMot}_k$ hence motives
have canonical Betti and Hodge realizations.




\begin{theorem}\label{MEC}There exists a unique
morphism of rings
$$
\chi_{c} :
K_{0} ({\rm Var}_k) \longrightarrow K_{0} ({\rm CHMot}_k)
$$
such that $\chi_{c} ([X]) = [h (X)]$
for $X$ projective and smooth.
\end{theorem}



Remark that $\chi_{c} ([\LL]) = \LL$.



\subsubsection{Counting points} Counting points also
yields additive invariants.
Assume $k = \FF_q$, then
$N_n : X \mapsto \vert X (\FF_{q^n}) \vert$ gives rise to an additive invariant
$N_n : {\rm Var}_k \rightarrow \ZZ$.
Similarly, if $R$ is (essentially) of finite type over $\ZZ$, for every maximal
ideal $\gP$ of $R$ with finite residue field $k_{\gP}$, we have 
an additive invariant
$N_{\gP} : {\rm Var}_R \rightarrow \ZZ$, which assigns to $X$
the cardinality of $(X \otimes k_{\gP})(k_{\gP})$.




\subsection{Bittner's Theorem}\label{bitt}We assume form now on that $k$ 
is a field of characteristic zero.
It is a rather straightforward consequence of Hironaka's theorem
that $K_0 (\Var_k)$ is generated by classes of smooth irreducible
projective  varieties. More subtle is the following
presentation by generators and relations of 
$K_0 (\Var_k)$
due to F. Bittner \cite{Bittner}.
We denote
by  $K_0^{\rm bl} (\Var_k)$
the quotient of the free abelian group
on isomorphism classes of irreducible smooth projective
varieties over $k$
by the relations 
$$
[{\rm Bl}_Y X] - [E] =  [X] - [Y],
$$
for $Y$ and $X$ irreducible
smooth projective over $k$,
$Y$ closed in $X$, ${\rm Bl}_Y X$ the blowup of $X$ with center
$Y$ and $E$ the exceptional divisor in ${\rm Bl}_Y X$.
As for $K_0 (\Var_k)$,
cartesian product induces a product on 
$K_0^{\rm bl} (\Var_k)$ which endows it with a ring structure.
There is a canonical ring morphism
$
K_0^{\rm bl} (\Var_k) \rightarrow K_0 (\Var_k),
$
which sends $[X]$ to $[X]$.


\begin{theorem}[Bittner \cite{Bittner}]\label{bi}Assume $k$
is of characteristic zero. The
canonical ring morphism
$$
K_0^{\rm bl} (\Var_k) \rightarrow K_0 (\Var_k)
$$
is an isomorphism.
\end{theorem}


The proof is based on Hironaka resolution of singularities
and the weak factorization theorem
of Abramovich, Karu, Matsuki and
W{\l}odarczyk \cite{wf} which we quote in  the following version:




\begin{theorem}[Weak factorization theorem]\label{wft}
Let $k$ be a field of characteristic zero.
Let $\phi : X_1 \dashrightarrow X_2$
be a birational map
between proper smooth irreducible varieties over $k$. Let $U \subset X_1$ 
be the largest open subset
on which $\phi$ is an isomorphism.
Then $\phi$ can be factored into a sequence of blowing
ups and blowing down
with smooth centers disjoint from $U$:
$\phi_i : V_{i - 1}\dashrightarrow V_i$, $i = 1, \dots, \ell$,
with $V_0 = X_1$, $V_{\ell} = X_2$,
with $\phi_i$ or $\phi_i^{-1}$ blowing ups with smooth centers away from $U$.
Moreover there exists $i_0$ such that
$V_i \dashrightarrow X_1$ is defined everywhere and projective for $i \leq i_0$
and
$V_i \dashrightarrow X_2$ is defined everywhere and projective for $i \geq i_0$.
\end{theorem}

Theorem \ref{bi} is a  very efficient  tool to provide
additive invariants. Indeed, it is enough to know the invariant for smooth projective
varieties and to check it behaves properly for blowing
ups with smooth centers. In particular it is now a straightforward
consequence of theorem \ref{bi}
(but using the full
strenght of weak factorization)
that the Hodge-Deligne polynomial
of
\ref{hdp} and the virtual motives of \ref{vm}
are well-defined additive invariants. 





\subsection{Grothendieck rings of first order theories}\label{g1t}The Grothendieck ring
$K_0 ({\rm Var}_k)$ may be generalized as follows to any
first order theory.
Let $\cL$ be a first order language and let $T$ be a theory in the
language 
$\cal L$.


We denote by $K_{0} (T)$ the quotient of the
free abelian group generated by
symbols $[\varphi]$ for $\varphi$ a formula in $\cL$
by the subgroup generated by the following
relations~
\begin{enumerate}
\item[(1)] If $\varphi$ is a formula in $\cL$ with free variables
$x = (x_{1}, \ldots, x_{n})$ and
$\varphi'$ is a formula in $\cL$ with free variables
$x' = (x'_{1}, \ldots, x'_{n'})$, then
$[\varphi] = [\varphi']$ if there exists a formula
$\psi$ in $\cL$, with free variables
$(x, x')$,
such that
$$
T \models [\forall x (\varphi (x) \rightarrow \exists! x' : (\varphi' (x')
\wedge \psi (x, x')))]
\wedge
[\forall x' (\varphi' (x') \rightarrow \exists! x : (\varphi (x)
\wedge \psi (x, x')))].$$
\item[(2)] \, $[\varphi \vee \varphi'] = [\varphi] + [\varphi'] - [\varphi
\wedge \varphi']$, for $\varphi$ and $\varphi'$  formulas in $\cL$.
\end{enumerate}
Furthermore one puts a ring structure on 
$K_{0} (T)$ by setting
\begin{enumerate}
\item[(3)] \, $[\varphi (x) ] \cdot [\varphi' (x')] =
[\varphi (x) \wedge  \varphi' (x')] $, if
$\varphi$ and $\varphi'$  are formulas in $\cL$ with disjoint free
variables $x$ and $x'$.
\end{enumerate}



For every interpretation of a theory $T_{1}$ in a theory
$T_{2}$ there is  a canonical morphism of rings
$K_{0} (T_{1}) \rightarrow K_{0} (T_{2})$, and this gives rise to a functor
from the category of theories in $\cL$, morphisms being given by
interpretation, to the category of commutative rings.



\medskip


If $k$ is a field and $T_{\rm ac}$ is 
the theory of algebraically closed fields containing $k$,
then $K_{0} (T_{\rm ac})$ is isomorphic to $K_{0} ({\rm Var}_k)$. If
$T_{\RR}$ is the theory
of real closed fields in the language of ordered rings, 
then $K_{0} (T_{\RR})$ is isomorphic to $\ZZ$.
Recently, Cluckers and Haskell \cite{CH} proved that the theory of
any fixed $p$-adic field, in the language of rings,
has trivial Grothendieck group.
In fact, Cluckers proved in \cite{raf} that for any two
$p$-adic semi-algebraic $X$ and $X'$
sets of the same dimension $d > 0$, there exists a semi-algebraic isomorphism
between $X$ and $X'$.


\section{Geometric motivic integration}




\subsection{ Arc spaces}Arc spaces are the $k [[t]]$-analogue of
$p$-adic points.
Let $k$ be a field of characteristic $0$. 
Many of the results presented in these lectures do not hold
anymore or become unknown
in positive characteristic.

For $n \geq 0,$ we introduce the space of $n$-arcs on $X,$ denoted by $\cL_n(X)$.
This is an algebraic variety which represents the functor:
$$  k-{\rm algebras} \longrightarrow \text{Sets}$$
$$ R \mapsto 
\text{Hom}_{k-schemes}(\Spec(R[t]/(t^{n+1}),X):=X(R[t]/(t^{n+1})). $$






For example when $X$ is an affine variety with equations $f_i (\vec x) = 0$,
$i= 1, \cdots, m$, $\vec x = (x_1,\cdots, x_r)$, then ${\cal L}_n(X)$
is given by the equations, in the variables $\vec a_0, \cdots, \vec a_n$,
expressing that $f_i (\vec a_0 + \vec a_1t + \cdots + \vec a_n t^n) \equiv 0
\mod t^{n+1}, i = 1,\cdots, m$.  



We have canonical
isomorphisms
$\cL_0(X)=X$ and $\cL_1(X)=TX$,
where $TX$ denotes the tangent space of the variety  $X$.





For $m \geq n$ there are canonical morphisms
$\theta_m^n : \cL_m(X) \rightarrow \cL_n(X)$. In general, when $X$ is not smooth,
they need nor to be surjective. When $X$ is smooth of dimension $d$,
$\theta_m^n$ is a locally trivial fibration for the Zariski topology with fiber
$\AA^{(m - n)d}$.



Taking the projective limit of these algebraic varieties ${\cal L}_n(X)$ we
obtain
the arc space ${\cal L}(X)$ of $X$. A priori this is just a pro-scheme,
but since the transition maps
$\theta_m^n$ are affine it is indeed a $k$-scheme.





In general, ${\cal L}(X)$ is not of finite type over $k$.
The $K$-rational points of ${\cal L} (X)$ are
the
$K[[t]]$-rational points of $X$.  These are called $K$-arcs on $X$.  For
example when $X$ is an affine variety with equations $f_i(\vec x) = 0,
i = 1, \cdots m, \vec x = (x_1,\cdots,x_r)$, then the $K$-rational points
of
${\cal L}(X)$ are the sequences $(\vec a_0, \vec a_1, \vec a_2, \cdots) \in
(K^n)^{\NN}$ satisfying $f_i (\vec a_0 + \vec a_1 t + \vec a_2t^2 +
\cdots )  = 0$,  for $i = 1,\cdots,m$. 
For every $n$ we have natural morphisms
$$\pi_n : {\cal L}(X) \rightarrow {\cal L}_n(X)$$
obtained by truncation.  For any arc $\gamma$ on $X$ (i.e. a $K$-arc for
some field $K$ containing $k$), we call $\pi_0(\gamma)$ the origin of the
arc
$\gamma$.



One can easily check that $\cL (X)$ represents the functor
$$  k-\text{algebras} \longrightarrow \text{Sets}$$
$$ R \mapsto  \text{Hom}_{k-schemes}(\Spec(R[[t]]),X):=X(R[[t]]). $$
It also represents the functor
$$  k-\text{schemes} \longrightarrow \text{Sets}$$
$$S \mapsto  \text{Hom}_{\text{locally ringed spaces}}((S, \cO_S [[t]]),X). $$


If $f : Y \rightarrow X$ is a morphism of varieties, we shall still 
denote by 
$f$ the corresponding morphism $\cL (Y) \rightarrow \cL (X)$.


\subsection{Completing $\cM_k$}We want to assign a measure
to subsets of $\cL (X)$. This measure will take values
in a ring related to $K_0 ({\rm Var}_k)$. In the analogy with $p$-adic integration,
$K_0 ({\rm Var}_k)$ is the analogue of $\ZZ$
and $\cM_k$ is the analogue of $\ZZ[p^{-1}]$ (the number of rational
points of the affine line over $\FF_p$ is $p$). Since in $\RR$, $p^{-n} \mapsto 0$
as $n \mapsto \infty$, we shall complete
$\cM_k$ is such a way that
$\LL^{-n} \mapsto 0$
as $n \mapsto \infty$.
This is achieved in the following way: 
we define $F^m \cM_k$ to be the subgroup
of
$\cM_k$ generated by elements of the form
$[S] \LL^{-i}$, with ${\rm dim}S - i \leq -m$.
We have $F^{m + 1} \subset F^m$,
$\LL^{-m} \in F^m$ and $F^n F^m \subset F^{n + m}$.
We denote by $\widehat \cM_k$ the completion of
$\cM_k$ with respect to that filtration.


A minor technical issue shows up here, since it is not known whether
the canonical morphism
$\cM_k \rightarrow \widehat \cM_k$ is injective or not.
Nevertheless, this is not too much a problem for applications by the following:


\begin{prop}Invariants ${\rm Eu} : \cM_k \rightarrow \ZZ$ (Euler number)
and $H:  \cM_k \rightarrow \ZZ [u, v, (uv)^{-1}]$ (Hodge polynomial) factor through
the image $\overline \cM_k$ of
$\cM_k$ in $\widehat \cM_k$.
\end{prop}


\begin{proof}Since ${\rm Eu} = H (1, 1)$, it is enough to prove
the result for $H$. 
But if $a$ is in $F^m \cM_k$, the total degree of $h (a)$ is $\leq - 2m$, so if $a$
belongs to the kernel $\cap_m F^m \cM_k$
of  $\cM_k \rightarrow \widehat \cM_k$, $h (a)$ should be zero.
\end{proof}


\subsection{Measurable sets}\label{123}For more details
about this section, see the appendix to \cite{compositio}.
Let $X$ be an algebraic variety over $k$
of dimension $d$, maybe singular.
By a cylinder 
in $\cL ( X)$, we mean a subset $A$ of $\cL (X)$
of the form $A = \pi_n^{-1} (C)$ with $C$ a constructible subset
of $\cL_n (X)$, for some $n$.
We say $A$ is stable (at level $n$) if furthermore
$\pi_{m + 1} (\cL (X)) \rightarrow \pi_{m} (\cL (X))$ is a piecewise Zariski fibration over
$\pi_m (A)$ with fiber $\AA^d_k$ for all $m \geq n$. By
a piecewise Zariski
fibration over
$\pi_m (A)$ we mean that there exists a finite  partition of $\pi_m (A)$
into locally closed subsets of $\cL_m (X)$ over which the morphism is
a locally trivial fibration for the Zariski topology.


If $A$ is a stable cylinder at level $n$, we set
$$
\tilde \mu (A) := [\pi_n (A) ] \LL^{- (n + 1) d}
$$
in $\cM_k$.
Remark that the stability condition insures that we would get the same value
by viewing $A$ as a stable cylinder at level $m$, $m \geq n$.
Also, it can be proved that if $X$ is smooth, all cylinders are stable.
In particular, in this case $\cL (X)$ itself is a stable
cylinder and $\tilde \mu (\cL (X)) = [X] \LL^{-d}$.


In general, we can assign to any cylinder
$A$ in $\cL (X)$ a measure $\mu (A)$ in $\widehat \cM_k$
by a limit process as follows:
for $e \geq 0$, set
$\cL^{(e)} (X) := \cL (X) \setminus \pi^{-1}_e (\pi_e (\cL (X_{\rm sing})))$,
where $X_{\rm sing}$ denote the singular locus of $X$ and we view 
$\cL (X_{\rm sing})$ as a subset of $\cL (X)$.
The set $\cL^{(e)} (X)$ should be viewed as $\cL (X)$ minus some tubular neighborhood
of the singular locus.
It can be proved that $A \cap 
\cL^{(e)} (X)$ is a stable cylinder
and that $\tilde \mu (A \cap 
\cL^{(e)} (X))$ does have a limit in $\widehat \cM_k$ as $e$ goes to $\infty$.
We denote this limit by  $\mu (A)$. This apply in particular to $A = \cL (X)$
when $X$ is not smooth.





We shall define
$$\vert \vert \_ \vert \vert : \widehat \cM_k : \rightarrow \RR_{\geq 0}$$
to be given by $\vert \vert a \vert \vert = 2^{-n}$
if $a \in F^n\widehat \cM_k$ and $a \notin F^{n + 1}\widehat \cM_k$, 
where $F^{\bullet} \widehat \cM_k$
denotes the induced filtration on $\widehat \cM_k$.




We shall say a subset $A$ of $\cL (X)$ is measurable
if, for every $\varepsilon >0$, there exists
cylinders $A_i (\varepsilon)$, $i \in \NN$,
such that $(A \cup A_0 (\varepsilon))
\setminus 
(A \cap A_0 (\varepsilon))$ is contained
in $\cup_{i \geq 1} A_i (\varepsilon)$,
and $\vert \vert \mu (A_i (\varepsilon)) \vert \vert \leq \varepsilon$,
for every $i \geq 1$.
Then one can show (cf. appendix to \cite{compositio}) that
$\mu (A) := \lim_{\varepsilon \mapsto 0} \mu (A_0 (\varepsilon))$
exists and is independent of the choice of the $A_i (\varepsilon)$'s.
We say that $A$ is strongly measurable
if moreover we can take $A_{0} (\varepsilon) \subset A$.


Let $A$ be a measurable subset of $\cL (X)$
and
$\alpha : A \rightarrow \ZZ \cup \{\infty\}$ 
be a function
such that all its fibers are measurable.
We shall say $\LL^{\-\alpha}$ is integrable
if the series
$$
\int_A \LL^{- \alpha} d \mu :=
\sum_{n \in \ZZ} \mu (A \cap \alpha^{-1} (n) ) \LL^{-n}
$$
is convergent in $\widehat \cM_k$.



\subsection{Semi-algebraic subsets}An important class
of measurable sets is that of semi-algebraic subsets of 
$\cL (X)$. We shall explain here only what are
semi-algebraic subsets of $\cL (\AA^n_k)$, the definition
for general $X$ being deduced by using charts
from the affine case.


We shall view points of $\cL (\AA^n_k)$ as $n$-uplets of formal power series.
A semi-algebraic subset
of $\cL (\AA^n_k)$ is a finite boolean combination of subsets
defined by
conditions of the form
\begin{align*}
\text{(1)}&&\ord f_1 (x_1, \ldots, x_m) \geq
\ord f_2 (x_1, \ldots, x_m) + L (\ell_1, \ldots, \ell_r)\\
\text{(2)}&&\ord f_1 (x_1, \ldots, x_m) \equiv
L (\ell_1, \ldots, \ell_r) \pmod d\\
\intertext{and}
\text{(3)}&&h (\ac (f_1 (x_1, \ldots, x_m)),
\ldots,
\ac (f_{m'} (x_1, \ldots, x_m))) = 0,
\end{align*}
where $f_i$ are polynomials with coefficients in  $k [[t]]$,
$h$ is a polynomial with coefficients in  $k$,
$L$ is a polynomial of degree $\leq 1$ over $\ZZ$, $d \in \NN$,
and $\ac (x)$ is the coefficient of lowest degree in $t$
of $x$ if $x \not= 0$, and is equal to 0 otherwise.
Here we use the convention that $\infty + \ell =  \infty$
and
$\infty \equiv \ell \; {\rm mod} \, d$, for all $\ell \in \ZZ$.
In particular the algebraic 
condition 
$f (x_1, \ldots, x_m) = 0$, for $f$ a
polynomial over $k [[t]]$, defines a semi-algebraic subset.



The following (consequence of a) quantifier elimination Theorem
of J. Pas \cite{Pas} is of fundamental use 
in the theory:


\begin{theorem}\label{Pas}Let $\pi : \AA_k^n
\rightarrow \AA^{n - 1}_k$ be the projection on the $n - 1$ first coordinates.
If $A$ is a semi-algebraic subset of
$\cL (\AA_k^n)$, then $\pi (A)$ is a 
semi-algebraic subset of
$\cL (\AA_k^{n - 1})$. 
\end{theorem}




One can prove that every semi-algebraic subset of $\cL (X)$
is strongly measurable. Furthermore, we have the following nice
description
of $\mu (A)$ in this case, which is an analogue of a $p$-adic result of
Oesterl\'e \cite{jo}, cf. \cite{inv}:

\begin{theorem}[Denef-Loeser]\label{mo}If $A$ is a semi-algebraic subset of $\cL (X)$,
with $X$ of dimension $d$,
then $\mu (A)$ is equal to limit of
$[\pi_n (A)] \LL^{- (n+ 1) d} $ in $\widehat \cM_k$.
\end{theorem}

Note that $[\pi_n (A)]$ in the above statement makes sense since
one can deduce from Pas' Theorem that
$\pi_n (A)$ is constructible.


\subsection{Pas quantifier elimination}\label{uuu}We give here the original statement
of Pas quantifier elimination. This subsection may be skipped at first reading
but it  will be used
later in the lectures.
Let $K$ be a valued field, with valuation
${\rm ord} : K \rightarrow \Gamma \cup \{\infty \}$,
where $\Gamma$ is an ordered abelian group.
We denote by $\cO_{K}$ the valuation ring, by $\cM_K$ the valuation ideal,
by $U$ the group of units in $\cO_{K}$, by $k$ the residue field, 
and by ${\rm Res} : \cO_{K} \rightarrow k$ the canonical projection.
We assume that $K$ has an angular component map.
By this we mean a map $\ac :
K \rightarrow k$ such that
$\ac 0 = 0$, the restriction of $\ac$ to
$K^{\times}$ is multiplicative and 
the restriction of $\ac$ to $U$
coincides with the restriction of ${\rm Res}$.
From now on we fix that angular component map
$\ac$.

We consider 3-sorted first order languages
of the form 
$$
\cL = (\LL_{K}, \LL_{k}, \LL_{\Gamma}, \ord, \ac),
$$
consisting of
\begin{enumerate}
\item[(i)] the language $\LL_{K} = \{+, -, \times, 0, 1\}$ of rings as 
valued
field sort,
\item[(ii)] the language $\LL_{k} = \{+, -, \times, 0, 1\}$ of rings as 
residue field sort,
\item[(iii)] a language $\LL_{\Gamma}$, which is an extension of the 
language $\{+, 0, \infty, \leq\}$ of ordered abelian groups with an 
element $\infty$
on top, as the value sort,
\item[(iv)] a function symbol $\ord$ from the valued field
sort to
the value sort, which stands for the valuation,
\item[(v)]  a function symbol $\ac$ from the valued field sort to
the residue field sort, which stands for the angular component map.
\end{enumerate}





In the following we shall assume that $K$ is henselian and that $k$ 
is of characteristic zero.
We consider $(K, k, \Gamma \cup \{\infty\}, \ord, \ac)$ as
a structure
for the language $\cL$, the interpretations of symbols being the 
standard ones. By an  henselian
$\cL$-extension of $K$, we mean an
extension
$(K', k', \Gamma' \cup \{\infty\}, \ord', \ac')$ of the structure
$(K, k, \Gamma \cup \{\infty\}, \ord, \ac)$ with respect to the language
$\cL$, with $K'$ a henselian valued field.
(By an extension, we mean a structure for the language $\cL$
which contains the original structure as a substructure.)
By abuse of language we shall
say that $K'$ is a henselian $\cL$-extension of $K$.




We may now state the quantifier elimination Theorem of Pas 
\cite{Pas}.




\begin{theorem}[Pas]\label{GPas}Let $K$ be a valued field
which satisfies the previous
conditions. For every  $\cal L$-formula $\varphi$ there exists 
an $\cal L$-formula $\varphi'$
without quantifiers over the valued field sort, such that
$\varphi$
is equivalent
in $K'$ to $\varphi'$, for every
henselian $\cL$-extension $K'$ of $K$.
\end{theorem}


In particular, when the value group is $\ZZ$, we shall use the language
$$
\cL_{\rm Pas} = (\LL_{K}, \LL_{k}, \cL_{\rm PR \infty}, \ord, \ac),
$$

where $\cL_{\rm Pres \infty} = \cL_{\rm Pres} \cup \{\infty\}$,
whith $ \cL_{\rm Pres}$
the Presburger language.




\subsection{Change of variables formula}

We have the following motivic analogue of
the $p$-adic change of variables formula \ref{pcvf}:



\begin{theorem}[Change of variables formula]\label{cvf}Let $X$
be an algebraic variety over $k$ of dimension $d$.
Let $h : Y \rightarrow X$ be a proper birational morphism. We assume $Y$
to be smooth. Let $A$ be  a subset 
of $\cL (X)$ such that $A$ and $h^{-1} (A) $ are strongly measurable.
Assume
$\LL^{- \alpha}$ is integrable on $A$.
Then
$$
\int_A \LL^{- \alpha} d \mu
=
\int_{h^{-1} (A)} \LL^{- \alpha \circ h - \ord h^* (\Omega^d_X)} d\mu.
$$
\end{theorem}

We should explain what is meant by  $ \ord h^{*} (\Omega^d_X)$.
Firstly, if $\cI$ is some ideal sheaf on $Y$, we denote by
$\ord \cI$ the function which to a arc $\varphi$ in $\cL (Y)$
assigns $\inf \ord g (\varphi)$ where $g$ runs over local sections of $\cI$ at
$\pi_0 (\varphi)$. We set $\Omega^d_X$ to be the $d$-th exterior power of
$\Omega^1_X$, the K\"ahler differentials. The image of
$h^* (\Omega^d_X)$ in $\Omega^d_Y$ is of the form
$\cI \Omega^d_Y$ and we set $ \ord h^{*} (\Omega^d_X) := \ord \cI$.



The proof of Theorem \ref{cvf}, given in section 3 of 
\cite{inv} (for $A$ semi-algebraic),
relies on the following geometric statement (Lemma 3.4 
of  \cite{inv}):


\begin{prop}[Denef-Loeser]\label{procvf}Let $X$
be an algebraic variety over $k$.
Let $h : Y \rightarrow X$ be proper birational morphism. We assume $Y$
to be smooth.
For $e$ and $e'$ in $\NN$, we set
$$
\Delta_{e, e'} :=
\Bigl \{ \varphi \in \cL (Y) \Bigm \vert
\ord^h{*} (\Omega^d_X) (\varphi) =
e \quad \text{and} \quad h (\varphi) \in \cL^{(e)} (X)\Bigr\}.
$$
Then there exists $c > 0$ such that, for $n \geq \sup (2e, e + ce')$,
\begin{enumerate}
\item[(1)]The image $\Delta_{e, e', n}$ of
$\Delta_{e, e'}$  in $\cL_n (Y)$ is a union of fibers of $h_n$, the morphism induced by $h$.
\item[(2)]The morphism
$h_n : \Delta_{e, e', n} \rightarrow h_n (\Delta_{e, e', n})$
is a piecewise Zariski fibration with fiber $\AA^e_k$.
\end{enumerate}
\end{prop}




\subsection{Some applications}We can  now reprove and reinterpret the results
in section \ref{prehistory} using motivic integration.

Let us begin by Batyrev's Theorem \ref{bat}.

\begin{theorem}[Kontsevich]\label{keq}Let $X$ and $X'$ be two proper smooth
varieties over $k$. Assume there are $K$-equivalent, i.e. that there 
exists
birational proper morphisms 
$h : Y \rightarrow X$ and $h' : Y \rightarrow X'$ with $Y$ 
smooth proper such that the invertible sheaves
$h^* (\Omega^n_X)$ and $h^* (\Omega^n_{X'})$ are isomorphic.
The $[X] = [X']$ in $\overline \cM_k$.
\end{theorem}


\begin{proof}Since $\int_{\cL (Y)} 
\LL^{- \ord h^* (\Omega^d_X)} d\mu
=
\int_{\cL (Y)} 
\LL^{- \ord h'{}^* (\Omega^d_{X'})} d\mu$, it follows from the change of variables 
formula that $\mu (\cL(X)) = \mu (\cL(X'))$, hence $[X] = [X']$, since $X$
and $X'$ are smooth.
\end{proof}


\begin{cor}Let $X$ and $X'$ be two proper smooth
varieties over $k$. Assume there are $K$-equivalent (this holds in particular if they are both
Calabi-Yau). Then they have the same Hodge numbers and Betti numbers.
\end{cor}


\begin{remark}Theorem \ref{keq} can also be proved as a consequence of the
weak factorization Theorem \ref{wft}.
\end{remark}








Now we shall give an intrisic meaning to the topological zeta function.
Let $X$ be a smooth algebraic variety of dimension $n$ over $k$
and $D$ a divisor on $X$. For any integer $m \geq 0$
we may consider $\cX_n := \{\varphi \in \cL_m (X) \vert \ord f (\varphi) =m\}$,
whet $f$ is a local equation for $d$ at $\pi_0 (\varphi)$.
We consider
$Z_{\rm mot, naive}(X, D)  (T)$ to be the formal series
$$
Z_{\rm mot, naive} (X, D)  (T) := \sum_{m \in \NN} [\cX_n] \LL^{-mn} T^m
$$
in $\cM_k [[T]]$. This is the naive motivic zeta function
attached to the pair $(X, D)$.

One deduces from Proposition \ref{procvf} the following formula
for $Z_{\rm mot, naive} (X, D)  (T)$ in terms of a log-resolution of $(X, D)$.

\begin{prop}[Denef-Loeser]\label{nmz}Let  $h : Y \rightarrow X$ be a log-resolution of $(X, D)$.
With the notations of \ref{tzf} we have
$$
Z_{\rm mot, naive} (X, D)  (T) = 
\LL^{-n} 
\sum_{I \subset J} [E_I^{\circ}]
\prod_{i \in I}\frac{(\LL- 1)\LL^{- \nu_i}T^{N_i}}{1 - \LL^{- \nu_i}T^{N_i}}.
$$
\end{prop}

Let us formally
evaluate $Z_{\rm mot, naive} (X, D)  (T)$ at $T = \LL^{-s}$ for $s$ an integer $ \geq 1$.
We obtain from the above Proposition a well defined element
$$
\LL^{-n} 
\sum_{I \subset J} [E_I^{\circ}]
\prod_{i \in I}\frac{(\LL- 1)\LL^{- (N_i s + \nu_i)}}{1 - \LL^{- (N_i s + \nu_i)}}.
$$
in the ring $\cM_{k, {\rm loc}}$ obtained from
$\cM_k$ by inverting the the
elements $[\PP^i_k] = 1 + \LL + \LL^2 + \cdots + \LL^i$, for $i =
1,2,3,\cdots$, where $\PP^i_k$ denotes the $i$-dimensional projective space
over $k$.
The ring morphism
${\rm Eu}  : \cM_k \rightarrow \ZZ$
extends uniquely to a ring morphism
${\rm Eu}  : \cM_{k, {\rm loc}} \rightarrow \QQ$.

Hence we deduce the following conceptual and intrinsic
interpretation of
$Z_{\rm top} (X, D) (s)$:

\begin{prop}[Denef-Loeser]For every integer $s \geq 1$,
\begin{equation}\label{rtt}
Z_{\rm top} (X, D) (s) 
={\rm Eu} (Z_{\rm mot, naive} (X, D)  (\LL^{-s})).
\end{equation}
\end{prop}

\begin{remark}One may also prove  Theorem \ref{ttt} by
using
the weak factorization Theorem \ref{wft} (which was not available at the time
of the first two proofs), but then one would miss 
the intrinsic interpretation (\ref{rtt}).
\end{remark}


Another important feature of $Z_{\rm mot, naive} $ is that it  ``contains'' also
the corresponding $p$-adic integrals. More precisely:


\begin{prop}[Denef-Loeser]\label{gyg}Let  $K$ be
a number field, set $X = \AA^n_K$ and let $D$ be the divisor of
a polynomial $f $ in $K [x_1, \dots, x_n]$.
For almost all $\gP$, $$N_{\gP} (Z_{\rm mot, naive} (X, D)  (\LL^{-s}))$$ is equal to 
the $p$-adic integral 
$$Z_{f, K_{\gP}} (s) := \int_{\cO_{\gP}^n} \vert f\vert_{\gP}^s \vert dx \vert_{\gP}.$$
Here, we extend $N_{\gP}$ to $\cM_K (\LL^{- s})$ by sending
$\LL^{- s}$ to $q^{-s}$, with $q$ the cardinal of $k_{\gP}$.
\end{prop}

\begin{proof}Follows directly from 
(\ref{red}) and Proposition \ref{nmz}.
\end{proof}

\subsection{Geometrization of $Q$}Let us now slighly generalize the Poincar\'e series $Q$ and $P$.
For $X$ a variety over a $\cO_K$, for $K$ a finite extension
of $\QQ_p$. 
We set
$N_m :=  \vert X (\cO_K / \varpi^{m + 1}) \vert$, for $m \geq 0$
and consider the series $Q (T)
:=  \sum_{m \geq 0}   N_m T^m$. 
The series $Q (T)$ is still rational (cf. \cite{meuser}, see also the
review MR 83g:12015).
Also, we denote by $\tilde N_m$ the cardinality of the image
of $X (\cO_K)$ in $X (\cO_K / \varpi^{m + 1})$. In other words,
$\tilde N_m$ is the number of points in
$X (\cO_K / \varpi^{m + 1})$ 
that may be lifted to actual points in 
$X (\cO_K)$ and we set $P (T)
:=  \sum_{m \geq 0}  \tilde N_m T^m$.
Denef's rationality proof extends to this setting.
When $X$ is defined by $f = 0$
in the affine space one recovers the previous definitions.


It follows from Proposition \ref{gyg} that, when $X$ is an hypersurface in the affine space
defined over some number field $K$,
$
N_{\gP} (Q_{\rm geom} (T)) = Q_{X \otimes \cO_{K_{\gP}}} (T).
$
forr almost all $\gP$.
This may be extended to any $X$ over a number field $K$ (cf. \cite{rat}).



\subsection{Geometrization of $P$}As a geometric analogue of the Serre series $P (T)$,
it is natural to consider, for a variety $X$ over a field $k$,
the generating series
$$
P_{\rm geom} (T) :=
\sum_{m \geq 0} [\pi_m (\cL (X))] \, T^m
$$
in
$\cM_k [[T]]$.
Here we should check that
the image $\pi_m (\cL (X))$ of $\cL (X)$ in $\cL_m (X)$ is a constructible subset
of $\cL_m (X)$.
This holds
thanks to Greenberg's Theorem on solutions of polynomial systems in Henselian rings
\cite{gree}, which states that $\pi_m (\cL (X))$ is equal to $\theta_n^m ( \cL_n(X))$
for some $n \geq m$, together with Chevalley's constructibility Theorem.









\begin{theorem}[Denef-Loeser \cite{inv}]\label{Pg}
Assume ${\rm char} k = 0$.
The series 
$P_{\rm geom} (T)$
in $\cM_k [[T]]$ is rational of the form
$$
\frac{R (T)}{\prod (1 - \LL^a T^b)},
$$
with $R (T)$ in 
$\cM_k [T]$, $a$ in $\ZZ$ and $b$ in $\NN \setminus \{0\}$.
\end{theorem}

The prove of this result follows similar lines than the proof of Theorem \ref{prat},
using motivic integration instead
of $p$-adic integration and Pas' quantifier elimination instead of Macintyre's.




\subsection{Towards $P_{\rm ar}$}When $X$ is defined over a number field
$K$, 
a quite natural guess would be, by
analogy with what we have seen
so far, that,
for almost
all finite places $\gP$,
$N_{\gP} (P_{\rm geom} (T)) = P_{X \otimes \cO_{K_{\gP}}} (T)$.
But such a statement cannot hold true as can be seen on some simple examples.
This is due to the fact that,
in the very definition of $P (T)$, one is concerned in not considering
extensions of the residue field, while in the definition
of $P_{\rm geom} (T)$ extensions of the residue field $k$
are allowed.
To remedy this, one needs to be more careful about
rationality issues concerning the residue field, and this will be the topic of the  next section.





\section{Assigning virtual Chow motives to formulas}


\subsection{Subassignments}
Fix a ring $R$. We denote
by $\textrm{Field}_R$
the category of $R$-algebras that are fields.
For an $R$-scheme $X$, we denote by $h_X$ the functor
which to a field $K$ in 
$\textrm{Field}_R$ assigns the set
$h_X (K) := X (K)$.
By a subassignment $h \subset h_X$
of 
$h_X$
we mean the datum, for every 
field $K$ in 
$\textrm{Field}_R$, of a subset $h (K)$ of 
$h_X (K)$. We stress that, contrarly to subfunctors,
no compatibility is required
between the various sets
$h (K)$.


All set theoretic constructions generalize in an obvious
way to the case of subassignments. For instance if $h$ and $h'$
are subassignments of $h_X$, then we denote by
$h \cap h'$ the subassignment $K \mapsto h (K) \cap h' (K)$, etc.


Also, if $\pi : X \rightarrow Y$ is a morphism
of $R$-schemes and $h$ is a 
subassignment of $h_X$, we define
the 
subassignment $\pi (h)$ of 
$h_Y$ by $\pi (h) (K) := \pi (h (K)) \subset h_Y (K)$.

\subsection{Definable subassignments}Let $R$
be a ring. 
By a ring formula $\varphi$  over $R$, we mean
a first order formula in the language of rings with coefficients
in $R$ and free variables $x_1, \dots x_n$.


To a  ring formula $\varphi$ over $R$ with
free variables $x_1, \dots x_n$
one assigns the subassignment $h_{\varphi}$
of $h_{\AA^n_R}$ defined by
\begin{equation}\label{hd}
h_{\varphi}( K) :=
\Bigl\{ (a_1, \dots, a_n) \in K^n\Bigm \vert
\varphi (a_1, \dots, a_n) \, \textrm{holds in } K \Bigr\} \subset
K^n = h_{\AA^n_R} (K).
\end{equation}


Such a subassignment 
of $h_{\AA^n_R}$ is called a definable subassignment.
More generally, using affine coverings, cf. \cite{def}, one defines 
definable subassignments of $h_X$ for $X$ a variety over $R$.



It is quite easy to show that if $\pi : X \rightarrow Y$
is an $R$-morphism of finite presentation,
$\pi (h)$ is a definable subassignment of $h_Y$
if $h$ is a definable subassignment of $h_X$.



In our situation, we are concerned with the
subassignment
$\pi (h_{\cL (X)}) \subset h_{\cL_n (X)}$.
Remark that
$\pi_n : \cL (X) \rightarrow \cL_n (X)$ is not of finite type.
Nevertheless, it follows from Pas' Theorem (cf. Proposition \ref{speci})
that
$\pi (h_{\cL (X)})$  is a definable subassignment of $h_{\cL_n (X)}$.


\subsection{Pseudo-finite fields}Let $\varphi$ be a formula over a number field $K$.
For almost all finite places $\gP$ with residue field
$k (\gP)$, one may extend the definition in
(\ref{hd}) to give a meaning
to $h_{\varphi}(k (\gP))$.
If $\varphi$ and $\varphi'$ are formulas over $K$,
we set $\varphi \equiv \varphi'$
if $h_{\varphi}(k (\gP)) =
h_{\varphi'}(k (\gP))$
for almost all finite places $\gP$.


It follows from a  fundamental result of 
J. Ax \cite{Ax} that
$\varphi \equiv \varphi'$
if and only if 
$h_{\varphi} (L) = h_{\varphi} (L')$
for every
pseudo-finite field $L$
containing $K$.
We recall that a pseudo-finite field
$L$ is a perfect
infinite field which has exactly one extension of each degree
and such that every absolutely irreducible variety over $L$
has a rational point
over
$L$. J. Ax proved \cite{Ax} that an infinite field $F$ is pseudo-finite
if and only if every sentence
in the first order language
of rings which is true in all finite fields is also true in $F$.
We recall also that the property of being a pseudo-finite field
is stable by ultraproducts.



\subsection{A brief review on quantifier elimination for Galois formulas}
Let $A$ be an integral and normal scheme.
A morphism of schemes
$h : C \rightarrow A$ is a
Galois cover if $C$ is integral,
$h$ is {\'e}tale (hence $C$ is normal)
and there is a
finite group $G = G (C  / A)$, the Galois group, acting on $C$ 
such that $A$  is isomorphic to the quotient $C / G$ 
in such a way that
$h$ is the composition of the quotient morphism with the 
isomorphism.
We say that the Galois cover $h : C \rightarrow A$
is colored
if $G (C  / A)$ is equipped with
a family ${\rm Con}$  of subgroups of
$G (C  / A)$ 
which is stable by conjugation under elements of
$G (C  / A)$.




Let $S$ be an integral normal scheme and let $X_{S}$ be a variety
over $S$.
A normal stratification
of $X_{S}$, 
$$
\cA = \langle X_{S}, C_{i} / A_{i} \, \vert \, i \in I \rangle , 
$$
is a partition of $X_{S}$
into a finite
set of integral and normal locally closed $S$-subschemes
$A_{i}$, $i \in I$, each equipped with a 
Galois cover $h_{i} : C_{i} \rightarrow A_{i}$.


A normal stratification $\cA$
may be augmented to a Galois stratification
$$
\cA 
= \langle X_{S}, C_{i} / A_{i}, {\rm Con} (A_{i})
\, \vert \, i \in I \rangle, 
$$
if for each $i \in I$, ${\rm Con} (A_{i})$
is a family of subgroups of
$G (C_{i}  / A_{i})$ 
which is stable by conjugation under elements in $G (C_{i}  / A_{i})$,
{\it i.e.} $(C_{i} \rightarrow A_{i}, {\rm Con} (A_{i}))$
is a colored Galois cover.



Let $U = {\rm Spec} \, R$
be an affine scheme, which we assume to be integral and normal.
For any variety $X_{U}$ over $U$
and any closed point $x$
of $U$, we denote by $\FF_{x}$ the residual field of $x$ on $U$
and by
$X_{x}$ the fiber of $X_{U}$ at $x$.
More generally, for any field $M$ containing $\FF_{x}$, we shall denote by
$X_{x, M}$ the fiber product of $X_{U}$ and ${\rm Spec} \, M$ over $x$.






Let $X_{U}$ be a variety over $U$.
Let $\cA 
= \langle X_{U}, C_{i} / A_{i}, {\rm Con} (A_{i}) \, \vert \,
i \in I \rangle$
be a Galois stratification of $X_{U}$ and let
$x$ be a closed point of $U$.
Let $M$ be a field containing $\FF_{x}$ and 
let $\aa$
be  an $M$-rational point of $X_{x}$ belonging
to $A_{i, x}$. We denote by ${\rm Ar} (C_{i} / A_{i}, x ,\aa)$
the conjugacy class of subgroups of $G (C_{i}  / A_{i})$
consisting of  the  decomposition subgroups at $\aa$. We shall
write
$$
{\rm Ar} (\aa) \subset {\rm Con} (\cA)
$$
for 
$${\rm Ar} (C_{i} / A_{i}, x ,\aa) \subset {\rm Con} (A_{i}).
$$
For $x$ a  closed point of $U$ and
$M$ a field containing $\FF_{x}$, we consider the subset
$$
Z (\cA, x, M) :=
\Bigl\{\aa \in X_{U} (M)
\Bigm \vert {\rm Ar} (\aa) \subset
{\rm Con} (\cA)
\Bigr\}
$$
of
$X_{U}(M)$.



Let
$\cA 
= \langle \AA^{m + n}_{U}, C_{i} / A_{i}, {\rm Con} (A_{i})
\, \vert \, i \in I \rangle$
be a Galois stratification of $\AA^{m + n}_{U}$.
Let $Q_{1}, \ldots, Q_{m}$ be quantifiers.
We denote by $\vartheta$  or by $\vartheta (\YY)$ the formal expression
$$
(Q_{1} X_{1}) \ldots (Q_{m} X_{m}) \, [{\rm Ar} (\XX, \YY) \subset
{\rm Con} (\cA)]
$$
with $\XX = (X_{1}, \ldots, X_{m})$ and
$\YY = (Y_{1}, \ldots, Y_{n})$. We call $\vartheta (\YY)$
a {\it Galois formula} over $R$ in the free variables $\YY$. 



Now to a Galois formula $\vartheta$, to a 
closed point $x$ of $U$ and to a field $M$ containing $\FF_{x}$,
one associates the subset
\begin{multline*}
Z (\vartheta, x, M) :=\\
\Bigl\{\bb = (b_{1}, \ldots, b_{n}) \in M^n
\Bigm \vert
(Q_{1} a_{1}) \ldots (Q_{m} a_{m}) \, [{\rm Ar} (\aa, \bb) \subset
{\rm Con} (\cA)]
\Bigr\}
\end{multline*}
of
$M^n$, where the quantifiers $Q_{1} a_{1}$, \dots, $Q_{m} a_{m}$ run
over $M$.





Let $\varphi (Y_{1}, \ldots, Y_{n})$ be a formula 
in the first order
language of rings with coefficients in the ring $R$ and free variables
$Y_{1}, \ldots, Y_{n}$.
For every 
closed point $x$ in $U$ and every field $M$ containing
$\FF_{x}$ we denote by 
$Z (\varphi, x, M)$ the subset of 
$M^n$ defined by the (image over $\FF_{x}$ of the) formula
$\varphi$.
Assume now $\varphi$ is
in prenex normal form,
{\it i.e.} a formula of the form
\begin{equation}\label{prenex}
(Q_{1} X_{1}) \ldots (Q_{m} X_{m}) \, \Bigl[
\bigvee_{i = 1}^{k} \bigwedge_{j = 1}^l f_{i, j}
(\XX, \YY) = 0 \wedge g_{i, j}
(\XX, \YY) \not= 0
\Bigr],
\end{equation}
with $f_{i, j}$ and $g_{i, j}$ in $R [\XX, \YY]$.
The formula between brackets defines
an $U$-construct\-ible subset $W$ of $\AA_{U}^{m + n}$ to which one 
associates a Galois stratification 
by taking any normal stratification
with all strata contained either in $W$ or in its complement,
by taking the identity morphism as Galois cover on each stratum, 
and taking for ${\rm Con} (A_{i})$ the family consisisting of
the group with one element 
if
$A_{i} \subset W$ and the empty family otherwise. In this way
one obtains a Galois formula $\vartheta$ over $R$ such that
$Z (\vartheta, x, M) = Z (\varphi, x, M)$ for every closed point $x$ in
$U$ and every field $M$
containing $\FF_{x}$.


There exists 
several versions of quantifier elimination for Galois formulas
\cite{F-S}, \cite{F-H-J}, \cite{F-J}. We shall use the
following one which
is a special case
of Proposition 25.9 of \cite{F-J}.

\begin{theorem}[Fried-Jarden]\label{pfelimination}Let $k$ be a field.
Let $\cA$
be a Galois stratification of $\AA^{m + n}_{k}$ and let  $\vartheta$  
be a
Galois formula
$$
(Q_{1} X_{1}) \ldots (Q_{m} X_{m}) \, [{\rm Ar} (\XX, \YY) \subset
{\rm Con} (\cA)]
$$
with respect to $\cA$.
There exists a  Galois stratification
$\cB$ of 
$\AA^{n}_{k}$ such that,
for every pseudo-finite field $F$  containing $k$,
$$Z (\vartheta, {\rm Spec} \, k, F)
=
Z (\cB, {\rm Spec} \, k, F).
$$
\end{theorem}




\begin{cor}\label{pf2.5}Let $\varphi (Y_{1}, \ldots, Y_{n})$ be a formula 
in the first order
language of rings with coefficients in a field $k$ and free variables
$Y_{1}, \ldots, Y_{n}$.
There exists a Galois stratification
$\cB$ of 
$\AA^{n}_{k}$ such that, 
for every pseudo-finite  field $F$ containing $k$,
$$Z (\varphi, {\rm Spec} \, k, F)
=
Z (\cB, {\rm Spec} \, k, F).
$$
\end{cor}




\subsection{Assigning virtual motives to formulas}
Let $k$ be a field of characteristic
zero.
Let us consider
the Grothendieck
ring $K_0 ({\rm PFF}_k)$ of the 
theory of pseudo-finite fields over $k$ as defined in \ref{g1t}.
It follows from \ref{vm} that we have a canonical morphism
$\chi_c :  K_0 ({\rm Var}_k) \rightarrow K_0 ({\rm CHMot}_k)$.
We shall denote by $K_0^{\rm mot} ({\rm Var}_k) $
the image of 
$K_0 ({\rm Var}_k)$
in
$K_0 ({\rm CHMot}_k)$ under this morphism.
Remark that the image of $\LL$
in
$K_0^{\rm mot} ({\rm Var}_k)$  is not a zero divisor since it is 
invertible in 
$K_0 ({\rm CHMot}_k)$.
We shall now explain the construction of a canonical ring morphism
$$ \chi_c : {\rm K}_0
({\rm PFF}_k ) \longrightarrow {\rm K}_0^{\rm mot} ({\rm Var}_{k} )
\otimes {\bf Q} $$
extending the one in \ref{vm}.

\begin{theorem}[Denef-Loeser \cite{def},\cite{icm}]\label{imp}Let $k$
be a field of characteristic zero.
There exists a unique ring morphism $$ \chi_c : {\rm K}_0
({\rm PFF}_k ) \longrightarrow {\rm K}_0^{\rm mot} ({\rm Var}_{k} )
\otimes {\bf Q} $$
satisfying  the following two properties:
\begin{enumerate}\item[(i)] For any formula $\varphi$ which is a conjunction of polynomial equations
over $k$, the element $\chi_c ([\varphi])$ equals the
class in ${\rm K}_0^{\rm mot} ({\rm Var}_{k}  ) \otimes {\bf Q}$ of the
variety defined
by $\varphi$.
\item[(ii)] Let $X$ be a normal affine irreducible variety over $k$, $Y$ an unramified
Galois cover of
$X$, and $C$ a cyclic subgroup of the Galois group G of $Y$ over $X$. For such data we
denote by
$\varphi_{Y,X,C}$ a ring formula, whose interpretation in any field $K$ containing
$k$, is the set of $K$-rational points on $X$ that lift to a geometric
point on $Y$ with decomposition group $C$ (i.e. the set of points on $X$ that lift to a $K$-rational
point of $Y/C$, but not to any $K$-rational point of $Y/C'$ with $C'$ a proper subgroup of
$C$). Then
\[
\chi_c ([\varphi_{Y,X,C} ]) = \frac{{\left| C \right|}}{{\left|
{{\rm N}_{G} (C)} \right|}}\chi_c ([\varphi_{Y,Y/C,C}
]),\]
where ${\rm N}_{G} (C)$ is the normalizer of $C$ in $G$.
\end{enumerate}
Moreover, when  $k$ is a number field,
for almost all finite places $\gP$,
$ N_{\gP} (\chi_c ([\varphi]))$ is
equal to the cardinality of $h_{\varphi} (k(\gP))$.
\end{theorem}



The above Theorem is a variant of results in \S 3.4  of \cite{def}.
A sketch of proof is given in \cite{icm}.
\begin{proof}[Some ingredients in the proof]
Uniqueness uses quantifier elimination for
pseudo-finite fields 
from which 
it follows by Corollary \ref{pf2.5} that ${\rm K}_0
({\rm PFF}_k )$ is generated as a group by
classes of
formulas of the form $\varphi _{Y,X,C}$.
Thus by (ii) we only have to
determine $\chi_c([\varphi _{Y,Y/C,C}])$, with $C$ a cyclic group.
But this follows
directly from the following recursion formula: 
\begin{equation}\label{rec}
\left| C \right|[Y/C] = \sum\limits_{A\,{\rm  subgroup\,  of }\, C} \left| A \right|
\chi_c ([\varphi_{Y,Y/A,A}]).
\end{equation}
This recursion formula is a direct consequence of (i), (ii), and the fact that
the formulas $\varphi _{Y,Y/C,A}$ yield a partition of $Y/C$.
The proof of existence
is based on work of del Ba{\~ n}o
Rollin and
Navarro Aznar
\cite{Rollin Aznar} who
associate to any representation over {\bf Q} of a finite group $G$ acting
freely on an affine variety $Y$ over $k$, an element in the Grothendieck group of Chow
motives over $k$. By linearity, we can hence associate to any {\bf Q}-central function
$\alpha$ on $G$ (i.e. a {\bf Q}-linear combination of characters of
representations of $G$ over {\bf Q}), an element $\chi_c(Y,\alpha )$ of that Grothendieck
group tensored with {\bf Q}. Using Emil Artin's Theorem, that any {\bf Q}-central function
$\alpha$ on $G$ is a {\bf Q}-linear combination of characters induced by trivial
representations of cyclic subgroups, one shows that
$\chi_c(Y,\alpha )\in {\rm K}_0^{\rm mot} ({\rm Var}_{k}  ) \otimes {\bf Q}$. For $X:=Y/G$
and $C$ any cyclic subgroup of $G$,
we define $\chi_c([\varphi _{Y,X,C}]) := \chi_c(Y,\theta)$, where $\theta$ sends $g\in G$ to 1
if the subgroup generated by $g$ is conjugate to $C$, and else to 0.
With some more work we prove that
the above definition of $\chi_c([\varphi _{Y,X,C}])$ extends by additivity to a
well-defined map
$ \chi_c : {\rm K}_0
({\rm PFF}_k ) \longrightarrow {\rm K}_0^{\rm mot} ({\rm Var}_{k} ) \otimes {\bf Q} $. 
\end{proof}


Clearly $\chi_c (\varphi)$ depends only on $h_{\varphi}$ and the construction
easily extends by additivity
to 
definable subassignments of $h_X$, for
any variety $X$ over $k$. So,
to any such
definable subassignment $h$, we may associate $\chi_c (h)$ in
${\rm K}_0^{\rm mot} ({\rm Var}_{k}) \otimes \QQ$.



The invariants ${\rm Eu}$ and $H$ being cohomological they factor through
$ {\rm K}_0^{\rm mot} ({\rm Var}_{k} )$.
One can show, cf. \cite{rat}, that for any 
definable subassignment $h$, ${\rm Eu} (\chi_c (h))$ belongs to $\ZZ$.
Such an integrality result does not hold for $H$ as shown by the following example:
Let $n$ be a integer $\geq 1$ and
assume $k$ contains
all $n$-roots of unity.
Consider the formula
$\varphi_n : (\exists y) (x = y^n \; \text{and} \;
x \not= 0)$ ; then $\chi_c (\varphi_n) = \frac{\LL -1}{n}$.
In particular
${\rm Eu} (\chi_c (\varphi_n)) = 0$
and
${H} (\chi_c (\varphi_n)) = \frac{uv -1}{n}$.
(This example contradicts the example on page
430 line -2 of \cite{def}
which is unfortunately incorrect.)



\section{Arithmetic motivic integration}
\subsection{The series $P_{\rm ar}$}We now consider
the series
$$
P_{\rm ar} (T) := \sum_{n \geq 0} \chi_c (\pi_n (h_{\cL (X)})) \, T^n
$$
in
${\rm K}_0^{\rm mot} ({\rm Var}_{k}) \otimes \QQ$.

\begin{theorem}[Denef-Loeser \cite{def}]\label{Pa}
Assume ${\rm char} k = 0$.
\begin{enumerate}
\item[1)]The series 
$P_{\rm ar} (T)$
in ${\rm K}_0^{\rm mot} ({\rm Var}_{k})\otimes \QQ$ is rational of the form
$$
\frac{R (T)}{\prod (1 - \LL^a T^b)},
$$
with $R (T)$ in 
$({\rm K}_0^{\rm mot} ({\rm Var}_{k}) \otimes \QQ )[T]$, $a$ in $\ZZ$ and $b$ in $\NN \setminus \{0\}$.
\item[2)] If $X$ is defined over
some number field $K$, then, for almost
all finite places $\gP$,
$$
N_{\gP} (P_{\rm ar} (T)) = P_{X \otimes \cO_{K_{\gP}}} (T).
$$
\end{enumerate}
\end{theorem}

Here we use implicitely
that $N_{\gP}$ factors through $ {\rm K}_0^{\rm mot} ({\rm Var}_{k} )$
which follows from the fact there exists,
by Grothendieck trace formula, a cohomological
expression for $N_{\gP}$.

The proof of Theorem \ref{Pa} relies on the theory of
arithmetic motivic integration we shall now explain.




\subsection{Definable subassignments of $h_{\cL (X)}$}We shall use now
the notations and definitions in \ref{uuu}, assuming
that $K = k ((t))$, that
$\kappa = k$, with $k$
a field of characteristic zero,
and that $\ord$ and $\ac$ have their classical meaning for 
formal power series.


\medskip 
Let $R$ be a subring of $k$.
By an $\cL_{\rm Pas}$-formula
with coefficients in $R$ in the valued field sort and in the residue
field sort, we mean a formula in the language obtained from
$\cL_{\rm Pas}$ by adding, for every element of $R$,
a new symbol to denote it in the valued field sort and in the residue
field sort. 
We shall consider $\cL_{\rm Pas}$-formulas
with coefficients in $R$ in the valued field sort and in the residue
field sort, free
variables $x_{1}, \ldots, x_{m}$ running
over the valued field sort and no free
variables running over
the residue field or the value sort. We shall call such formulas
formulas on $R [[t]]^m$ for short







\medskip
One may deduce the
following statement  of Ax/Ax-Kochen-Er{\v s}ov type 
from
the Pas Theorem.



\begin{prop}\label{axax}Let $R$ be a
normal domain 
of finite type over $\ZZ$ with field of fractions $k$.
Let $\sigma$ be a sentence
in the language $\cL_{{\rm Pas}}$ with coefficients in $R$
in the valued field sort and in the residue
field sort. The following
statements are equivalent:
\begin{enumerate}
\item[(1)] The sentence
$\sigma$ is true in $F ((t))$ for every pseudo-finite field $F$
containing
$k$.    
\item[(2)]
There exists $f$ in $R \setminus \{0\}$
such that, for every closed point $x$ in ${\rm Spec} \, R_{f}$,
the sentence $\sigma$ is true in $\FF_{x}((t))$.
\end{enumerate}
If, furthermore, $k$ is a finite extension of $\QQ$, the previous statements
are also equivalent to the following:
\begin{enumerate}
\item[(3)]There exists $f$ in $R \setminus \{0\}$, multiple of the
discriminant of $k / \QQ$,
such that, for every closed point $x$ in ${\rm Spec} \, R_{f}$,
the sentence $\sigma$ is true in $k_{x}$,
\end{enumerate}
where $k_{x}$ denotes the completion of $k$ at $x$. Remark that, 
the extension $k / \QQ$ being non ramified at $x$,
the field $k_{x}$ admits a canonical uniformizing parameter, hence
also
a canonical
angular component map.
\end{prop}



Let $k$ be a field
and let $X$ be a variety over $k$. 
We consider the functor
$h_{\cL (X)} :  K \mapsto X (K[[t]])$ from ${\rm Field}_{k}$
to the category of sets.

Let $\varphi$ be a formula on $k [[t]]^m$.
For every field $K$ in ${\rm Field}_{k}$, denote by 
$Z (\varphi, K[[t]]) $ the subset of
of all $x$ in
$K[[t]]^{m} = \AA^{m}_{k} (K [[t]])$
for which  $\varphi (x)$ is true
in $K((t))$.
This defines a subassignment
$K \mapsto Z (\varphi, K[[t]]) $
of the functor $h_{\cL (\AA^{m}_{k})}$.
We call such a subassignment a definable subassignment of 
$h_{\cL (\AA^{m}_{k})}$.

By using affine coverings, one may also define
definable subassignments of $h_{\cL (X)}$, for
$X$ any variety over $k$.








We shall denote by ${\rm Def}_{k} (\cL (X))$ the set of
definable subassignments of $h_{\cL (X)}$. Clearly ${\rm Def}_{k} (\cL
(X))$ is stable
by finite intersection and finite union and by taking complements.




For $n$ in $\NN$, recall the  canonical truncation 
morphism $\pi_{n} : \cL (X) \rightarrow \cL_{n} (X)$.
Hence if $h$ is a subassignment of $h_{\cL(X)}$ 
(resp. of $h_{\cL_{n} (X)}$)
we may consider
$\pi_{n} (h) : K \mapsto \pi_{n} (h (K))$
(resp. $\pi_{n}^{-1} (h) : K \mapsto \pi_{n}^{-1} (h (K))$)
which is a subassignment of $h_{\cL_{n} (X)}$
(resp. of $h_{\cL (X)}$).



\begin{prop}\label{speci}Let $h$ be  a definable
subassignment of $h_{\cL(X)}$. Then, for every 
$n$ in $\NN$, 
$\pi_{n} (h)$ is a definable
subassignment of $h_{\cL_{n} (X)}$ 
and
$\pi_{n}^{-1} \pi_{n} (h)$ is a definable
subassignment of $h_{\cL (X)}$.
\end{prop}


\subsection{Arithmetic motivic integration}Now let us explain briefly how arithmetic motivic integration
is constructed.
We shall denote by $\cM^{\rm mot}_k$
the image of 
$\cM_k$
in
$K_0 ({\rm CHMot}_k)$ by the morphism $\chi_c$.
We endow $\cM^{\rm mot}_k$ with the filtration $F^{\bullet}$,
image by $\chi_c$ of the filtration $F^{\bullet}$ on $\cM_k$
and we denote by $\widehat \cM^{\rm mot}_k$ the completion of
$\cM^{\rm mot}_k$ with respect to that filtration.



Arithmetic motivic integration will assign to subassignments $h$ of $h_{\cL (X)}$
a measure $\nu (h)$ in $\widehat \cM^{\rm mot}_k \otimes \QQ$.
The idea of the construction is very much the same as the one in \ref{123},
starting from stable cylinders. 
Since we are concerned only with definable subassignments, we
shall not talk about  measurable subassignments here.
For a definable subassignment $h$ of $\cL (X)$ which is a stable cylinder
(this is defined 
in a completely similar way 
than in the geometric case), 
the sequence $\chi_c (\pi_n (h)) \LL^{- (n+ 1) d}$
has a  constant value $\tilde \nu (h)$ in
$\cM^{\rm mot}_k \otimes \QQ$ for large $n$, with $d$ the dimension of $X$.




\begin{def-theorem}[Denef-Loeser]\label{ami}There exists a unique
mapping
$$\nu : {\rm Def}_{k} (\cL (X))
\longrightarrow
\widehat \cM^{\rm mot}_k \otimes \QQ$$
satisfying the following properties.
\begin{enumerate}
\item[(1)]If  $h$ is a stable cylinder which is a definable subassignment of $h_{\cL (X)}$,
then
$\nu (h)$
is equal to the image
of
$\tilde \nu (h)$ in $\widehat \cM^{\rm mot}_k \otimes \QQ$.
\item[(2)]If $h$ and $h'$ are definable subassignments of $h_{\cL (X)}$, then
$$\nu (h \cup h') =
 \nu (h) +  \nu (h') -
 \nu (h \cap h').
$$
\item[(3)]If $h (E) = h' (E)$ for every pseudo-finite field $E$
containing $k$,
then $ \nu (h) =   \nu (h')$.
\item[(4)]Let  $h$ be a definable subassignment of $h_{\cL (X)}$.
If there exists a subvariety $S$ of $X$ with ${\rm dim} \, S \leq d - 1$
such that $h \subset h_{\cL (S)}$, then
$\nu (h) = 0$.
\item[(5)]Let $h_{n}$ be a definable partition of a definable subassignment
$h$
with parameter $n \in \NN$.
Then the series $\sum_{n \in \NN} \nu (h_{n})$
is convergent in
$\widehat \cM^{\rm mot}_k \otimes \QQ$
and 
$$
\nu (h) = \sum_{n \in \NN} \nu (h_{n}).
$$
\item[(6)]Let  $h$ and $h'$ be definable subassignments of $h_{\cL (X)}$.
Assume
$h \subset h'$.
If $ \nu (h') $ belongs to $F^{e} \widehat \cM^{\rm mot}_k \otimes \QQ$, then $\nu (h)$
also belongs to $F^{e} \widehat \cM^{\rm mot}_k \otimes \QQ$.
(Here $F^{\bullet} \widehat \cM^{\rm mot}_k \otimes \QQ$ denotes the filtration induced
by $F^{\bullet}$ on $ \widehat \cM^{\rm mot}_k \otimes \QQ$.)
\end{enumerate}
We call $\nu (h)$ the arithmetic motivic volume
of $h$.
\end{def-theorem}


We have the following analogue of Theorem \ref{mo}:
\begin{theorem}Let $X$ be a variety over $k$ of dimension $d$.
Let  $h$ be a definable subassignment of $h_{\cL (X)}$. Then 
$$
\lim_{n \rightarrow \infty}
\chi_{c} (\pi_{n} (h)) \, \LL^{-(n +
1) d}
= 
\nu (h)
$$ in
$\widehat \cM^{\rm mot}_k \otimes \QQ$.
\end{theorem}

Also, if $\alpha : h \rightarrow \NN$ is a definable function
on the definable subassignment $h$, meaning
that, for every field
$K$ containing $k$, we have a function $\alpha (K)  : h (K) \rightarrow \NN$
and that the graph of these functions are definable in $\cL (X) \times \NN$,
we may consider the  integral
$\int_h \LL^{- \alpha} d \nu$.
In particular,
we have a direct analogue of
the change of variables formula (Theorem \ref{cvf})
for arithmetic motivic integration,
with a similar proof relying on Proposition \ref{procvf}.


We have also general rationality Theorems, for which we refer to
\S\kern .15em 7 of \cite{def}. 
Using Proposition \ref{axax}, one may also prove general specialization results
of arithmetic integrals to $p$-adic ones (cf. \S\kern .15em 8 of \cite{def}).
In particular Theorem \ref{Pa} concerning $P_{\rm ar}$
may be obtained as a consequence of these rationality and specialization
statements.
Here is a typical example of such a statement:

\begin{theorem}[Denef-Loeser \cite{def}]\label{int}
Let $K$ be a number field. Let $\varphi$ be
a first order formula in the language
$\cL_{\rm Pas}$  with coefficients
in $K$ and free variables 
$x_1$, \dots, $x_n$. Let $f$ be a polynomial
in $K [x_1, \dots, x_n]$.
For 
$\got P$ a finite place of $K$, denote by 
$K_{\got P}$ the completion of $K$
at $\got P$. For almost all  $\got P$, 
applying the operator $N_{\gP}$  to the motivic integral $\int_h \LL^{- s (\ord f)} d \nu$
gives the $p$-adic integral
$$
\int_{h_{\varphi} (K_{\got P})} |f|_{\got P}^s |d x|_{\got P}
$$
\end{theorem}


Let us remark that this result is sufficient
to get the specialization statement in Theorem \ref{Pa} about $P_{\rm ar}$.
Indeed,
one may as well assume $f$ is a definable function in Theorem \ref{int},
since by a graph construction one may always replace $f$ by a coordinate.



\subsection{``All natural $p$-adic integrals are motivic''}Theorem \ref{int} is an illustration of the principle
``All natural $p$-adic integrals are motivic''. It 
applies in particular to integrals occuring
in $p$-adic harmonic analysis, like orbital integrals.
This has led recently Tom Hales \cite{H1}
to propose that
many of the basic objects in
representation theory should be  motivic in nature and to
develop 
a beautiful conjectural program aiming to the determination
of the virtual Chow motives
that should control the behavior of orbital integrals
and leading to a motivic fundamental lemma  (see
\cite{GH} and \cite{H2} for recent progress on these questions).















%\subsection{Epilogue}In these notes we did not discuss 

\bibliographystyle{amsplain}
\begin{thebibliography}{SGA}

\bibitem{wf}D. Abramovich, K. Karu, K. Matsuki,
J. W{\l}odarczyk,
\textit{Torification and factorization of birational
maps},
J. Amer. Math. Soc. \textbf{15} (2002), 531--572.




\bibitem{Ax} J. Ax, 
\textit{The elementary theory of finite fields}, 
Ann. of Math.
\textbf{88} (1968), 239--271.



\bibitem{inj}
J. Ax, 
\textit{Injective endomorphisms of varieties and schemes}, 
Pacific J. Math. 
\textbf{31} (1969), 1--7.
14.15



\bibitem{batyrev}
V. Batyrev,
\textit{Birational Calabi-Yau $n$-folds have equal Betti numbers},
New trends in algebraic geometry (Warwick, 1996), 1--11, 
London Math. Soc. Lecture Note Ser., 264, 
Cambridge Univ. Press, Cambridge, 1999. 


\bibitem{Bittner}F. Bittner,
\textit{The universal Euler characteristic
for varieties of characteristic zero},
math.AG/0111062, to appear in Compositio Math.



\bibitem{Rollin Aznar} S. del Ba{\~ n}o Rollin, V. Navarro Aznar,
\textit{On the motive of a quotient variety},
{Collect. Math.} \textbf{49} (1998),
203--226.



\bibitem{CH}R. Cluckers, D. Haskell,
\textit{Grothendieck rings of $\ZZ$-valued fields},
Bull. Symbolic Logic \textbf{7} (2001), 262--269.


\bibitem{raf}R. Cluckers,
\textit{Classification of semi-algebraic $p$-adic sets up to
semi-algebraic bijection}, J. Reine Angew. Math.
\textbf{540} (2001), 105--114. 

\bibitem{denef}J. Denef,
\textit{The rationality of the Poincar{\'e}
series associated to the $p$-adic points on a variety}, 
Invent. Math. \textbf{77} (1984), 1--23. 


\bibitem{D85}
J. Denef,
\textit{On the evaluation of certain $p$-adic integrals},
S\'eminaire de th\'eorie des nombres, Paris
1983--84, 25--47, Progr. Math., \textbf{59}, Birkh\"auser
Boston, Boston, MA, 1985


\bibitem{Dcell}
J. Denef,
\textit{$p$-adic semi-algebraic sets and cell decomposition},
J. Reine Angew. Math. \textbf{369} (1986), 154--166.

\bibitem{D87}
J. Denef,
\textit{On the degree of Igusa's local zeta function}, 
Amer. J. Math. \textbf{109} (1987), 991--1008.

\bibitem{D2000}
J. Denef,
\textit{Arithmetic and geometric applications of quantifier elimination for valued fields}, Model
theory, algebra, and geometry, 173--198, Math. Sci. Res. Inst. Publ., 39, Cambridge Univ. Press, Cambridge, 2000.


\bibitem{jams92}J. Denef, F. Loeser,
\textit{Caract\'eristiques d'Euler-Poincar\'e,
fonctions z\^etas locales et modifications analytiques},
J. Amer. Math. Soc. \textbf{5} (1992), 705--720.



\bibitem{jag}J. Denef, F. Loeser,
\textit{Motivic Igusa zeta functions},
J. Algebraic Geom.,
\textbf{7} (1998),
505--537.


\bibitem{inv}J. Denef, F. Loeser,
\textit{Germs of arcs on singular algebraic varieties
and motivic integration}, Invent. Math., \textbf{135} (1999), 201--232.

\bibitem{def}
J. Denef, F. Loeser,
\textit{Definable sets, motives and $p$-adic integrals},
J. Amer. Math. Soc. \textbf{14} (2001), 429--469.

\bibitem{compositio}J. Denef, F. Loeser,
\textit{Motivic integration, quotient singularities and the McKay correspondence},
Compositio Math. \textbf{131} (2002), 267--290.



\bibitem{icm}J. Denef, F. Loeser,
\textit{Motivic Integration and the Grothendieck Group of Pseudo-Finite Fields},
Proceedings of the International Congress of Mathematicians, Beijing 2002,
Volume 2, 13--23, Higher Education Press, Beijing 2002.


\bibitem{rat}J. Denef, F. Loeser,
\textit{On some
rational generating series occuring in arithmetic geometry},
math.NT/0212202.
to appear in the volume
Geometric Aspects of Dwork's Theory.


\bibitem{dwork}B. Dwork,
\textit{On the rationality of the zeta function of an algebraic variety},
Amer. J. Math. \textbf{82} (1960), 631--648.



\bibitem{F-H-J}
M. Fried, D. Haran, M. Jarden,
\textit{Galois stratifications over Frobenius fields},
Adv. in  Math., 
\textbf{51}
(1984),
1--35.

\bibitem{F-J} M. Fried, M. Jarden,\textit{Field arithmetic}, Ergebnisse der Mathematik und
ihrer Grenzgebiete (3), Springer-Verlag, Berlin, 1986.  ISBN: 3-540-16640-8.

\bibitem{F-S}
M. Fried, G. Sacerdote,
\textit{Solving diophantine problems over all residue class fields
of a number field and all finite fields},
Ann. Math.
\textbf{100} (1976),
203--233.


\bibitem{GS}
H. Gillet, C. Soul{\'e},
\textit{Descent, motives and $K$-theory},
J. Reine Angew. Math.
\textbf{478}
(1996),
127--176.



\bibitem{GH}J. Gordon, T. Hales,
\textit{Virtual transfer factors},
math.RT/0209001.






\bibitem{gree}M. Greenberg,
\textit{Rational points in henselian discrete valuation rings},
Inst. Hautes {\'E}tudes Sci. Publ. Math.
\textbf{31}
(1966), 59--64.


\bibitem{GN}
F. Guill\'{e}n, V. Navarro Aznar,
\textit{Un crit\`{e}re d'extension d'un foncteur d\'{e}fini sur les
sch\'{e}mas lisses},
Inst. Hautes {\'E}tudes Sci. Publ. Math.
\textbf{95}
(2002), 1--91.





\bibitem{gs}\textit{Correspondance Grothendieck-Serre}, edited by
P. Colmez and J.-P. Serre, Documents Math\'ematiques \textbf{2},
Soci\'et\'e math\'ematique de France, 2001.



\bibitem{GSS}
F. Grunewald, D. Segal, G. Smith,
\textit{Subgroups of finite index in nilpotent groups},
Invent. Math. \textbf{93} (1988), 185--223.


\bibitem{H1}
T. Hales, \textit{Can $p$-adic integrals be computed?},
math.RT/0205207.



\bibitem{H2}
T. Hales, \textit{Orbital Integrals are Motivic},
math.RT/0212236.







\bibitem{igusa}
J. Igusa,
\textit{Complex powers and asymptotic expansions. II. Asymptotic expansions},
 J. Reine
Angew. Math. \textbf{278/279} (1975), 307--321. 


\bibitem{igusabook}J. Igusa,
\textit{An introduction to the theory of local zeta functions},
AMS/IP Studies in Advanced Mathematics, 14. 
American Mathematical Society, Providence, RI;
International Press, Cambridge, MA, 2000.

%\bibitem{kapranov}
%M. Kapranov,
%\textit{The elliptic curve in the $S$-duality theory and Eisenstein series for
%Kac-Moody groups},
%math.AG/0001005.


\bibitem{maxim}
M. Kontsevich, Lecture at Orsay, December 7, 1995.


%\bibitem{ll}
%M. Larsen, V. Lunts,
%\textit{Motivic measures and stable birational geometry},
%math.AG/0110255 


%\bibitem{rigid}
%F. Loeser, J. Sebag,
%\textit{Motivic integration
%on smooth rigid varieties and invariants of degenerations}



\bibitem{angus}
A. Macintyre,
\textit{On definable subsets of $p$-adic fields},
J. Symbolic Logic \textbf{41} (1976), 
605--610. 

\bibitem{meuser}
D. Meuser, \textit{On the rationality of certain generating functions},
Math. Ann. \textbf{256} (1981), 303--310. 


\bibitem{jo}J. Oesterl\'e,
\textit{R\'eduction modulo $p^{n}$ des sous-ensembles analytiques ferm\'es
de $\ZZ^{N}_{p}$},
Invent. Math., \textit{66} (1982), 325--341.







\bibitem{Pas}
J. Pas,
\textit{Uniform $p$-adic cell decomposition and local zeta functions},
J. Reine Angew. Math.,
\textbf{399}
(1989),
137--172.

\bibitem{poonen}
B. Poonen,
\textit{The Grothendieck ring of varieties is not a domain},
Math. Res. Lett. \textbf{9} (2002), 493--497.



\bibitem {Scholl}
A. Scholl,
\textit{Classical motives},
in
\textsl{Motives}, U. Jannsen, S. Kleiman, J.-P. Serre Ed., 
Proceedings of Symposia in Pure Mathematics, Volume 55 Part 1 (1994),
163--187.

%\bibitem{serre}J.-P. Serre,
%\textit{Classification des vari\'et\'es analytiques $p$-adiques
%compactes},
%{Topology} \textbf{3} (1965), 409--412.












\end{thebibliography}


\end{document}

