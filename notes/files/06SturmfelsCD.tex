\documentclass[12pt]{amsart}
\usepackage{amsthm}
\usepackage{amssymb}
\usepackage{amsmath}
\usepackage{amsfonts,color}
\usepackage[dvips]{epsfig}

\textwidth 14cm
\textheight 21cm
\oddsidemargin=1.2cm 
\evensidemargin=1.2cm

\newcommand{\baseRing}[1]{\ensuremath{\mathbb{#1}}}
\newcommand{\Z}{\baseRing{Z}}
\newcommand{\C}{\baseRing{C}}
\newcommand{\N}{\baseRing{N}}
\newcommand{\R}{\baseRing{R}}
\newcommand{\Q}{\baseRing{Q}}
\newcommand{\T}{\baseRing{T}}
\newcommand{\D}{\Delta}

\def\P{\baseRing{P}}

\newcommand{\BB}{\mathcal{B}}
\newcommand{\CC}{\mathcal{C}}
\newcommand{\GG}{\mathcal{G}}
\newcommand{\HH}{\mathcal{H}}
\newcommand{\LL}{\mathcal{L}}
\newcommand{\MM}{\mathcal{M}}
\newcommand{\NN}{\mathcal{N}}


\theoremstyle{plain}
\newtheorem{theorem}{Theorem}[section]
\newtheorem{lemma}[theorem]{Lemma}
\newtheorem{corollary}[theorem]{Corollary}
\newtheorem{proposition}[theorem]{Proposition}
\newtheorem{construction}[theorem]{Construction}
\newtheorem{conjecture}[theorem]{Conjecture}

%\newenvironment{pf}{\noindent {\bf Proof.}}{\hfill $\Box$\vspace{0.3cm}}
%\newenvironment{explrm}{\begin{expl} \rm}{\end{expl}}
%\newenvironment{remrm}{\begin{rem} \rm}{\end{rem}}
%\newenvironment{dfrm}{\begin{df} \rm}{\end{df}}


\theoremstyle{definition}
\newtheorem{definition}[theorem]{Definition}
\newtheorem{observation}[theorem]{Observation}
\newtheorem{remark}[theorem]{Remark}
\newtheorem{example}[theorem]{Example}
\newtheorem{algorithm}[theorem]{Algorithm}


\def\theenumi{\roman{enumi}}
\renewcommand{\labelenumi}{\theenumi.}

\numberwithin{equation}{section}

\def\codim{ {\rm codim}\, }

\sloppy


\def\pstexInput#1{%
   \begin{picture}(0,0)%
     \special{psfile=#1.pstex}%
   \end{picture}%
   \input{#1.pstex_t}%
   }
%%%%%%%%%%%%%%%%%%%%%%%%%%%%%%%%%%%%%%%%%%%%%%%%%%%%%%%%%%%%%%%%%%%%%%%%
%%%%%%%%%%%%%%%%%%%%%%%%%%%%%%%%%%%%%%%%%%%%%%%%%%%%%%%%%%%%%%%%%%%%%%%%
%%%%%%%%%%%%%%%%%%%%%%%%%%%%%%%%%%%%%%%%%%%%%%%%%%%%%%%%%%%%%%%%%%%%%%%%

\begin{document}
\title{Discriminants, Resultants  \\ and Their Tropicalization}

\author{Bernd Sturmfels}

\centerline{Arizona Winter School 2006}

\maketitle

The aim of this course is to introduce discriminants and resultants,
 in the sense of Gel'fand, Kapranov and Zelevinsky \cite{GKZ},
 with emphasis on the tropical approach which was developed
 by  Dickenstein, Feichtner and the lecturer \cite{DFS}.
 
 \bigskip
 
{\bf Lecture 1: A-discriminants.}
Every configuration $A$ of lattice points defines a projective
toric variety $X_A$, whose dual variety $X_A^*$ is typically
a hypersurface, known as the {\em $A$-discriminant}.
We show how many classical discriminants and
classical resultants arise as special cases of this construction.

\medskip

{\bf Lecture 2: Degree Formulas.}
We present various known formulas for the degree
and Newton polytope of the $A$-discriminant. In the case
of resultants,  this degree involves mixed volumes \cite{BS},
and is closely related 
to determinantal formulas for eliminating 
variables from systems of polynomial equations.
For arbitrary $A$-discriminants, a positive
 degree formula was recently given in \cite{DFS}.

\medskip

{\bf Lecture 3: Tropical Varieties.} This lecture
 assumes familiarity with matroids and Gr\"obner bases,
 and it gives an otherwise self-contained introduction to
 tropical algebraic geometry. Software tools for computing
 arbitrary tropical varieties will be discussed briefly  \cite{BJSST}.
  We then show how to compute the
degree and the toric degenerations of a projective variety 
from its tropicalization, and how to tropicalize
the image of a map given by
monomials in linear forms.

\medskip

{\bf Lecture 4: Tropical Horn Uniformization.}
Kapranov's Horn uniformization \cite{K} parametrizes the
$A$-discriminant by monomials in linear forms.
From this we derive that the tropical $A$-discriminant
is the Minkowski sum of the co-Bergman fan of $A$
and the row space of $A$. This explains the degree formulas
discussed in Lecture 2, and it gives an algorithm
for computing its Newton polytope.
  We also  relate this to the combinatorial aspects
  of  Gel'fand-Kapranov-Zelevinsky theory
(regular triangulations, secondary polytopes \cite{GKZ}).


\bigskip

{\bf Project: Mixed Discriminants.}
Given a sparse system of $n$ polynomials in $n$ variables
with indeterminate  coefficients, their {\em mixed discriminant}
is the unique irreducible polynomial in the coefficients
which vanishes when the system has a double root.
The aim of this project is to find a formula for the degree 
and (Newton polytope) of the mixed discriminant, at least
when $n=2$. As an example consider the 
following system of two equations in $x$ and~$y$:
\begin{eqnarray*}
& \, c_1 x^2 y^{53} + 
c_2 x^3 y^{47} + 
c_3 x^5 y^{43} + 
c_4 x^7 y^{41}\,\,\,
\quad = \quad 0 , \\ &
c_5 x^{11} y^{37} + 
c_6 x^{13} y^{31} + 
c_7 x^{17} y^{29} + 
c_8 x^{19} y^{23}
\quad = \quad 0. 
\end{eqnarray*}
If the coefficients $c_1,c_2,\ldots,c_8$ are random complex
numbers then this system has $???$ distinct roots in $(\C^*)^2$.
The vanishing of the mixed discriminant is the condition
for this system to have a double root.
It is a homogeneous polynomial of degree $???$ in
the unknowns $c_1,c_2,\ldots,c_8$. Can
{\bf you} figure out what the two integers 
indicated by the question marks ``$ ???$'' are~?
If yes, then this AWS student project is the one for {\bf you}.
To get our discussions started, please e-mail me your answers
right away to \
{\tt bernd@math.berkeley.edu}.

\bigskip \bigskip \bigskip

\begin{thebibliography}{10}

\bibitem{BJSST} T.~Bogart, A.~Jensen, D.~Speyer, B.~Sturmfels and R.~Thomas: 

{\em Computing tropical varieties}; preprint,
{\tt http://front.math.ucdavis.edu/math.AG/0507563}.



\bibitem{DFS} A.~Dickenstein, E.-M.~Feichtner and B.~Sturmfels:
{\em Tropical Discriminants}; in preparation,  preprint
will appear on the {\tt ArXiV} in October 2005.

\bibitem{GKZ}  I.M.~Gelfand, M.M.~Kapranov, A.V.~Zelevinsky:
{\em  Discriminants, Resultants, and Multidimensional Determinants\/};
Birkh\"auser, Boston, MA, 1994. 

\bibitem{K} M.M.~Kapranov: {\em A characterization of 
$A$-discriminantal hypersurfaces in terms of the logarithmic Gauss map\/};
Mathematische Annalen  290  (1991),  no. 2, 277--285. 

\bibitem{BS} B.~Sturmfels: {\em On the Newton polytope of the resultant\/}; 
 Journal of Algebraic Combinatorics  3  (1994), 207--236.

\end{thebibliography}
\end{document}
