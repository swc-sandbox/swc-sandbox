\documentclass[11pt]{amsart}
\usepackage{amsmath,amsthm,amscd,amssymb}
\usepackage{latexsym}
\usepackage{graphicx}

%\pagestyle{empty}

\begin{document}

\title{P-adic Modular Forms by Kevin Buzzard - Lecture 1}

\maketitle

\newtheorem{lem}{Lemma}
\newtheorem{defn}{Definition}
\newtheorem{conj}{Conjecture}

\section{History}

\medskip
\noindent
Serre and Katz (1972):
Understand congruences mod $p$ and mod $p^n$ between modular forms
(see Antwerp).\\
Serre (1972):
Application to $p$-adic $L$-functions.\\
Katz (1972):
Wanted conceptual explanation of Atkin's work and Serre's paper.\\
Hida (early 80's):
``Ordinary'' subspace of space of $p$-adic modular forms, and
constructed families.\\
Gouvea (1987):
Makes conjectures that Mazur's deformation rings correspond to Hida's
ring of Hecke operators.\\
Wiles and Taylor-Wiles (1994):
Deformation rings = Hecke rings\\
Taylor et al:
Applications to conjecture of Artin.\\
Gouvea-Mazur:
Conjecture ``forms come in big families''\\
Coleman:
Uses rigid analytic approach, proved that forms come in small
families.\\
Wan, Smithline, Emerton (more recently)\\
Stein (more recently):
Enables us to do calculations.

\medskip

\section{Gouvea-Mazur Conjecture}

Let $p$ be prime, $N\geq 1$ an integer prime to $p$ and
$S_k(\Gamma_0(Np))$ the space of cusp forms of weight $k$ and level $Np$
(think of $N$, $p$ fixed with $k$ varying). Consider $U_p$, the Hecke
operator, acting on $S_k(\Gamma_0(Np))$, the characteristic
polynomial of $U_p$ lies in $\mathbb Z[X]$, think of the roots,
$\lambda_1,...,\lambda_n$ of this polynomial as elements of
$\overline{\mathbb Q}_p$ so that they
have valuations. For $\alpha\in\mathbb Q$ define
\begin{align*}
d(k,\alpha) = \#\{\lambda_i : v(\lambda_i)=\alpha \}
\end{align*}

\begin{conj}
Let $k_1,k_2\in\mathbb Z$, $\alpha\in\mathbb Q$, $\alpha\geq 0$ and
assume that\\
(i) $k_1,k_2 > \alpha + 1$\\
(ii) $k_1 \equiv k_2 \bmod (p-1)p^r$ for some $r\in\mathbb Z, r\geq\alpha$\\
then $d(k_1,\alpha) = d(k_2,\alpha)$.
\end{conj}

\medskip

One approach:\\
Remove hypothesis (i) but instead let $d(k,\alpha)$ be the number of
eigenvalues, with valuation $\alpha$, of $U_p$ acting on huge space of
``overconvergent $p$-adic
modular forms of weight $k$''.
\begin{center}
Classical mod forms of wt $k$ $\subset$ Overconvergent
$p$-adic mod forms of wt $k$
\end{center}

\section{Modular Forms}

\begin{defn}
A modular form of level 1 and weight $k\in\mathbb Z$ is an analytic
function $f:\mathcal H \to \mathbb C$ such that
\begin{align*}
f\biggl( {a\tau + b \over c\tau + d} \biggr) = (c\tau +d)^kf(\tau)
\text{ for all } \begin{pmatrix} a&b \\ c&d \end{pmatrix}\in
SL_2(\mathbb Z)
\end{align*}
and f satisfies boundedness conditions.
\end{defn}

If $k=0$ the only modular forms are constants. One could weaken
the boundedness conditions by, e.g., allowing $f$ to have some poles,
and then one gets a whole lot of interesting functions (e.g. $j$,
which has a "pole at infinity"). For $k=0$ a modular form is a
function on $SL_2(\mathbb Z)\setminus\mathcal H$, however if $k\ne 0$
then one can't think of a modular form as a function on $SL_2(\mathbb
Z)\setminus\mathcal H$ because of the $(c\tau +d)^k$ factor.\\

$SL_2(\mathbb Z)\setminus\mathcal H$ is a parameter space for elliptic
curves:\\
$\tau\in\mathcal H$ : $\mathbb C/\langle 1,\tau\rangle$ is a
1-dimensional complex torus.
\begin{align*}
\mathbb C/\langle 1,\tau\rangle\cong \mathbb C/\langle 1,\sigma\rangle 
&\iff \lambda\langle 1,\sigma\rangle = \langle 1,\tau\rangle \text{ for
  some } \lambda\in\mathbb C\\
&\iff \lambda = c\tau + d , \sigma\lambda = a\tau + b\\
&\iff \sigma = {a\tau + b \over c\tau + d} \text{ for some } \begin{pmatrix} a&b \\ c&d \end{pmatrix}  \in SL_2(\mathbb Z)
\end{align*}
So let's remember $\lambda$ by considering functions $f$ sending a
pair $(E,\omega)$ to a complex number $f(E,\omega)$ where $E$ is a
complex 1-dimensional torus and $\omega$ is a non-vanishing global differential
such that $f(E,\lambda\omega) = \lambda^{-k}f(E,\omega)$
A modular form of weight $k$ is an analytic rule sending $(E,\tau)$ to
$f(E,\tau)$ such that $f(E,\lambda\omega) = \lambda^{-k}f(E,\omega)$
and $f$ satisfies boundedness conditions.\\

That's ``encoded'' the functional equation, now we wish to encode the
word ``analytic'' by allowing families of tori.\\

If $S$ is a base complex manifold and $\pi:T\to S$ is a family of tori
then we should try and make sense of $f(T/S,\omega)$ where $\omega$ is
now a family of non-vanishing differentials: $f(T/S,\omega)$ should be
an analytic function $S\to\mathbb C$ and if $s\in\mathbb C$ then
$f(T/S,\omega)(s) = f(T_s,\omega_s)$.\\

This is now getting very non-computable but also much more algebraic.\\

\begin{defn}
Let $R_0$ be a ring (commutative with identity). A modular form of
level 1 and weight $k$ defined over $R_0$ is a rule, which to every
pair $(E/R,\omega)$ where,\\
(1) $R$ is an $R_0$-algebra\\
(2) $E/R$ is an elliptic curve, $\pi : E\to Spec(R)$\\
(3) $\omega\in H^0(E,\Omega^1)$ is a nowhere vanishing differential\\
an element $f(E/R,\omega)\in R$ such that\\
(a) $f(E/R,\omega)$ only depends on the isomorphism class of the
data\\
(b) if $E/R_1$ and $E'/R_2$ are elliptic curves, and we have
$\beta:Spec(R_2)\to Spec(R_1)$, then we can form the pullback
$\beta^*E$ of $E$ to $Spec(R_2)$ along $\beta$, and
if we have an isomorphism from $E'$ to $\beta^*E$ then the data
of the f's should match up too.\\
(c) $f(Tate(q),\omega_{can})\in R_0[[q]]$ (boundedness conditions)\\
(d) $f(E/R,\lambda\omega) = \lambda^{-k}f(E/R,\omega)$ for all $\lambda\in R^*$
\end{defn}

Explanation of (c):
There is an elliptic curve called $Tate(q)$ defined over
the $p$-adic completion of $R_0((q))$, with a canonical nowhere-vanishing
differential $\omega_{can}$, and by definition $f(Tate(q),\omega_{can})$ will
be in the $p$-adic completion of $R_0((q))$, and the assertion is that
it has to be in the much smaller ring $R_0[[q]]$ (the point is that
we want to rule out poles at infinity).\\


Non-computable definition: a modular form is a well behaved rule on
(Elliptic curves, differentials).

\bigskip

\begin{flushright}
David Whitehouse\\
California Institute of Technology\\
March 26, 2001
\end{flushright}

\end{document}








