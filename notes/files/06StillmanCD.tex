\magnification=\magstep1
\centerline{\bf Computing with cohomology in algebraic geometry}
\bigskip
\centerline{Mike Stillman (mike@math.cornell.edu)}

\bigskip
In modern algebraic geometry, cohomology is an important tool for many
different kinds of problems: it gives rise to numerical invariants of
algebraic varieties and it may be used to find tangent spaces of
deformation spaces and parameter spaces, among many other
applications.

\smallskip%\noindent
In this series of lectures, we introduce computing sheaf cohomology on
projective varieties and schemes, and illustrate its use in examples
and computational applications.  Our main goals are (1) to bring to
life the abstract cohomology machinery by learning how to compute
cohomology and use it computationally in examples, and (2) to present
the state of the art methods for computing sheaf cohomology.
 
\smallskip%\noindent
The suggested project (described later) will have us working on an
unsolved problem, that is interesting by itself, but it also ties into
the larger goal of "mapping the Hilbert scheme".

\bigskip\noindent
{\bf Lecture 1}: We show how to think of sheaves on projective schemes in a
computational manner, and show how to compute the most important
sheaves associated to a projective variety: tangent and cotangent
sheaves, normal sheaves, canonical sheaves, and line bundles and
divisors.

It is amazing that one can compute all of these, using only Groebner
bases and some byproducts of computing Groebner bases.

\bigskip\noindent
{\bf Lecture 2}: The Cech cohomology method of defining sheaf cohomology
translates into a reasonable method for computing sheaf cohomology.
We describe Serre's methods for computing sheaf cohomology: as direct
limits of Ext modules, and via local duality as the dual of an Ext
module.  We apply these methods to some of the sheaves that we
considered in the first lecture.

\bigskip\noindent
{\bf Lecture 3}: Each sheaf on projective space gives rise by an explicit
method to an exact complex (infinite in both directions) of free
E-modules, where E is the exterior algebra.  (This is called the 
Berstein-Gelfand-Gelfand correspondence).  The resulting complex is called
the Tate resolution, and can be computed using computer algebra
systems, such as Macaulay2.  It has many amazing properties and
applications, including that the cohomology of the original sheaf
and all of its twists appears in the Tate resolution!  This turns out
to be an excellent method to compute sheaf cohomology.  This approach
was pioneered by Eisenbud, Floystad and Schreyer.  In this lecture we
present this technique and its application to computing sheaf
cohomology.

\bigskip\noindent
{\bf Lecture 4}: We apply the above technique to examples.  Depending on the
interest and background of the students, we will consider (1)
the Beilinson monad, which is a very interesting method which can be
used to (attempt to) construct varieties with specific cohomology, (2)
resultants and Chow varieties, (3) computing higher direct image
sheaves, and applications, or (4) computing the "Hodge diamond" of a variety.

\bigskip\noindent
{\bf Reading list and prerequisites}

\medskip%\noindent
A knowledge of sheaves and schemes is not necessary for these lectures, nor
for the project,  The only prerequisite as far as this goes is an understanding that 
cohomology of sheaves is important and carries interesting geometric information.

\smallskip%\noindent
In order to prepare for these lectures, I suggest reading about Groebner 
bases and some of their applications (especially syzygies and free resolutions).  
An excellent readable introduction can be found in the book  
``Ideals, Varieties, and Algorithms'', by Cox, Little and O'Shea.

\smallskip%\noindent
Since we will be using Macaulay2 throughout the lectures and 
projects, it is worth downloading the latest version from 
our web site: {\tt http://www.math.uiuc.edu/Macaulay2} and playing with it.
Closer to the winter school, I will place a Macaulay 2 tutorial 
on the winter school web site, along with some exercises for you to develop
familiarity with the system.

\smallskip%\noindent
Although knowledge of sheaves is not completely necessary, it is
useful.  I suggest Serre's FAC (Faisceaux Algebrique Coherent) paper
from 1955 (this one is in French).  Hartshorne's book "Algebraic
Geometry" has a good introduction to sheaves, at the beginning of
Chapter 2.  However, our view of sheaves will be far more explicit and
computational.  So even if you don't know much about sheaves, these
lectures should be understandable.  The most important part is an
understanding that sheaves and their cohomology is important in the
first place!

\bigskip\noindent
{\bf Suggested Project} 

\medskip
 If $I$ is a homogeneous ideal in the polynomial ring $S =
k[x_0,...,x_n]$, then the normal sheaf $N$ of the projective variety
$X = V(I) \subset {\bf P}^n$ corresponds to the graded $S$-module
$Hom(I/I^2,S/I)$.  The normal sheaf contains a large amount of
information about "nearby" varieties (i.e. deformations).

\smallskip%\noindent
In the special, yet very interesting case when $I$ is an ideal
generated by monomials, we will attempt to find formulas for the
dimensions of the cohomology groups $\dim H^0(N)$ and $\dim H^1(N)$
for certain classes, or certain examples of monomial ideals.  No
general formulas for these dimensions are known.

\smallskip%\noindent
Besides being an open problem that can be attacked using computational
methods, these dimensions are interesting in "mapping the Hilbert
scheme".  The Hilbert scheme is a projective algebraic set (scheme)
whose points are in 1-1 correspondence with homogeneous (and
saturated) ideals $I$.  A path on the Hilbert scheme is a deformation,
or family of varieties.  For example, a Groebner basis computation
gives rise to a path which connects your original ideal to its initial
ideal of monomials.  Because of this, it is very interesting to
understand how the monomial ideals sit on the Hilbert scheme.  They
are in some ways the "backbone" or "subway stops" which allow you to
move around the Hilbert scheme.  The cohomology groups we will
consider in this project give geometric information about how the
ideal sits on the Hilbert scheme.

\smallskip%\noindent
For example, $\dim H^0(N)$ is the dimension of the Zariski tangent
space at the point of the Hilbert scheme at the point corresponding to
the ideal $I$.  $H^1(N)$ contains the obstructions for this Hilbert
scheme to be smooth at this point (so if it is zero, the Hilbert
scheme is smooth at that point).  In general, cohomology groups carry
interesting and subtle geometric information, in this case about
deformations of the algebraic variety $X$.

\bigskip\noindent
{\bf Other possible projects} 

\medskip
There are many other projects that one could suggest, depending on the
background and interests of the participants.  For example, suppose
you are given equations for a "mystery" variety, and you wish to
determine some structural information about it (e.g. If it is a
surface, where does it fit in the Kodaira classification of surfaces.
Or, if it is a rational surface, how is it obtained from ${\bf P}^2$
by a series of blow-up and blow-downs of points).  In this possible
project, we would be given equations for a variety, and we would
compute its cohomology, and attempt to understand the structure of the
variety.


\end
