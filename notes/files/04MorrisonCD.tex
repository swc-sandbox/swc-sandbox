\documentclass[12pt]{article}

\begin{document}

\centerline{\bf Instanton Sums and Monodromy}

\bigskip

\centerline{David R. Morrison}
\centerline{Duke University}

\bigskip

\bigskip

String theorists are interested in several quantum field theories and
string theories associated with Calabi--Yau threefolds.  The most 
straightforward of these to formulate is the two-dimensional quantum
field theory which governs the physics of a string propagating on
the Calabi--Yau threefold.  However, making calculations directly with
that quantum field theory is quite difficult.  To the extent that
they can be made, some amazing things have been discovered about 
Calabi--Yau threefolds, including the famous counting of rational
curves (as ``instantons'') on the quintic threefold.

In 1992, Witten proposed an alternative way to study these quantum
field theories for certain Calabi--Yau threefolds, by means of his
``gauged linear sigma model'' (GLSM).  The connection to the Calabi--Yau 
threefold 
is less direct, but the model has some features which make calculations
easier in many cases.  For example, the instantons in this model can
often be calculated explicitly with some ease, although they have
a different mathematical interpretation (i.e., they do not directly
count rational curves on the Calabi--Yau threefold).  When gathered 
into an appropriate ``instanton sum,'' they are expected to express
intrinsic information about the physical model, and so be ultimately
related to the famous curve-counting problem.  Many remarkable features 
of Calabi--Yau threefolds and their moduli spaces can be seen from
calculation involving these GLSM instantons.

The course will describe the mathematical aspects of these GLSM theories,
focusing on instantons and also on monodromy of various structures
defined over the moduli space.  The basic reference is \cite{summing},
although that paper was not written with a mathematical audience in mind.

The student project will study these 
 GLSM instanton sums, and other features 
of the moduli
space, explicitly in the case of a particular Calabi--Yau
threefold (the so-called (2,86) model).
The geometry of the (2,86) model is described many places, such
as \cite{bhole}.
Some aspects of the GLSM for this model were discussed in \cite{confinement}.

Useful mathematical orientation is provided in \cite{predictions},
section 3.3 of \cite{looking}, and \cite{aspects}.

\begin{thebibliography}{10}

\bibitem{summing}
{D. R. Morrison and M. R. Plesser}, {\em Summing the instantons: Quantum cohomology and
  mirror symmetry in toric varieties}, Nuclear Phys. B {\bf 440} (1995),
  279--354, {\tt hep-th/9412236}.


\bibitem{bhole}
B. R. Greene, D. R. Morrison, and A. Strominger, {\em Black hole condensation and the
  unification of string vacua}, Nuclear Phys. B {\bf 451} (1995), 109--120,
  {\tt hep-th/9504145}.

\bibitem{confinement}
B.~R.~Greene, D. R. Morrison, and C.~Vafa, {\em A geometric realization of confinement},
  Nuclear Phys. B {\bf 481} (1996), 513--538, {\tt hep-th/9608039}.




\bibitem{predictions}
D. R. Morrison, {\em Making enumerative predictions by means of mirror symmetry}, Mirror
  Symmetry {II} (B.~Greene and S.-T. Yau, eds.), International Press,
  Cambridge, 1997, pp.~457--482, {\tt alg-geom/9504013}.

\bibitem{looking}
D. R. Morrison, {\em Through the looking glass}, Mirror Symmetry {III} (D.~H. Phong, L.~Vinet,
  and S.-T. Yau, eds.), AMS/IP Studies in Advanced Mathematics, vol.~10,
  American Mathematical Society and International Press, 1999, pp.~263--277,
  {\tt alg-geom/9705028}.

\bibitem{aspects}
D. R. Morrison, {\em
Geometric aspects of mirror symmetry}, Mathematics Unlimited -- 2001 and Beyond (B. Enquist and W. Schmid, eds.), Springer-Verlag, 2001, pp. 899-918, {\tt math.AG/0007090}.

\end{thebibliography}

\end{document}
