\documentclass[11pt]{article}
\def\Spec{{\mathrm Spec}}

\begin{document}
\centerline{\bf Galois representations in fundamental groups 
 and their Lie algebras
}
\medskip
\centerline{Makoto MATSUMOTO}
We treat the following subjects (possibly not all).
A fundamental reference to the fundamental group is SGA1
(``Rev\^{e}tement Etales et Groupe Fondamental'' 
by A. Grothendieck and Mme M. Raynaud, 1971).

\begin{description}
\item{I} Algebraic fundamental group. 

Definition of algebraic fundamental group.
Use of ``etale coverings'' and ``fiber functors'', instead of
the usual notion of ``path modulo homotopy.''

Two keys:
\begin{itemize}
\item Comparison theorem with the topological fundamental groups.
This allows us to study algebraic fundamental groups 
using classical topology.

\item The algebraic fundamental group of a field is
its absolute Galois group. This connects algebraic fundamental groups
with number theory. 
\end{itemize}

\item{II}
Galois representation on fundamental groups, as monodromy.

The absolute Galois group of the base field acts on the algebraic
fundamental group. This can be regarded as an analogy to 
the geometric monodromy.

\item{III}
Computation using Puiseux Series. 

Compute the Galois action using analytic continuation.
Introduce the notion of tangential base points (and tangential morphisms).

\item{IV}
Soul\'e's cocycle and Deligne-Ihara conjecture.

Formulate Deligne-Ihara's conjecture about 
the Galois-action on the Lie algebra of 
the fundamental group of
projective line minus three points. Explain briefly
about a partial solution to this conjecture.
\end{description}

A possible subject to work is 
to give a computational direct proof of the 
fact that Soul\'e's cocycle appears
in the representation on the fundamental group
of projective line minus three points.
\end{document}
 