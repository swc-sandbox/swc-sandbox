\documentclass{article}

\title{Guide to the Arizona Winter School}
\date{}

\begin{document}
\maketitle


Welcome to the Arizona Winter School! The next five days are going to
be intense, so before you plunge in, look over these suggestions on
how to get the most out of them. We've divided the suggestions into
two sections, one for students and one for speakers (feel free to
cheat and look at the other section if you want).


\section{For students}
\label{sec:students}


The most important thing to remember is that the Arizona Winter School
is for you, not for the speakers, postdocs, or other people attending
the workshop. The courses, readings, projects, and help sessions have
been designed for you. Take part in them.


\subsection{How it all works}
\label{sec:how-it-all}

There are four topics for the School. For each topic we have invited
one or two experts to give a series of talks, for which lecture notes
are posted in advance. We have also asked the speakers  to
propose in advance a related project on which some of you will be
working. You
will be assigned in groups to one of the projects, will work on it
with the corresponding speaker (your team leader)
both before and during the school, and will make a presentation with
your team members at the end.  Even if you are not assigned to a project, you are welcome to join in and work on the problem (but in most cases you will not be part of the formal presentation). 

\subsection{Before the school (i.e., now!)}
\label{sec:readings}

Read the project descriptions and the lecture notes that have been
posted on the AWS web site. Read at least the ones for your group, and
all of them if at all possible. Follow up on the references cited, and discuss
them with your team  members. Start to think about the projects
that have been proposed. If you don't understand, email your team
leader and your team members with your
questions.


You aren't going to have a lot of time to do all this during the
school itself. Come as prepared as you possibly can and you will get a
lot more out of it. If you feel overwhelmed by the project
description, show it and the lecture notes to your advisor and ask for
suggested background reading. If you have time, organize a local
seminar at your institution on background material for one or more of
the topics. 


\subsection{Lectures}
\label{sec:lectures}
Read the notes in advance!
Don't be afraid to ask questions during the lectures. Grab the front
rows; they are for you. If you don't want to ask a question during the
lecture, go up to the speaker afterwards. If there are points you
don't understand, ask the speaker to clarify them.


\subsection{Working groups}
\label{sec:working-groups}
There's a good chance the some of you will get stuck at some point
during your work on your project. Don't think that you are the only
one! There are plenty of others in the same boat (and more who have
been in it in previous years and come back for a more relaxing
experience). There are many people you can ask for help: your team
leader, your fellow team members, a friendly postdoc or senior
graduate student who happens to be floating around, one of the
organizers of the school. Go back and look at the lecture notes and
papers that you read in preparation for the school, and see if they
shed any new light on the problem.


\subsection{Evening sessions}
\label{sec:even-quest-sess}
These are where most of the work and learning happens. Take advantage
of them.  Ask one of the speakers to expand on that day's topic, or to
give a preview of what is coming. Grab your team leader for an
extended work session.  Winter School alums will probably come down to
watch you suffer; make them work by helping you (you'll be able to
pick them by the fond smile of reminiscence on their faces).

``Ombudspeople" (typically a Winter School alum and a Southwestern Center member) will be available at the evening sessions and will try to resolve any issues that come up.  If you are having trouble, but not quite sure who to ask or how to ask about it, then they are the people to go to.


\subsection{Professional development component}
\label{sec:prof-devel-comp}
These activities are helpful and there is no exam on them. Take
advantage of them.



\subsection{Presentations}
\label{sec:presentations}
You will not have a lot of time, so make your presentation as
efficient as possible. Practice your presentation with other members
of the group. Most novice speakers make the mistake of preparing too
much material; don't try to fill up all the time available. You will
have questions from the audience, and it generally takes longer to
explain something than it does to think it through in your head. If
you have messy details to report---don't report them. Summarize the
key points, or put them on an overhead slide. Coordinate with your
team members so that you use the same notation and don't have to
repeat it. 



\subsection{What to do in your spare time}
\label{sec:what-do-your}
You don't need to worry about this, you won't have any.


\section{For speakers}
\label{sec:speakers}


\subsection{Lectures}
\label{sec:lectures-1}
The most important thing to remember is that the Arizona Winter School
is not for the big shots sitting in the front row (they shouldn't even
be there),  it is for the students. There is a wide range of levels, and we
want to serve them all. Many AWS speakers have made the mistake of
preparing too much material for the time available. It is better to
make the talks clear and understandable, and use the evening question
sessions for filling in extra details.


\subsection{Working groups}
\label{sec:working-groups-1}
A team of graduate students will be assigned the problems you
proposed. You are their team leader. You are responsible for getting
them through the project, and preparing them to make a coherent
presentation on their work at the end of the school. Take the time to
get to know them by email before the conference. If there are some who
seem less prepared, suggest readings to them. Meet with your students
early during the school, and set up a regular system of work sessions
with them. In addition to
mathematical help, they may well need help on how to prepare a
presentation. Serious attention to the team projects can pay off well;
in the past, some of the projects have produced publishable work.


\subsection{Evening question sessions}
\label{sec:even-work-sess}
The evening question sessions are a crucial part of your job; that's
where students who didn't understand a point in your lecture can ask
you about it, and that's where your team members will get guidance
from you. You might want to enlist the help of a senior graduate
student or postdoc who is not directly involved in your team. 


\subsection{What to do in your spare time}
\label{sec:what-do-your-1}
Alas, you won't have any of this either. 
\end{document}



%%% Local Variables: 
%%% mode: latex
%%% TeX-master: t
%%% End: 

% LocalWords:  postdocs AWS postdoc alums devel comp sess
