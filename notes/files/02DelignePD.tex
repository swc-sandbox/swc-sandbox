\documentclass[12pt,leqno]{article}
\usepackage{amsthm,amsbsy,amsfonts,amssymb,amsmath}
\usepackage{latexsym}
\usepackage{fontenc}
%\usepackage[mathscr]{euscript}

\setlength{\textwidth}{6.5in}
\setlength{\textheight}{8.5in}
\setlength{\topmargin}{0pt}
\setlength{\oddsidemargin}{0pt}
\setlength{\evensidemargin}{0pt}
\setlength{\headheight}{0pt}
\setlength{\headsep}{0pt}


\newcommand{\dbC}{{\mathbb{C}}} %Blackboard Bold
\newcommand{\dbG}{{\mathbb{G}}}
\newcommand{\dbP}{{\mathbb{P}}}
\newcommand{\dbQ}{{\mathbb{Q}}}
\newcommand{\dbR}{{\mathbb{R}}}

%\newcommand{\grA}{{\mathfrak{A}}}  %Gothic or German
%\newcommand{\grB}{{\mathfrak{B}}}
%\newcommand{\grC}{{\mathfrak{C}}} 
%\newcommand{\grD}{{\mathfrak{D}}}
%\newcommand{\grG}{{\mathfrak{G}}} 

%\newcommand{\ScrA}{{\mathscr{A}}} %Script 
%\newcommand{\ScrB}{{\mathscr{B}}}
%\newcommand{\ScrC}{{\mathscr{C}}} 
%\newcommand{\ScrL}{{\mathscr{L}}} 
%\newcommand{\ScrS}{{\mathscr{S}}}

\newcommand{\Ge}{{{\mathchoice{\,{\scriptstyle\ge}\,}
{\,{\scriptstyle\ge}\,}{\,{\scriptscriptstyle\ge}\,}
{\,{\scriptscriptstyle\ge}\,}}}}

\font\titlefont=cmss10 scaled\magstep2
\font\sectionfont=cmss10 scaled\magstep1

\newcommand{\dspace}{\lineskip=2pt
     \baselineskip=18pt\lineskiplimit=0pt}
\newcommand{\SubSet}{\raise2pt\hbox{$\,\scriptstyle
     \subset\,$}}
\newcommand{\otimesop}{\operatornamewithlimits{\otimes}\limits}

\title{\titlefont Pierre Deligne\\Arizona Winter School 2002}
\author{}
\date{}

\overfullrule=5pt
\begin{document}

\maketitle

\bigskip
\dspace
\noindent
{\sectionfont Project description}


Compute the $\zeta^{(p)}(s,t)$ mentioned
above, or make an educated guess, and verify
that they satisfy the identities known for the
$\zeta(s,t)$.

The ``known identities'' I have in mind are:

\medskip\noindent
{\sectionfont (a)}\enspace
Let $\Delta$ be the coproduct on $\dbQ\ll
e_0,e_1\gg$ for which the $e_i$ are primitive:
$\Delta e_i=e_i\otimes 1+1\otimes e_i$.
It induces a coproduct on $\dbQ\ll
e_0,e_q\gg/\dbQ\ll e_0,e_1\gg e_0+e_1\dbQ
\ll e_0,e_1\gg$.
In this quotient,
$$
g_{\dbR}:= \sum\zeta(s_1,\dotsc,s_r)
e_0^{s_1-1}e_1\ldots e_0^{s_r-1}e_1
$$
is group like: $\Delta
g_{\dbR}=g_{\dbR}\otimes g_{\dbR}$.

\medskip\noindent
{\sectionfont (b)}\enspace
$\zeta(s)\zeta(t)=\zeta(s,t)+\zeta(t,s)+
\zeta(s+t)$\quad (an instance of
$\sum\limits_{n,m}=\sum\limits_{n>m}
+\sum\limits_{m>n}+\sum\limits_{n=m}$)

\medskip
The action of Frobenius is defined in 
[1], \S{11}.
A more natural definition is given in 
[2], but I don't expect it to make
computations easier.
The tangential base point required (tangent
vector $1$ at $0$) is explained in
[1], \S{15}.
A rather unsatisfactory computation of the
$\zeta^{(p)}(s)$ is essentially the content of
[1], 19.6, 19.7, and amounts to
$\zeta^{(p)}(s)$ being the following
regularization of
$$
-p^s\sum\limits_{p\nmid n}\frac{1}{n^s}\,\,:
$$
write ${{\sum}'}$ for a sum extended only to $n$
prime to $p$ and for $\ell$ prime to $p$ write
formally
$$
\ell^{1-s}{{\sum}'}\frac{1}{n^s}=\ell{{\sum}'}
\frac{1}{(\ell n)^s}=\sum\limits_{\alpha^\ell=1}
{{\sum}'}\frac{\alpha^n}{n^s}\,\,,\quad \text{hence}
$$
$$
{{\sum}'}\frac{1}{n^s}=(\ell^{1-s}-1)^{-1}
\sum\limits_{\alpha^\ell=1,\alpha\not=1}
\mathop{{\sum}'}_{p\nmid n}
\frac{\alpha^n}{n^s}.
$$
It remains to regularize
${{\sum}'}\frac{z^n}{n^s}$ for $z$ not
congruent to $1\mod\,p$.
Let $(n\mod\,p^N)$ denote the residue of $n$
$\mod\,p^N$.
Define
\begin{align*}
{{\sum}'}\frac{z^n}{n^s} &=\lim\limits_{N}
{{\sum}'}\frac{z^n}{(n\mod\,p^N)^s}=
\lim\limits_{N}\sum\limits_{k}z^{p^N\cdot k}
\mathop{{\sum}'}_1^{p^N} \frac{z^n}{n^s}\\
&:=\lim\limits_{N}(1-z^{p^N})^{-1}
\mathop{{\sum}'}_1^{p^N}\frac{z^n}{n^s}.
\end{align*}


\bigskip
\begin{description}
\item[{[1]}]
P. Deligne,
{\it Le groupe fondamental de la droite
projective moins trois points},
in: Galois Groups over $\dbQ$, MSRI Publ. {\bf
16} (1989), p.~79--297.

\smallskip
\item[{[2]}]
B. Chiarellotto and B. Le Stum,
{\it $F$-isocristaux unipotents},
Comp. Math. {\bf 116} (1999), p.~81--110.
\end{description}

\bigskip
\dspace
\noindent
{\sectionfont Additional comment:}

The problem I ask is something which, using
results of Voevodsky as a black box, I can
prove is true (for multizeta in general, not
just the case asked). I think it would be
instructive (and could lead to a better
understanding of relations between classical
multidzeta) to have a direct proof. I tried to
ask the simplest case, but some variants more
complicated to formulate, involving roots of
1, could be easier to handle and as
instructive (cf. the fact that zeta has a pole
at 1, but Dirichlet $L$-functions don't).


The relation with ``periods'' is that
multizeta are periods, attached to some
motive, and crystalline Frobenius are another
way to look at the same motive. The same group
should control the algebraic relations between
periods, and where crystalline Frobenius sits.



\end{document}

