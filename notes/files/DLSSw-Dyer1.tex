\documentclass[12pt]{article}

\usepackage{amssymb}
\usepackage{amsmath}

\def\bA{{\mathbf A}}
\def\bC{{\mathbf C}}
\def\bG{{\mathbf G}}
\def\bP{{\mathbf P}}
\def\bQ{{\mathbf Q}}
\def\bR{{\mathbf R}}
\def\bZ{{\mathbf Z}}

\def\fO{{\mathfrak O}}
\def\fp{{\mathfrak p}}
\def\fR{{\mathfrak R}}

\def\ga{{\alpha}}
\def\gG{{\Gamma}}
\def\gk{{\kappa}}
\def\gl{{\lambda}}
\def\gs{{\sigma}}

\def\sB{{\mathcal B}}
\def\sC{{\mathcal C}}
\def\sF{{\mathcal F}}
\def\sN{{\mathcal N}}
\def\sS{{\mathcal S}}

\newfont{\wncyr}{wncyr10 at 12pt}
\def\Sha{\mbox{\wncyr Sh}}

\def\beq{\begin{equation} \label}
\def\ble{\begin{lemma} \label}
\def\bpr{\begin{question} \label}
\def\bth{\begin{theorem} \label}
\def\ele{\end{lemma}}
\def\epr{\end{question}}
\def\eth{\end{theorem}}

\newtheorem{theorem}{Theorem}
\newtheorem{lemma}{Lemma}
\newtheorem{question}{Question}

\def\half{{\textstyle{\frac{1}{2}}}}


\begin{document}
\begin{center}
\textbf{DIOPHANTINE EQUATIONS: PROGRESS AND PROBLEMS}
\end{center}


\vskip 0.1in


\noindent 1. \emph{Introduction}. \newline
A \emph{Diophantine problem} over $\bQ$ is concerned with the solutions
either in $\bQ$ or in $\bZ$ of a finite system of polynomial equations
\beq{E1} F_i(X_1,\ldots,X_n)=0 \quad (1\leq i\leq m) \end{equation}
with coefficients in $\bQ$. Without loss of generality we can obviously
require the coefficients to be in $\bZ$. A system (\ref{E1}) is also called
a system of \emph{Diophantine equations}. Often one will be interested in a
family of such problems rather than a single one; in this case one requires
the coefficients of the $F_i$ to lie in some $\bQ(c_1,\ldots,c_r)$, and one
obtains an individual problem by giving the $c_j$ values in $\bQ$. Again one
can get rid of denominators.
Some of the most obvious questions to ask about such a family are:
\begin{description}
\item[(A)] Is there an algorithm which will determine, for each assigned set
of values of the $c_j$, whether the corresponding Diophantine problem has
solutions, either in $\bZ$ or in $\bQ$?
\item[(B)] For values of the $c_j$ for which the system is soluble, is there
an algorithm for exhibiting a solution?
\end{description}
For individual members of such a family, it is also natural to ask:
\begin{description}
\item[(C)] Can we describe the set of all solutions, or even its structure?
\item[(D)] Is the phrase `density of solutions' meaningful, and if so, what
can we say about it?
\end{description}
The attempts to answer these questions have led to the introduction of new
ideas and these have generated new questions. On some of them I expect progress
within the next decade, and I have restricted myself
to these in the text below. Progress
in mathematics usually means proven results; but there are cases where even a 
well justified conjecture throws new light on the structure of the
subject. (For similar reasons, well motivated computations can be helpful; but
computations not based on a deep feeling for the structure of the subject have
generally turned out to be a waste of time.)
But I have not included those problems (such as the Riemann Hypothesis and the
Birch/Swinnerton-Dyer conjecture) on which I do not expect further progress
within so short a timescale.

Though the study of solutions in $\bZ$ and in $\bQ$ may look very similar (and
indeed were believed for a long time to be so), it now appears that they are
actually very different and that the theory for solutions in $\bQ$ has much
more structure than that for solutions in $\bZ$. The main reason for this seems
to be that in the rational case the system (\ref{E1}) defines a variety in
the sense of algebraic geometry, and many of the tools of that discipline can
be used; in contrast, in the integral case we do not know how to make
significant use of such
tools. However, for varieties of degree greater than 2 it is only
in low dimension that we yet know enough of the geometry for it to be useful.

Uniquely, the Hardy-Littlewood method is useful both for integral and for
rational problems; it was designed for integral
problems but can also be applied to rational problems in projective space,
because then the $F_i$ in (\ref{E1}) are homogeneous and it does not
matter whether we treat the variables $X_\nu$ as integral or rational. There
is a brief discussion of this method in \S8, and a comprehensive
survey in [42].

Denote by $V$ the variety defined by the equations (\ref{E1}) and let $V'$
be any variety birationally equivalent to $V$ over $\bQ$. The problem of
finding solutions of (\ref{E1}) in $\bQ$ is the same as that of finding
rational points on $V$, which is almost the same as that of 
finding rational points on $V'$. Hence (except possibly for Question (D) above)
one expects the properties of the rational solutions of
(\ref{E1}) to be essentially determined by the birational equivalence class
of $V$; and the way in which algebraic geometers classify varieties should
provide at least a first rough guide to the classification of Diophantine
problems --- though they mainly study birational equivalence over $\bC$
rather than over $\bQ$. But it does at the moment seem that the
geometric classification needs some
modifications and refinements if it is to fit the number-theoretic results
and conjectures.

Without loss of
generality we can assume that $V$ is absolutely irreducible. For if $V$ has
proper components defined over $\bQ$ it is enough to ask the questions above
for each of the proper components; and if $V$ is the union of varieties
conjugate over $\bQ$ then any rational point on $V$ lies on the intersection
of these conjugates, which is a proper subvariety of $V$. Since we can
desingularize $V$ by a birational transformation defined over $\bQ$,
it is natural to concentrate on the case when $V$ is
projective and nonsingular.

The definitions and the questions above can be generalized to an arbitrary
algebraic number field and the ring of integers in it; the answers are
usually known or conjectured to be essentially the same as over $\bQ$ or
$\bZ$, though the proofs can be very much harder. (But there are exceptions;
for example, the modularity of elliptic curves only holds over $\bQ$.)
The questions above can also be posed for other fields of
number-theoretic interest --- in particular for finite fields and for
completions of algebraic number fields --- and when one studies Diophantine
problems it is essential to consider these other fields also.
If $V$ is defined over a field $K$, the set of points on $V$ defined over $K$
will always be denoted by $V(K)$. If $V(K)$ is not empty we say that $V$ is
\emph{soluble in} $K$. In the special case where $K=k_v$, the completion of
an algebraic number field $k$ at the place $v$, we also say that $V$ is
\emph{locally soluble at} $v$. From now on we denote by
$\bQ_v$ any completion of $\bQ$; thus $\bQ_v$ means $\bR$ or some $\bQ_p$.

One major reason for considering solubility in complete fields and in finite
fields
is that a necessary condition for (\ref{E1}) to be soluble in $\bQ$, for
example, is that it is soluble in every $\bQ_v$.
The condition of solubility in every $\bQ_v$ is computationally decidable;
see \S2. Moreover, at least for primes $p$
for which the system (\ref{E1}) has good reduction mod $p$, the first step
in deciding solubility in $\bQ_p$ is to decide whether the reduced
system is soluble in the finite field GF$(p)$ of $p$ elements.

Geometers are used to studying varieties
over non-algebraically-closed fields $k$; what makes Diophantine problems
special is the number-theoretic nature of the fields $k$.
But it seems that only a few of the properties peculiar to such fields
are useful in this context, so that a geometer need not learn much
number-theory in order to work on Diophantine problems. On the other hand,
a number-theorist would be wise to learn quite a lot of geometry.

Diophantine problems were first
introduced by Diophantus of Alexandria, the last of the great Greek
mathematicians, who lived at some time between 300 B.C. and 300 A.D.; but
he was handicapped by having only one letter available to represent
variables, all the others being used in classical Greek
to represent specific numbers.
Individual Diophantine problems were studied by such great mathematicians
as Fermat, Euler and Gauss. But it was Hilbert's address to the International
Congress in 1900 which started the development of a systematic theory. His
tenth problem asked:
\begin{quote}
Given a Diophantine equation with any number of unknown
quantities
and with rational integral numerical coefficients: to devise a process
according to which it can be determined by a finite number of operations
whether the equation is soluble in rational integers.
\end{quote}
Most of the early work on Diophantine equations was concerned with rational
rather than integral solutions; presumably Hilbert posed this problem in
terms of integral solutions because such a process for integral solutions
would automatically provide the corresponding process for rational solutions
also, by restricting to the special case when the equations are homogeneous.
In those confident days before the First World War,
it was assumed that such an process
must exist; but in 1970 Matijasevi\v{c} showed that this was impossible.
Indeed he exhibited a polynomial $F(c;x_1,\ldots,x_n)$ such that
there cannot exist an algorithm which will decide for every given $c$ whether
$F=0$ is soluble in integers. His proof is part of the great program on
decidability initiated by G\"odel; good
accounts of it can be found in [10], pp 323-378 or [9]. The corresponding
question for
rational solutions is still open; I am among the very few who believe
that it may have a positive answer.

But even if the answer to the analogue of Hilbert's tenth problem for rational
solutions is positive, one must expect that a separate algorithm will be needed
for each kind of variety. Thus we shall need not one algorithm but an infinity
of them. So number theorists depend on the development by geometers of an
adequate classification of varieties. At the moment, such a classification is
reasonably complete for curves and surfaces, but it is still fragmentary even
in dimension 3; so number theorists have to concentrate on curves and surfaces,
and on certain particularly simple kinds of variety in higher dimension.

\bigskip

\noindent 2. \emph{The Hasse Principle and the Brauer-Manin obstruction}.
\newline
Let $V$ be a variety defined over $\bQ$. If $V$ is locally soluble at every
place of $\bQ$, we say that it satisfies the \emph{Hasse
condition}. If $V(\bQ)$ is not empty then $V$ certainly satisfies
the Hasse condition. What makes this remark valuable is that the Hasse
condition is computable --- that is, one can decide in finitely many steps
whether a given $V$ satisfies the Hasse condition. This follows from the
next two lemmas.
\ble{L1} Let $W$ be an absolutely irreducible variety of dimension $n$ defined
over the finite field
$k={\mathrm{GF}}(q)$. Then $N(q)$, the number of points on $W$ defined over
$k$, satisfies
\[ |N(q)-q^n|<Cq^{n-1/2} \]
where the constant $C$ depends only on the degree and dimension of $W$ and is
computable.
\ele
This follows from the Weil conjectures, for which see \S3; but weaker results
which are adequate for the present application were known much earlier. Since
the singular points of $W$ lie on a proper subvariety, there are at most
$C_1q^{n-1}$ of them, where $C_1$ is also computable. It follows that if $q$
exceeds a computable bound depending only on the degree and dimension of $W$
then $W$ contains a nonsingular point defined over $k$.

Now let $V$ be an absolutely irreducible variety defined over $\bQ$. If $V$ has
good reduction at $p$, which happens for all but a finite computable set of
primes $p$, denote that reduction by $\tilde{V}_p$. If $p$ is large enough, it
follows from the remarks above that $\tilde{V}_p$ contains a nonsingular
point $Q_p$ defined over GF$(p)$. The result which follows, which is known as
Hensel's Lemma though the idea of the proof goes back to Sir Isaac Newton,
now shows that $V$ contains a point $P_p$ defined over $\bQ_p$.
\ble{L2} Let $V$ be an absolutely irreducible variety defined over $\bQ$ and
let $\tilde{V}_p$ be its reduction ${\mathrm{mod~}}p$. If $\tilde{V}_p$
contains a nonsingular point $Q_p$ defined over ${\mathrm{GF}}(p)$ then
$V$ contains a nonsingular point $P_p$ defined over $\bQ_p$ whose reduction
${\mathrm{mod~}}p$ is $Q_p$.
\ele
In view of this, to decide whether $V$ satisfies the Hasse condition one only
has to check individually solubility in $\bR$ and in finitely many $\bQ_p$.
Each of these checks can be shown to be a finite process,
using ideas similar to those in the proof of Lemma \ref{L1}.

A family $\sF$ of varieties is said to satisfy the \emph{Hasse Principle} if
every $V$ contained in $\sF$ and defined over $\bQ$
which satisfies the Hasse condition
actually contains at least one point defined over $\bQ$. Again, a family $\sF$
is said to admit \emph{weak approximation} if every $V$ contained in $\sF$
and defined over $\bQ$, and such that $V(\bQ)$ is not empty, has the following
property: given any finite set of places $v$ and corresponding non-empty sets
$\sN_v\subset V(\bQ_v)$ open in the $v$-adic topology, there is a point $P$
in $V(\bQ)$ which lies in each
of the $\sN_v$. In the special case when $\sF$ consists of a single variety
$V$, and $V(\bQ)$ is not empty, we simply say that $V$ admits weak
approximation. In contrast to the Hasse condition, whether $V$ admits
weak approximation is in general not computable; for an important exception,
see [39].

The most important families which are known to have either of these properties
(and which actually have both) are the families of quadrics of any given
dimension; this was
proved by Minkowski for quadrics over $\bQ$ and by Hasse for quadrics
over an arbitrary algebraic
number field. But many families, even of very simple varieties, do not
satisfy either the Hasse Principle or weak approximation. (For example, neither
of them holds for nonsingular
cubic surfaces.) It is therefore natural to ask
\bpr{Q1} For a given family $\sF$, what are the obstructions to the Hasse
Principle and to weak approximation?
\epr
For weak approximation there is a variant of this question which may be both
more interesting and easier to answer. For another way of stating weak
approximation on $V$ is to say that if $V(\bQ)$ is not empty then it is dense
in the adelic space $V(\bA)=\prod_vV(\bQ_v)$. This suggests the following:
\bpr{Q19} For a given $V$, or family $\sF$, what can be said about the closure
of $V(\bQ)$ in the adelic space $V(\bA)$?
\epr
However, there are families for which Question \ref{Q1} does not seem to be a
sensible
question to ask; these probably include for example all families of
varieties of general type. So one should back up Question \ref{Q1} with
\bpr{Q2} For what kinds of families is either part of Question $\ref{Q1}$ a
sensible question to ask?
\epr

The only systematic obstruction to the Hasse Principle which is known is the
Brauer-Manin obstruction, though obstructions can be found in the literature
which are not Brauer-Manin. Let $A$ be a \emph{central simple algebra} ---
that is, a simple algebra which is finite dimensional over a field $K$ which
is its centre. Each such algebra consists, for fixed $D$ and $n$, of all
$n\times n$ matrices with elements in a division algebra $D$ with centre $K$.
Two
central simple algebras over $K$ are \emph{equivalent} if they have the same
underlying division algebra. Formation of tensor products over $K$ gives the
set of equivalence classes the structure of a commutative group, called the
\emph{Brauer group} of $K$ and written Br$(K)$. There is a
canonical isomorphism $\imath_p:{\mathrm{Br}}(\bQ_p)\simeq\bQ/\bZ$ for each
$p$; and there
is a canonical isomorphism $\imath_\infty:{\mathrm{Br}}(\bR)\simeq\{0,\half\}$,
the nontrivial division algebra over $\bR$ being the classical quaternions.

Let $B$ be an element of Br$(\bQ)$; tensoring $B$ with any $\bQ_v$ gives rise
to an
element of Br$(\bQ_v)$, and this element is trivial for almost
all $v$. There is an exact sequence
\[ 0\rightarrow{\mathrm{Br}}(\bQ)\rightarrow\bigoplus{\mathrm{Br}}(\bQ_v)
\rightarrow\bQ/\bZ\rightarrow0, \]
due to Hasse, in which the third map is the sum of the $\imath_v$; it tells us
when a set of elements, one in each Br$(\bQ_v)$ and almost all
trivial, can be generated from some element of Br$(\bQ)$.

Now let $V$ be a complete nonsingular variety defined over $\bQ$ and $A$ an
\emph{Azumaya algebra}
on $V$ --- that is, a simple algebra with centre $\bQ(V)$ which has a good
specialization at every point of $V$. The group of equivalence classes of
Azumaya algebras on $V$ is denoted by Br$(V)$.
If $P$ is any point of $V$, with field
of definition $\bQ(P)$, we obtain a simple algebra $A(P)$ with centre $\bQ(P)$
by specializing at $P$. For all but finitely many $p$, we have $\imath_p(A(
P_p))=0$ for all $p$-adic points $P_p$ on $V$. Thus a necessary condition for
the existence of a rational point $P$ on $V$ is that for every $v$ there
should be a $v$-adic point $P_v$ on $V$ such that
\beq{E14} \sum\imath_v(A(P_v))=0 \quad {\mathrm{for~all~}} A. \end{equation}
Similarly, a necessary condition for $V$ with $V(\bQ)$ not empty to admit weak
approximation is that (\ref{E14}) should hold for all Azumaya algebras $A$ and
all adelic points $\prod_vP_v$. In each case this is the \emph{Brauer-Manin
condition}. It is clearly unaffected if we add
to $A$ a constant algebra --- that is, an element of Br$(\bQ)$. So what we are
really interested in is Br$(V)/$Br$(\bQ)$.

All this can be put into highbrow language. Even without any hypotheses on $V$,
there is an injection of Br$(V)$ into the \'{e}tale cohomology group
H$^2(V,\bG_m)$; and if for example $V$ is a complete nonsingular surface, this
injection is an isomorphism. If we write
\[ {\mathrm{Br}}_1(V)={\mathrm{ker(Br}}(V)\rightarrow{\mathrm{Br}}(\bar{V}))
={\mathrm{ker(H}}^2(V,\bG_m)\rightarrow{\mathrm{H}}^2(\bar{V},\bG_m)), \]
there is a filtration
\[ {\mathrm{Br}}(\bQ)\subset{\mathrm{Br}}_1(V)\subset{\mathrm{Br}}(V). \]
Here only the abstract structure of Br$(V)/$Br$_1(V)$ is known; and in general
there is no known way of finding Azumaya algebras which represent nontrivial
elements of this quotient, though in a particular case Harari [20] has
exhibited a Brauer-Manin obstruction coming from such an algebra. In contrast,
there is an isomorphism
\[ {\mathrm{Br}}_1(V)/{\mathrm{Br}}(\bQ)\simeq{\mathrm{H}}^1({\mathrm{Gal}}
(\bar{\bQ}/\bQ),{\mathrm{Pic}}(V\otimes\bar{\bQ})), \]
and this is computable in both directions provided Pic$(V\otimes\bar{\bQ})$
is known. (For details of this, see [8].)

There is no known systematic way of determining Pic$(V\otimes\bar{\bQ})$ for
arbitrary $V$, and there is strong reason to suppose that this is really a
number-theoretic rather than a geometric problem. If $V$ is defined over $\bQ$
(rather than over an arbitrary algebraic number field) there is a tentative
algorithm, depending on the Birch/Swinnerton-Dyer conjecture,
for determining an algebraic number field $K$ (depending on $V$)
such that Pic$(V\otimes\bar{\bQ})=$Pic$(V\otimes K)$, and this may be the
right first step towards determining Pic$(V\otimes\bar{\bQ})$; but one hopes
not, because even for so elementary a variety
as a cubic surface we may need to have $[K:\bQ]\geq51840$. It seems to me
likely that a better approach to this question will be through the Tate
conjectures, for which see \S3; but this is a very long-term prospect.
However, it is usually possible to determine Pic$(V\otimes\bar{\bQ})$ for any
particular $V$ that one is interested in.
\bpr{Q3} Is there a general algorithm (even conjectural)
for determining ${\mathrm{Pic}}
(V\otimes\bar{\bQ})$ for varieties $V$ defined over an algebraic
number field?
\epr

Lang has conjectured that if $V$ is a variety of general type defined over an
algebraic number field $K$ then there is a finite union $\sS$ of proper
subvarieties of $V$ such that every point of $V(K)$ lies in
$\sS$. (Faltings' theorem, for which see \S4, is the special case of this for
curves.) This raises another question, similar to Question \ref{Q3} but
probably easier:
\bpr{Q8} Is there an algorithm for determining ${\mathrm{Pic}}(V)$ where $V$
is a variety defined over an algebraic number field?
\epr

There are very few families for which the Brauer-Manin obstruction can be
nontrivial but for which
it has been shown that it is the only obstruction to the Hasse
principle. (See however [12] and, subject to Schinzel's hypothesis, [37]
and [14].) It is generally believed that the Brauer-Manin obstruction is
indeed the
only obstruction to the Hasse principle for rational
surfaces --- that is, surfaces birationally equivalent to $\bP^2$ over
$\bar{\bQ}$. On the other hand, Skorobogatov ([33], and see also [34]) has
exhibited an obstruction to the Hasse principle on a bielliptic surface which
is definitely not Brauer-Manin.
\bpr{Q4} Is the Brauer-Manin obstruction the only obstruction
to the Hasse principle for all unirational (or all Fano) varieties?
\epr
We can of course ask a similar question for weak approximation.
The major difficulty in addressing such questions for a given family $\sF$
is that we do not know how to deduce anything useful from the fact that there
is no Brauer-Manin obstruction. The proofs of such results as are known all
break up into two parts:
\begin{description}
\item[(i)] Assuming that $V$ in $\sF$ satisfies the Hasse condition, one finds
a necessary and sufficient condition for $V$ to have a rational point, or to
admit weak approximation.
\item[(ii)] One then shows that this necessary and sufficient condition is
equivalent to the Brauer-Manin condition.
\end{description}
I know of no families for which it has been possible to carry out the first
part of this programme but not the second. But there are families for which it
has been possible to find a sufficient condition for solubility (additional to
the Hasse condition) which appears rather weak but which is
definitely stronger than the Brauer-Manin condition. However, such a condition
is usually not necessary and
the gap should be attributed to clumsiness in the proof.
\bpr{Q5} When the Brauer-Manin condition is trivial, how can one make use of
this fact?
\epr

\bigskip

\noindent 3. \emph{Zeta-functions and L-series}. \newline
Let $W\subset\bP^n$ be a nonsingular and absolutely irreducible projective
variety of dimension $d$ defined over the finite field $k=$GF$(q)$,
and denote by $\phi(q)$ the Frobenius automorphism of $W$ given by
\[ \phi(q): (x_0,x_1,\ldots,x_n)\mapsto(x_0^q,x_1^q,\ldots,x_n^q). \]
For any $r>0$ the fixed points of $(\phi(q))^r$ are precisely the points of
$W$ which are defined over GF$(q^r)$; suppose that there are $N(q^r)$ of them.
Although the context is totally different, this is almost the formalism of the
Lefschetz Fixed Point theorem, since for geometric reasons each of these fixed
points has multiplicity $+1$. This analogy led Weil to conjecture that there
should be a cohomology theory applicable in this context. This would
imply that there were finitely many complex numbers $\ga_{ij}$ such that
\beq{E2} N(q^r)=\sum_{i=0}^{2d}\sum_{j=1}^{B_i}(-1)^i\ga_{ij}^r \quad
{\mathrm{for~all~}} r>0, \end{equation}
where $B_i$ is the dimension of the $i$th cohomology group of $W$ and the
$\ga_{ij}$ are the characteristic roots of the map induced by $\phi(q)$ on
the $i$th cohomology. For each $i$ duality
implies that $B_i=B_{2d-i}$ and the $\ga_{2d-i,j}$
are a permutation of the $q^d/\ga_{ij}$. If we define the local zeta-function
$Z(t,W)$ by either of the equivalent relations
\[ \log Z(t)=\sum_{r=1}^\infty N(q^r)t^r/r \quad {\mathrm{or}} \quad
tZ'(t)/Z(t)=\sum_{r=1}^\infty N(q^r)t^r, \]
then (\ref{E2}) is equivalent to
\[ Z(t)=\frac{P_1(t,W)\cdots P_{2d-1}(t,W)}
{P_0(t,W)P_2(t,W)\cdots P_{2d}(t,W)} \]
where $P_i(t,W)=\prod_j(1-\ga_{ij}t)$. Each $P_i(t,W)$ must have coefficients
in $\bZ$, and the analogue of the Riemann hypothesis is that $|\ga_{ij}|=
q^{i/2}$. (For a fuller account of Weil's conjectures and their motivation,
see the excellent survey [23].) All this has now been proved, the main
contributor being Deligne.

\medskip

Now let $V$ be a nonsingular and absolutely irreducible projective variety
defined over an algebraic number field $K$. If $V$ has good reduction at a
prime $\fp$ of $K$ we can form $\tilde{V}_\fp$, the reduction of $V$ mod $\fp$,
and hence form the $P_i(t,\tilde{V}_\fp)$. For $s$ in $\bC$, we can now define
the $i$th global L-series $L_i(s,V)$
of $V$ as a product over all places of $K$, the factor at a prime $\fp$ of
good reduction being $(P_i(q^{-s},\tilde{V}_\fp))^{-1}$ where
$q=$Norm$_{K/\bQ}\fp$. The
rules for forming the factors at the primes of bad reduction and at the
infinite places can be found in [31]. These L-series of course depend on $K$
as well as on $V$. In particular, $L_0(s,V)$
is just the zeta-function of the algebraic number field $K$.

To call a function $F(s)$ a (global) zeta-function or L-series carries
with it certain implications:
\begin{itemize}
\item $F(s)$ must be the product of a Dirichlet series and possibly some
Gamma-functions, and the half-plane of absolute convergence for the Dirichlet
series must have the form $\fR s>\gs_0$ with $2\gs_0$ in $\bZ$.
\item The Dirichlet series must be expressible as an Euler product $\prod_p
f_p(p^{-s})$ where the $f_p$ are rational functions.
\item $F(s)$ must have an analytic continuation to the entire $s$-plane as a
meromorphic function, for all of whose poles $s$ is in $\bZ$.
\item There must be a functional equation relating $F(s)$ and $F(2\gs_0-1-s)$.
\item The zeroes of $F(s)$ in the critical strip $\gs_0-1<\fR s<\gs_0$ must
lie on $\fR s=\gs_0-\half$.
\end{itemize}
In our case, the first two implications are trivial; and fortunately one is
not expected to prove the last three, but only to state them as conjectures.
The last one is the Riemann Hypothesis, which appears to be out of reach even
in the simplest case, which is the classical
Riemann zeta-function; and the third and
fourth have so far only been proved in a few favourable cases.
\bpr{Q6} Can one extend the list of $V$ for which analytic continuation and
the functional equation can be proved? (It seems likely that any proof of
analytic continuation will carry a proof of the functional equation with it.)
\epr

It has been said about the zeta-functions of algebraic number fields that
`the zeta-function knows everything about the number field; we just have to
prevail on it to tell us'. If this is so, we have not yet unlocked the
treasure-house. Apart from the classical formula which relates $hR$ to
$\zeta_K(0)$ all that has so far been proved are certain results of Borel [6]
which
relate the behaviour of $\zeta_K(s)$ near $s=1-m$ for integers $m>1$
to the K-groups of $\fO_K$.
I would be reluctant to claim that the L-series of a variety $V$ contains all
the information which one would like to have about the number-theoretical
properties of
$V$; but one might hope that when a mysterious number turns up in
the study of Diophantine problems on $V$, some L-series contains information
about it.

Suppose for convenience that $V$ is defined over $\bQ$, and let its dimension
be $d$. Even for varieties with $B_1=0$ we do not expect a product like
\beq{E10} \prod_pN(p)/p^d \quad {\mathrm{or}} \quad
\prod_pN(p)\left/\left(\frac{p^{d+1}-1}{p-1}\right)\right. \end{equation}
to be necessarily absolutely convergent. But in some contexts there is a
respectable expression which is formally equivalent to one of these, with
appropriate modifications of the factors at the bad primes. The idea that
such an expression should have number-theoretic significance goes back to
Siegel (for genera of quadratic forms) and Hardy and Littlewood (for what they
called the \emph{singular series}). Using the ideas above, we are led to
replace the study of the products (\ref{E10}) by a study of the behaviour of
$L_{2d-1}(s,V)$ and $L_{2d-2}(s,V)$ near $s=d$. By duality, this is the same
as studying $L_1(s,V)$ near $s=1$ and $L_2(s,V)$ near $s=2$. The information
derived in this way appears to relate to the Picard group of $V$, defined as
the group of divisors defined over $\bQ$ modulo linear equivalence. By
considering simultaneously both $V$ and its Picard variety (the abelian
variety which parametrises divisors algebraically equivalent to zero modulo
linear equivalence), one concludes that $L_1(s,V)$ must be associated with
the Picard variety and $L_2(s,V)$ with the group of divisors modulo algebraic
equivalence --- that is, with the N\'{e}ron-Severi group of $V$. These remarks
motivate the weak forms of the Birch/Swinnerton-Dyer conjecture
(for which see \S4) and the case $m=1$ of
the Tate conjecture below. For the strong forms (which give
expressions for the leading coefficients of the relevent Laurent series
expansions) heuristic arguments are less convincing; but one can formulate
conjectures for these coefficients by asking what other mysterious numbers
turn up in the
same context and should therefore appear in the formulae for the leading
coefficients.

The weak form of the Tate conjecture asserts that the
order of the pole of $L_{2m}(s,V)$ at $s=m+1$ is equal to the rank of the
group of classes of $m$-cycles on $V$ defined over $K$, modulo algebraic
equivalence; it is a natural generalization of the case $m=1$ for which
the heuristics have just been shown. For a more detailed account of both of
these,
including the conjectural formulae for the leading coefficients,
see [41] or [36].
\bpr{Q7} What information about $V$ is contained in its L-series?
\epr

There is in the literature a beautiful edifice of conjecture, lightly
supported by evidence, about the behaviour of the $L_i(s,V)$
at integral points.  The principal architects of this edifice are Beilinson,
Bloch and Kato. Beilinson's conjectures relate to the order and leading
coefficients of the Laurent series expansions of the $L_i(s,V)$ about integer
values of $s$; in them
the leading coefficients are treated as elements of $\bC^*/
\bQ^*$. (For a full account see [28] or [22].) Bloch and Kato ([4] and
[5]) have strengthened these conjectures by treating the leading coefficients
as elements of $\bC^*$. But I do not believe that anything like the full story
has yet been revealed.

\bigskip

\noindent 4. \emph{Curves}. \newline
The most important invariant of a curve is its genus. In the language of
algebraic geometry over $\bC$, curves of genus 0 are called \emph{rational},
curves of genus 1 are called \emph{elliptic} and curves of genus greater than
1 are \emph{of general type}. But note that for a number theorist an elliptic
curve is a curve of genus 1 with a distinguished point $P_0$ on it, both being
defined over the ground field $K$. The effect of this is that the points on an
elliptic curve form an abelian group with $P_0$ as its identity
element, the sum of $P_1$ and $P_2$ being the
other zero of the function (defined up to multiplication by a constant) with
poles at $P_1$ and $P_2$ and a zero at $P_0$.

A canonical divisor on a curve $\gG$ of genus 0 has degree $-2$; hence by the
Riemann-Roch theorem $\gG$ is birationally equivalent over the ground
field to a conic. The Hasse
principle holds for conics, and therefore for all curves of genus 0; this gives
a complete answer to Question (A) at the beginning of these notes. But it does
not give an answer to Question (B). Over $\bQ$, a very simple answer to
Question (B) is as follows:
\bth{T1} Let $a_0,a_1,a_2$ be nonzero elements of $\bZ$. If the equation
\[ a_0X_0^2+a_1X_1^2+a_2X_2^2=0 \]
is soluble in $\bZ$, then it has a solution for which each $a_iX_i^2$ is
absolutely bounded by $|a_0a_1a_2|$.
\eth
Siegel [32] has given an answer to Question (B) over arbitrary algebraic
number fields, and Raghavan [27] has generalized Siegel's work to quadratic
forms in more variables.

The knowledge of one rational point on $\gG$ enables us to transform $\gG$ 
birationally into a line; so there is a parametric solution which gives
explicitly all the points on $\gG$ defined over the ground field. This
answers Question (C).

\medskip

If $\gG$ is a curve of general type defined over an algebraic number field $K$,
Mordell conjectured and Faltings proved that $\gG(K)$ is finite; and a number
of other proofs have appeared since then. But it does not seem that any of them
enable one to compute $\gG(K)$, though some of them come tantalizingly close.
For a survey of several such proofs, see [15].
\bpr{Q10} Is there an algorithm for computing $\gG(K)$ when $\gG$ is a curve
of general type defined over an algebraic number field $K$?
\epr

\medskip

The study of rational points on elliptic curves is now a major industry,
almost entirely separate from the study of other Diophantine problems.
If $\gG$ is an elliptic curve defined over an algebraic number field $K$,
the group $\gG(K)$ is called the \emph{Mordell-Weil group}. Mordell proved that
$\gG(K)$ is finitely generated; Weil's contribution was to extend this result
to all Abelian varieties. Thanks to Mazur (see [25]) the theory of the torsion
part of the Mordell-Weil
group is now reasonably complete; but for the non-torsion part all that was
known before 1960 is that $\gG(K)$ can be embedded into a certain group
which is finitely generated and computable. The process involved, which is
known as the method of infinite descent, goes back to Fermat; for use in \S6
I shall illustrate it below in a particularly simple case. By means of this
process
one can always compute an upper bound for the rank of the Mordell-Weil group
of any particular $\gG$, and the upper bound thus obtained can frequently
be shown to be equal to the actual rank by exhibiting enough elements
of $\gG(K)$. It was also conjectured that
the difference between the upper bound thus computed and the actual rank was
always an even integer, but apart from this the actual rank was
mysterious. This not wholly satisfactory state of
affairs has been radically changed by the Birch/Swinnerton-Dyer conjecture,
the weak form of which is described at the end of this section.

I now turn to the situation in which $\gG$ is a curve of genus 1 defined over
$K$ but not necessarily containing a point defined over $K$. Let $J$ be the
Jacobian of $\gG$, defined as a curve whose points are in one-one
correspondence with the divisors of degree 0 on $\gG$ modulo linear
equivalence. Then $J$ is also a curve of genus 1 defined over $K$, and $J(K)$
contains the point which corresponds to the trivial divisor. So $J$ is an
elliptic curve in our sense.

Conversely, if we fix an elliptic curve $J$ defined over $K$ we can consider
the equivalence classes (for birational equivalence over $K$) of curves $\gG$
of genus 1 defined over $K$ which have $J$ as Jacobian. For number theory, the
only ones of interest are those which contain points defined over each
completion $K_v$. These form a commutative torsion group, called
the \emph{Tate-Shafarevich group} and usually denoted by $\Sha$;
the identity element of this group is the
class which contains $J$ itself, and it consists of those $\gG$ which have $J$
as Jacobian and which contain a point defined over $K$. (The simplest example
of a nontrivial element of a Tate-Shafarevich group is the curve
\[ 3X_0^3+4X_1^3+5X_2^3=0 \quad {\mathrm{with~Jacobian}} \quad
Y_0^3+Y_1^3+60Y_2^3=0.) \]
Thus for curves of
genus 1 the Tate-Shafarevich group is by definition the obstruction to the
Hasse principle.

Suppose in particular that the elliptic curve $J$ is defined over $\bQ$ and
has the form
\[ Y^2=(X-c_1)(X-c_2)(X-c_3) \]
where the distinguished point is taken to be the point at infinity.
The three points $(c_i,0)$ on $J$ have order 2; they are called the
2-\emph{division points}. To any rational point $(x,y)$
on $\gG$ there exist $m_1,m_2,m_3$ and $y_1,y_2,y_3$
such that
\[ m_i(x-c_i)=y_i^2 \quad {\mathrm{for}} \quad i=1,2,3; \]
here the $m_i$ are really elements of $\bQ^*/\bQ^{*2}$ but it is convenient to
treat them as square-free integers. We must have $m_1m_2m_3=m^2$ for some
integer $m$, and $my=y_1y_2y_3$. Conversely the equations
\[ Y_i^2=m_i(X-c_i)\;\; (i=1,2,3) \quad {\mathrm{and}} \quad mY=Y_1Y_2Y_3 \]
for any $m,m_i$ with $m_1m_2m_3=m^2$
define a curve $\sC=\sC(m_1,m_2,m_3)$ and a four-to-one map $\sC\rightarrow J$.
If $\sC(\bQ)$ is not empty, its image under this map is a coset of $2J(\bQ)$
in $J(\bQ)$, and we obtain all such cosets in this way. Thus we could find
$J(\bQ)$ if we could decide which $\sC$ are soluble in $\bQ$.

After a change of variables
we can assume that the $c_i$ are in $\bZ$.
Define the \emph{good primes} for $J$ as those which do not divide
$(c_1-c_2)(c_2-c_3)(c_3-c_1)$; then it is not hard to show that $\sC$ is
locally soluble at
all good primes if and only if all the $m_i$ are units at all good primes.
So there are only finitely many $\sC$ whose solubility in $\bQ$
is at all hard to decide.

The curves $\sC$ obtained in this way are called 2-\emph{coverings} of $J$,
and the process of obtaining them is called a 2-\emph{descent}.
They form a group under multiplication of the corresponding triples
$(m_1,m_2,m_3)$. The finite subgroup consisting of those 2-coverings which
are everywhere locally soluble is called the 2-\emph{Selmer group}. It is
easily computable; and since there is a canonical embedding of $J(\bQ)/2J(\bQ)$
into the 2-Selmer group, this provides an upper bound for the rank of $J(\bQ)$.
The descent process can be continued, though with somewhat
greater difficulty; for 4-descents see [11]. One can also carry out
2-descents for the more general elliptic curve
\[ Y^2=X^3+aX^2+bX+c; \]
but in order to do this one requires information about the splitting field of
the right hand side.

The weak form of the Birch/Swinnerton-Dyer conjecture states that the rank of
the Mordell-Weil group of an elliptic curve $J$
is equal to the order of the zero of $L_1(s,J)$ at $s=1$; the conjecture
also gives an explicit formula for the leading coefficient of the power series
expansion at that point. Note that this point is in the critical strip, so
that the conjecture pre-supposes the analytic continuation of $L_1(s,J)$.
At present there are two well-understood 
cases in which analytic continuation is known: when
$K=\bQ$, so that $J$ can be parametrised by means of modular functions, and
when $J$ admits complex multiplication. In consequence, these two cases are
likely to be easier than the general case; but even here I do not expect much
further progress in the next decade. In each of these two cases, if one
assumes the Birch/Swinnerton-Dyer conjecture one can derive an algorithm for
finding the Mordell-Weil group and the order of the Tate-Shafarevich group;
and in the first of the two cases this algorithm has been implemented by
Gebel. (See [16].) Without using the Birch/Swinnerton-Dyer conjecture,
Heegner long ago produced a way of generating a point on $J$ whenever $K=\bQ$
and $J$ is modular; and Gross
and Zagier ([18] and [19]) have shown that this point has infinite order
precisely when $L'(1,J)\neq0$. Building on their work, Kolyvagin (see [17])
has shown the following.
\bth{T5} Suppose that the Heegner point has infinite order; then the group
$J(\bQ)$ has rank $1$ and $\Sha(J)$ is finite.
\eth
Kolyvagin [24] has also obtained sufficient conditions for both $J(\bQ)$ and
$\Sha(J)$ to be finite. The following result is due to Nekovar and Plater.
\bth{T6} If the order of $L(s,J)$ is odd then either $J(\bQ)$ is infinite or
$\Sha(J)\{p\}$ is infinite for every good ordinary $p$.
\eth
If $J$ can be parametrized by modular
functions for some arithmetic subgroup of SL$_2(\bR)$ then analytic
continuation and the functional equation follow; but there is not even a
plausible conjecture identifying the $J$ which have this property, and there
is no known analogue of Heegner's construction.

In the complex multiplication case, what is
known is as follows.
\bth{T2} Let $K$ be an imaginary quadratic field and $J$ an elliptic curve
defined and admitting complex multiplication over $K$. If $L(1,J)\neq0$, then
\begin{description}
\item[(i)] $J(K)$ is finite;
\item[(ii)] for every prime $p>7$ the $p$-part of $\Sha(J)$
has the order predicted by the Birch/Swinnerton-Dyer conjecture.
\end{description}
\eth
Here (i) is due to Coates and Wiles, and (ii) to Rubin. For an account of the
proofs, see [29]. Katz has generalized (i) and part of (ii) to behaviour over
an abelian extension of $\bQ$, but with the same $J$ as before.

In general we do not know how to compute $\Sha$. It is
conjectured that it is always finite; and indeed this assertion can be
regarded as part of the Birch/Swinnerton-Dyer conjecture, for the formula
for the leading coefficient of the power series for $L_1(s,J)$
at $s=1$ contains the
order of $\Sha(J)$ as a factor. If indeed this order is finite,
then it must be a square; for Cassels has proved the existence of a
nonsingular skew-symmetric bilinear form on $\Sha$
with values in $\bQ/\bZ$. Thus
finiteness implies that if $\Sha$ contains at most $p-1$ elements of order
exactly $p$ for some prime $p$ then it actually contains no
such elements; hence an element which is killed by $p$ is trivial,
and the curves of genus 1 in that equivalence class
contain points defined over $K$. For use later, we state the case $p=2$ as a
lemma.
\ble{L4} Suppose that $\Sha(J)$ is finite and the
quotient of the $2$-Selmer group of $J$ by its soluble elements has order at
most $2$; then that quotient is actually trivial.
\ele
This result will play a crucial role in \S\S 6 and 7.


\bigskip

\noindent 5. \emph{Generalities about surfaces}. \newline
Over $\bC$ a full classification of surfaces can be found in [1]. A first
coarse classification is given by the \emph{Kodaira dimension} $\gk$, which
for surfaces can take the values $-\infty,0,1$ or 2. What also seems to be
significant for the number theory (and cuts across this classification)
is whether the surface is \emph{elliptic} --- that is, whether over $\bC$
there is a map $V\rightarrow C$ for some curve $C$ whose general fibre
is a curve of genus 1. The case when the map $V\rightarrow C$ is defined over
the ground field $K$ and $C$ has genus 0 is discussed in \S6; in this case the
Diophantine problems for $V$ are only of interest when $C(K)$ is nonzero,
in which case $C$ can be identified with
$\bP^1$. When $C$ has genus greater than
1, the map $V\rightarrow C$ is essentially unique and it and $C$ are therefore
both defined over $K$. By Faltings' theorem, $C(K)$ is then finite;
thus each point of $V(K)$ lies on one of a finite set of fibres, and it is
enough to study these. In contrast,
we know nothing about the case when $C$ is elliptic.

The surfaces with $\gk=-\infty$ are precisely the \emph{ruled surfaces} ---
that is, those which are birationally equivalent over $\bC$ to $\bP^1\times C$
for some curve $C$. Among these, by far the most interesting are the
\emph{rational surfaces}, which are birationally equivalent to $\bP^2$ over
$\bC$.

Surfaces with $\gk=0$ fall into four families:
\begin{itemize}
\item Abelian surfaces. These are the analogues in two dimensions of elliptic
curves, and there is no reason to doubt that their number-theoretical
properties simply generalize those of elliptic curves.
\item K3 surfaces, including in particular Kummer surfaces. Some but not all
K3 surfaces are elliptic.
\item Enriques surfaces, whose number theory has been very little studied.
Enriques surfaces are necessarily elliptic.
\item bielliptic surfaces.
\end{itemize}

Surfaces with $\gk=1$ are necessarily elliptic.

Surfaces with $\gk=2$ are called \emph{surfaces of general type} --- which in
mathematics is generally a derogatory phrase. About them there is currently
nothing to say beyond Lang's conjecture stated in \S2.

\medskip

In the next two sections I shall outline what can at present be said about
rational surfaces and K3 surfaces respectively; these appear to be the two
most interesting families of surfaces for the
number-theorist. In both cases many of the most
recent results depend on one or both of two major conjectures. One of these
(for the reason given near the end of \S4) is
the finiteness of the Tate-Shafarevich group; the other is
Schinzel's Hypothesis, which we now describe. It gives a conjectural answer
to the following question: given finitely many polynomials $F_1(X),\ldots,
F_n(X)$ in $\bZ[X]$ with positive leading coefficients, is there an
arbitrarily large integer $x$ at which they all take prime values? There are
two obvious obstructions to this:
\begin{itemize}
\item One or more of the $F_i(X)$ may split in $\bZ[X]$.
\item There may be a prime $p$ such that for any value of $x$ mod $p$ at
least one of the $F_i(x)$ is divisible by $p$.
\end{itemize}
Clearly the second obstruction can only happen for $p\leq\sum\deg(F_i)$.
Schinzel's Hypothesis is that these are the only obstructions: in other words,
if neither of them happens then we can choose an arbitrarily large $x$ so
that every $F_i(x)$ is a prime. There are various more complicated variants of
this hypothesis (including ones in other algebraic
number fields), but they all follow
fairly easily from the hypothesis in its original form.

No one in his right mind would attempt to prove Schinzel's Hypothesis; indeed
one instance is the notoriously intractable twin primes problem,
which is the special case when the $F_i$ are the two polynomials $X+1$ and
$X-1$. But probabilistic arguments suggest that the hypothesis is in fact
true. At the very least it would be perverse to look for counter-examples to
results which have been proved subject to Schinzel's Hypothesis.

\bigskip

\noindent 6. \emph{Rational surfaces}. \newline
From the number-theoretic point of view, there are two kinds of rational
surface:
\begin{itemize}
\item Pencils of conics, given by an equation of the form
\beq{E3} a_0(u,v)X_0^2+a_1(u,v)X_1^2+a_2(u,v)X_2^2=0 \end{equation}
where the $a_i(u,v)$ are homogeneous polynomials of the same degree. Pencils
of conics can
be classified in more detail according to the number of bad fibres.
\item Del Pezzo surfaces of degree $d$, where $0<d<9$. Over $\bC$, such a
surface is obtained by blowing up $(9-d)$ points of $\bP^2$ in general
position. It is known that Del Pezzo surfaces of degree $d>4$ satisfy the
Hasse principle and weak approximation; indeed those of degree 5 necessarily
contain rational points. Del Pezzo surfaces of degree 2 or 1
have no aesthetic
merits and have attracted little attention; it seems sensible to ignore
them until the problems coming from those of degrees 4 and 3 have been solved.
The Del Pezzo surfaces of degree 3 are the nonsingular cubic
surfaces, which have an enormous but largely irrelevent literature,
and those of degree 4 are the nonsingular intersections of
two quadrics in $\bP^4$. For historical reasons, attention has been
concentrated on the Del Pezzo
surfaces of degree 3; but the problems presented by those
of degree 4 are necessarily simpler.
\end{itemize}

\medskip

We consider first pencils of conics, and assume that (\ref{E3}) is defined
over $\bQ$, the argument for an arbitrary algebraic number field not being
essentially different. We can require the coefficients of the
$a_i(u,v)$ to be in $\bZ$. Since the Hasse principle holds for conics, it is
enough to choose $u=u_0,v=v_0$ in such a way that (\ref{E3}) is locally
soluble at $2,\infty$ and all the odd primes which divide any of the
$a_i(u_0,v_0)$. As it stands, this appears to involve arguing in a circle;
the way to make the argument respectable is as follows.

Assume that (\ref{E3}) is everywhere locally soluble.
By absorbing suitable factors into the $X_i$, we can ensure that the $a_i(u,v)$
are square-free and coprime. To achieve this, we have to drop the condition
that the $a_i(u,v)$ are all of the same degree; but it is still true that
their degrees are all even or all odd. Denote by $\sB$ the set of bad places,
which turns out to consist of $2,\infty$, the primes which divide the
discriminant of $a_0(u,v)a_1(u,v)a_2(u,v)$ and the primes which do not exceed
the degree of that product. Let $\sS$ be the space of all pairs of coprime
integers $u_0,v_0$, with the topology induced by the places of $\sB$; and let
$\sS_0$ be the subset of $\sS$ consisting of the points at which (\ref{E3}) is
locally soluble at every place in $\sB$. By hypothesis, $\sS_0$ is not empty;
and it is open in $\sS$. To obtain solubility in $\bQ$, we have to choose
$u_0,v_0$ in $\sS_0$ so that (\ref{E3}) is locally soluble at each good prime
$p_0$ which divides one of the $a_i(u_0,v_0)$; for solubility at the other
good primes is trivial, and we have already taken care of the bad places.
Let $c(u,v)$ be the irreducible
factor of that one of the $a_i(u,v)$ for which $p_0|c(u_0,v_0)$, and to fix
ideas assume that $c(u,v)$ divides $a_2(u,v)$; here $c(u,v)$ is unique
because $p_0$ does not divide the discriminant of $a_0a_1a_2$. The condition
of local solubility at
$p_0$ is
\beq{E5} (a_0(u_0,v_0)a_1(u_0,v_0),c(u_0,v_0))_{p_0}=+1 \end{equation}
where the outer bracket is the Hilbert symbol. So a necessary condition for
the solubility of (\ref{E3}) is that all the conditions like
\beq{E6} \prod(a_0(u_0,v_0)a_1(u_0,v_0),c(u_0,v_0))_p=+1 \end{equation}
hold simultaneously
for some $(u_0,v_0)$ in $\sS_0$, where the product is taken over all $p$
which divide $c(u_0,v_0)$.

What is unexpected is that this turns out to be useful, because of the
following lemma. The proof of the lemma
is straightforward, since the function $\Phi$
behaves like a quadratic residue symbol and can be evaluated by a Euclidean
algorithm process very like that which is used for such symbols.
\ble{L3} Let $F(u,v),G(u,v)$ be homogeneous polynomials in $\bZ[u,v]$, with
$\deg(F)$ even. Let $\sB$ be a finite set of places of $\bQ$ which contains
$2,\infty$ and all the primes which divide the discriminant of $FG$. For
any coprime $u_0,v_0$ in $\bZ$, write
\beq{E4} \Phi(\sB;F,G;u_0,v_0)=\prod(F(u_0,v_0),G(u_0,v_0))_p \end{equation}
where the outer bracket on the right is the Hilbert symbol and the product is
taken over all primes $p$ not in $\sB$ such that $p|G(u_0,v_0)$.
Then $\Phi(u_0,v_0)$ is continuous in the topology on $\sS$, and computable.
\ele
In this result we take $F=a_0a_1,G=c$; we noted above that $\deg(a_0a_1)$ is
necessarily even. It follows that a necessary condition for the
solubility of (\ref{E3}) is that there is a point $(u_0,v_0)$ in $\sS_0$
such that $\Phi(u_0,v_0)=+1$ for all $\Phi$ which can be generated from
(\ref{E3}) in this way.
This condition is computable, and it is unsurprising (though not obvious)
that it turns out to be equivalent to the Brauer-Manin condition for
(\ref{E3}).

If one assumes Schinzel's Hypothesis, this condition is also
sufficient. For suppose that $u_0,v_0$ have been so chosen that there is only
one good prime $p_0$ which divides $c(u_0,v_0)$; then the product in (\ref{E6})
reduces to the left hand side of (\ref{E5}), and so (\ref{E5}) holds for this
prime. Now choose an open set $\sN\subset\sS_0$ such that (\ref{E6}) holds
throughout $\sN$ for each $c(u,v)$; by a slightly modified version
of Schinzel's Hypothesis we can choose $(u_0,v_0)$ in $\sN$
so that every $c(u_0,v_0)$ is
the product of one good prime and possibly some factors in $\sB$. As $c$ runs
through all irreducible factors of $a_1a_2a_3$, $p_0$ runs through all those
primes for which we have to verify (\ref{E5}). Thus (\ref{E3}) is everywhere
locally soluble for the pair $u_0,v_0$, and therefore globally soluble. With
minor modifications, the same argument shows that (subject to Schinzel's
Hypothesis) the Brauer-Manin obstruction is also the only obstruction to weak
approximation.

If there is no Brauer-Manin obstruction, this construction finds infinitely
many conics in the pencil which
contain rational points. But, somewhat unexpectedly, even if we know some
conics of the pencil which are soluble, without Schinzel's Hypothesis we do
not know how to generate more such conics.
\bpr{Q11} Given a pencil of conics and finitely many conics in the pencil each
of which contains rational points, can we generate further conics of the
pencil which contain rational points without using Schinzel's Hypothesis?
\epr

\medskip

If we know even one rational point on a Del Pezzo surface $V$ of degree 3 or 4,
we can obtain an infinity of curves of genus 0 each of which lies on $V$,
though they will be singular and for degree 3
it will usually not be true that each point of $V(\bQ)$ lies on at least
one curve of the family. But without
such a point, the best we can do is to find on $V$ a family of curves of
genus 1. At first sight, it would seem that in these circumstances nothing
resembling the argument above can be applied;
for an essential component of that
argument was that conics satisfy the Hasse principle, and this is not true
for curves of genus 1. However Lemma \ref{L4} provides us with a way round
this obstacle.

The arguments involved are applicable to some surfaces $V$ which are not
necessarily rational, but which are elliptic with a fibration
$V\rightarrow\bP^1$. Consider a pencil of curves $\sC_\gl$ of genus 1, each of
which is a 2-covering of its Jacobian $J_\gl$. If we can choose $\gl$ in
such a way that $\sC_\gl$ is everywhere locally soluble and at least half
the elements of the 2-Selmer group of $J_\gl$
are soluble (and if we assume the finiteness of the relevent Tate-Shafarevich
group), then it will follow from Lemma \ref{L4} that $\sC_\gl$ is soluble.
For this machinery to have any chance of working, we must be able to implement
the 2-descent on $J_\gl$ in a manner which is uniform in $\gl$. This more or
less requires $J_\gl$ to have its 2-division points defined over $\bQ(\gl)$ and
therefore to have the form
\beq{E7} Y^2=(X-c_1(\gl))(X-c_2(\gl))(X-c_3(\gl)) \end{equation}
where the $c_i(\gl)$ are in $\bQ(\gl)$; but an additional trick, given in [2],
enables the method to be used even if
$J_\gl$ has just one 2-division point in $\bQ(\gl)$.

The details of this method are unattractive, but the strategy is as follows.
(For a full account, see [13].)
Without loss of generality we can assume that the $c_i(\gl)$ are in $\bZ[\gl]$.
For any particular integral value  $\gl_0$ of
$\gl$, the bad places for the equation (\ref{E7})
are the bad places for the system together with the primes which divide one
of the $c_i(\gl_0)-c_j(\gl_0)$. It was explained in \S4 how to implement the
2-descent for (\ref{E7}). We shall eventually use Schinzel's Hypothesis
to choose $\gl_0$ so that the value of each irreducible factor of any
$c_i(\gl)-c_j(\gl)$ at $\gl=\gl_0$ is the product of some bad primes for the
system with one good prime. We call the latter a \emph{Schinzel prime}; though
it is a good prime for the system, it is
a bad prime for the curve obtained by writing $\gl=\gl_0$ in (\ref{E7}). The
effect of restricting $\gl_0$ in this way
is that we know those 2-coverings of (\ref{E7}) for $\gl=\gl_0$
which are locally soluble at all good primes for the curve; they form a finite
group of 2-coverings $\sC'_\gl$ of $J_\gl$ which does not depend on the choice
of $\gl_0$ provided it satisfies the condition above.

This group certainly contains the original $\sC_\gl$ and the 2-coverings which
correspond to the 2-division points. The next step, which involves a
sophisticated analysis of the 2-descent process and also requires us to
introduce additional well chosen bad primes for the
system, is to find local conditions
on $\gl_0$ at the bad primes of the system which ensure that for $\gl=\gl_0$
\begin{itemize}
\item the only elements of this group which are locally soluble at all the bad
places of the system lie in the subgroup generated by the original $\sC_\gl$
and the 2-coverings generated by the 2-division points; and
\item the original $\sC_\gl$ is locally soluble at all the bad places of the
system.
\end{itemize}
This is not always possible; if it is impossible, that provides an obstruction
to this method of attack on the problem
though not necessarily to the solubility of the system. In general this
obstruction is not much stronger than the Brauer-Manin obstruction, and in some
cases they are provably the same; so this
is not too serious a blemish on the method. If we achieve
the two properties above then solubility at the Schinzel
primes turns out to be
automatic. (This is an example of what seems to be a rather
general phenomenon,
that the number of related things which go wrong must be even.) With our
enlarged set of bad places for the system, we now choose $\gl_0$ to satisfy
the local conditions and the Schinzel condition in the previous paragraph. By
Lemma \ref{L4}, the curve (\ref{E7}) is soluble for this value of $\gl$.
But in contrast to what happens for pencils of conics, this kind of
argument appears to provide no information about weak approximation.
\bpr{Q12} Can one modify the method above so that it works without any
assumption about the $2$-division points of $J_\gl$?
\epr

Unfortunately it is not known (and probably is not even true) that an arbitrary
Del Pezzo surface of degree 4 contains a pencil of curves of genus 1 of this
particular type --- and the situation for Del Pezzo surfaces of degree 3 is
much worse, in that the natural curves to consider are 3-coverings
rather than 2-coverings.

However, for Del Pezzo surfaces of degree 4 some progress has been made.
Salberger
and Skorobogatov [30] have shown that the Brauer-Manin obstruction is the only
obstruction to weak approximation. As for the Hasse principle, the present
situation is as follows. A Del Pezzo surface of degree 4 defined over
the algebraic number field $K$ has the form
\[ V:F_1(X_0,\ldots,X_5)=F_2(X_0,\ldots,X_5)=0, \]
where the coefficients of $F_1$ and $F_2$ are in $K$.
By a linear transformation defined over an extension $K_1$ of odd degree over
$K$, we can separate off one of the variables; and over $K_1$ we can find a
pencil of curves $\sC_\gl$ of genus 1 on $V$ for which each $J_\gl$ has one
2-division point defined over $K_1(\gl)$. 
Using the trick described in [2], and subject to an obstruction typical for
the method, it can now be shown that $V$ contains a point defined over
$K_1$. A straightforward geometric argument, which does not rely on $K$ being
an algebraic number field, now shows that $V$ contains a point
defined over $K$. Unfortunately the overall arguments are so complicated
(and so unnatural) that it is not clear whether the obstruction to the method
is still simply the Brauer-Manin obstruction to the solubility of $V$ over $K$;
but even if it is stronger, it is not much stronger.

To use and then collapse a field extension in this way is a device which
probably has a number of uses. For such a collapse step to be feasible, the
degree of the field extension needs to be prime to the degree of the variety;
and this leads one to phrase the same property somewhat differently.
\bpr{Q13} Let $V$ be a variety defined over a field $K$, not necessarily of
a number-theoretic kind. For what families of $V$ is it true that if $V$
contains
a $0$-cycle of degree $1$ defined over $K$ then it contains a point defined
over $K$?
\epr
As stated above, this is true for Del Pezzo surfaces of degree 4.
For pencils of conics it is in general false, even for algebraic number fields
$K$. For Del Pezzo surfaces of degree 3 the question is open: I expect it to
be true for algebraic number fields $K$ but false for general fields.

A variant of the method above can be applied to diagonal cubic surfaces
\beq{E12} V:a_0X_0^3+a_1X_1^3=a_2X_2^3+a_3X_3^3, \end{equation}
subject to one very counterintuitive condition, which is that $K$, the field of
definition of $V$, must not contain the primitive cube roots of unity.
Write $V$ in the form
\beq{E8} a_0X_0^3+a_1X_1^3=\gl Y^3, \quad a_2X_2^3+a_3X_3^3=\gl Y^3
\end{equation}
where $\gl$ is at our disposal. We now have two pencils of
curves of genus 1, each of which is a $\sqrt{-3}$-covering
of its Jacobian; and we have to apply the method simultaneously to both
curves. This introduces considerable additional complications, for which
see [40]; and the obstruction to the method, though weak, is certainly
strictly stronger than the Brauer-Manin obstruction. The reason for the
condition on $K$ is that otherwise the curves (\ref{E8}) would admit complex
multiplication, and the latter acts on the $\sqrt{-3}$-Selmer groups; thus the
order of the latter would always be an odd power of 3, whereas it has to be
reduced to 9 for the method to work. (Here one factor 3 arises because of the
3-division points on the Jacobian defined over $\bQ(\sqrt{-3})$.) On the other
hand, in this argument we only need apply
Schinzel's Hypothesis to the single polynomial $X$, so that it can be replaced
by Dirichlet's theorem on primes in arithmetic progression.

All this relates to Question (A). For Question (B) the only known results
are for quadrics, for which see the remarks after Theorem \ref{T1}. It seems
reasonable to ask whether there is an analogous result for other kinds of
rational surfaces; this is another problem for which the first step should
probably be to use numerical search to generate a plausible conjecture. For
this purpose, one needs to examine a system with not too many parameters;
and this leads to the following question:
\bpr{Q9} For the surface $V$ given by $(\ref{E12})$ with the $a_i$ in $\bZ$,
is there a polynomial $P$ in the $|a_i|$ such that if $V$ is soluble in $\bZ$
then it has such a solution for which each summand is absolutely bounded by
$P$?
\epr

The ideal answer to Question (C) would be to provide a birational map
$V\rightarrow\bP^2$ defined over $\bQ$. However, it can be shown that such a
map exists for nonsingular cubic surfaces $V$ if and only if $V(\bQ)$ is not
empty and $V$ contains a divisor defined over $\bQ$ which consists of 2, 3
or 6 skew lines. (For Del Pezzo surfaces of degree 4, the second condition
must be replaced by the statement that $V$ contains a divisor defined over
$\bQ$ which consists of one or more skew lines.) For Ch\^{a}telet surfaces,
which have the form
\[ X_2(X_0^2-cX_1^2)=f(X_2,X_3) \]
where $c$ is a non-square and $f$ is homogeneous cubic, it is known that
there is a finite set of parametric solutions (each in 4 inhomogeneous
variables) such that each point of $V(\bQ)$ is represented by one of them,
though in an infinity of different ways.
But in general more than one parametric solution is needed, and parametric
solutions in only 2 variables cannot play a useful part in the process.
\bpr{Q21} Is there a larger class of cubic surfaces (ideally, the class of
all nonsingular cubic surfaces) for which analogous results hold?
\epr

The question of parametric solutions is linked to the idea of
\emph{R-equivalence}.
Let $V$ be a variety defined over $\bQ$; then R-equivalence is defined as the
finest equivalence relation such that two points given by the same parametric
solution are equivalent. Alternatively, it is the finest equivalence relation
such that for any map $f:\bP^1\rightarrow V$ and points $P_1,P_2$, all
defined over $\bQ$, the points $f(P_1)$ and $f(P_2)$ are equivalent. A good
deal is known about R-equivalence on cubic surfaces; in particular, it is
shown in [39] that the closure of an R-equivalence class in $V(\bA)$ is
computable, and that the closures of two R-equivalence classes are either
the same or disjoint.
\bpr{Q22} Is the set of R-equivalence classes on a nonsingular cubic surface
finite? Can there be two distinct R-equivalence classes which have the same
closure in $V(\bA)$?
\epr

\bigskip

\noindent 7. K3 \emph{surfaces and Kummer surfaces}. \newline
K3 surfaces are the simplest kind of variety about whose number-theoretic
properties very little is known; indeed they still present many unsolved
problems even
to the geometer. There are infinitely many families of K3 surfaces;
the simplest of them, and the only one which will be considered in the
present article, consists of all nonsingular quartic surfaces. An important
special type of K3 surfaces consists of the \emph{Kummer surfaces}, a phrase
which can carry either of two related meanings:
\begin{itemize}
\item The quotient of an Abelian surface $A$ by the automorphism $-1$; this
has 16 singular points corresponding to the 16 2-division points of $A$.
\item The nonsingular surface obtained by blowing up the 16 singular points
in the previous definition.
\end{itemize}
One advantage of Kummer surfaces in comparison
with general K3 surfaces is that for the
former it is easy to determine Pic$(\bar{V})$.

Some K3 surfaces contain one or more pencils of curves of genus 1, and these
pencils may even be of the kind discussed in the previous section;
but one should not
confine one's attention to K3 surfaces with this additional property. For
the time being, there is
merit in concentrating on diagonal quartics
\beq{E13} V:a_0X_0^4+a_1X_1^4+a_2X_2^4+a_3X_3^4=0, \end{equation}
because these contain few enough parameters to make systematic numerical
experimentation possible. However, the number theory of such surfaces may be
exceptional, because the geometry certainly is. Indeed
Pic$(\bar{V})$ has rank 20, which is the largest possible value for any
K3 surface, and it is generated by the classes of the 48 lines on $\bar{V}$;
moreover $V$ is a Kummer surface up to isogeny, and indeed is the Kummer
surface of $E\times E$ where $E$ is an elliptic curve which admits complex
multiplication. One consequence of this is that $V$ is rigid in the sense of
algebraic geometry.

There is an obvious map from $V$ to the quadric surface
\[ W:a_0Y_0^2+a_1Y_1^2+a_2Y_2^2+a_3Y_3^2=0. \]
If $a_0a_1a_2a_3$ is a square, each of the two families of lines on $W$ is
defined over the ground field, and each such line lifts to a curve of genus 1
on $V$; moreover the Jacobians of these curves have the form (\ref{E7}), so
that the methods of the previous section can be applied.

Martin Bright [7] has computed and tabulated Br$_1(V)/$Br$(\bQ)$ for all
$V$ of the form (\ref{E13}); it is necessary to do this by computer, because
there are 546 distinct cases. I had previously shown in [38] that
over $\bQ$ the Brauer-Manin obstruction is the only obstruction to the Hasse
principle in the most general case in which $a_0a_1a_2a_3$ is a square. (Most
general in this context means that none of the $\pm a_ia_j$ is a square and
$a_0a_1a_2a_3$ is not a fourth power.) It seems reasonable to hope that the
same property will still hold in all the cases for which $a_0a_1a_2a_3$ is a
square; but there are too many of them
to examine individually. On the other hand, there is strong numerical
evidence that when $a_0a_1a_2a_3$ is not a square the obstruction coming
from Br$_1(V)$
is not in general the only obstruction to the Hasse principle.
\bpr{Q14} What is the additional obstruction in this case?
\epr
One particularly interesting example is the surface
\beq{E9} X_0^4+2X_1^4=X_2^4+4X_3^4; \end{equation}
this has two obvious rational points, but appears to have no others.

\bigskip

\noindent 8. \emph{Density of rational points}. \newline
So far I have ignored Question (D). It differs from the others in that it is
not a birational question, but is associated with a particular embedding
of the variety $V$ in
projective space. For simplicity we work over $\bQ$. A point $P$ in $\bP^n$
defined over $\bQ$ has a representation $(x_0,\ldots,x_n)$ where the $x_i$ are
integers with no common factor; and this representation is unique up to
changing the signs of all the $x_i$. We define the \emph{height} of $P$ to be
$\max|x_i|$; a linear transformation on the ambient space multiplies heights
by numbers which lie between two positive constants depending on the linear
transformation. Denote by $N(H,V)$ the number of points of $V(\bQ)$ whose
height is less than $H$; then it is natural to ask how $N(H,V)$ behaves as
$H\rightarrow\infty$.
This is the core question for the Hardy-Littlewood method, which when it is
applicable is the best (and often the only) way of proving that $V(\bQ)$ is
not empty. In very general circumstances that method provides estimates of the
form
\[ N(H,V)={\mathrm{~leading~term~+~error~term}}. \]
The leading term is usually the same as one would obtain by probabilistic
arguments. But such results are only valuable when it can be shown that the
error
term is small compared to the leading term, and to achieve this the dimension
of $V$ needs to be large compared to its degree. The extreme case of this is 
the following theorem, due to Birch [3].
\bth{T3} Let $r_1,\ldots,r_m$ be positive odd integers, not necessarily all
different. Then there exists $N_0(r_1,\ldots,r_m)$ with the following property.
For any $N\geq N_0$ let $F_i(X_0,\ldots,X_N)$ be homogeneous polynomials with
coefficients in $\bZ$ and
$\deg F_i=r_i$ for $i=1,\ldots,m$. Then the $F_i$ have a common nontrivial
zero in $\bZ^N$.
\eth
The proof falls into two parts. First, the Hardy-Littlewood method is used to
prove the result in the special case when $m=1$ and $F_1$ is diagonal ---
that is, to show that if $r$ is odd and $N\geq N_1(r)$ then
\[ c_0X_0^r+\ldots+c_NX_N^r=0 \]
has a nontrivial integral solution. Then the general case is reduced to this
special case by purely elementary methods.
The requirement that all the $r_i$ should be odd arises from difficulties
connected with the real place; over a totally complex algebraic number field
there is a similar theorem for which the $r_i$ can be any positive integers.
\bpr{Q20} In Theorem $\ref{T3}$, can the condition that all the $r_i$ are odd
be replaced by the requirement on the $F_i$ that the projective variety given
by $F_1=\ldots=F_m=0$ has a nonsingular real point?
\epr

The Hardy-Littlewood method was designed for a single equation in which the
variables are separated --- for example, an equation of the form
\[ f_1(X_1)+\ldots+f_N(X_N)=c \]
where the $f_i$ are polynomials, the $X_i$ are integers, and one wishes to
prove solubility for all integers $c$, or all large enough $c$, or almost all
$c$. But it has also been applied both to several simultaneous equations and
to equations in which the variables are not separated. The following theorem
of Hooley [21] is much the most impressive result in this direction.
\bth{T4} Homogeneous nonary cubics over $\bQ$ satisfy both the Hasse
principle and weak approximation.
\eth

It appears that the Hardy-Littlewood method can only work for families for
which $N(H,V)$ is asymptotically equal to its probabilistic value; in
particular it seems unlikely that it can be made to work for families for
which weak approximation fails. Manin has put forward a conjecture about the
asymptotic density of rational solutions for certain geometrically interesting
families of varieties for which weak approximation is unlikely to hold: more
precisely, for Fano varieties embedded in $\bP^n$ by means of
their anticanonical divisors. For simplicity, we describe his conjecture only
for Del Pezzo surfaces $V$ of degrees 3 and 4.
To ask about $N(H,V)$ is now the wrong question,
for $V$ may contain lines $L$ defined over $\bQ$, and for any
line $N(H,L)\sim AH^2$ for some nonzero constant $A$. This is much greater
than the order-of-magnitude estimate for $N(H,V)$ given by a
probabilistic argument. For the latter suggests an estimate
$AH\prod(N(p)/(p+1))$, where the product is taken over all primes less than
a certain bound which depends on $H$. In view of what is said in \S3,
this product ought to be
replaced by something which depends on the behaviour of $L_2(s,V)$
near $s=1$. More precisely, the way in which the leading term in the
Hardy-Littlewood method is obtained suggests that here we should take
$s-1$ to be comparable with $(\log H)^{-1}$. Remembering the
Tate conjecture, this gives the right hand side of (\ref{E11}) as a
conjectural estimate for $N(H,V)$. But if this argument were valid, $L$
would contain more rational points than $V$, even though $V\supset L$.
Manin's way to resolve this absurdity is to study not $N(H,V)$ but $N(H,U)$,
where
$U$ is the open subset of $V$ obtained by deleting the 27 or 16 lines on $V$.
Manin conjectured that
\beq{E11} N(H,U)\sim AH(\log H)^{r-1} {\mathrm{~where~}} r
{\mathrm{~is~the~rank~ of~Pic}}(V); \end{equation}
and Peyre [26] has given a conjectural formula for $A$. (But note that there
exist
Fano varieties of dimension greater than 2 for which (\ref{E11}) is certainly
false; and it is not clear how Manin's conjecture should be modified to
cover such cases.) Various people have proved
this conjecture
for the cone $X_0X_1X_2=X_3^3$, and Rudge has sketched a proof for the
singular cubic surface
\[ X_0X_1X_2+X_0X_1X_3+X_0X_2X_3+X_1X_2X_3=0, \]
to which attention had been drawn by Birch.
\bpr{Q15} Are there nonsingular Del Pezzo surfaces of degree $3$ or $4$ for
which the Manin conjecture can be proved?
\epr
In the first instance, it would be wise to address this problem under rather
restrictive hypotheses about Pic$(V)$, not least because the Brauer-Manin
obstruction to weak approximation occurs in the conjectural formula for $A$
and therefore the problem is likely to be easier for families of $V$ for
which weak approximation holds. A one-sided estimate for one such family is
given in [35].

For Del Pezzo surfaces, the value of $c$ for which
$N(H,U)\sim AH(\log H)^c$ is defined by the geometry rather than by the
number theory, though that is not true of $A$. For other varieties, the
corresponding statement need no longer be true. We start with curves. For a
curve of genus 0 and degree $d$, we have $N(H,V)\sim AH^{2/d}$; and for a
curve of genus greater than 1 Faltings' theorem is equivalent to the
statement that $N(H,V)=O(1)$. But if $V$ is an elliptic curve then $N(H,V)\sim
A(\log H)^{r/2}$ where $r$ is the rank of the Mordell-Weil group. (For elliptic
curves there is a more canonical definition of height, which is invariant
under bilinear transformation; this is used to prove the result above.)

For pencils of conics, Manin's question is probably not the best one to ask,
and it would be better to proceed as follows. A pencil of conics is a surface
$V$ together with a map $V\rightarrow\bP^1$ whose fibres are conics. Let
$N^*(H,V)$ be the number of points on $\bP^1$ of height less than $H$ for
which the corresponding fibre contains rational points.
\bpr{Q16} What is the conjectural estimate for $N^*(H,V)$ and under what
conditions can one prove it?
\epr
It may be worth asking the same questions for pencils of curves of genus
1.

For surfaces of general type, Lang's conjecture implies that questions about
$N(H,V)$ are really questions about certain curves on $V$; and for Abelian
surfaces (and indeed Abelian varieties in any dimension) the obvious
generalisation of the theorem for elliptic curves holds. But K3 surfaces
pose new problems --- and not ones on which any practicable amount of
computation is likely to shed light. If $V$ is a K3 surface, then we have to
study not $N(H,V)$ but $N(H,U)$ where $U$ is obtained from $V$ by deleting the
curves of genus 0 on $V$ defined over $\bQ$, of which there may be an infinite
number. One can expect
that $N(H,U)\sim A(\log H)^c$ for some constants $A$ and $c$; and
it seems reasonable to hope that $c$ will be a half-integer. The surface
(\ref{E9}) suggests that we can have $c=0$, and it must be certain (though
perhaps difficult to prove) that $c$ can sometimes be strictly positive.
\bpr{Q17} Can the value of $c$ be obtained from the L-series $L_2(s,V)$?
\epr
\bpr{Q18} If $V$ is a Kummer surface obtained from the Abelian surface $A$,
is $c$ related to the rank of the Mordell-Weil group of $A$?
\epr


\bigskip

\begin{center} REFERENCES \end{center}

\noindent [1] A.Beauville, \textit{Complex Algebraic Surfaces} (2nd ed.,
Cambridge, 1996). 

\noindent [2] A.O.Bender and Sir Peter Swinnerton-Dyer, Solubility of certain
pencils of curves of genus 1, and of the intersection of two quadrics in
$\bP^4$, Proc. London Math. Soc (3) 83(2001), 299-329.

\noindent [3] B.J.Birch, Homogeneous forms of odd degree in a large number
of variables, Mathematika 4(1957), 102-105.

\noindent [4] S.J.Bloch, \textit{Higher regulators, algebraic $K$-theory and
zeta-functions of elliptic curves}, CRM Monograph Series 11 (AMS, 2000).

\noindent [5] S.J.Bloch and K.Kato, $L$-functions and Tamagawa numbers of
motives, in \textit{The Grothendieck Festschrift}, vol. I, pp. 333-400
(Birkhauser, 1990).

\noindent [6] A.Borel, Cohomologie de $SL_2$ et valeurs de fonctions
zeta aux points entiers, Ann. Sc. Norm. Pisa (1976), 613-636.

\noindent [7] M.Bright, Ph.D. dissertation, (Cambridge, 2002).

\noindent [8] M.Bright and Sir Peter Swinnerton-Dyer, Computing the
Brauer-Manin obstructions, Math. Proc. Cambridge Phil. Soc. (to appear).

\noindent [9] \textit{Hilbert's Tenth Problem: Relations with Arithmetic and
Algebraic Geometry} (ed. Jan Denef et al), Contemporary Mathematics, vol. 270.

\noindent [10] \textit{Mathematical Developments arising from Hilbert
Problems}, AMS Symposia in Pure Mathematics, Vol XXVIII (ed. F.E.Browder),
(Providence, 1976).

\noindent [11] J.W.S.Cassels, Second descents for elliptic curves, J. Reine
Angew. Math. 494(1998), 101-127.

\noindent [12] J-L.Colliot-Th\'{e}l\`{e}ne, J-J.Sansuc and Sir Peter
Swinnerton-Dyer, Intersections of two quadrics and Chatelet surfaces,
J. Reine Angew. Math. 373(1987), 37-107 and 374(1987), 72-168.

\noindent [13] J-L.Colliot-Th\'{e}l\`{e}ne, A.N.Skorobogatov and Sir Peter
Swinnerton-Dyer, Hasse principle for pencils of curves of genus one whose
Jacobians have rational 2-division points, Invent. Math. 134(1998),
579-650.

\noindent [14] J-L.Colliot-Th\'{e}l\`{e}ne and Sir Peter Swinnerton-Dyer,
Hasse principle and weak approximation for pencils of Severi-Brauer and
similar varieties, J. Reine Angew. Math., 453(1994) 49-112.

\noindent [15] \textit{Rational Points}, ed. G.Faltings and G. W\"{u}stholz
(3rd ed., Vieweg, 1992).

\noindent [16] J.Gebel and H.G.Zimmer, Computing the Mordell-Weil Group of
an Elliptic Curve over $\bQ$, in \textit{Elliptic Curves and Related Topics}
(ed. H.Kisilevsky and M. Ram Murthy), CRM Proceedings and Lecture Notes, vol 4,
pp 61-83 (Amer. Math. Soc., 1994).

\noindent [17] B.Gross, Kolyvagin's work on modular elliptic curves, in
\textit{L-functions and Arithmetic} ed. J.Coates and M.J.Taylor (Cambridge,
1991).

\noindent [18] B.Gross, W.Kohnen and D.Zagier, Heegner points and derivatives
of $L$-series II, Math. Ann. 278(1987), 497-562.

\noindent [19] B.Gross and D.Zagier, Heegner points and derivatives of
$L$-series, Invent. Math. 84(1986), 225-320.

\noindent [20] D.Harari, Obstructions de Manin "transcendantes",
in \textit{S\'{e}minaire de Th\'{e}orie des Nombres de Paris 1993-1994},
ed. S.David, (Cambridge, 1996).

\noindent [21] C.Hooley, On nonary cubic forms, J. Reine Angew. Math.,
386(1988), 32-98 and 415(1991), 95-165 and 456(1994), 53-63.

\noindent [22] W.W.J.Hulsbergen, \textit{Conjectures in Arithmetic
Algebraic Geometry}, (Vieweg, 1992).

\noindent [23] S.L.Kleiman, Algebraic cycles and the Weil conjectures,
in \textit{Dix expos\'{e}s sur la cohomologie des sch\'{e}mas},
ed. A.Grothendieck and N.H.Kuiper (North-Holland, 1968).

\noindent [24] V.A.Kolyvagin, Finiteness of $E(\bQ)$ and $\Sha(E/\bQ)$
for a class of Weil curves, Izv. Akad. Nauk SSSR 52(1988).

\noindent [25] B.Mazur, Rational isogenies of prime degree, Inv. Math.
44(1978), 129-162.

\noindent [26] E.Peyre, Hauteurs et mesures de Tamagawa sur les
vari\'{e}t\'{e}s de Fano, Duke Math. J. 79(1995), 101-218.

\noindent [27] S.Raghaven, Bounds for minimal solutions of Diophantine
equations, G\"{o}ttinger Nachr. (1975), 109-114.

\noindent [28] \textit{Beilinson's Conjectures on Special Values of
$L$-Functions}, ed. M.Rapaport, N.Schappacher and P.Schneider (Academic
Press, 1988).

\noindent [29] K.Rubin, Elliptic Curves with Complex Multiplication and
the Conjecture of Birch and Swinnerton-Dyer, in \textit{Arithmetic Theory
of Elliptic Curves} (ed. C.Viola), pp. 167-234 (Springer Lecture Notes
1716 (1999)).

\noindent [30] P.Salberger and A.N.Skorobogatov, Weak approximation for
surfaces defined by two quadratic forms, Duke J. Math. 63(1991), 517-536.

\noindent [31] J-P.Serre, Facteurs locaux des fonctions z\^{e}ta des
vari\'{e}t\'{e}s alg\'{e}briques (d\'{e}finitions et conjectures),
S\'{e}minaire Delange-Pisot-Poitou 1969/70, exp. 19.

\noindent [32] C.L.Siegel, Normen algebraischer Zahlen, (Werke, Band IV,
250-268).

\noindent [33] A.N.Skorobogatov, Beyond the Manin obstuction, Inv. Math.
135(1999), 399-424.

\noindent [34] A.N.Skorobogatov, \textit{Torsors and Rational Points},
(Cambridge, 2002).

\noindent [35] J.B.Slater and Sir Peter Swinnerton-Dyer, Counting points on
cubic surfaces I, Ast\'{e}risque 251(1998), 1-11.

\noindent [36] P.Swinnerton-Dyer, The Conjectures of Birch and
Swinnerton-Dyer, and of Tate, in \textit{Proceedings of a Conference on Local
Fields} (ed. T.A.Springer), Driebergen 1966, pp. 132-157 (Springer, 1967).

\noindent [37] Sir Peter Swinnerton-Dyer, Rational points on pencils of
conics and on pencils of quadrics, J. London Math. Soc. (2) 50(1994),
231-242.

\noindent [38] Sir Peter Swinnerton-Dyer, Arithmetic of diagonal quartic
surfaces II, Proc. London Math. Soc. (3) 80(2000), 513-544.

\noindent [39] Sir Peter Swinnerton-Dyer, Weak approximation and
$R$-equivalence on Cubic Surfaces, in \textit{Rational points on algebraic
varieties} (ed. E.Peyre and Y.Tschinkel) pp. 357-404 (Birkh\"{a}user, 2001).

\noindent [40] Sir Peter Swinnerton-Dyer, The solubility of diagonal cubic
surfaces, Ann. Scient. \'{E}c. Norm. Sup. (4) 34(2001), 891-912.

\noindent [41] J.T.Tate, On the conjectures of Birch and Swinnerton-Dyer
and a geometric analog, S\'{e}m. Bourbaki 306(1966).

\noindent [42] R.C.Vaughan, \textit{The Hardy-Littlewood method}, (2nd ed.,
Cambridge, 1997).

\end{document}

