\documentclass{article}
\title{Lecture 3}
\author{Kevin Buzzard
\thanks{The author would like to thank David Whitehouse for supplying
him with a copy of the notes for the lecture}
}
\usepackage{amsmath} % do we need this? Contains \projlim, for example.
%\usepackage{amssymb} % do we need this? Contains \nmid, for example.
%
\newcommand{\C}{\mathbf C}
\newcommand{\Cp}{\C_p}
\newcommand{\E}{\mathcal E}
\newcommand{\Ebar}{\overline{E}}
\newcommand{\F}{\mathbf F}
\newcommand{\Ftbar}{\overline{\F}_2}

\renewcommand{\O}{\mathcal O}
\newcommand{\OK}{\O_K}
\newcommand{\Q}{\mathbf{Q}}
\newcommand{\Qp}{\Q_p}
\newcommand{\Qt}{\Q_2}
\newcommand{\Qpbar}{\overline{\Q}_p}
\newcommand{\Qtbar}{\overline{\Q}_2}
\newcommand{\Z}{\mathbf{Z}}
%
\DeclareMathOperator{\Spec}{Spec}
%
\usepackage{amsthm}
\newtheorem*{theoremn}{Theorem}

% I guess those will have numbers. Use a * to kill the numbers.

\begin{document}
\maketitle % puts the title in, and the author if there is one.

\section{Maps between spaces of forms}
Recall that we have been thinking of modular forms as rules defined
on test objects $(E/R,\omega)$ or $(E/R,\omega,Y)$. Hence
if one has a ``natural'' map $F$ which sends a test object to
another test object, then $F$ induces a natural map between spaces of modular
forms: if $f$ is a modular form, then one can define a modular form
$F^*f$ as being the rule sending a test object $T$ to $f(F(T))$. One needs
to check that this rule satisfies the axioms, but if $F$ is sufficiently
natural then this kind of check should be very
straightforward.\footnote{One could in fact define ``natural'' as meaning
``such that this check works''!} We use the upper star
notation because one easily checks that the map on forms goes the other
way to the map $F$. We now give some concrete examples of this phenomenon.

\begin{itemize}
\item If $(E/R,\omega,Y)$ is a $\rho$-overconvergent test object,
then forgetting $Y$ gives us a classical test object $(E/R,\omega)$.
Hence if $f$ is a classical modular form, one can define a
$\rho$-overconvergent form as being the rule sending $(E/R,\omega,Y)$
to $f(E/R,\omega)$. This way we get a natural map from classical
forms to $\rho$-overconvergent forms.

\item if $r\in R_0$ and $\rho_1=r\rho_2$, and if $(E/R,\omega,Y)$ is
a $\rho_2$-overconvergent test object, then $(E/R,\omega,rY)$ is
a $\rho_1$-overconvergent test object, and hence $r$ gives us a natural
map from the space of $\rho_1$-overconvergent forms to the space of
$\rho_2$-overconvergent forms.
\end{itemize}
One very interesting collection of maps between spaces of forms,
namely the Hecke operators, do not quite fit into this framework, but
are only a mild generalisation of it, as we shall now see.

\section{Hecke operators}

If $(E/R,\omega)$ is a classical test object, and $l$ is a prime,
then for a finite locally-free subgroup scheme $C\subset E$
of order $l$ defined over $R$, we can form the quotient curve $E/C$
and we have a natural projection map $\pi:E\to (E/C)$, an isogeny
of degree~$l$. One can form the dual isogeny
$\pi^\vee:(E/C)\to E$. It is often true that $(E/C,(\pi^\vee)^*\omega)$
is another classical test object---the only troublesome point is that
the differential might vanish, but this will not happen if, for example,
$R$ is a $\Z[1/l]$-algebra, as then $\pi$ and its dual will both be etale.
Let us hence assume that $R$ is a $\Z[1/l]$-algebra.

There could be, in general, more than one subgroup of $E$ of order~$l$,
and hence we do not yet have a ``natural'' rule sending one test object
to another, as in the previous
section. However, we get around this difficulty by simply considering
\emph{all} subgroups at once! Before we make this rigorous, we recall
some facts about the group scheme $(\Z/l\Z)^2$, considered as an etale
group scheme over $\Z[1/l]$. This group scheme is essentially just
a copy of the abelian group $(\Z/l\Z)^2$ over each point of $\Spec(\Z[1/l])$,
and one can easily check that it has precisely $l+1$ locally free subgroups
of order $l$, corresponding to our usual intuition from group theory.
Let us label these subgroups $C_1$, $C_2,\ldots,C_{l+1}$. Note that
if $R$ is any $\Z[1/l]$-algebra then the base extensions $C_i/R$
are still subgroups of $(\Z/l\Z)^2/R$, and we shall refer to these
groups as $C_i$ for short.

Let~$f$ be a modular form of weight~$k$, defined over
a $\Z[1/l]$-algebra $R_0$. We will define a new modular form $T_lf$ as follows:
If $(E/R,\omega)$ is a test object, then $E[l]$ will be locally isomorphic,
in the etale topology, to $(\Z/l\Z)^2$. More concretely, this implies that
there will be a finite etale over-ring $R'\supset R$ such that over
$R'$, $E[l]$ becomes isomorphic to $(\Z/l\Z)^2$.
Choose such an isomorphism. Let $C_1$, $C_2,\ldots,C_{l+1}$ be the
corresponding $l+1$ subgroups of $E[l]\cong(\Z/l\Z)^2$,
and define $T_lf(E/R,\omega)=
l^{k-1}\sum_{i=1}^{l+1}f((E/C_i)/R',(\pi_i^\vee)^*\omega)$.
Here $\pi_i$ denotes the projection $E\to E/C_i$.

This definition has two subtle problems associated
to it, one of which I was not in fact aware of before the Arizona
Winter School, and I shall sketch how one gets around these problems. The first
is that we chose an isomorphism $E[l]\cong(\Z/l\Z)^2$ over $R'$. If $\Spec(R')$
is not connected then there is the issue that different isomorphisms
will yield different choices of $C_1,\ldots,C_{l+1}$. So one has to check
that different choices yield the same result. Fortunately, this is not
difficult to do, because one can reduce to the case of a local ring, and the
spectrum of a local ring is connected. We thank Bjorn Poonen for pointing
out this subtlety, and Brian Conrad for explaining how to get around it.

The second problem is that we extended our base from $R$ to $R'$,
and hence it looks like $T_lf(E/R,\omega)$ will be an element of $R'$
rather than $R$. One can use a generalisation of Galois theory,
or what the experts would call ``a descent argument'', to prove that
$T_lf(E/R,\omega)$ is in fact in~$R$.

The above discussion yields a map $T_l$ from the space of classical
weight~$k$ modular forms over $\Z[1/l]$ to itself, and also
a map $T_l$ from the space of $p$-adic modular forms to itself,
as long as $l\not=p$. In the $p$-adic setting there is also a very
important Hecke operator at~$p$, but its definition is slightly more
subtle and we shall come back to it later. As a brief summary of the
problem, what will happen is that for an elliptic curve defined over
a $p$-adic ring, it is frequently the case that not all subgroups
of order~$p$ are the same---one of them is more ``canonical'' than the
others. We can define a Hecke operator $U_p$ by quotienting out
by the $p$ non-canonical subgroups of order~$p$. We now make all
this more precise.

\section{A measure of supersingularity on an elliptic curve}

For simplicity now, let $K$ be a finite extension of~$\Q_p$,
Let $\O_K$ denote the integers of~$K$. There is a valuation
map $v:\O_K\backslash\{0\}\to\Q_{\geq0}$, normalised so
that $v(p)=1$.

Let $R$ denote $\O_K/p\O_K$. Note that $R$ may well not be the
residue field of $K$---in fact this is exactly the point: if $K$
is highly ramified, then $R$ will contain lots of nilpotent elements.
The valuation map above induces $v:R\backslash\{0\}\to[0,1)\cap\Q$,
with the property that $v(ur)=v(r)$ for all $u\in R^\times$.

Let $E/K$ be an elliptic curve with good reduction. 
By definition of ``good reduction'',
there is an elliptic curve $\E/\O_K$ with generic fibre~$E$.
Define $\Ebar/R$ to be the base change of $\E$ to~$R$. The
$R$-module $H^0(\Ebar,\Omega^1_{\Ebar/R})$ is projective of rank~1,
and hence free of rank~1, over~$R$. If $\omega$ is an $R$-basis for
this module, then by definition, $\omega$ is a non-vanishing
differential. Furthermore, such $\omega$ exist, and are unique up to
multiplication by an element of~$R^\times$. 

If $A$ denotes the Hasse invariant, then $A(\Ebar,\omega)\in R$ is
an element which is either equal to~0, or has a valuation which
is independent of choice of non-vanishing~$\omega$. Let us say
that $E$ is ``very supersingular'' if $A(\Ebar,\omega)=0$, and
that $E$ is ``not too supersingular'' otherwise.

Assume that $E$ is not too supersingular. Then $v(A(\Ebar,\omega))$
is independent of choice of $\omega$, and is a rational in $[0,1)$.
Define $v(E)$ to be this rational. By the definition of the Hasse
invariant, $v(E)=0$ iff $E$ has good
ordinary reduction. For completion, define $v(E)=0$ if $E$
has bad reduction.

This definition gives us another way of understanding the ``$Y$''
part of the definition of an overconvergent test object, in some
simple cases: if $R_0$ is the integers in a finite extension of $\Q_p$,
and $0\not=\rho\in R_0$ with $0\leq v(\rho)<1$, then for a test
object $(E/R_0,\omega,Y)$ we have $YE_{p-1}(E,\omega)=\rho$
and this implies that $v(E)\leq v(\rho)$. On the other hand,
if $(E/R_0,\omega)$ is a classical test object, then $Y$ will
exist making $(E/R_0,\omega,Y)$ a $\rho$-overconvergent test object
iff $v(E)\leq v(\rho)$, because if the inequality holds then one
can define $Y=\rho/E_{p-1}(E,\omega)$. 

More generally, if $R$ is an arbitrary $p$-adically complete $R_0$-algebra,
then a $\rho$-overconvergent test object defined over~$R$ can be thought
of, loosely speaking, as a family of elliptic curves $E$ all of
which have $v(E)\leq v(\rho)$. In fact, this can be made more rigorous,
as we are about to see.

\section{The rigid-analytic viewpoint}

This section is rather vague, because I did not want to get bogged down
with the details of the foundations of rigid analysis. The reader is
hence asked to take on board the fact that there is a good $p$-adic
analogue of the theory of Riemann surfaces, namely the theory of
rigid-analytic curves.

Let $N$ be an integer prime to~$p$.
The modular curve $X_1(N)$ parameterises (generalised) elliptic curves
equipped with a point of order~$N$. If $\E$ is the universal elliptic
curve over $X_1(N)$, then one can define a sheaf $\omega$ on $X_1(N)$
as being the pushforward of the differentials on $\E/X_1(N)$ (and being
careful at cusps). This sheaf is locally free of rank 1, and one can
think of a classical modular form as being a section of $\omega^{\otimes k}$.

The problem comes when one wants to start ``throwing away'' elliptic curves.
For example, let us try and consider only the ordinary locus of $X_1(N)$,
that is, let us consider $X_1(N)$ over, say, $\Qpbar$, and let us consider
the locus of points $X_1(N)^{\mathrm{ord}}$
which correspond to curves with good ordinary,
or multiplicative, reduction. This set contains infinitely many points,
as does its complement. Hence there is no way that this set can possibly
be the set of points of some kind of subvariety of $X_1(N)$, as any
non-trivial closed subvariety of a curve is finite, and any non-trivial
open subvariety has finite complement.

Fortunately, if one words over a complete base field like $\Q_p$ or $\C_p$,
then $X_1(N)^{\mbox{ord}}$ has the structure of a \emph{rigid-analytic
space}. What is happening here is that $X_1(N)^{\mathrm{ord}}$ is
some kind of $p$-adic analogue of a Riemann surface. The theory of
rigid analytic spaces is set up from scratch in the book ``non-Archimedean
analysis'' by Bosch, Guentzer and Remmert, and in several other places,
but the reader with less
patience can find a summary of the theory in Peter Schneider's article
in the 1996 Durham proceedings. Let us just think of these things
as being $p$-adic analogues of Riemann surfaces, and let us use the
theory of complex analytic geometry as a guide to what we can do. From
this viewpoint, $X_1(N)^{\mathrm{ord}}$ is an open subvariety of $X_1(N)$,
and it will inherit an analytic sheaf $\omega^{\mathrm{an}}$ of rank~1.
The theory of rigid spaces is precisely what
one needs to give a good geometric feel to the theory of $p$-adic
modular forms. For example, one can check that if $K$ is a finite
extension of~$\Qp$ with integers $\OK$ then the global sections
of $(\omega^{\mathrm{an}})^{\otimes k}$ over $K$
are precisely the $1$-overconvergent modular forms defined over $\OK$.

More generally, if $0\leq r<1$ is rational, one can define $X_1(N)_{\geq r}$
as $X_1(N)$ with all points corresponding to elliptic curves $E$ which
are either much too supersingular, or have $v(E)>r$, removed. Although
it is slightly dangerous to draw a picture of a $p$-adic Riemann
surface, one can think of these objects as looking rather like classical
Riemann surfaces with small discs removed. This is because the regions
of $X_1(N)$ corresponding to elliptic curves with supersingular reduction
are the preimages in the generic fibre of the supersingular points,
and one can easily be convinced that the pre-image of a smooth point in the
special fibre is a disc in the generic fibre (for example, consider the
projective line over $\C_p$: the pre-image of the origin in the special
fibre is $\{z\in\Cp:|z|<1\}$). 

One can easily analyse these so-called ``supersingular discs''. 
If one chooses a trivialisation of $\omega^{p-1}$ on each disc,
then the form $E_{p-1}$ gives a parameter on these discs, which can now
be thought of as open discs with radius~1. The
elliptic curves with $0<v(E)<1$ are the curves on the boundary of
these discs, and $v$ can be thought of as the valuation of the parameter.
The closed disc of radius $1/p$ with centre the zero of $E_{p-1}$
is the region consisting of elliptic curves which are ``much too
supersingular''. The space $X_1(N)_{\geq r}$ corresponds to $X_1(N)$
with open discs radius $p^{-r}$ removed.

Again, one has the powerful geometric definition of a $\rho$-overconvergent
form of weight~$k$ over $K$, a finite extension of $\Q_p$:
it is a section of $(\omega^{\mathrm{an}})^{\otimes k}$
on $X_1(N)_{\geq r}$, where $r=v(\rho)$. Note that
if we stick to $\rho$ with $v(\rho)<1$ then the definition does not
even depend on a choice of lifting of the Hasse invariant, and in particular
this method gives us a means of avoiding the thorny problems associated
with lifting the Hasse invariant to characteristic zero in many cases
of interest---one simply has to lift the Hasse invariant on each
supersingular disc, which is possible even if $p$ is small.

\section{Canonical subgroups and the $U$ operator}

We now come back to the Hecke operator at~$p$. We start with a
specific example which the author finds very illuminating, because
it really shows a concrete example of the canonical subgroup of an
elliptic curve.

If $a\in\Qtbar$ with $|a|\leq1$ then define the elliptic curve
$$E_a:y^2+y+axy=x^3+x^2.$$
One can reduce this curve mod the prime above~2, and there are
two cases: if $|a|=1$ then $E_a$ reduces to the curve
$y^2+y+\bar{a}xy=x^3+x^2$,
which is an ordinary elliptic curve over $\Ftbar$. On the other hand,
if $|a|<1$ then $E_a$ reduces to $y^2+y=x^3+x^2$, which is supersingular.

Let's put this curve into canonical form: define $Y=y+
\frac{1}{2}\left(1+ax\right)$
and the equation for $E_a$ becomes $Y^2=f(x)$, where
$$f(x)=x^3+\left(\frac{a^2}{4}+1\right)x^2+\frac{a}{2}x+\frac{1}{4}.$$
The points of order~2 on~$E_a$ correspond to the roots of~$f(x)$.
What are the valuations of these roots? This is easy to establish
via the theory of the Newton Polygon.

If $|a|=1$ then the valuations of the coefficients of $f(x)$
are $0,-2,-1,-2$ respectively, and hence $f$ has one root with valuation $-2$
and two roots with valuation~0. The root with valuation $-2$
is of course the one that reduces to the point at infinity in the
reduction map, and indeed one expects exactly one non-zero point
to have this property because the 2-torsion in generic fibre has order~4,
and the 2-torsion in the special fibre has order only~2.

In the supersingular reduction case, the special fibre has no 2-torsion
at all. However, if $|a|=1-\epsilon$ with $\epsilon$ small, then
a similar Newton polygon argument shows that one of the roots of $f$
has valuation $-2+2\epsilon$ and the other two have valuation $-\epsilon$.
All three roots have negative valuation, as expected, but one sticks
out like a sore thumb. This point generates the so-called ``canonical
subgroup'' of $E_a$.

Finally, if $|a|$ is very small, then all three roots have valuation
$-2/3$ and it is hard to distinguish between them in any canonical
manner.

One should think of $a$ as being a function on $X_0(1)$, with $|a|<1$
on the supersingular locus and $|a|\geq1$ on the ordinary locus. 
The area where $|a|=1-\epsilon$ with $\epsilon$ small then corresponds
to the region near the boundary of the supersingular disc, and the
example shows that for elliptic curves near the boundary of the disc,
even though they have supersingular reduction, they still have
a canonical subgroup of order~2.

This example is a special case of the following phenomenon (whose proof
is just a long elaboration of what we have just seen above):

\begin{theoremn} If $K$ is a finite extension of $\Q_p$ and $E/K$
is an elliptic curve with $v(E)<\frac{p}{p+1}$ then $E$ has a canonical
subgroup of order~$p$. Furthermore, this canonical subgroup varies
smoothly as $E$ varies smoothly, and hence can be defined for a family
of elliptic curves over a $p$-adic base.

\end{theoremn}

\end{document}

