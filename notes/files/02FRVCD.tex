\centerline{Arizona Winter School 2002}
\centerline{Fernando Rodriguez Villegas}
\centerline{Course and Project Description}
\bigskip
The Mahler measure of a non-zero Laurent polynomial 
 $P \in {\bf C}[x_1,x_1^{-1},\cdots,x_N,x_N^{-1}]$ is defined by
$$
m(P) = \int_0^1 \cdots \int_0^1 \, \log \left| P(e^{2\pi i \theta_1}
\, , \ldots , \; 
      e^{2\pi i\theta_n}) \right| d\theta_1 \cdots d\theta_n .
$$
  This quantity arises naturally, for example, in the computation of
  heights of subvarieties of tori.

     In these lectures I will describe how the Mahler measure is
  related to the regulator of the Bloch--Beilinson conjectures.
  Concretely, we will discuss what is the theoretical framework behind
  identities like the one discovered by Smyth
$$  
   m(x + y + 1) = L'(\chi,-1), 
$$
  where $\chi$ is the quadratic Dircihlet character of conductor 3. Our
  tour will involve K-theory, the dilogarithm and hyperbolic
  3-manifolds.

\bigskip\noindent
{\bf Problems:}\medskip
\item{1.} Let $E= X_1(11)$, an elliptic curve defined over $\bf Q$ of
     conductor 11 with minimal Weierstrass model $y^2+y=x^3-x^2$.
     Find two rational functions $f$ and $g$ on $E$ such that the
     symbol $\{f,g\}$ is a non-torsion integral element of $K_2(E)$
     (See the article by D. Ramakrishnan, "Regulators, Algebraic
     Cycles, and Values of $L$-functions", in the book Algebraic
     K-theory and Algebraic Number Theory, Contemporary Mathematics
     83, pp. 183--310, (1989), Amer. Math. Soc. Providence,
     R.I. (M. R. Stein and R. K. Dennis eds.).)

\item{2.} Let 
$$\eqalign{
        P &= (x+y+1)(x+1)(y+1)+xy \cr
        Q&= y^2+(x^2+2x-1)y+x^3.}
$$
     Prove that
$$
         m(P) = 7/5 m(Q)
$$

\end
