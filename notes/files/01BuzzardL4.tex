%&LaTeX

\documentclass{article}%[12pt,twoside,a4paper]{article}

\usepackage{amsmath,amsopn,latexsym,amssymb,amscd,amsthm}

\setlength{\oddsidemargin}{0in}

\setlength{\evensidemargin}{0in}

\setlength{\textwidth}{6.5in}

\setlength{\topmargin}{-.20in}

\setlength{\headheight}{0.1in}

\setlength{\headsep}{0in}

\setlength{\textheight}{8.4in}

\setlength{\parindent}{0.1in}

\author{K. Buzzard}

\title{p-adic Modular Forms}

\date{3/13/01}

\newtheorem{theorem}{Theorem}

\newtheorem{prop}[theorem]{Proposition}

\newtheorem{lem}[theorem]{Lemma}

\newtheorem{cor}[theorem]{Corollary}

\newtheorem{rmk}{Remark}

\newtheorem{thm}{Theorem}

\newtheorem{Def}[theorem]{Definition}
\newcommand{\bw}{{\bf w}}

\newcommand{\br}{{\bf R}}

\newcommand{\bO}{{\bf 0}}

\newcommand{\bx}{{\bf x}}

\newcommand{\by}{{\bf y}}

\newcommand{\BZ}{{\mathbb{Z}}}

\newcommand{\BP}{{\mathbb{P}}}

\newcommand{\BQ}{{\mathbb{Q}}}

\newcommand{\BF}{{\mathbb{F}}}

\newcommand{\BB}{{\mathbb{B}}}

\newcommand{\BA}{{\mathbb{A}}}

\begin{document}

\maketitle

Let $E/R/R_0$ be an elliptic curve over an $R_0$-algebra $R$, where
$R_0=\mathcal{O}_K$ with

$[K:\BQ_p]<\infty$. Now consider $E/K$, then we have two cases:
\begin{equation}
v(E)\in
\begin{cases}\text{not defined} & \text{if $E$ is very supersingular}\\
\text{[0,1)}\cap\BQ & \text{otherwise}
\end{cases}
\end{equation}

\begin{thm}(\text{Katz-Lubin})\,\, If 

\begin{equation}
v(E)<
\begin{cases}\frac{p}{p+1} & \text{if $p\geq5$}\\
\frac{p}{2(p+1)} & \text{if $p=3$}\\
\frac{p}{4(p+1)} & \text{if $p=2$}
\end{cases}
\end{equation}

then $E$ has a "canonical" subgroup of ord=p.

\end{thm}



\begin{rmk}

$v(E)=0\Leftrightarrow$ $E$ has ordinary reduction, and then the canonical
subgroup is

just the kernel of the reduction map on its $p$-torsions.

\end{rmk}



Assume $v(\rho)<c_p$, where $c_p$ denotes the number on the right of (2)
corresponding to
different $p$'s. If $(E/R,\omega,Y)$ is a $\rho$-overconvergent test object,
then $v(E_K)\leq v(\rho)<c_p$.
So $E$ has a canonical subgroup $H$, and $(E/R,\omega,H)$ is a classical
test object 
plus a subgroup of order $p$. A rule on these objects is a classical
modular form of level $p$. Hence we get a map from classical modular forms
of level $p$
over $K_0$ to $\rho$-overconvergent forms of level 1.
So we also have a $U_p$ operator acting on the $\rho$-overconvergent forms.
If $f$ is a
$\rho$-overconvergent, then
%$$ U_p(f)(E/K,\omega,Y)=\sum_{{C\subset E}\genfrac \text{C cyclic of order
%p}}
%$$

\begin{rmk}
Let $E/K$ have $v(E)<c_p$, and $H$ be the canonical subgroup, then\\
(1) If $C$ is a subgroup of order $n$ with $(n,p)=1$ then $v(E/C)=v(E)$,\\
(2) If $C$ is not canonical then $v(E/C)=\frac1pv(E)$,\\
(3) If $v(E)<\frac1pc_p$ then $v(E/C)=pv(E)$, so in fact $U_p$ maps
$\rho$-overconvergent forms to $\rho^P$-overconvergent forms.
\end{rmk}
\begin{Def}
$$\mathbb{M}_k(K_0,\rho)=(\rho-{\rm overconvergent\: forms\: of\: weight\:
{\it k}\: defined\: over\:} R_0)\otimes K_0.$$ Then $\mathbb{M}_k(K_0,\rho)$
is a $p$-adic Banach space over $K_0$.
\end{Def}

As the remark indicates, we will have Hecke operators $T_l$ for $l\neq p$
acting on $\mathbb{M}_k(K_0,\rho)$, and $U_p$:
$\mathbb{M}_k(K_0,\rho)\to\mathbb{M}_k(K_0,\rho^p)$.

While at the same time there is a natural inclusion
$$\mathbb{M}_k(K_0,\rho^p)\longrightarrow\mathbb{M}_k(K_0,\rho)$$
where $v(\rho)<\frac1pc_p$.


Hence we get a map 
$$U_p:\mathbb{M}_k(K_0,\rho)\longrightarrow\mathbb{M}_k(K_0,\rho)$$


One can also get $U_p(\sum a_nq^n)=\sum a_{np}q^n$.



\begin{rmk}
$T_l$'s are continuous. $U_p$ is even better than that! Let $V$ be a big
infinite
dimensional $p$-adic Banach space, and assume $e_1,e_2,\ldots$ is a
countable Banach basis
of $V$. Then every $v\in V$ can be  written uniquely as 
$$v=\sum a_ie_i,\text{with $a_m\to0$, $a_n\in K_0$}$$
Let $T:V\to V$ be a continuous operator, and $T(e_i)=\sum c_{ji}e_j$. So
$c_{ji}$ is
the matrix of $T$ with respect to the basis. Then the queation is: does this
matrix have
a trace? Of course one cannot expect an affirmative answer in general as the
identity
matrix has no trace.\\
But the operator $T:e_i\to p^ie_i$ of $V$ has a trace=$\sum
p^i=\frac{p}{1-p}$.
\end{rmk}

Now denote $\mathcal{L}(V,V)$=continuous linear maps:$V\to V$.
$\mathcal{L}(V,V)$ inherits a norm from $V$. Let $F$ be the subspace
consisting of the maps whose image is finite dimensional. We define compact
operators to be the closure of these $F$'s.

Compact operators have traces, and even better, they have a spectral theory.
Now say $C$ is a compact linear operator, i.e. $C=\lim_{n\to\infty}C_n$,
where $C_n:V\to V$
have finite dimensional images. Put 
$$P_n(X)=\text{det}(I-XC_n)=1-t_nX+\dots+(-1)^n\text{det}(C_n)X^n$$

then $P_n$'s converge to a power series $P\in K_0[[X]]$ called the
characteristic power
series of $C$.

{\bf Example:}Let $C_n=\begin{pmatrix}a & 0\\ 0 & 1\end{pmatrix}$, $C=\lim
C_n$.
Then 
$$P_n(X)=\prod_{i=1}^{n}(1-p^iX)$$
therefore 
$$P(X)=\prod_{i=1}^{\infty}(1-p^iX)\in K_0[[X]]$$
and $P(x)$ converges ofor any $x\in K_0$.

\vspace{.2in}
Now we have a very nice result
\begin{thm}
If $v(\rho)\in (0,\frac1pc_p)$, then
$U_p:\mathbb{M}_k(K_0,\rho)\longrightarrow\mathbb{M}_k(K_0,\rho)$ is
compact.
\end{thm}

\vspace{0.2in}
{\bf Re-interpretation of G-M:} Fix $\rho$ such that $0<v(\rho)<\frac1pc_p$.
Recall that
$M_k(\Gamma_0(p),K_0)$ denotes the classical modular forms with weight $k$
of level $p$
over $K_0$. Then we have a $U_p$-covariant linear injection
$$M_k(\Gamma_0(p),K_0)\longrightarrow\mathbb{M}_k(K_0,\rho)$$
 
$M_k(\Gamma_0(p),K_0)=(\text{old part})\oplus(\text{new part})$. $U_p$ acts
differently
on these two parts: \\
(1) if $f\in$(old part), then $U_p(f)=a_pf$ and $U_p$ has eigenvalues as
roots of
$X^2-a_pX+p^{k-1}$, both of which have valuation$\leq k-1$,\\
(2) if $f\in$(new part), then $U_p$ has eigenvalues $\pm p^{\frac{p-2}2}$.
Therefore
if $\lambda$ is a $U_p$-eigenvalue on the classical forms, then
$v(\lambda)\leq k-1$.
The converse is almost true!

\begin{thm}[Coleman]
Assume $f\in\mathbb{M}_k(K_0,\rho)$ is an eigenform for $U_p$, $T_l$, and
the $U_p$-eigenvalue
is $\lambda$. If $v(\lambda)<k-1$ then $f\in$ the image of
$M_k(\Gamma_0(p),K_0)$.
\end{thm}

\noindent {\bf Definition.} $v(\lambda)$ is called the slope of the
overconvergent form $f$.

Hence one can retrieve classical forms as being "overconvergent forms of
small slope".

\noindent {\bf Gouvea-Mazur Conjecture.} Let $k\in 2\BZ$, $\alpha\in\BQ$, 
$\mathbb{M}_k(K_0,\rho)$, and $d(k,\alpha)=\sharp\{$eigenvalues of $U_p$
with valuation
$\alpha\}$. Then $k_1\equiv k_2$ (mod $(p-1)p^m$), for $m\geq\alpha$, will
imply that
$d(k_1,\alpha)=d(k_2,\alpha)$.

\vspace{.2in}
\begin{thm}[Coleman]
If $P_k(X)$=char power series of $U_p$ acting on $\mathbb{M}_k(K_0,\rho)$,
then
$P_k$ varies analytically with $k$.
\end{thm}

This theorem implies that $d(k,\alpha)$ is a "locally constant" function of
$k$.

\begin{prop}
If $k_1\equiv k_2$ (mod $(p-1)p^m$), and $\alpha<O(\sqrt{m})$, then
$d(k_1,\alpha)=d(k_2,\alpha)$.
\end{prop}
 
\vspace{.2in}
\noindent {\bf Example of the Spectrum of $U_p$.} 

Let's seek the structure of $U_2$ on $\mathbb{M}_0(K_0,\rho)$ (i.e. $k=0,
N=1$). Let
the char power series of $U_2$ be 
$$\sum_{n\geq0}a_nX^n=\prod_{i\geq0}(1-\lambda_iX).$$
The question is: what are the valuations of $\lambda_i$?

Inspired by a method of Kilford, we find that:

\begin{thm}(Buzzard, Calegari)
The valuations are 3,7,13,15,17,$\ldots$, where the ith term is given by
$$1+2v_2\left(\frac{(3i)!}{i!}\right).$$
\end{thm}
\begin{proof}
Let's write down a basis for $\mathbb{M}_0(K_0,\rho)$ (the basis depends on
$\rho$ although the characteristic p.s. of $\rho$ does not), say,
$$1,\alpha f,\alpha^2 f^2,\alpha^3 f^3,\cdots$$
where 

$$f=\frac{\Delta(q^2)}{\Delta(q)}=q+24q^2+\cdots$$
and $\alpha=\alpha(\rho), \alpha\in\bar{\mathbb{Q}}_2,|\alpha|<1$.



The matrix of $U_2$ is: 
$$U_2(f^m)=\sum_{n=\lceil\frac{m}2\rceil}^{2m}s_{m,n}f^n$$

where
$$s_{m,n}=2^{8n-4m-1}\cdot3m(m+n-1)!/(2n-m)!(2m-n)!$$
Write $U_2=A\cdot B$, where $A$ is lower triangular, $B$ is upper
triangular, with 1's on both diagonals. Actually we can compute the entries
$A_{ij}$ and $B_{ij}$. 

Now let $A=C\cdot D$ with $D$ diagonal, then 
$$D_{ii}=2^{1+2v((3i)!/i!)}$$
Once we  take $\alpha=2^6$: it changes $C_{ij}$ and $B_{ij}$ by
$2^{6(j-i)}$. Then the following lemma  concludes the proof. \end{proof} 

\begin{lem}
After  making the change if $C\equiv B\equiv\text{Id}\hspace{.1in}\text{mod
2}$, then the slopes of the characteristic power series of $U_2$ and $D$ are
the same.
\end{lem}


\end{document}


