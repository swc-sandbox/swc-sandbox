\documentclass{article}

\usepackage{amssymb}

\begin{document}



\Large
\begin{center}
\underline{\bf {A History of Interactions between}}

\underline{\bf{ Logic and
Number Theory}}

\vspace{3mm}
A. Macintyre

\end{center}
\normalsize



\section{Outline}

Interactions between logic and number theory have almost 
always
involved \underline{definability}.



In an earlier period the relevant logical component was 
recursion
theory (decidability and undecidability). For ${\mathbb Z}$ 
the
central issue was Hilbert's 10th Problem, and the central 
result
is that recursively enumerable relations on ${\mathbb Z}$ are
existentially definable. The highpoint of definability theory 
in
${\mathbb Q}$ remains Julia Robinson's, that ${\mathbb Z}$ is
$\Pi_3$-definable in ${\mathbb Q}$. Whether ${\mathbb Z}$ is
existentially definable in ${\mathbb Q}$ is unknown (if it 
is,
Hilbert's 10th Problem for ${\mathbb Q}$ is undecidable).



Recursion theory is thus very relevant for the logic of
\underline {global} fields and their rings of integers.



In contrast, model theory is much more relevant for the logic 
of
local fields, and for those areas of number theory with a
geometric aspect.



The locally compact completions of number fields have all
undergone  fruitful model-theoretic analyses. Thus Tarski 
(1930's)
obtained the classical results on definitions in ${\mathbb 
C}$ and
${\mathbb R}$, while not till the 1960's did Ax-Kochen-Ersov 
obtain
analogous results for p-adic fields (and for many Henselian
fields). The completions are all decidable, but nowadays one 
gives
more importance to the definability aspect of the above 
analyses.
One obtains normal forms for definitions, links between the
geometry of the set and the form of its definition, and 
various
uniformities for number of connected components, and in 
${\mathbb Q}_p$ for the form of various integrals (cf.~Loeser's course).



In the 1980's the ring of all algebraic integers was 
shown
(via a local-to-global principle involving earlier work in
algebraically closed fields with valuation) to have a very 
clear
definability theory, and in particular to have the analogue 
of
Hilbert's 10th Problem decidable.


In the 1960's and 70's there were several developments in
\underline{pure} model theory that led some time later to
interactions with number theory. The first was the work of
Robinson school on model completion, existentially closed
structures, and forcing methods. The theories of the 
completions
${\mathbb C}$, ${\mathbb R}$, ${\mathbb Q}_p$ all arise 
naturally in
this settings. But other \underline{theories} emerge, 
without
natural models, but which were to be key components in
significant interactions in the 1990's. One is the theory of
differentially closed fields (to be involved in the Mordell-
Lang
conjecture, cf.~the Pillay-Scanlon course), and another (not
discovered till 1990) is the theory ACFA of generic 
automorphisms
(to be involved in the ``logical" approach to the Manin-
Mumford
conjecture). Other theories of this type relate to the 
lifting of
Frobenius to the Witt vectors.



The other model-theoretic development, certainly deeper qua 
model
theory, originated with Morley's (1965) work initiating
model theoretic stability. The 1965 paper gave a suggestive
topological setting for first-order definability,and 
initiated a
systematic study of general notions around the geometrical 
ideas
of \underline{dimension} and \underline {independence}. Although the
ensuing \underline{stability theory} applies only to ${\mathbb C}$
among the completions of number fields, latter day ``local" 
versions
of it have been involved in most recent interactions of logic 
and
number theory (cf.~Pillay-Scanlon).



The other turning point in the 60's was Ax's work on the
elementary theory of finite fields,where logic was seen to
interact with Weil's Riemann Hypothesis for curves, and with
Cebotarev's Theorem. The new theories could be construed as
completions of Robinson type, and their definability was
suggestively analyzed in terms of \underline{Galois
Stratification} (cf.~Loeser's lectures). Moreover, one was 
soon led
to model theoretic questions about \underline{absolute Galois
groups}, and those are related to a vision of Grothendieck 
(see
Pop's lectures). (This is by no means the only case where 
ideas of
Grothendieck, the ``logician" of Bourbaki, have had fundamental
logical content.)



Already in Ax-Kochen-Ersov one had seen the power of the 
impulse ``Let $p$ go to 0."  After Ax's work one had richer environments 
for
this idea. In particular, one could gradually approach the 
study
of the Frobenius $x^p$ as $p$ tends to 0. Here one makes 
essential
contact with the Weil Conjectures and the cohomological 
methods
used in their proof. Old themes of Robinson, on bounds in the
theory of ideals in polynomial rings, reappear in the wider
setting of Intersection Theory and Weil Cohomology, and 
relate to
the Grothendieck Standard Conjectures.


Another grand conjecture, that of Schanuel on transcendence 
of
values of the complex exponential, has recently begun to 
interact
with logic. It was first seen in connection with the 
decidability
of the real and p-adic exponentials, and more recently in a
profound definability-theoretic study by Zilber of the
\underline{undecidable} complex exponential. Zilber's work
interacts naturally with diophantine geometry, and with old 
work of
Ax (cf.~Pillay-Scanlon).
 

\section{Structure of lectures}

\begin{enumerate}

\item

Definability
in the fields ${\mathbb C}$, ${\mathbb R}$ and ${\mathbb Q}_{p}$.

Uniformities, with special
reference on those in $p$.
\item
Model theory.
 New theories. Ax's work.
\item
From pseudofinite to ACFA. Galois groups and logic. Manifestations of
Frobenius. 
\item
Analytic aspects. Schanuel's Conjecture.
\end{enumerate}

\section{Prerequisites}
Basics of first-order logic, algebraic geometry, and number
theory. (Roughly as for Poonen's course).

\section{Project}
The completions of number fields are naturally united in
the adele construction. 25 years ago Weispfenning gave a first analysis
of definability in the adeles, in the spirit of Feferman-Vaught
Theorems. In view of the deepening of our understanding of definability
in the interim (e.g., in the work of Denef and Loeser),  it seems natural to
go back and write an up-to-date account, paying attention to
measure-theoretic and analyic uniformities. For example, what exactly do
the model theoretic uniformities for $p$-adic integrals mean in an adelic
setting? The goal is to give a Feferman-Vaught analysis for the analytic
structure of the adeles.

I am currently having trouble tracking down the reference for
Weispfenning's original work, but am confident of locating it before
long.

\begin{thebibliography}{10}

\bibitem{fj}
Fried, Michael D., and Jarden, Moshe 
{\em Field arithmetic}.
\newblock Ergebnisse der Mathematik
   und ihrer Grenzgebiete (3) [Results in Mathematics and Related Areas
(3)], 11.
   Springer-Verlag, 1986.
\bibitem {m1}  
   Macintyre, Angus 
   {\em Twenty years of $p$-adic model theory}. 
   \newblock Logic
colloquium
   '84 (Manchester, 1984), 121--153, Stud. Logic Found. Math., 120,
North-Holland,
   Amsterdam, 1986.
   \bibitem{mw1}
Macintyre, Angus; Wilkie, A. J. {\em On the decidability of the real
exponential
   field}. 
   \newblock Kreiseliana, 441--467, A K Peters, Wellesley, MA, 1996.
   \bibitem{ma}
Marker, David Model Theory:An Introduction.
\newblock Graduate Texts in Mathematics
217, Springer Verlag, 2002.
\bibitem{hand}
   {\em Handbook of mathematical logic}. Edited by Jon Barwise. With the
   cooperation of H. J. Keisler, K. Kunen, Y. N. Moschovakis and A. S.
Troelstra. 
\newblock Studies in Logic
   and the Foundations of Mathematics, Vol. 90. North-Holland Publishing
Co., Amsterdam-New
   York-Oxford, 1977.

\end{thebibliography}





\end{document}
