\documentclass[12pt,a4paper]{article}
\usepackage{amsmath}
\usepackage{amsfonts}
\usepackage{amssymb}
\long\def\proof#1{\removelastskip\vskip\baselineskip\relax\noindent{\it
Proof\if!#1!\else\ \ignorespaces#1\fi.\ }\ignorespaces}
\newcommand{\nullset}{\varnothing}
\newcommand{\legsm}[2]{\mbox{$\left(\frac{#1}{#2}\right)$}}
\newcommand{\leg}[2]{\mbox{$\left(\dfrac{#1}{#2}\right)$}}
\newcommand{\isom}{\simeq}
\newcommand{\LR}{\longrightarrow}
\newcommand{\gothr}{{\mathfrak r}}
\newcommand{\m}{{\mathfrak m}}
\newcommand{\n}{{\mathfrak n}}
\newcommand{\SQ}{SQ}
\newcommand{\odd}{{\rm odd}}
\newcommand{\qq}{Q_2(K)}
\DeclareMathOperator{\disc}{disc}
\DeclareMathOperator{\Ker}{Ker}
\renewcommand{\Im}{\mbox{\rm Im}}
\renewcommand{\Re}{\mbox{\rm Re}}
\DeclareMathOperator{\Coker}{Coker}
\DeclareMathOperator{\diag}{diag}
\DeclareMathOperator{\sign}{sign}
\DeclareMathOperator{\End}{End}
\DeclareMathOperator{\lcm}{lcm}
\DeclareMathOperator{\Gal}{Gal}
\DeclareMathOperator{\rk}{rk}
\DeclareMathOperator{\Hom}{Hom}
\DeclareMathOperator{\Tr}{Tr}
\DeclareMathOperator{\Res}{Res}
\DeclareMathOperator{\Art}{Art}
\DeclareMathOperator{\St}{St}
\DeclareMathOperator{\GL}{GL}
\DeclareMathOperator{\PGL}{PGL}
\DeclareMathOperator{\SL}{SL}
\DeclareMathOperator{\PSL}{PSL}
\DeclareMathOperator{\Aut}{Aut}
\DeclareMathOperator{\NO}{{\cal N}}
\DeclareMathOperator{\NCO}{{\lceil{\cal N}\rceil}}
\DeclareMathOperator{\N}{{\NO\!}}
\DeclareMathOperator{\NC}{{\NCO\!}}
\DeclareMathOperator{\CR}{{\cal R}}
\newcommand{\op}{{\rm (}}
\newcommand{\cp}{\/{\rm )\,}}
\newcommand{\cps}{\/{\rm )}}
\newcommand{\Nc}{{\N\c}}
\newcommand{\abs}[1]{\left\lvert#1\right\rvert}
\newcommand{\norm}[1]{\left\lVert#1\right\rVert}
\newcommand{\pa}{\partial{\hbox{}}}
\newcommand{\pd}[2]{\dfrac{\partial#1}{\partial#2}}
\newcommand{\psmm}[4]{\left(\begin{smallmatrix}{#1}&{#2}\\{#3}&{#4}\end{smallmatrix}\right)}
\newcommand{\dpk}[1]{\left\lfloor\dfrac{#1}{p^k}\right\rfloor}
\newcommand{\ddk}[1]{\left\lfloor\dfrac{#1}{2^k}\right\rfloor}
\newcommand{\ov}[1]{\overline{\vphantom{T}#1}}
\newcommand{\Q}{{\mathbb Q}}
\newcommand{\Z}{{\mathbb Z}}
\newcommand{\R}{{\mathbb R}}
\newcommand{\F}{{\mathbb F}}
\newcommand{\C}{{\mathbb C}}
\renewcommand{\P}{{\mathbb P}}
\newcommand{\X}{{\cyr X}}
\renewcommand{\v}{v_{\mathfrak p}}
\newcommand{\z}{{\zeta_\ell}}
\newcommand{\fsubp}{{\mathfrak f}_{\rm p}}
\newcommand{\ba}{{\mathbf a}}
\newcommand{\bb}{{\mathbf b}}
\newcommand{\bh}{{\mathbf h}}
\newcommand{\bm}{{\mathbf m}}
\newcommand{\br}{{\mathbf r}}
\newcommand{\bs}{{\mathbf s}}
\newcommand{\bv}{{\mathbf v}}
\newcommand{\bx}{{\mathbf x}}
\newcommand{\by}{{\mathbf y}}
\newcommand{\bighat}{\hat}
\renewcommand{\th}{\theta}
\newcommand{\la}{\lambda}
\newcommand{\om}{\omega}
\newcommand{\Om}{\Omega}
\newcommand{\oa}{\omega_1}
\newcommand{\ob}{\omega_2}
\newcommand{\D}{\Delta}
\newcommand{\tors}{{\rm tors}}
\newcommand{\eqt}{E(\mathbb Q)_\tors}
\newcommand{\ap}{\alpha_p}
\renewcommand{\a}{{\mathfrak a}}
\renewcommand{\b}{{\mathfrak b}}
\renewcommand{\c}{{\mathfrak c}}
\newcommand{\ac}{{\cal A}}
\newcommand{\bc}{{\mathfrak B}}
\newcommand{\mc}{{\mathfrak M}}
\newcommand{\zc}{{\cal Z}}
\newcommand{\ic}{{\cal I}}
\newcommand{\Is}{{\cal I}^q}
\newcommand{\jc}{{\cal J}}
%\newcommand{\qc}{{\cal Q}}
\newcommand{\qc}{{\mathfrak Q}}
%\newcommand{\cc}{{\cal C}}
\newcommand{\cc}{{\mathfrak C}}
\newcommand{\dc}{{\cal D}}
\newcommand{\lcg}{{\cal L}_K}
\newcommand{\gc}{{\cal G}_{\cc^2}(K)}
\newcommand{\gck}{{\cal G}_{\Z_k}(K)}
\newcommand{\efk}{{\mathfrak e}}
\newcommand{\f}{{\mathfrak f}}
\newcommand{\GP}{{\mathfrak P}}
\newcommand{\GQ}{{\mathfrak Q}}
\newcommand{\GS}{{\mathfrak S}}
\newcommand{\CA}{{\cal A}}
\newcommand{\CB}{{\cal B}}
\newcommand{\CC}{{\cal C}}
\newcommand{\CE}{{\cal E}}
\newcommand{\FF}{{\cal F}}
\newcommand{\CX}{{\cal X}}
\newcommand{\CZ}{{\cal Z}}
\newcommand{\al}{\alpha}
\newcommand{\be}{\beta}
\newcommand{\ga}{\gamma}
\newcommand{\de}{\delta}
\newcommand{\fgg}{{\mathfrak g}}
\newcommand{\h}{{\mathfrak h}}
\newcommand{\p}{{\mathfrak p}}
\newcommand{\vd}{v_{\p}(\mathfrak d)}
\newcommand{\Np}{{\N\p}}
\newcommand{\q}{{\mathfrak q}}
\newcommand{\gd}{{\mathfrak d}}
\newcommand{\Gd}{{\mathfrak D}}
\newcommand{\eps}{\varepsilon}
\newcommand{\myproof}[1]{{\it Proof #1. \/}}
\newcommand{\Proof}{{\it Proof. \/}}
\newcommand{\squareforqed}{\hbox{\rlap{$\sqcap$}$\sqcup$}}
\newcommand{\qed}{\ifmmode\squareforqed\else{\unskip\nobreak\hfil
\penalty50\hskip1em\null\nobreak\hfil\squareforqed
\parfillskip=0pt\finalhyphendemerits=0\endgraf}\fi}
\newcommand{\eqed}{\hfil\hbox{}\parfillskip=0pt\finalhyphendemerits=0\endgraf}
\newcommand{\fp}{\qed\removelastskip\vskip\baselineskip\relax}
\newtheorem{theorem}{Theorem}[section]
\newtheorem{corollary}[theorem]{Corollary}
\newtheorem{proposition}[theorem]{Proposition}
\newtheorem{lemma}[theorem]{Lemma}
\newtheorem{definition}[theorem]{Definition}
\newcommand{\nli}{\newline\indent}
\renewcommand{\pmod}[1]{\allowbreak\ ({\rm{mod}}\,\,#1)}
\newcommand{\sd}{\!\vartriangle\!}
\newcommand{\ssd}{\vartriangle}

\begin{document}
\pagestyle{plain}

\title{Explicit Methods for Solving Diophantine Equations}
\author{Henri Cohen,\\
Laboratoire A2X, U.M.R. 5465 du C.N.R.S.,\\
Universit\'e Bordeaux I, 351 Cours de la Lib\'eration,\\
33405 TALENCE Cedex, FRANCE}
\maketitle

\begin{abstract}
We give a survey of some classical and modern methods for solving Diophantine
equations.
\end{abstract}

\section{Introduction to Diophantine Equations}

The study of \emph{Diophantine equations} is the study of solutions of
polynomial equations or systems of equations in integers, rational numbers, or 
sometimes more general number rings. It is one of the oldest branches of number
theory, in fact of mathematics itself, since its origins can be found in texts
of the ancient Babylonians, Chinese, Egyptians, Greeks,... One of the
fascinations of the subject is that the problems are usually easy to state,
but more often than not very difficult to solve, and when they can be
solved sometimes involve extremely sophisticated mathematical tools.

Perhaps even more importantly, mathematicians must often invent or extensively
develop entirely new tools to solve the number-theoretical problems,
and these become in turn important branches of mathematics per se, which
often have applications in completely different problems than the one from
which they originate.

\smallskip

For many more details and examples, see Chapters 6 and 14 (pages
327 to 443 and 1011 to 1040) of the accompanying pdf file.

\subsection{Examples of Diophantine Problems}

Let me give four examples. The first and most famous is ``Fermat's last 
theorem'' (FLT), stating that for $n\ge3$, the curve 
$x^n+y^n=1$ has no rational points other than the ones with $x$ or $y$ equal 
to $0$ (this is of course equivalent to the usual statement).
\footnote{Incidentally, this is the place to destroy the legend concerning 
this statement, which has produced an enormous number of ``Fermatists'' 
claiming to have found an ``elementary'' proof that Fermat may have found 
himself: Fermat made this statement in
the margin of his copy of the book by Diophantus on number theory (at the
place where Diophantus discusses Pythagorean triples, see below), and claimed
to have found a marvelous proof and so on. However, he wrote this statement
when he was young, never claimed it publicly, and certainly never imagined
that it would be made public, so he forgot about it. It \emph{may} be
possible that there does exist an elementary proof (although this is
unlikely), but we can be positively sure that Fermat did not have it, otherwise
he would at least have challenged his English colleagues as was the custom
at that time.}

In the nineteenth century, thanks in particular to the work of E.~Kummer
and P.-G.~Lejeune-Dirichlet,
the theorem was proved for quite a large number of values
of $n$, including all $n\le100$. Together with the theory of quadratic
forms initiated by A.-M.~Legendre and especially by C.-F.~Gauss, 
one can 
without exaggeration say that this single problem gave rise to algebraic number
theory (rings, ideals, prime ideals, principal ideals, class numbers, units,
Dirichlet series, $L$-functions,...) As is well-known, although these
methods were pushed to the extreme in the twentieth century, they did not
succeed in solving the problem completely. The next progress on FLT came from
algebraic geometry thanks to the work of G.~Faltings who proved the so-called
Mordell conjecture, which in particular implies that for a \emph{fixed} $n\ge3$
the number of solutions to the Fermat equation is finite. However it was only 
thanks to the work of several mathematicians starting with 
Y.~Hellegouarch and G.~Frey, and culminating with the work of K.~Ribet, 
then finally of A.~Wiles (helped for a crucial part by R.~Taylor),
that the problem
was finally completely solved by using completely different tools from those
of Kummer (and even Faltings): elliptic curves, 
Galois representations and modular forms.
Although these subjects were not initiated by FLT, their development was 
certainly accelerated by the impetus given by FLT. In particular, thanks to
the work of Wiles the complete proof of the Taniyama--Weil conjecture was
obtained a few years later by C.~Breuil, 
B.~Conrad, F.~Diamond and R.~Taylor. This latter result can be considered in 
itself a more important (and certainly a more useful) theorem than FLT.

\medskip

A second rather similar problem whose history is slightly different is
\emph{Catalan's conjecture}. This states that when $n$ and $m$ are greater or
equal to $2$, the only solutions in nonzero integers $x$ and $y$ of the
equation $x^m-y^n=1$ come from the equality $3^2-2^3=1$. This problem can
be naturally attacked by the standard methods of algebraic number theory
originating in the work of Kummer. However, it came as a surprise that
an elementary argument due to Cassels (see Theorem \ref{catca0}) shows 
that the ``first case'' is impossible, in other words that if $x^p-y^q=1$ with 
$p$ and $q$ primes then $p\mid y$ and $q\mid x$.
The next important result due to R.~Tijdeman using Baker's theory of linear 
forms in logarithms of algebraic numbers was that the total
number of quadruplets $(m,n,x,y)$ satisfying the required conditions is
finite. Note that the proof of this finiteness result is completely different 
from Faltings's proof of the corresponding one for FLT, and in fact in the
latter his result did not imply the finiteness of the number of triples
$(x,y,n)$ with $n\ge3$ and $xy\ne0$ such that $x^n+y^n=1$.

Until the end of the 1990's the situation was quite similar to that of
FLT before Wiles: under suitable conditions on the nondivisibility of
the class number of cyclotomic fields, the Catalan equation was known to have 
no nontrivial solutions. It thus came as a total surprise that in 1999
P.~Mih\u{a}ilescu proved that if Catalan's equation $x^p-y^q=1$ with $p$ and 
$q$ odd primes has a solution then $p$ and $q$ must satisfy the
so-called double Wieferich condition $p^{q-1}\equiv1\pmod{q^2}$ and
$q^{p-1}\equiv1\pmod{p^2}$. These conditions were known before him, but he
completely removed the conditions on class numbers. The last step was again
taken by Mih\u{a}ilescu in 2001, who finished the proof of Catalan's 
conjecture. His proof was improved and simplified by several people, including 
in particular Yu.~Bilu and H.~W.~Lenstra.

The remarkable thing about the final proof is that it \emph{only} uses
algebraic number theory techniques on cyclotomic fields. However it uses
a large part of the theory, including the relatively recent theorem of
F.~Thaine, which has had some very important applications elsewhere.
It does not use any computer calculations, while the initial proof did. 

\medskip

A third example is the \emph{congruent number problem},
stated by Diophantus in the fourth century
A.D. The problem is to find all integers $n$ (called congruent numbers) which 
are equal to the area of a \emph{Pythagorean triangle}, 
i.e., a right-angled triangle with all three 
sides rational. Very simple algebraic transformations show that $n$ is 
congruent if and only if the Diophantine equation
$y^2=x^3-n^2x$ has rational solutions other than those with $y=0$.
The problem was in an ``experimental'' state until the 1970's, more precisely
one knew the congruent or noncongruent nature of numbers $n$ up to a few 
hundred (and of course of many other larger numbers). Remarkable progress was 
made on this problem by J.~Tunnell in 1980 using the 
theory of modular forms, and especially of modular forms of half-integral 
weight. In effect, he completely solved the problem,
by giving an easily checked criterion for $n$ to be a congruent number,
assuming a weak form of the Birch--Swinnerton-Dyer conjecture.
This conjecture (for which a prize money of 1 million U.S. dollars has been 
offered by the Clay foundation) is probably
one of the most important, and also one of the most beautiful conjectures in 
all of mathematics in the twenty-first century.

\medskip

A fourth important example is the Weil conjectures. These have to do with the 
number of solutions
of Diophantine equations \emph{in finite fields}. Looking at Diophantine
equations \emph{locally}, and in particular over finite fields, is usually
a first important step in its study. Let us give a simple example.
Let $N(p)$ be the number of solutions modulo $p$ of the equation
$y^2=x^5-x$. Then $|N(p)-p|$ can never be very large compared to $p$, more
precisely $|N(p)-p|<4\sqrt{p}$, and the constant $4$ is best possible.
This result is already quite nontrivial, and the general study of the number
of points on \emph{curves} culminated with work of A.~Weil in 1949 proving 
that this phenomenon occurs for all (nonsingular) curves and many other
results besides. It was then natural to ask the question for surfaces,
and more generally varieties of any dimension. This problem (in a very precise
form, which in particular implied excellent bounds on the number of solutions)
became known as the Weil conjectures. A general strategy for solving these
conjectures was put forth by Weil himself, but the achievement of this goal
was only made possible by an amazing amount of work by numerous people.
It included the creation of modern algebraic geometry by A.~Grothendieck 
and his students (the famous EGA and SGA treatises). The Weil conjectures were 
finally solved by P.~Deligne in the early 1970's, exactly following Weil's 
strategy, but using all the tools developed since.

\medskip

Note that for instance Waring's problem (given an integer $k\ge2$, find the 
smallest integer $g(k)$ such that any nonnegative integer can be represented 
as a sum of $g(k)$ nonnegative $k$-th powers) or variations, will not be
considered as Diophantine equations in this course since the equation is
not fixed.

\subsection{Introduction to Local Methods}

As is explicit or implicit in all of the examples given above (and in fact
in all Diophantine problems), it is essential to start by studying a
Diophantine equation \emph{locally}, in other words prime by prime (we will
see later precisely what this means). Let $p$ be a prime number, and let 
$\F_p\isom\Z/p\Z$ be the prime finite field with $p$ elements. We can begin by 
studying our problem in $\F_p$ (i.e., modulo $p$), and this can already be 
considered as the start of a local study. This is sometimes sufficient, but 
usually not, so we refine the study by considering the equation
modulo $p^2$, $p^3$ and so on, i.e., by working in
$\Z/p^2\Z$, $\Z/p^3\Z$... An important discovery, made by K.~Hensel at the
beginning of the twentieth century, is that it is possible to regroup all
these rings with zero divisors into a single object, called the $p$-adic 
integers, and denoted by $\Z_p$, which is an integral domain. Not only do we
have the benefit of being able to work conveniently with all the congruences
modulo $p$, $p^2$, $p^3$,... simultaneously, but we have the added benefit
of having \emph{topological properties} which add a considerable number of
tools that we may use, in particular \emph{analytic methods} (note that this 
type of limiting construction is very frequent in mathematics, with the same 
type of benefits). When we say that we study our Diophantine problem locally at
$p$, this means that we study it in $\Z_p$, or in the field of fractions 
$\Q_p$ of $\Z_p$.

\smallskip

Let us give simple but typical examples of all this. Consider first the 
Diophantine equation $x^2+y^2=3$ to be solved in rational numbers or,
equivalently, the Diophantine equation $x^2+y^2=3z^2$ to be solved in rational
integers. We may assume that $x$ and $y$ are coprime (exercise). 
Looking at the equation modulo $3$, i.e., in the field $\F_3$, we see that it 
has no solution ($x^2$ and $y^2$ are congruent to $0$ or $1$ modulo $3$, hence
$x^2+y^2$ is congruent to $0$ modulo $3$ if and only if $x$ and $y$ are
both divisible by $3$, excluded by assumption). Thus, our initial Diophantine
equation does not have any solution. 

We are here in the case of a \emph{quadratic} Diophantine equation. It is 
crucial to note that this type of equation can \emph{always} be solved by
local methods. In other words, either we can find a solution to the equation
(often helped by the local conditions), or it is possible to prove 
that the equation does not have any solutions using positivity conditions
together with congruences as above (or, equivalently, real and $p$-adic 
solubility). This is the so-called \emph{Hasse principle},
a nontrivial theorem which is valid for a \emph{single quadratic} Diophantine 
equation, but is in general not true for higher degree equations or for 
systems of equations.

\smallskip

Consider now the Diophantine equation $x^3+y^3=1$ to be solved in nonzero
rational numbers or, equivalently, the Diophantine equation $x^3+y^3=z^3$ to be
solved in nonzero rational integers. Once again we may assume that $x$, $y$,
and $z$ are pairwise coprime. It is natural to consider once more the problem 
modulo $3$. Here, however, the equation has nonzero solutions (for example
$1^3+1^3\equiv2^3\pmod{3}$). We must go up one level, and consider the equation
modulo $9=3^2$ to obtain a partial result: since it is easily checked that
an integer cube is congruent to $-1$, $0$ or $1$ modulo $9$, if we exclude
the possibility that $x$, $y$, or $z$ is divisible by $3$ then we see 
immediately that the equation does not have any solution modulo $9$, hence
no solution at all. Thus we have proved that if $x^3+y^3=z^3$, then one of
$x$, $y$, and $z$ is divisible by $3$. This is called solving the first case 
of FLT for the exponent $3$. To show that the equation has no solutions at all,
even with $x$, $y$, or $z$ divisible by $3$, is more difficult and 
\emph{cannot} be shown by congruence conditions alone. Indeed it is easy to
show that the equation $x^3+y^3=z^3$ has a solution with $xyz\ne0$ in every 
$p$-adic field, hence modulo $p^k$ for any prime number $p$ and any exponent 
$k$ (and it of course has real solutions). Thus, the Hasse principle clearly 
fails here since the equation does not have any solution in rational integers.
In that case, it is necessary to use additional \emph{global} arguments, whose 
main tools are those of algebraic number theory developed by Kummer et al.~in
the nineteenth century, and in particular class and unit groups, which are 
objects of a strictly \emph{global} nature.

\section{Use of Local Methods}

\subsection{A $p$-adic Reminder}

In this section it would be useful for the reader to have some basic knowledge
of the field of $p$-adic numbers $\Q_p$ and its ring of integers $\Z_p$. We 
briefly recall without proof what is needed:

\begin{itemize}\item A homogeneous equation with integer coefficients has
a nontrivial solution modulo $p^n$ for all $n\ge0$ if and only if it has a
nontrivial solution in $\Z_p$ (or in $\Q_p$ by homogeneity).
\item There is a canonical integer-valued valuation $v_p$ on $\Q_p^*$ such
that, if $x\in\Q$ then $v_p(x)$ is the unique integer such that $x/p^{v_p(x)}$
can be written as a rational number with denominator and numerator not 
divisible by $p$.
\item The elements of $\Q_p$ such that $v_p(x)\ge0$ are called $p$-adic
integers, they form a local ring $\Z_p$ with maximal ideal $p\Z_p$, the
invertible elements of $\Z_p$, called $p$-adic units, are those $x$ such that
$v_p(x)=0$, and if $x\in\Q_p^*$ we have the canonical decomposition
$x=p^{v_p(x)}y$ where $y$ is a $p$-adic unit.
\item If $a\in\Q$ is such that $v_p(a)\ge0$ and if $v_p(x)\ge1$ then the
power series $(1+x)^a$ converges. On the other hand, if $v_p(a)<0$ then
the power series converges for $v_p(x)\ge |v_p(a)|+1$ when $p\ge3$, and
for $v_p(x)\ge |v_p(a)|+2$ when $p=2$. In all cases, it converges to its
``expected'' value, for instance if $m\in\Z\setminus\{0\}$ then 
$y=(1+x)^{1/m}$ satisfies $y^m=1+x$.
\item Hensel's lemma (which is nothing else than Newton's method). We will 
need only the following special case: let $f(X)\in\Q_p[X]$ be a polynomial,
and assume that $\al\in\Q_p$ is such that $v_p(f(\al))\ge1$ and
$v_p(f'(\al))=0$. Then there exists $\al^*\in\Q_p$ such that $f(\al^*)=0$
and $v_p(\al^*-\al)\ge1$, and $\al^*$ can easily be constructed algorithmically
by using Newton's iteration.
\end{itemize}


 If this is not the case, however, the reader can 
translate a statement such as $C(\Q_p)\neq\emptyset$ as meaning that 
$C(\Z/p^n\Z)\neq\emptyset$ for all $n$ (the equations that we will study here 
are homogeneous, so that there is no real difference between $C(\Q_p)$ and 
$C(\Z_p)$). Furthermore, consider the following statement (which will be used
below): if $u\equiv1\pmod{16\Z_2}$ there exists $y\in\Z_2$ such that $u=y^4$.
The proof is as follows: if we write $u=1+x$, then $y=(1+x)^{1/4}$ is obtained
by using the binomial expansion, and it converges $2$-adically because
$16\mid x$. In ``elementary'' terms, this means that if we call 
$y_n\in\Q$ the truncation of the expansion of $(1+x)^{1/4}$ to $n$ terms,
then the $2$-adic valuation of $u-y_n^4$ tends to infinity with $n$. All
these statements are very easy to prove.

\subsection{The Fermat Quartics $x^4+y^4=cz^4$}

Although this is certainly not the simplest kind of Diophantine equation,
we begin by studying in detail the so-called \emph{Fermat quartics} 
$x^4+y^4=cz^4$ because they involve very interesting notions.

Thus, let $c\in\Z$, and denote by $\CC_c$ the projective curve defined 
by the equation $x^4+y^4=cz^4$. To find all the integers satisfying this
equation or, equivalently, the rational points on $\CC_c$, we may clearly 
assume that $c>0$ and that $c$ is not divisible by a fourth 
power strictly greater than $1$.

The general philosophy concerning local solubility is that it is in general
possible to decide \emph{algorithmically} whether a given equation is locally
soluble or not, and even whether it is \emph{everywhere} locally soluble,
in other words soluble over $\Q_p$ for all $p$, and also over $\R$. For
\emph{families} of equations, such as the ones we have here, it is also
usually possible to do this, as we illustrate in this subsection by giving
a necessary and sufficient sufficient for local solubility.

\begin{proposition} 
\begin{enumerate}\item $\CC_c(\Q_2)\neq\emptyset$ if and only if $c\equiv1$ or
$2$ modulo $16$.
\item If $p$ is an odd prime divisor of $c$, then $\CC_c(\Q_p)\neq\emptyset$ if
and only if $p\equiv1\pmod8$.
\item If $p\equiv3\pmod4$ is a prime not dividing $c$ then 
$\CC_c(\Q_p)\neq\emptyset$.
\item If $p\ge37$ is a prime not dividing $c$ then $\CC_c(\Q_p)\neq\emptyset$.
\item $\CC_c(\Q_{17})\neq\emptyset$.
\item Let $p\in\{5,13,29\}$ be a prime not dividing $c$. Then
\begin{enumerate}
\item $\CC_c(\Q_5)\neq\emptyset$ if and only if $c\not\equiv3$ or $4$ modulo $5$.
\item $\CC_c(\Q_{13})\neq\emptyset$ if and only if $c\not\equiv7$, $8$ or $11$ modulo $13$.
\item $\CC_c(\Q_{29})\neq\emptyset$ if and only if $c\not\equiv4$, $5$, $6$, $9$,
$13$, $22$ or $28$ modulo $29$.
\end{enumerate}\end{enumerate}\end{proposition}

\Proof If $(x:y:z)\in \CC_c(\Q_p)$, we may clearly assume that $x$, $y$ and 
$z$ are $p$-adic integers and that at least one is a $p$-adic unit (in
other words with $p$-adic valuation equal to $0$, hence invertible in $\Z_p$).
If $p\nmid c$, reduction modulo $p$ gives a projective curve $\ov{\CC_c}$ over 
$\F_p$, which is smooth (nonsingular) if $p\neq2$.

\smallskip

(1). Let $u$ be a $2$-adic unit. I claim that $u\in\Q_2^4$ if and only if
$u\equiv1\pmod{16\Z_2}$. Indeed, if $v$ is a $2$-adic unit we can write
$v=1+2t$ with $t\in\Z_2$, and 
$$v^4=1+8t+24t^2+32t^3+16t^4\equiv1+8(t(3t+1))\equiv1\pmod{16}\;.$$
Conversely, if $u\equiv1\pmod{16\Z_2}$ we write $u=1+x$ with $v_2(x)\ge4$,
and it is easy to check that the binomial expansion for $(1+x)^{1/4}$ converges
for $v_2(x)\ge4$.

Now assume that $x^4+y^4=cz^4$. Since $v_2(c)\le3$, either $x$ or $y$ is
a $2$-adic unit. It follows that $x^4+y^4\equiv1$ or $2$ modulo $16$, hence
$z$ is a $2$-adic unit, so that $c\equiv1$ or $2$ modulo $16$ as claimed.
Conversely, if $c\equiv1\pmod{16}$ then $c=t^4$ by my claim above, so that
$(t:0:1)\in \CC_c(\Q_2)$, while if $c\equiv2\pmod{16}$, then $c-1=t^4$ for some
$t$, hence $(t:1:1)\in \CC_c(\Q_2)$, proving (1).

\smallskip

(2). Assume that $p\mid c$ is odd. Since $v_p(c)\le3$, $x$ and $y$ are 
$p$-adic units, so that $-1$ is a fourth power in $\F_p$. If $g$ is a generator
of the cyclic group $\F_p^*$, then $-1=g^{(p-1)/2}$, hence $-1$ is a fourth
power in $\F_p$ if and only if $p\equiv1\pmod8$. If this is the case, let
$x_0\in\Z$ such that $x_0^4\equiv-1\pmod{p}$. By Hensel's lemma (which is
trivial here since the derivative of $X^4+1$ at $x_0$ is a $p$-adic unit),
there exists $x\in\Z_p$ such that $x^4=-1$, so that $(x:1:0)\in \CC_c(\Q_p)$,
proving (2).

\smallskip

The following lemma shows that for the remaining $p$ it is sufficient to
consider the equation in $\F_p$.

\begin{lemma} Let $p\nmid 2c$ be a prime number. Then 
$\CC_c(\Q_p)\neq\emptyset$
if and only if $\ov{\CC_c}(\F_p)\neq\emptyset$. In particular if 
$p\not\equiv1\pmod8$ then 
$$\CC_c(\Q_p)\neq\emptyset\text{\quad if and only if\quad}c\bmod p\in\F_p^4+\F_p^4\;.$$
\end{lemma}

\Proof One direction is clear. Conversely, assume that
$\ov{\CC_c}(\F_p)\neq\emptyset$, and let $(x_0:y_0:z_0)$ with $x_0$, $y_0$, 
and $z_0$ not all divisible by $p$ such that 
$x_0^4+y_0^4\equiv cz_0^4\pmod{p}$.
Since $p\nmid c$, either $p\nmid x_0$ or $p\nmid y_0$. Assume for instance
that $p\nmid x_0$, and set $f(X)=X^4+y_0^4-cz_0^4$. Clearly $v_p(f'(x_0))=0$
and $v_p(f(x_0))\ge1$, so that by Hensel's lemma there exists $t\in\Q_p$ such 
that $f(t)=0$, hence $(t:y_0:z_0)\in \CC_c(\Q_p)$, proving the converse.

Finally, assume that $p\not\equiv1\pmod8$. If $x^4+y^4\equiv cz^4\pmod{p}$ 
with $x$ or $y$ not divisible by $p$, we cannot have $p\mid z$ otherwise 
$x^4\equiv-y^4\pmod{p}$ so that $-1$ is a fourth power modulo $p$, a 
contradiction. Thus $p\nmid z$, hence
$(xz^{-1})^4+(yz^{-1})^4\equiv c\pmod{p}$, finishing the proof of the lemma.\fp

\smallskip

(3). Let $p\nmid c$, $p\equiv3\pmod4$. I claim that there exist $x$ and $y$
in $\Z$ such that $x^4+y^4\equiv c\pmod{p}$. Indeed, in a finite field $\F$
any element is a sum of two squares (in characteristic 2 any element is a 
square so the result is trivial, otherwise if $q=|\F|$ then there are $(q+1)/2$
squares hence $(q+1)/2$ elements of the form $c-y^2$, so the two sets have
a nonempty intersection). Thus there exist $u$ and
$v$ such that $c\equiv u^2+v^2\pmod{p}$. However, when $p\equiv3\pmod4$ we 
have ${\F_p^*}^2={\F_p^*}^4$: indeed we have a trivial inclusion, and the 
kernel of the map $x\mapsto x^4$ from $\F_p^*$ into itself is $\pm1$, so that
$|{\F_p^*}^4|=(p-1)/2=|{\F_p^*}^2|$, proving the equality. Thus $c=x^4+y^4$,
as claimed, and the above lemma proves (3).

\smallskip

(4). If $p\nmid 2c$ the curve $\ov{\CC_c}$ is smooth and absolutely 
irreducible. Its genus is at most equal to $3$ (in fact equal), so that by the
Weil bounds we know that $|\ov{\CC_c}(\F_p)|\ge p+1-6p^{1/2}$. This is 
strictly positive (for $p$ prime) if and only if $p\ge37$, so that (4) follows
from the above lemma.

\smallskip

(5) and (6). Thanks to the above cases, it remains to consider the primes
$p$ not dividing $c$ such that $3\le p\le 31$ and $p\equiv1\pmod4$, in
other words $p\in\{5,13,17,29\}$. For such a $p$, $-1$ is a fourth power
modulo $p$ only for $p=17$. In that case, Hensel's lemma as usual shows that
there exists $t\in\Q_{17}^4$ such that $-1=t^4$, proving (5) in this case.
Otherwise, we compute that
$$\F_5^4=\{0,1\},\quad\F_{13}^4=\{0,1,3,9\},\quad\F_{29}^4=\{0,1,7,16,20,23,24,25\}\;,$$
and we deduce the list of nonzero elements of $\F_p^4+\F_p^4$, proving (6).\fp

\smallskip

\noindent
{\bf Remark.} In the above proof we have used the Weil bounds, which are not
easy to prove, even for a curve. In the present case, however, the equations
being diagonal it is not difficult to prove these bounds using \emph{Jacobi
sums}.

\begin{corollary}\label{eclocsol} The curve $\CC_c$ is everywhere locally 
soluble \op i.e., has points in $\R$ and in every $\Q_p$\cp if and only if 
$c>0$ and the following conditions are satisfied.
\begin{enumerate}\item $c\equiv1$ or $2$ modulo $16$.
\item $p\mid c$, $p\neq2$ implies $p\equiv1\pmod8$.
\item $c\not\equiv3$ or $4$ modulo $5$.
\item $c\not\equiv7$, $8$ or $11$ modulo $13$.
\item $c\not\equiv4$, $5$, $6$, $9$, $13$, $22$ or $28$ modulo $29$.
\end{enumerate}\end{corollary}

\Proof Clear.\fp

As an interesting consequence, we give the following.

\begin{corollary} For all primes $p$ such that $p\equiv1\pmod{1160}$ the
curve $\CC_{p^2}$ is everywhere locally soluble, but is not globally soluble.
\end{corollary}

\Proof It is clear that the above conditions are satisfied modulo $16$,
$5$, and $29$, and also modulo $13$ since $7$, $8$, and $11$ are nonquadratic 
residues modulo $13$. On the other hand a classical and easy result of
Fermat states that the equation $x^4+y^4=Z^2$ does not have any nontrivial 
solutions, so this is in particular the case for our equation 
$x^4+y^4=(pz^2)^2$.\fp

Since by Dirichlet's theorem on primes in arithmetic progressions
there exist infinitely many primes $p\equiv1\pmod{1160}$ this
corollary gives infinitely many examples where everywhere local solubility
does \emph{not} imply global solubility. 

An equation (or system of equations) is said to satisfy the 
\emph{Hasse principle} if everywhere
local solubility implies global solubility. Important examples are given
by \emph{quadratic forms} thanks to the Hasse--Minkowski theorem which tells
us that a quadratic form has nontrivial solutions over $\Q$ if and only if
it is everywhere locally soluble. Unfortunately the Hasse principle is usually
not valid, and the above result gives infinitely counterexamples.

\smallskip

The study of the global solubility of Fermat quartics is harder and will
be considered later.

\subsection{Fermat's Last Theorem (FLT)}

Recall that FLT states that the equation $x^n+y^n=z^n$ has no integral
solutions with $xyz$ for $n\ge3$, in other words that the curve $x^n+y^n=1$
has no other rational points than those with $x$ or $y$ equal to $0$.
Note that, although these points are easy (!) to spot, they are \emph{not}
trivial, and this makes the problem more difficult than if there were none
at all.

\smallskip

Thanks to Fermat's impossibility result on the equation $x^4+y^4=z^2$,
it is immediate to see that we may reduce to equations of the form
$x^p+y^p=z^p$ with $p\ge3$ prime, and $x$, $y$, and $z$ pairwise coprime
integers. As we have seen in the introduction, it is convenient to separate 
FLT into two subproblems: FLT I deals with the case where $p\nmid xyz$,
and FLT II with the case $p\mid xyz$. Intuitively FLT I should be simpler
since the statement is now that there exist \emph{no} solutions at all, and
indeed this is the case. We will not embark on a study of FLT, but in this
section we mention what can be said using only local methods.
We begin by the following.

\begin{proposition} The following three conditions are equivalent.
\begin{enumerate}\item There exists three $p$-adic units $\al$, $\be$, and
$\ga$ such that $\al^p+\be^p=\ga^p$ \op in other words FLT I is soluble 
$p$-adically\cps.
\item There exists three integers $a$, $b$, $c$ in $\Z$ such that $p\nmid abc$
with $a^p+b^p\equiv c^p\pmod{p^2}$.
\item There exists $a\in\Z$ such that $a$ is not congruent to $0$ or $-1$
modulo $p$ with $(a+1)^p\equiv a^p+1\pmod{p^2}$.
\end{enumerate}\end{proposition}

\Proof From the binomial theorem it is clear that if $u\equiv1\pmod{p\Z_p}$
then $u^p\equiv1\pmod{p^2\Z_p}$. Thus if $u\equiv v\pmod{p\Z_p}$ and $u$ and 
$v$ are $p$-adic units, then $u^p\equiv v^p\pmod{p^2\Z_p}$. We will use
this several times without further mention. Taking $a$, $b$ and $c$ to be
residues modulo $p$ of $\al$, $\be$ and $\ga$ thus shows that (1) implies (2).
Conversely, assume (2). We would like to apply Hensel's lemma. However, the
congruence is not quite good enough, so we have to do one step by hand. Let
$a^p+b^p=c^p+kp^2$ for some $k\in\Z$, and set $d=c+kp$, so that $p\nmid d$.
Then by the binomial theorem $d^p\equiv c^p+kp^2c^{p-1}\pmod{p^3}$, so that
$$a^p+b^p-d^p\equiv kp^2(1-c^{p-1})\equiv0\pmod{p^3}$$ 
since $p\nmid c$. We can now apply Hensel's lemma
to the polynomial $f(X)=(X^p+b^p-d^p)/p$ and to $\al=a$: we have
$v_p(f'(a))=v_p(a^{p-1})=0$ since $p\nmid a$, while $v_p(f(a))\ge2$ by
the above, so Hensel's lemma is applicable, proving (1).

Clearly (3) implies (2). Conversely, assume (2), i.e., that 
$c^p\equiv a^p+b^p\pmod{p^2}$ with $p\nmid abc$. In particular 
$c\equiv a+b\pmod p$. Thus, if we set $A=ba^{-1}$ modulo $p$, then by the
above remark $A^p\equiv b^pa^{-p}\pmod{p^2}$ and 
$(A+1)^p\equiv c^pa^{-p}\pmod{p^2}$, so that $(A+1)^p\equiv A^p+1\pmod{p^2}$,
proving (3) and the proposition.\fp

\begin{corollary} FLT I cannot be proved by congruence conditions \op i.e., 
$p$-adically\cp if and only if condition \op 3\cp of the proposition is 
satisfied for some $a$ such that $1\le a\le (p-1)/2$.\end{corollary}

\Proof Indeed, condition (3) is invariant when we change $a$ modulo $p$,
and also under the change $a\mapsto p-1-a$, so the result is clear.\fp

\begin{corollary}\label{localFLTI} If for all $a\in\Z$ such that 
$1\le a\le (p-1)/2$ we have $(a+1)^p-a^p-1\not\equiv0\pmod{p^2}$, then the 
first case of FLT is true for $p$.\end{corollary}

\Proof Indeed, if $a^p+b^p=c^p$ with $p\nmid abc$ then condition (2) of the
proposition is satisfied, hence by (3), as above there exists $a$ such that 
$1\le a\le (p-1)/2$ with $(a+1)^p-a^p-1\equiv0\pmod{p^2}$, proving the 
corollary.\fp

For instance, thanks to this corollary we can assert that FLT I is true 
for $p=3$, $5$, $11$, $17$, $23$, $29$, $41$, $47$, $53$, $71$, $89$, $101$, 
$107$, $113$, $131$, $137$, $149$, $167$, $173$, $191$, $197$, which are
the prime numbers less than $200$ satisfying the condition of the corollary.

\smallskip

Using global methods, and in particular the Eisenstein reciprocity law,
one can prove that it is sufficient to take $a=1$ in the above corollary, in 
other words that FLT I is true as soon as $2^p-2\not\equiv0\mod{p^2}$. This
result is due to Wieferich.

\section{Naive Factorization over $\Z$}

We now start our study of global (as opposed to local) methods, and begin
by the most naive approach, which sometimes work: factorization of the
equation over $\Z$. We give two important examples where the results are
quite spectacular: once again a result on FLT I, called Wendt's criterion.
What is remarkable about it, apart from the simplicity of its proof, is that
it is highly probable that is applicable to \emph{any} prime number $p$,
and this would give an alternate proof of FLT I if this could be shown. 
Unfortunately, to \emph{prove} that it is indeed applicable to all $p$ would
involve proving results in \emph{analytic number theory} which are at present
totally out of reach. The second example is the theorem of Cassels on Catalan's
equation.

\subsection{Wendt's Criterion for FLT I}

\begin{proposition}[Wendt]\label{wendt} Let $p>2$ be an odd prime, and 
$k\ge1$ be an integer. Assume that the following conditions are satisfied.
\begin{enumerate}\item $k\equiv\pm2\pmod{6}$.
\item $q=kp+1$ is a prime number.
\item $q\nmid(k^k-1)R(X^k-1,(X+1)^k-1)$, where $R(P,Q)$ denotes the resultant
of the polynomials $P$ and $Q$.
\end{enumerate}
Then FLT I is valid, in other words if $x^p+y^p+z^p=0$ then $p\mid xyz$.
\end{proposition}

\Proof Assume that $x^p+y^p+z^p=0$ with $p\nmid xyz$ and as usual $x$, $y$,
and $z$ pairwise coprime. We can write
$$-x^p=y^p+z^p=(y+z)(y^{p-1}-y^{p-2}z+\cdots+z^{p-1})\;.$$
Clearly the two factors are relatively prime: we cannot have $p\mid (y+z)$
otherwise $p\mid x$, and if $r\neq p$ is a prime dividing both factors
then $y\equiv -z\pmod{r}$ hence the second factor is congruent to
$py^{p-1}$ modulo $r$, and since $r\neq p$ we have $r\mid y$, hence $r\mid z$
contradicting the fact that $y$ and $z$ are coprime. Since $p$ is odd 
(otherwise we would have to include signs), it follows that there 
exist coprime integers $a$ and $s$ such that $y+z=a^p$ and 
$y^{p-1}-y^{p-2}z+\cdots+z^{p-1}=s^p$. By symmetry, there exist $b$ and $c$
such that $z+x=b^p$ and $x+y=c^p$.

Consider now the prime $q=kp+1$. The Fermat equation implies that
$$x^{(q-1)/k}+y^{(q-1)/k}+z^{(q-1)/k}\equiv0\pmod{q}\;.$$
I claim that $q\mid xyz$. Indeed, assume by contradiction that $q\nmid xyz$,
and let $u=(x/z)^{(q-1)/k}\bmod{q}$, which makes sense since $q\nmid z$.
Since $q\nmid x$ we have $u^k-1\equiv0\pmod{q}$. On the other hand
$u+1\equiv-(y/z)^{(q-1)/k}\pmod{q}$, and since $k$ is even and $q\nmid y$
we deduce that $(u+1)^k-1\equiv0\pmod{q}$. It follows that the polynomials
$X^k-1$ and $(X+1)^k-1$ have the common root $u$ modulo $q$, contradicting
the assumption that $q\nmid R(X^k-1,(X+1)^k-1)$.

Thus $q\mid xyz$, and by symmetry we may assume for instance that $q\mid x$. 
Thus
\begin{align*}0\equiv 2x&=(x+y)+(z+x)-(y+z)=c^p+b^p+(-a)^p\\
&=c^{(q-1)/k}+b^{(q-1)/k}+(-a)^{(q-1)/k}\pmod{q}\;.\end{align*}
As above, it follows that $q\mid abc$. Since $q\mid x$ and $x$, $y$ and $z$
are pairwise coprime, we cannot have $q\mid b^p=z+x$ or $q\mid c^p=x+y$.
Thus $q\mid a$. It follows that $y\equiv -z\pmod{q}$, hence
$s^p\equiv py^{p-1}\pmod{q}$. On the other hand $y=(x+y)-x\equiv c^p\pmod{q}$,
so that
$$s^{(q-1)/k}=s^p\equiv p c^{((q-1)/k)(p-1)}\pmod{q}\;,$$
and since $q\nmid c$ we have $p\equiv d^{(q-1)/k}\pmod{q}$ with 
$d=s/c^{p-1}$ modulo $q$. Since $a$ and $s$ are coprime we have $q\nmid s$ 
hence $q\nmid d$, so $p^k\equiv1\pmod{q}$. Since $k$ is even it follows that
$$1=(-1)^k=(kp-q)^k\equiv k^kp^k\equiv k^k\pmod{q}\;,$$
contradicting the assumption that $q\nmid k^k-1$.\fp

Note that we have not used explicitly the assumption that 
$k\not\equiv0\pmod{6}$. However, if $k\equiv0\pmod{6}$ then $\exp(2i\pi/3)$ is
a common root of $X^k-1$ and $(X+1)^k-1$ in $\C$, hence the resultant of
these polynomials is equal to $0$ (over $\C$, hence over any ring), so that
the condition on $q$ can never be satisfied. In other words (2) and (3) 
together imply (1).

A computer search shows that for every prime $p\ge3$ up to very large bounds
we can find an integer $k$ satisfying the conditions of the proposition, and 
as mentioned at the beginning of this section it can reasonably be conjectured
that such a $k$ always exists, so that in practice FLT I can always be checked
thanks to this criterion. Of course, thanks to the work of Wiles et al., this
is not really necessary, but it shows how far one can go using very elementary
methods.

\smallskip

A special case of Wendt's criterion due to S.~Germain was stated and proved
some years before:

\begin{corollary}\label{sger} Let $p>2$ be an odd prime, and assume that 
$q=2p+1$ is also a prime. Then FLT I is valid, in other words if 
$x^p+y^p+z^p=0$ then $p\mid xyz$.\end{corollary}

\Proof Since for $k=2$ we have $(k^k-1)R(X^k-1,(X+1)^k-1)=-3^2$,
the condition of the proposition is $q\neq3$, which is always true.\fp

\subsection{Special Cases of the Equation $y^2=x^3+t$}

This subsection is meant to give additional examples, but should be considered
as supplementary exercises, and skipped on first reading.

The equation $y^2=x^3+t$ is famous, and has been treated by a wide variety
of methods. In the present subsection, we give two class of examples which
can easily be solved by factoring over $\Z$.

\begin{proposition}\label{a38b2} Let $a$ and $b$ be odd integers such that 
$3\nmid b$, and assume that $t=8a^3-b^2$ is squarefree but of any sign. Then 
the equation $y^2=x^3+t$ has no integral solution.
\end{proposition}

Note that the case $t=7$ of this proposition was already posed by Fermat
to his English contemporaries.

\Proof We rewrite the equation as
$$y^2+b^2=(x+2a)((x-a)^2+3a^2)\;.$$
Note that $x$ must be odd otherwise $y^2=x^3+t\equiv t\equiv7\pmod{8}$, which 
is absurd. Since $a$ is also odd it follows that $(x-a)^2+3a^2\equiv3\pmod4$,
and since this is a positive number (why is this needed?) this implies that 
there exists a prime $p\equiv3\pmod4$ dividing it to an \emph{odd} power. 
Thus $y^2+b^2\equiv0\pmod{p}$. Since $-1$ is not a square in $\F_p$ when
$p\equiv3\pmod4$, this implies that $p$ divides $b$ and $y$. I claim that 
$p\nmid x+2a$. Indeed, since $(x-a)^2+3a^2=(x+2a)(x-4a)+12a^2$
the condition $p\mid x+2a$ would imply $p\mid 12a^2$, hence either 
$p\mid a$ or $p=3$ ($p=2$ is impossible since $p\equiv3\pmod4$). But $p\mid a$
implies $p^2\mid t=8a^3-b^2$, a contradiction since $t$ is squarefree, and
$p=3$ implies $3\mid b$, which has been excluded, proving my claim. Thus the
$p$-adic valuation of $y^2+b^2$ is equal to that of $(x-a)^2+3a^2$ hence is 
odd, a contradiction since this would again imply that $-1$ is a square
in $\F_p$.\fp

Another similar result is the following.

\begin{proposition}\label{a38b22} Let $a$ be an odd integer, let $b$ be an
integer such that $3\nmid b$, and assume that $t=a^3-4b^2$ is squarefree, not
congruent to $1$ modulo $8$, but of any sign. Then the equation $y^2=x^3+t$ 
has no integral solution.
\end{proposition}

\Proof I claim that $x$ is odd. Indeed, otherwise, since $t$ is odd, $y$
would be odd, hence $y^2\equiv1\pmod{8}$, hence $t\equiv1\pmod{8}$, 
contradicting our assumption. Thus $x$ is odd and $y$ is even. Writing
$y=2y_1$ we obtain 
$$4(y_1^2+b^2)=x^3+a^3=(x+a)(x(x-a)+a^2)\;.$$
Since $x-a$ is even and $a$ is odd, it follows that $4\mid x+a$. Writing
$x+a=4x_1$, we obtain
$$y_1^2+b^2=x_1((4x_1-a)(4x_1-2a)+a^2)=x_1(16x_1^2-12ax_1+3a^2)\;.$$
Since $a$ is odd we have $16x_1^2-12ax_1+3a^2\equiv3\pmod{4}$, hence as in the 
preceding proof there exists a prime $p\equiv3\pmod4$ dividing it to an odd
power. As above, this implies that $p$ divides $y_1$ and $b$. I claim that
$p\nmid x_1$. Indeed, otherwise $p\mid3a^2$, hence either $p\mid a$ or
$p=3$. As above $p\mid a$ is impossible since it implies $p^2\mid t$, a
contradiction since $t$ is squarefree, and $p=3$ implies $3\mid b$, which
has been excluded. Thus the $p$-adic valuation of $y_1^2+b^2$ is odd,
a contradiction since this would imply that $-1$ is a square in $\F_p$.\fp

A computer search shows that the squarefree values of $t$ such that $|t|\le100$
which can be treated by these propositions are the following: $-97$, $-91$, $-79$, $-73$, $-71$, $-65$, $-57$, $-55$, $-43$, $-41$, $-37$, $-33$, $-31$, $-17$, $-15$, $-5$, $-3$, $7$, $11$, $13$, $23$, $39$, $47$, $53$, $61$, $67$, $83$, $87$, and
$95$. Another computer search finds solutions for the following squarefree
values of $t$ such that $|t|\le100$:
$-95$, $-89$, $-87$, $-83$, $-79$, $-74$, $-71$, $-67$, $-61$, $-55$, $-53$, $-47$, $-39$, $-35$, $-26$, $-23$, $-19$, $-15$, $-13$, $-11$, $-7$, $-2$, $-1$, $1$, $2$, $3$, $5$, $10$, $15$, $17$, $19$, $22$, $26$, $30$, $31$, $33$, $35$, $37$, $38$, $41$, $43$, $55$, $57$, $65$, $71$, $73$, $79$, $82$, $89$, $91$, $94$, and $97$. This still leaves $45$ squarefree values of $t$, which can all
be treated by other methods.

\subsection{Introduction to Catalan's Equation}

Catalan's conjecture, now a theorem, is the following:

\begin{theorem}[Mih\u{a}ilescu]\label{mihathm} If $n$ and $m$ are greater than
or equal to $2$ the only nonzero integral solutions to $$x^m-y^n=1$$
are $m=2$, $n=3$, $x=\pm3$, $y=2$.\end{theorem}

This conjecture was formulated by Catalan in 1844 and received much attention.
It was finally solved in 2002 by P.~Mih\u{a}ilescu. Complete proofs are 
available on the Web (see in particular \cite{Mis}), and at least 
two books are being written on the subject.

The goal of this section is to prove a result on this equation, due to Cassels,
which has been crucial for the final proof.

The cases $m=2$ or $n=2$, which are \emph{not} excluded, are treated
separately. The proof of the case $n=2$, in other words of the equation 
$x^m-y^2=1$, is due to V.-A.~Lebesgue in 1850. It is not difficult, but
uses the factoring of the equation over $\Z[i]$, and not over $\Z$. Knowing 
this, the reader can try his/her hand at it, perhaps after reading the
section dealing with factoring over number fields.

On the other hand, quite surprisingly, the proof of the case $m=2$, in other
words of the equation $x^2-y^n=1$ is considerably more difficult, and was
only obtained in the 1960's by Ko Chao. In retrospect, it could have been
obtained much earlier, since it is an easy consequence of a theorem of Nagell
which only uses the structure of the unit group in a real quadratic 
\emph{order}.

Once these two cases out of the way, it is clear that we are reduced to
the equation $x^p-y^q=1$, where $p$ and $q$ are odd primes. Before continuing,
we make the trivial but crucial observation that this equation is now
symmetrical in $p$ and $q$, in that if $(p,q,x,y)$ is a solution, then
$(q,p,-y,-x)$ is also a solution, since $p$ and $q$ are odd.

\smallskip

Cassels's results, which we will prove in this section, involve factoring
the equation over $\Z$, clever reasoning, and an analytic method called
``Runge's method'', which boils down to saying that if $x\in\R$ is such that
$|x|<1$ and $x\in\Z$ then $x=0$. Even though this looks like a triviality,
all proofs using Diophantine approximation techniques (of which Runge's method
is one) boil down to that. As an exercise, find all integer solutions to
$y^2=x^4+x^3+x^2+x+1$ by introducing the polynomial $P(x)$ such that
$P(x)^2-(x^4+x^3+x^2+x+1)$ has lowest degree.

\smallskip

For the proof of Cassel's results we will need one arithmetic and two
analytic lemmas.


\begin{lemma}\label{binpq} Set $w(j)=j+v_q(j!)$. Then 
$q^{w(j)}\binom{p/q}{j}$ is an integer not divisible by $q$, and
$w(j)$ is a strictly increasing function of $j$.
\end{lemma}

\Proof If $\ell$ is a prime number different from $q$ we know that
$\binom{p/q}{j}$ is an $\ell$-adic integer (note that this is \emph{not}
completely trivial), and it is an immediate exercise that its $q$-adic 
valuation is equal to $-w(j)$, proving the first assertion. Since
$w(j+1)-w(j)=1+v_q(j+1)\ge1$ the second assertion is also clear.\fp

The first analytic result that we need is the following.

\begin{lemma}\label{lemcas1}\begin{enumerate}\item For all $x>0$ we have
$(x+1)\log(x+1)>x\log(x)$.
\item Let $b\in\R_{>1}$. The function $(b^t+1)^{1/t}$ is a decreasing function
of $t$ from $\R_{>0}$ to $R_{>0}$ and the function $(b^t-1)^{1/t}$ is an 
increasing function of $t$ from $\R_{>0}$ to $R_{>0}$.
\item Assume that $q>p\in\R_{>0}$. If $a\in\R_{\ge1}$ then 
$(a^q+1)^p<(a^p+1)^q$ and if $a\in\R_{>1}$ then $(a^q-1)^p>(a^p-1)^q$.
\end{enumerate}\end{lemma}

\Proof Easy undergraduate exercise, left to the reader.\fp

The second analytic result that we need is more delicate.

\begin{lemma}\label{lemcas2} Assume that $p>q$, set $F(t)=((1+t)^p-t^p)^{1/q}$,
let $m=\lfloor p/q\rfloor+1$, and denote by $F_m(t)$ the sum of the terms of 
degree at most equal to $m$ in the Taylor series expansion of $F(t)$ around 
$t=0$. Then for all $t\in\R$ such that $|t|\le1/2$ we have
$$|F(t)-F_m(t)|\le\dfrac{|t|^{m+1}}{(1-|t|)^2}\;.$$
\end{lemma}

I could leave the proof as an exercise, but since it is not entirely trivial
I prefer to give it explicitly. I am indebted to R.~Schoof for it.

\Proof Set $G(t)=(1+t)^{p/q}$. It is clear that the Taylor coefficients
of $F(t)$ and $G(t)$ around $t=0$ are the same to order strictly less than 
$p$, and in particular to order $m$ since $m\le p/3+1<p$ (since $p\ge5$). In
what follows, assume that $|t|<1$. By the Taylor--Lagrange formula
applied to the functions $x^{1/q}$ and $G(x)$ respectively there exist $t_1$
and $t_2$ such that
\begin{align*}|F(t)-F_m(t)|&\le|F(t)-G(t)|+|G(t)-F_m(t)|\\
&\le\dfrac{|t|^p}{q}t_1^{1/q-1}+|t|^{m+1}\dfrac{1}{(m+1)!}G^{(m+1)}(t_2)\\
&\le\dfrac{|t|^p}{q}t_1^{1/q-1}+|t|^{m+1}\binom{p/q}{m+1}(1+t_2)^{p/q-m-1}\;,\end{align*}
with $t_1$ between $(1+t)^p$ and $(1+t)^p-t^p$, and $t_2$ between $0$ and $t$.
Now note that $p/q<m\le p/q+1$, so that $-1\le p/q-m<0$ and for all $j\ge1$
$0<p/q-(m-j)=j-(m-p/q)<j$ hence
$$0<\prod_{1\le j\le m}(p/q-(m-j))<\prod_{1\le j\le m}j=m!\;.$$
It follows that
$$\left|\binom{p/q}{m+1}\right|=\dfrac{(m-p/q)}{m+1}\dfrac{\prod_{1\le j\le m}(p/q-(m-j))}{m!}\le\dfrac{1}{m+1}\;.$$
Since $1/q-1<0$ and $p/q-m-1<0$ we must estimate $t_1$ and $1+t_2$ from below. 
If $t>0$ both $(1+t)^p$ and $(1+t)^p-t^p$ are greater than $1$, so 
$t_1>1>1-t^p$. If $t<0$ then $(1+t)^p=(1-|t|)^p$ and
$(1+t)^p-t^p=(1-|t|)^p+|t|^p>(1-|t|)^p$, so that $t_1>(1-|t|)^p$ in all cases.
On the other hand we have trivially $|1+t_2|\ge1-|t|$. Putting everything
together we obtain
$$|F(t)-F_m(t)|\le\dfrac{|t|^p}{q}(1-|t|)^{-p+p/q}+\dfrac{|t|^{m+1}}{m+1}(1-|t|)^{p/q-m-1}\;.$$
The above inequality is valid for all $t$ such that $|t|<1$. If we assume that
$|t|\le1/2$ then $|t|^{p-m-1}\le(1-|t|)^{p-m-1}$ (since $m\le p-1$), hence
$|t|^p(1-|t|)^{-p+p/q}\le|t|^{m+1}(1-|t|)^{p/q-m-1}$. It follows that
$$|F(t)-F_m(t)|\le \left(\dfrac{1}{q}+\dfrac{1}{m+1}\right)|t|^{m+1}(1-|t|)^{p/q-m-1}\;.$$
Since $p/q-m-1\ge-2$ and $1/q+1/(m+1)\le1$ the lemma follows.\fp

\subsection{Cassels's Results on Catalan's Equation}

\begin{lemma}\label{lemma1} Let $p$ be prime, let $x\in\Z$ be such that 
$x\ne1$, and set $r_p(x)=(x^p-1)/(x-1)$.
\begin{enumerate}\item If $p$ divides one of the numbers $(x-1)$ or $r_p(x)$
it divides both.
\item If $d=\gcd(x-1,r_p(x))$ then $d=1$ or $d=p$.
\item If $d=p$ and $p>2$, then $r_p(x)\equiv p\pmod{p^2}$.
\end{enumerate}\end{lemma}

\Proof Expanding $r_p(x)=((x-1+1)^p-1)/(x-1)$ by the binomial theorem we
can write
$$r_p(x)=(x-1)^{p-1}+p+(x-1)\sum_{k=1}^{p-2}\binom{p}{k+1}(x-1)^{k-1}$$
and all three results of the lemma immediately follow from this and the
fact that $p\mid\binom{p}{k+1}$ for $1\le k\le p-2$. Note that (3) is
trivially false for $p=2$.\fp

\begin{corollary}\label{corcop} Let $(x,y,p,q)$ be such that $x^p-y^q=1$.
Then $\gcd(r_p(x),x-1)=p$ if $p\mid y$ and $\gcd(r_p(x),x-1)=1$ 
otherwise.\end{corollary}

\Proof Since $y^q=(x-1)r_p(x)$ it follows that $p\mid y$ if and only if
$p$ divides either $x-1$ or $r_p(x)$, hence by the above lemma,
if and only if $\gcd(r_p(x),x-1)=p$.\fp

We can now state and prove Cassels's results.

\begin{theorem}[Cassels]\label{catca0} Let $p$ and $q$ be primes, and 
let $x$ and $y$ be nonzero integers such that $x^p-y^q=1$. Then $p\mid y$
and $q\mid x$.\end{theorem}

Before proving this theorem, we state and prove its most important corollary.

\begin{corollary}\label{catcas} If $x$ and $y$ are nonzero integers and
$p$ and $q$ are odd primes such that $x^p-y^q=1$ there exist
nonzero integers $a$ and $b$, and positive integers $u$ and $v$ with
$q\nmid u$ and $p\nmid v$ such that
\begin{align*}x&=qbu,\ x-1=p^{q-1}a^q,\ \dfrac{x^p-1}{x-1}=pv^q,\\
y&=pav,\ y+1=q^{p-1}b^p,\ \dfrac{y^q+1}{y+1}=qu^p\;.\end{align*}
\end{corollary}

\Proof Since $p\mid y$, by the above corollary we have $\gcd(r_p(x),x-1)=p$,
so by Lemma \ref{lemma1} (3) we have $r_p(x)\equiv p\pmod{p^2}$, and in
particular $v_p(r_p(x))=1$. Thus the relation $y^q=(x-1)r_p(x)$ implies that 
there exist integers $a$ and $v$ with $p\nmid v$ such that $x-1=p^{q-1}a^q$, 
$r_p(x)=pv^q$, hence $y=pav$, and since $r_p(x)>0$, we also have $v>0$. This 
shows half of the relations of the theorem, and the other half follow by 
symmetry, changing $(x,y,p,q)$ into $(-y,-x,q,p)$ and noting that $p$ and $q$
are odd.\fp

The proof of Cassels's Theorem \ref{catca0} is split in two, according to
whether $p<q$ or $p>q$. We begin with the case $p<q$ which is considerably
simpler.

\begin{proposition}\label{plq} Let $x$ and $y$ be nonzero integers and $p$ and 
$q$ be odd primes such that $x^p-y^q=1$. Then if $p<q$ we have $p\mid y$.
\end{proposition}

\Proof Assume on the contrary that $p\nmid y$. It follows from Corollary
\ref{corcop} that $x-1$ and $r_p(x)$ are coprime, and since their product
is a $q$-th power, they both are. We can thus write $x-1=a^q$ for some
integer $a$, and $a\ne0$ (otherwise $y=0$) and $a\ne-1$ (otherwise $x=0$),
hence $(a^q+1)^p-y^q=1$. Consider the function $f(z)=(a^q+1)^p-z^q-1$,
which is trivially a decreasing function of $z$. Assume first that $a\ge1$.
Then $f(a^p)=(a^q+1)^p-a^{pq}-1>0$ by the binomial expansion, while
$f(a^p+1)=(a^q+1)^p-(a^p+1)^q-1<0$ by (3) of Lemma \ref{lemcas1}. Since $f$
is strictly decreasing it follows that $y$ which is such that $f(y)=0$ is
not an integer, a contradiction. Similarly, assume that $a<0$, so that in fact
$a\le-2$, and set $b=-a$. Then since $p$ and $q$ are odd 
$f(a^p)=(a^q+1)^p-a^{pq}-1=-((b^q-1)^p-b^{pq}+1)>0$ by the binomial expansion,
while $f(a^p+1)=(a^q+1)^p-(a^p+1)^q-1=-((b^q-1)^p-(b^p-1)^q+1)<0$ again by (3)
of the Lemma \ref{lemcas1} since $b>1$. Once again we obtain a contradiction,
proving the proposition.\fp

The following corollary, essentially due to S.~Hyyr\"o, will be used for
the case $p>q$.

\begin{corollary}\label{hyr} With the same assumptions as above \op and in
particular $p<q$\cp we have $|y|\ge p^{q-1}+p$.\end{corollary}

\Proof Since by the above proposition we have $p\mid y$, as in Corollary 
\ref{catcas} we deduce that there exist integers $a$ and $v$ with $a\ne0$ and
$v>0$ such that $x-1=p^{q-1}a^p$, $(x^p-1)/(x-1)=pv^q$ and $y=pav$. Set 
$P(X)=X^p-1-p(X-1)$.
Since $P(1)=P'(1)=0$, it follows that $(X-1)^2\mid P(X)$, hence that
$(x-1)\mid (x^p-1)/(x-1)-p=p(v^q-1)$. Since $p^{q-1}\mid x-1$ it follows that 
$v^q\equiv1\pmod{p^{q-2}}$. However the order of the multiplicative group
modulo $p^{q-2}$ is equal to $p^{q-3}(p-1)$, and since $q>p$ this is coprime
to $q$. As usual this implies that $v\equiv1\pmod{p^{q-2}}$.

On the other hand, I claim that $v>1$. Indeed, assume otherwise that $v=1$,
in other words $x^{p-1}+\cdots+x+1=p$. If $x>1$ then $2^{p-1}>p$ so this is
impossible. Since $p$ and $q$ are odd primes and $a\ne0$ we have 
$|x-1|=p^{q-1}|a|^p\ge9$, hence when $x\le1$ we must have in fact $z=-x\ge8$.
But then since $p-1$ is even we have
$$p=z^{p-1}-z^{p-2}+\cdots+1\ge z^{p-1}(z-1)\ge z^{p-1}\ge2^{p-1}\;,$$
a contradiction which proves my claim. Since $v\equiv1\pmod{p^{q-2}}$, it 
follows that $v\ge p^{q-2}+1$, hence $|y|=pav\ge pv\ge p^{q-1}+p$, proving the
corollary.\fp

We now prove the more difficult case $p>q$ of Cassels's theorem.

\begin{proposition} Let $x$ and $y$ be nonzero integers and $p$ and $q$ be odd
primes such that $x^p-y^q=1$. Then if $p>q$ we have $p\mid y$.
\end{proposition}

\Proof We keep all the notation of Lemma \ref{lemcas2} and begin as for the 
case $p<q$ (Proposition \ref{plq}): assuming by contradiction that $p\nmid y$ 
and using Corollary \ref{corcop}, we deduce that there exists 
$a\in\Z\setminus\{0\}$ such that $x-1=a^q$, hence $y^q=(a^q+1)^p-1$,
so that $y=a^pF(1/a^q)$. Thus if we set $z=a^{mq-p}y-a^{mq}F_m(1/a^q)$ we have
$z=a^{mq}(F(1/a^q)-F_m(1/a^q))$. Applying Lemma \ref{lemcas2} to $t=1/a^q$
(which satisfies $|t|\le1/2$ since $a\ne\pm1$) we obtain
$$|z|\le\dfrac{|a|^q}{(|a|^q-1)^2}\le\dfrac{1}{|a|^q-2}\le\dfrac{1}{|x|-3}\;.$$
By Taylor's theorem we have $t^mF_m(1/t)=\sum_{0\le j\le m}\binom{p/q}{j}t^{m-j}$, and by Lemma \ref{binpq} $D=q^{m+v_q(m!)}$ is a common denominator of
all the $\binom{p/q}{j}$ for $0\le j\le m$. It follows that 
$Da^{mq}F_m(1/a^q)\in\Z$, and since $mq\ge p$ that $Dz\in\Z$.
We now estimate the size of $Dz$. By Hyyr\"o's Corollary \ref{hyr} (with
$(p,q,x,y)$ replaced by $(q,p,-y,-x)$) we have $|x|\ge q^{p-1}+q\ge q^{p-1}+3$,
so by the above estimate for $|z|$ we have
$$|Dz|\le\dfrac{D}{|x|-3}\le q^{m+v_q(m!)-(p-1)}\;.$$
Now for $m\ge1$ we have $v_q(m!)<m/(q-1)$, and since $m<p/q+1$ we have
$$m+v_q(m!)-(p-1)<m\dfrac{q}{q-1}-(p-1)=\dfrac{3-(p-2)(q-2)}{q-1}\le0$$
since $q\ge3$ and $p\ge5$ (note that it is essential that the above inequality
be strict). Thus $|Dz|<1$, and since $Dz\in\Z$, it follows that $Dz=0$. However
note that
$$Dz=Da^{mq-p}y-\sum_{0\le j\le m}D\binom{p/q}{j}a^{q(m-j)}\;,$$
and by Lemma \ref{binpq} we have 
$$v_q\left(\binom{p/q}{j}\right)<v_q\left(\binom{p/q}{m}\right)=v_q(D)$$ 
for $0\le j\le m-1$, so that
$0=Dz\equiv D\binom{p/q}{m}\not\equiv0\pmod{q}$ by the same lemma. This
contradiction finishes the proof of the proposition hence of Cassels's 
theorem.\fp

\section{Factorization over Number Fields}

Although factorization over $\Z$ can sometimes give interesting results, it
is in general much more fruitful to factor over a \emph{number field}. In fact,
as already mentioned, the theory of number fields, essentially algebraic
number theory, arose mainly from the necessity of inventing the tools 
necessary to solve Diophantine equations such as FLT. 

\subsection{An Algebraic Reminder}

The prerequisites for this section is any classical course on algebraic
number theory. We briefly review what we will need.

\begin{itemize}\item A number field $K$ is a finite extension of $\Q$. By the
primitive element theorem it can always be given as $K=\Q(\al)$, where
$\al$ is a root of some nonzero polynomial $A\in\Q[X]$.
\item An algebraic \emph{integer} is a root of a \emph{monic} polynomial
with integer coefficients. The element $\al$ such that $K=\Q(\al)$ can
always be chosen to be an algebraic integer. The set of algebraic integers
of $K$ forms a ring, which we will denote by $\Z_K$, which contains (with
finite index) $\Z[\al]$, when $\al$ is chosen to be an algebraic integer. It 
is a free $\Z$-module of rank $n=[K:\Q]$, and a $\Z$-basis of $\Z_K$ is
called an integral basis.
\item The ring $\Z_K$ is a Dedekind domain. Whatever that means, the main
implication for us is that any fractional ideal can be decomposed uniquely
into a power product of prime ideals. This is in fact the main motivation.
Note the crucial fact that $\Z[\al]$ is \emph{never} a Dedekind domain
when it is not equal to $\Z_K$, so that prime ideal decomposition does
not work in $\Z[\al]$.
\item If $p$ is a prime number, let $p\Z_K=\prod_{1\le i\le g}\p_i^{e_i}$
be the prime power decomposition of the principal ideal $p\Z_K$. The ideals
$\p_i$ are exactly the prime ideals ``above'' (in other words containing)
$p$, the $e_i$ are called the ramification indexes, the field $\Z_K/\p_i$
is a finite field containing $\F_p=\Z/p\Z$, and the degree of the finite
field extension is denoted by $f_i$. Finally we have the important relation
$\sum_{1\le i\le g}e_if_i=n=[K:\Q]$.
\item The class group $Cl(K)$ defined as the quotient of the group of
fractional ideals by the group of principal ideals, is a finite group whose
cardinality is often denoted $h(K)$.
\item The unit group $U(K)$, in other words the group of invertible elements
of $\Z_K$, or again the group of algebraic integers of norm equal to $\pm1$,
is a finitely generated abelian group of rank $r_1+r_2-1$, where $r_1$ and
$2r_2$ are the number of real and complex embeddings, respectively. Its
torsion subgroup is finite and equal to the group $\mu(K)$ of roots of unity
contained in $K$.
\item A quadratic field is of the form $\Q(\sqrt{t})$, where $t$ is a
squarefree integer different from $1$. Its ring of integers is either
equal to $\Z[\sqrt{t}]=\{a+b\sqrt{t},\ a,b\in\Z\}$ when $t\equiv2$ or $3$
modulo $4$, or is the set of $(a+b\sqrt{t})/2$, where $a$ and $b$ are
integers having the same parity.
\item A cyclotomic field is a number field of the form $K=\Q(\z)$, where
$\z$ is a primitive $m$-th root of unity for some $m$. The main result
that we will need is that the ring of integers of a cyclotomic field is
equal to $\Z[\z]$, and no larger.\end{itemize}

\subsection{FLT I}

We begin by the historically most important example, that of FLT I. In
retrospect, Kummer's criterion that we will prove below does not seem to be 
very interesting since it has infinitely many exceptions, while
the much more elementary criterion of Wendt has probably none. However
congruence methods or approaches a la Wendt are totally useless for the
\emph{second case} of FLT, and in that case Kummer's methods can be adapted,
with some difficulty, and in fact Kummer's initial criterion remains valid.

\smallskip

In the sequel, we let $\z=\zeta_p$ be a primitive $p$-th root of unity in $\C$,
we let $K=\Q(\z)$, and we recall that the ring of integers of $K$ is equal to
$\Z[\z]$. We set $\pi=1-\z$, and recall that the ideal $\pi\Z_K$ is a prime
ideal such that $(\pi\Z_K)^{p-1}=p\Z_K$, and $p$ is the only prime number
ramified in $K$. The first successful attacks on FLT were based on the 
possibility of unique factorization in $\Z[\z]$. Unfortunately this is true for
only a limited number of small values of $p$. With the work of E.~Kummer it was
realized that one could achieve the same result with the much weaker hypothesis
that $p$ does not divide the class number $h_p$ of $\Z_K$.
Such a prime is called a \emph{regular} prime.
Note that it is known that there are infinitely many irregular (i.e., 
nonregular) primes, but that it is unknown 
(although widely believed) that there are infinitely many regular primes. In 
fact, there should be a positive density equal to $1-1/e$ of regular primes 
among all prime numbers. The irregular primes below $100$ are $p=37$, $59$, 
and $67$.

\smallskip

Let us begin by considering the case where there is unique factorization
in $\Z[\z]$, so as to see below the ``magic'' of ideals. We prove the following
lemma.

\begin{lemma}\label{tech1} Assume that $\Z[\z]$ has unique factorization,
in other words that it is a principal ideal domain. If $x^p+y^p=z^p$ with
$p\nmid xyz$ then there exists $\al\in\Z[\z]$ and a unit $u$ of $\Z[\z]$ such
that
$$x+y\z=u\al^p\;.$$
\end{lemma}

\Proof As usual we may assume that $x$, $y$, and $z$ are pairwise coprime.
The equation $x^p+y^p=z^p$ can be written
$$(x+y)(x+y\z)\cdots(x+y\z^{p-1})=z^p\;.$$
I claim that the factors on the left are pairwise coprime (this makes sense,
since $\Z[\z]$ is a PID): indeed, if some prime element $\om$ divides 
$x+y\z^i$ and $x+y\z^j$ for $i\neq j$, it divides also $y(\z^i-\z^j)$ and 
$x(\z^j-\z^i)$, hence $\z^i-\z^j$ since $x$ and $y$ are coprime. Since
the norm of $\z^i-\z^j$ is equal to $p$ it follows that $\om\mid p$, so that 
$\om\mid z$, hence $p\mid z$, contrary to our hypothesis. We thus have a 
product of pairwise coprime elements in $\Z[\z]$ which is equal to a $p$-th 
power. Since $\Z[\z]$ is a PID, it follows that each of them is a 
$p$-th power, up to multiplication by a unit, proving the lemma.\fp

Unfortunately this lemma is not of much use since $\Z[\z]$ is a PID for only
a small finite number of primes $p$. This is where \emph{ideals} come in handy,
since in the ring of integers of a number field there is always unique 
factorization of ideals into prime ideals:

\begin{lemma} The result of the above lemma is still true if we only assume
that $p\nmid h_p$, in other words that $p$ is a regular prime.\end{lemma}

\Proof The above proof is valid verbatim if we replace ``prime element''
by ``prime ideal'', so each ideal $\a_i=(x+y\z^i)\Z_K$ is equal to the $p$th 
power of an ideal, say $\a_i=\b_i^p$. Now comes the crucial 
additional step: since the class number $h_p$ is finite, we know that any
ideal raised to the $h_p$th power is a principal ideal. Thus, both
$\b_i^p=(x+y\z^i)\Z_K$ and $\b_i^{h_p}$ are principal ideals. Since
$p\nmid h_p$ there exists integers $u$ and $v$ such that $up+vh_p=1$, so
that $\b_i=(\b_i^p)^u(\b_i^{h_p})^v$ is also a principal ideal. Thus, if
we write $\b_1=\al\Z_K$, we have 
$$\a_1=(x+y \z)\Z_K=\al^p\Z_K=\b_1^p\;.$$
Since two generators of a principal ideal differ multiplicatively by a unit,
it follows that $x+y\z=u\al^p$ for some unit $u$.\fp

\begin{proposition}\label{globalFLTI} If $p\ge3$ is a regular prime then FLT I 
holds.\end{proposition}

\Proof First note that if $p=3$ and $p\nmid xyz$ we have $x^3$, $y^3$, and
$z^3$ congruent to $\pm1$ modulo $9$, which is impossible if $x^3+y^3=z^3$,
so we may assume that $p\ge5$. By the above lemma, there exists $\al\in\Z[\z]$
and a unit $u$ such that $x+y\z=u\al^p$. Denote complex conjugation by 
$\ov{\ }$. An elementary but crucial lemma on cyclotomic fields asserts
that if $u$ is a unit of $\Z[\z]$, then $u/\ov{u}$ is a root of unity, and
since the only roots of unity are $\pm\z^m$ for some $m$, we have
$u/\ov{u}=\eta=\pm z^m$. Recall that we have set $\pi=1-\z$. Thus
$\pi\mid(\z^j-\z^{-j})$ for all $j$, so that for any
$\be\in\Z[\z]$ we have $\ov{\be}\equiv\be\pmod{\pi}$, hence
$\ov{\al}\equiv\al\pmod{\pi}$. Since $\pi\nmid z$, it follows that
$\pi\nmid\al$, hence $\ov{\al}/\al\equiv1\pmod{\pi}$. Using the binomial
expansion and the fact that $\pi^{(p-1)}\mid p\Z_K$, we deduce that
$(\ov{\al}/\al)^p\equiv1\pmod{\pi^p}$. Dividing $x+\z y$ by its complex
conjugate (and remembering that both are coprime to $\pi$), we obtain
$(x+\z y)/(x+\z^{-1}y)\equiv\eta\pmod{\pi^p}$, in other words
$$x+\z y-\eta(x+\z^{-1}y)\equiv0\pmod{\pi^p}\;.$$
I claim that $m=1$. Indeed, assume otherwise. If $m=0$ we multiply the above 
congruence by $\z$, and if $m=p-1$ we multiply it by $\z^2$, otherwise
we do nothing. Thus we see that there exists a polynomial $f(T)\in\Z[T]$ of 
degree at most equal to $p-2\ge3$ (since we have assumed $p\ge5$), not 
divisible by $p$, and such that $f(\z)\equiv0\pmod{\pi^p}$. Set $g(X)=f(1-X)$.
It is also of degree at most
equal to $p-2$ and not divisible by $p$, and $g(\pi)\equiv0\pmod{\pi^p}$.
However it is clear that different monomials in $g(\pi)$ have valuations
which are noncongruent modulo $p-1$, hence are distinct, a contradiction.
It follows that $m=1$, proving my claim. Thus $\eta=\pm\z$, and our congruence
reads $x+\z y\mp (x\z+y)=(x\mp y)(1\mp\z)\equiv0\pmod{\pi^p}$
hence $x\mp y\equiv0\pmod{p}$. We cannot have $x+y\equiv0\pmod p$, otherwise
$p\mid z$. Thus $y\equiv x\pmod p$. We may now apply the same reasoning to
the equation $(-x)^p+z^p=y^p$ and deduce that $-z\equiv x\pmod{p}$. It follows
that $0=x^p+y^p-z^p\equiv 3x^p\pmod{p}$, and since $p\nmid x$, we obtain $p=3$
which has been excluded and treated directly, finishing the proof of FLT I 
when $p$ is a regular prime.\fp

For instance, the irregular primes less than or equal to $200$ are
$p=37$, $59$, $67$, $101$, $103$, $131$, $149$, $157$, so that FLT I is true
up to $p=200$ for all but those primes. This of course also follows (for all
primes) from Wendt's criterion.

\smallskip

Asymptotically, it is conjectured that the proportion of regular prime
numbers is equal to $\exp(-1/2)=0.607\dots$, although it is not even known
that there are infinitely of them, while it is easy to show that there are
infinitely many irregular primes.

To finish this section on FLT, note that with more work it is possible to
extend Kummer's theorem verbatim to FLT II.

\subsection{The Equation $y^2=x^3+t$ Revisited}

We come back to this equation which we have already solved in many cases above,
but now using the techniques of algebraic number theory. Note that most of
what we are going to say also applies to the more general equations
$y^2=x^p+t$ with $p\ge3$ prime.

\begin{proposition}\label{y2t3} Let $t$ be a squarefree negative integer
not congruent to $1$ modulo $8$ and such that $3$ does not divide
the class number of the imaginary quadratic field $\Q(\sqrt{t})$.
\begin{enumerate}\item When $t\equiv2$ or $3$ modulo $4$ then if $t$ is 
not of the form $t=-(3a^2\pm1)$ the equation $y^2=x^3+t$ has no integral 
solutions. If $t=-(3a^2+\eps)$ with $\eps=\pm1$, the integral solutions are 
$x=4a^2+\eps$, $y=\pm(8a^3+3\eps a)$.
\item When $t\equiv5\pmod8$ then if $t$ is not of the form 
$t=-(12a^2-1)$ or $-(3a^2\pm8)$, both with $a$ odd, the equation $y^2=x^3+t$ 
has no integral solutions. If $t=-(12a^2-1)$ with $a$ odd, the integral 
solutions are $x=16a^2-1$, $y=\pm(64a^3-6a)$. If $t=-(3a^2+8\eps)$ with 
$\eps=\pm1$ and $a$ odd, the integral solutions are $x=a^2+2\eps$, 
$y=\pm(a^3+3a\eps)$.\end{enumerate}
\end{proposition}

Note that the case $t=-2$ of the above equation was already solved by Fermat, 
who also posed it as a challenge problem to his English contemporaries.

\Proof Let $(x,y)$ be a solution to the equation $y^2=x^3+t$. I first
claim that $x$ is odd. Indeed, is $x$ is even then $y$ is odd (since otherwise
$4\mid t$, contradicting the fact that $t$ is squarefree), hence
$t=y^2-x^3\equiv1\pmod{8}$, contradicting the assumption of the proposition.
In the quadratic field $K=\Q(\sqrt{t})$ we factor our equation as
$(y-\sqrt{t})(y+\sqrt{t})=x^3$. I claim that the ideals generated by the two 
factors on the left are coprime. Indeed, assume otherwise, and let $\q$ be a 
prime ideal of $\Z_K$ dividing both factors. It thus divides their sum and 
difference, hence if $q$ is the prime number below $\q$ we have $q\mid 2y$ 
and $q\mid 2t$. Since we have seen that $x$ is odd $\q$ cannot be above $2$, 
so $q\mid\gcd(y,t)$, hence $q\mid x$ so $q^2\mid t$, contradicting the fact 
that $t$ is squarefree and proving my claim. Since the product of the two 
coprime ideals $(y-\sqrt{t})\Z_K$ and $(y+\sqrt{t})\Z_K$ is a cube, it follows
that $(y+\sqrt{t})\Z_K=\a^3$ for some ideal $\a$ of $\Z_K$. As in the proof
of FLT I for regular primes, since we have assumed that $3$ does not divide
the class number of $K$ it follows that $\a$ itself is a principal ideal,
hence that there exists a unit $u\in K$ such that $y+\sqrt{t}=u\al^3$. 
However, since $K$ is an \emph{imaginary} quadratic field, there are not many 
units, and more precisely the group of units is $\{\pm1\}$ except for $t=-1$ 
and $t=-3$ for which it has order $4$ and $6$ respectively. Thus the apart
from the case $t=-3$, the order of the group of units not divisible by $3$,
hence any unit is a cube, so in these cases we are reduced to the equation 
$y+\sqrt{t}=\al^3$ with $\al\in\Z_K$. We postpone for later the special
case $t=-3$. Since the ring of integers of a quadratic field is well known,
we can write $\al=(a+b\sqrt{t})/d$ with $a$ and $b$ integral, where 
either $d=1$, or, only in the case $t\equiv5\pmod8$, also $d=2$ and $a$ and $b$
odd. Expanding the relation $y+\sqrt{t}=\al^3$ gives the two equations
$$d^3y=a(a^2+3b^2t)\text{\quad and\quad}d^3=b(3a^2+b^2t)\;.$$
Note that we may assume $a\ge0$ since changing $a$ into $-a$ does not change
the second equation, and changes $y$ into $-y$ in the first. From the second 
equation we deduce that $b\mid d^3$, and since $b$ is coprime
to $d$ this means that $b=\pm1$. It follows that $d^3=\pm(3a^2+t)$
and $d^3y=a(a^2+3t)$. Separating the cases $d=1$ and $d=2$ (in which case
$a$ must be odd), and using the formula 
$x=\N_{K/\Q}(\al)=(a^2-b^2t)/d^2=(a^2-t)/d^2$ proves the proposition for
$t\ne-3$.

Consider now the case $t=-3$. We have seen above that $y+\sqrt{t}=u\al^3$
for some unit $u$. Thus either we are led to the equations of the proposition 
(if $u=\pm1$), or there exists $\eps=\pm1$ such that 
$y+\sqrt{t}=((a+b\sqrt{t})/2)^3(-1+\eps\sqrt{t})/2$. 
Equating coefficients of $\sqrt{t}$ gives
$$16=\eps(a^3-9b^2a)-3b(a^2-b^2)\;.$$
If $a\equiv0\pmod3$, the right hand side is divisible by $3$, a 
contradiction. If $b\equiv0\pmod3$, the right hand side is congruent to
$\pm1$ modulo $9$ since a cube is such, again a contradiction. Thus neither
$a$ nor $b$ is divisible by $3$, hence $a^2\equiv b^2\equiv1\pmod3$,
so the right hand side is still congruent to $\pm1$ modulo $9$, a
contradiction once again, so there are no solutions for $t=-3$.\fp

\noindent
{\bf Remarks.}\begin{enumerate}
\item When $t$ is not squarefree, it not difficult to obtain similar, but
more complicated results.
\item In the other cases that we have not treated ($t\equiv1\pmod8$, $t>0$, 
or $3$ dividing the class number of $\Q(\sqrt{t})$) the problem is 
considerably more difficult but can be solved for a \emph{given} value of
$t$ by the use of so-called Thue equations.\end{enumerate}

\section{The Super-Fermat Equation}

An equation of the form $x^p+y^q=z^r$, where $p$, $q$, and $r$ are given 
positive exponents (greater than or equal to $2$, otherwise there is no 
problem) is called a super-Fermat equation, and we search for integral 
solutions.
Note that the equation is not homogeneous, so some new phenomena appear.
In particular, it is now reasonable to \emph{add} the supplementary condition
that $x$, $y$, and $z$ be pairwise coprime (in the homogeneous case this
could be assumed without loss of generality). Indeed, as an easy but important
exercise, the reader is invited to show that for instance if the exponents
$p$, $q$, and $r$ are pairwise coprime, there exist an infinity of solutions
to the equation. We thus make the coprimeness assumption from now on.

A detailed study of what is known on these equations is fascinating, but we
will have to restrict to a few facts. The behavior of the solution set
depends in an essential way on the quantity $\chi=1/p+1/q+1/r$ associated
to the equation. It can be shown that if $\chi>1$ there exist infinitely
many (coprime) solutions, which can be given by a finite number of explicitly
given disjoint parametric families. For $\chi=1$ there are only finitely
many (known) solutions, although if we had taken different coefficients in
front of $x^p$, $y^q$, and $z^r$, there could be infinitely many. Finally,
for $\chi<1$ it is known that there are finitely many solutions, but not
effectively. For instance it is widely believed that the equation $x^3+y^5=z^7$
has no coprime solutions, but this problem seems presently out of reach.

We will give a few examples of complete parametric solutions, and an
example for $\chi<1$ where it is not too difficult to give the solution set,
using tools that we will only introduce later.

\subsection{The Equations $x^2+y^2=z^2$ and $x^2+3y^2=z^2$}

We begin by a very simple and classical result.

\begin{proposition}\label{pytheq222} The general coprime integer solution to
the equation $x^2+y^2=z^2$ is given by the two disjoint parametrizations
$$(x,y,z)=(2st,s^2-t^2,\pm(s^2+t^2))\text{\quad and\quad}(x,y,z)=(s^2-t^2,2st,\pm(s^2+t^2))\;,$$
where $s$ and $t$ are two coprime integers of opposite parity.\end{proposition}

\Proof Exchanging if necessary $x$ and $y$, we may assume that $x$ is even,
hence $y$ and $z$ are odd. Also, changing signs if necessary we may assume that
$x$, $y$, and $z$ are nonnegative. Since $z$ and $y$ are coprime, so are 
$(z-y)/2$ and $(z+y)/2$ (consider the sum and difference), hence from the 
equation $(x/2)^2=((z-y)/2)((z+y)/2)$ we deduce that $(z-y)/2$ and $(z+y)/2$
are both squares (since they are nonnegative). The proposition follows
by setting $(z-y)/2=t^2$ and $(z+y)/2=s^2$, which are coprime and of opposite
parity.\fp

Note that the change of signs of $x$ and/or $y$ are accounted for by
a change of sign of $s$ or the exchange of $s$ and $t$.

\begin{proposition}\label{pytheq3} The general coprime integer solution of the
equation $x^2+3y^2=z^2$ is given by the two disjoint parametrizations
$$(x,y,z)=(\pm(s^2-3t^2),2st,\pm(s^2+3t^2))\;,$$
where $s$ and $t$ are coprime integers of opposite parity such that $3\nmid s$,
and
$$(x,y,z)=(\pm(s^2+4st+t^2),s^2-t^2,\pm2(s^2+st+t^2))\;,$$
where $s$ and $t$ are coprime integers of opposite parity such that
$s\not\equiv t\pmod{3}$.\end{proposition}

\Proof I first claim that $x$ is odd. Indeed, if $x$ is even $y$ and $z$ are
odd, so that $x^2=z^2-3y^2\equiv6\pmod{8}$, which is absurd. We write 
$3y^2=(z-x)(z+x)$. Since $x$ and $z$ are coprime the GCD of $z-x$ and $z+x$ is 
either equal to $1$ (when $z$ is even) or to $2$ (when $z$ is odd). Assume 
first that $\gcd(z-x,z+x)=1$, so that $z$ is even. Changing $x$ into
$-x$ and $z$ into $-z$ if necessary, there exist integers $a$ and $b$, 
necessarily coprime, such that $z+x=3a^2$, $z-x=b^2$, and $y=ab$. Since $z$ is
even and $x$ odd, $a$ and $b$ are both odd, so that if we write $s=(a+b)/2$ 
and $t=(a-b)/2$ we obtain $z=2(s^2+st+t^2)$ and $x=s^2+4st+t^2$, giving the 
second parametrization. Since $a$ and $b$ are odd, $s$ and $t$ have opposite
parity, and since $z+x$ and $z-x$ are coprime we have $3\nmid b=s-t$.
Assume now that $\gcd(z-x,z+x)=2$, so that $z$ is odd and $y$ is even.
Writing $3(y/2)^2=((z-x)/2)((z+x)/2)$ and changing once again the signs
of $x$ and $z$, there exist coprime integers $s$ and $t$ such that
$(z+x)/2=s^2$ and $(z-x)/2=3t^2$, giving the first parametrization. Since $x$
and $z$ are odd $s$ and $t$ have opposite parity, and since $(z+x)/2$ and
$(z-x)/2$ are coprime we have $3\nmid s$.\fp

\subsection{The Equation $x^3+y^2=z^2$ and $x^2+y^2=z^3$}

\begin{proposition}\label{prop322} The general coprime integer solution of the
equation $x^3+y^2=z^2$ is given by the two disjoint parametrizations
$$(x,y,z)=(s(s^2+3t^2),t(3s^2+t^2),(s-t)(s+t))\;,$$ where 
$s\not\equiv t\pmod2$, and
$$(x,y,z)=(\pm(2s^3+t^3),2s^3-t^3,2ts)\;,$$
where $2\nmid t$.\end{proposition}

\Proof Here we simply write $(z-y)(z+y)=x^3$, and separate the cases where
$z$ and $y$ have opposite or the same parity. The details are left as an
exercise for the reader.\fp

\begin{proposition}\label{prop223} The general coprime integer solution of the
equation $x^2+y^2=z^3$ is given by the parametrization
$$(x,y,z)=(s(s^2-3t^2),t(3s^2-t^2),s^2+t^2)\;,$$
where $s$ and $t$ are coprime integers of opposite parity.
\end{proposition}

\Proof Here we work in the PID $\Z[i]$. Set $a=x+iy$, $b=x-iy$ so that 
$ab=z^3$. If we had $x\equiv y\equiv1\pmod2$, we would have 
$z^3\equiv 2\pmod8$, which is impossible. Since $x$ and $y$ are coprime it
follows that $x$ and $y$ have opposite parity and $a$ and $b$ are coprime in 
the PID $\Z[i]$. It follows that there exist $\al=s+it\in\Z[i]$ and some
unit $u$ of $\Z[i]$ such that $x+iy=u\al^3$. Since the unit group has order
$4$ every unit is a cube, so that, changing $\al$ if necessary we can
write $x+iy=\al^3$, hence $x-iy=\ov{\al}^3$, $z=\al\ov{\al}$, giving the
parametrization of the proposition. It is immediate to see that the condition
that $x$ and $y$ be coprime is equivalent to $s$ and $t$ being coprime of
opposite parity.\fp

\subsection{The Equation $x^2+y^4=z^3$}\label{sec234p}

We note that here we cannot have $x$ and $y$ both odd, otherwise 
$z^3\equiv2\pmod8$, absurd. We work in $\Z[i]$ and factor the equation as 
$(x+iy^2)(x-iy^2)=z^3$. Since $x$ and $y$ are coprime and not both odd, 
$x+iy^2$ and $x-iy^2$ are coprime in $\Z[i]$. Thus there exists $\al\in \Z[i]$
such that $x+iy^2=\al^3$, hence $x-iy^2=\ov{\al}^3$, $z=\al\ov{\al}$, where the
possible power of $i$ can be absorbed in $\al$. We write $\al=u+iv$, so that
$z=u^2+v^2$, $x=u^3-3uv^2$, and $y^2=3u^2v-v^3$. Thus, we must solve this 
equation. Note that since $x$ and $y$ are coprime, we have $\gcd(u,v)=1$
and $u$ and $v$ have opposite parity. We write $y^2=v(3u^2-v^2)$ and
consider two cases.

\smallskip

\noindent
{\bf Case 1: $3\nmid v$}

\smallskip

Then $v$ and $3u^2-v^2$ are coprime, hence $v=\eps a^2$, $3u^2-v^2=\eps b^2$,
$y=\pm ab$ with $\eps=\pm1$, and then $a$ and $b$ are coprime, $b$ is odd,
and $3\nmid ab$. We note that $3u^2-v^2\equiv-(u^2+v^2)\equiv-1\pmod4$ since
$u$ and $v$ have opposite parity, hence we must have $\eps=-1$, so the 
equations to be solved are $v=-a^2$ and $3u^2=v^2-b^2$. Since $3\nmid v$ and 
$3\nmid b$, changing if necessary $b$ into $-b$, we may assume that 
$3\mid v-b$, so the second equation is $u^2=((v-b)/3)(v+b)$. Note that $v$ and
$b$ are coprime. I claim that $v$ is odd. Indeed, otherwise $a$ is even,
hence $4\mid v=-a^2$, hence $v^2-b^2\equiv7\pmod8$, while $3u^2\equiv3\pmod8$,
a contradiction. Thus $v$ is indeed odd, so $u$ is even and $v-b$ and $v+b$
are even with $(v-b)/2$ and $(v+b)/2$ coprime. Thus we can write
$v-b=6\eps_1c^2$, $v+b=2\eps_1d^2$, $u=2cd$ (where the sign of $u$ can be
removed by changing $c$ into $-c$) with $c$ and $d$ coprime, and $3\nmid d$.
Thus $v=\eps_1(3c^2+d^2)$, $b=\eps_1(d^2-3c^2)$, and since $v=-a^2$ we have
$\eps_1=-1$, the last remaining equation to be solved is the second degree 
equation $d^2+3c^2=a^2$. Proposition \ref{pytheq3} gives us \`a priori the two
parametrizations $d=\pm(s^2-3t^2)$, $c=2st$, $a=\pm(s^2+3t^2)$
with coprime integers $s$ and $t$ of opposite parity such that $3\nmid s$,
and $d=\pm(s^2+4st+t^2)$, $c=s^2-t^2$, $a=\pm2(s^2+st+t^2)$, with coprime
integers $s$ and $t$ of opposite parity such that $s\not\equiv t\pmod3$.
However, since $v=-a^2$ is odd, $a$ is odd hence this second parametrization
is impossible. Thus there only remains the first one, so replacing everywhere 
gives the first parametrization

$$\begin{cases}
x=4ts(s^2-3t^2)(s^4+6t^2s^2+81t^4)(3s^4+2t^2s^2+3t^4)&\\
y=\pm(s^2+3t^2)(s^4-18t^2s^2+9t^4)&\\
z=(s^4-2t^2s^2+9t^4)(s^4+30t^2s^2+9t^4)\;,&
\end{cases}$$
where $s\not\equiv t\pmod2$ and $3\nmid s$.
%[12*t*s^11+44*t^3*s^9+792*t^5*s^7-2376*t^7*s^5-1188*t^9*s^3-2916*t^11*s,s^6-15*t^2*s^4-45*t^4*s^2+27*t^6,s^8+28*t^2*s^6-42*t^4*s^4+252*t^6*s^2+81*t^8]

\smallskip

\noindent
{\bf Case 2: $3\mid v$}

\smallskip

Set $w=v/3$. Then $3\nmid u$, $w$ and $u^2-3w^2$ are coprime, hence 
$v=\eps 3a^2$, $u^2-3w^2=\eps b^2$, $y=\pm 3ab$ with $\eps=\pm1$,
and then $a$ and $b$ are coprime and $b$ is odd. Since $u$ and $v$ (hence $w$)
have opposite parity, we have $u^2-3w^2\equiv u^2+w^2\equiv1\pmod4$, hence
we must have $\eps=1$, so the equations to be solved are $w=a^2$ and
$u^2-3w^2=b^2$. Proposition \ref{pytheq3} tells us that there exist coprime 
integers $c$ and $d$ of opposite parity 
such that either $u=c^2+3d^2$, $w=2cd$, $b=c^2-3d^2$ with $3\nmid c$, or 
$u=2(c^2+cd+d^2)$, $w=c^2-d^2$, $b=c^2+4cd+d^2$ with $c\not\equiv d\pmod3$,
where the signs can be absorbed as usual either by changing $x$ into $-x$
or $b$ into $-b$. Thus in the first case the final equation to be solved is
$2cd=a^2$, so that there exists coprime $s$ and $t$ with $3\nmid s$ such that
either $c=2s^2$, $d=t^2$, $a=\pm2st$ and $t$ odd, or $c=s^2$, $d=2t^2$,
$a=\pm2st$ and $s$ odd. Replacing everywhere gives the second and third
parametrizations:

$$\begin{cases}
x=\pm(4s^4+3t^4)(16s^8-408t^4s^4+9t^8)&\\
y=6ts(4s^4-3t^4)&\\
z=16s^8+168t^4s^4+9t^8\;,&
\end{cases}$$
where $t$ is odd and $3\nmid s$.
%[64*s^12-1584*t^4*s^8-1188*t^8*s^4+27*t^12,24*t*s^5-18*t^5*s,16*s^8+168*t^4*s^4+9*t^8]

$$\begin{cases}
x=\pm(s^4+12t^4)(s^8-408t^4s^4+144t^8)&\\
y=6ts(s^4-12t^4)&\\
z=s^8+168t^4s^4+144t^8\;,&
\end{cases}$$
where $s$ is odd and $3\nmid s$.
%[s^12-396*t^4*s^8-4752*t^8*s^4+1728*t^12,6*t*s^5-72*t^5*s,s^8+168*t^4*s^4+144*t^8]

\smallskip

In the second case the final equation to be solved is $c^2-d^2=a^2$ with $c$ 
and $d$ of opposite parity, hence with $a$ odd, so that by the solution to the
Pythagorean equation there exists coprime integers $s$ and $t$ of opposite 
parity such that $c=s^2+t^2$, $d=2st$, $a=s^2-t^2$ with $s\not\equiv t\pmod3$
Replacing everywhere gives the fourth and final parametrization:

$$\begin{cases}
x=\pm2(s^4+2ts^3+6t^2s^2+2t^3s+t^4)(23s^8-16ts^7-172t^2s^6-112t^3s^5&\\
\phantom{=\pm2(}\kern90pt-22t^4s^4-112t^5s^3-172t^6s^2-16t^7s+23t^8)&\\
y=3(s-t)(s+t)(s^4+8ts^3+6t^2s^2+8t^3s+t^4)&\\
z=13s^8+16ts^7+28t^2s^6+112t^3s^5+238t^4s^4&\\
\phantom{=\pm2(}\kern90pt+112t^5s^3+28t^6s^2+16t^7s+13t^8\;,&
\end{cases}$$
where $s\not\equiv t\pmod2$ and $s\not\equiv t\pmod3$.
%[46*s^12+60*t*s^11-132*t^2*s^10-1012*t^3*s^9-2574*t^4*s^8-2376*t^5*s^7-1848*t^6*s^6-2376*t^7*s^5-2574*t^8*s^4-1012*t^9*s^3-132*t^10*s^2+60*t^11*s+46*t^12,3*s^6+24*t*s^5+15*t^2*s^4-15*t^4*s^2-24*t^5*s-3*t^6,13*s^8+16*t*s^7+28*t^2*s^6+112*t^3*s^5+238*t^4*s^4+112*t^5*s^3+28*t^6*s^2+16*t^7*s+13*t^8]

We have thus shown the following theorem:

\begin{theorem} The equation $x^2+y^4=z^3$ in integers $x$, $y$, $z$ with
$\gcd(x,y)=1$ can be parametrized by one of the above four parametrizations,
where $s$ and $t$ denote coprime integers with the indicated congruence 
conditions modulo $2$ and $3$. In addition these parametrizations are disjoint,
in that any solution to our equation belongs to a single parametrization.
\end{theorem}

%fa1(pol)=local(fa,lfa,dd);fa=factor(subst(pol,t,1));lfa=length(fa[,1]);for(i=1,lfa,dd=poldegree(fa[i,1]);fa[i,1]=t^dd*subst(fa[i,1],s,s/t));[content(pol),fa]
%fac(v)=vector(3,i,fa1(v[i]))

\subsection{An Example with $1/p+1/q+1/r<1$: the Equation $x^6-y^4=z^2$}

As already mentioned, the cases $1/p+1/q+1/r<1$ are considerably more
difficult, essentially because they can be reduced to finding rational points
on curves of genus $1$ or higher. We give one example of this. To treat it,
we will need to find all the rational points on two elliptic curves. In the
cases that we will consider this can be done using \emph{$2$-descent} methods.
All this is implemented in an extremely useful program {\tt mwrank} of
J.~Cremona, that we will therefore use as a black box. We will see later
a sketch of how it works.

\begin{proposition} The equation $x^6-y^4=z^2$ has no solution in nonzero
coprime integers $x$, $y$, $z$.\end{proposition}

\Proof Thanks to Proposition \ref{prop223}, we see that $x^6-y^4=z^2$ is 
equivalent to $x^2=s^2+t^2$, $y^2=s(s^2-3t^2)$, $z=t(3s^2-t^2)$ where $s$ and 
$t$ are coprime integers of opposite parity. By Proposition \ref{pytheq222},
up to exchange of $s$ and $t$ the first equation is equivalent to
$s=2uv$, $t=u^2-v^2$, $x=\pm(u^2+v^2)$, where $u$ and $v$ are coprime integers
of opposite parity. We consider both cases.

\noindent
{\bf Case 1: $2\mid s$}

Set $a=u+v$, $b=u-v$, which are coprime and both odd. Then $s=(a^2-b^2)/2$
and $t=ab$, so the last equation to be solved can be written
$8y^2=(a^2-b^2)(a^4-14a^2b^2+b^4)$. Since $b$ is odd, we can set
$Y=y/b^3$, $X=a^2/b^2$, and we obtain the equation
$8Y^2=(X-1)(X^2-14X+1)$ or, equivalently $Y_1^2=(X_1-2)(X_1^2-28X_1+4)$
after multiplying by $8$ and setting $Y_1=8Y$ and $X_1=2X$. This is the
equation of an elliptic curve in Weierstrass form, and the {\tt mwrank} 
program or $2$-descent methods tell us that the only rational
point has $Y_1=0$, which does not correspond to a solution of our equation.

\noindent
{\bf Case 2: $2\nmid s$}

Here $s=u^2-v^2$, $t=2uv$, so that the last equation to be solved can be
written $y^2=(u^2-v^2)(u^4-14u^2v^2+v^4)$. We cannot have $v=0$, otherwise
$t=0$ hence $z=0$, which is impossible. Thus, we can set $Y=y/v^3$,
$X=u^2/v^2$ and we obtain the elliptic curve $Y^2=(X-1)(X^2-14X+1)$.
The {\tt mwrank} program again tells us that the only rational point
has $y=0$, which does not correspond to a solution of our equation.\fp

\section{Introduction to Elliptic Curves}

It is of course out of the question in this short text to explain the
incredibly rich theory of elliptic curves. As we have done above with
$p$-adic numbers and with algebraic number theory, we simply recall without
proof a number of basic definitions and facts.

For many more details and examples, see Chapter 8 (pages 495 to 587)
of the accompanying pdf file.

\subsection{An Elliptic Curve Reminder}

\begin{itemize}\item The ``abstract'' definition of an elliptic curve is
a curve of genus $1$ together with a point defined on the base field.
In practice, an elliptic curve can be given in a number of ways: the simplest
is as a simple Weierstrass equation $y^2=x^3+ax^2+bx+c$, or as a
generalized Weierstrass equation $y^2+a_1xy+a_3y=x^3+a_2x^2+a_4x+a_6$
(the numbering is canonical!), together with the condition that the curve be
nonsingular, condition which is understood in the subsequent examples. More
generally it can be given as a nonsingular plane cubic, as a hyperelliptic
quartic $y^2=a^2x^4+bx^3+cx^2+dx+e$, as the intersection of two quadrics,
and so on. All these other realizations can algorithmically be transformed
into Weierstrass form, so we will assume from now on that this is the case.
\item The set of projective points of an elliptic curve (in the case of
$y^2=x^3+ax^2+bx+c$, these are the affine points plus the point at infinity
$\O$ with projective coordinates $(0:1:0)$) form an abelian \emph{group}
under the secant and tangent method of Fermat (if you do not know what this
is, here is a brief explanation: if $P$ and $Q$ are distinct points on the
curve, draw the line joining $P$ and $Q$; it meets the curve in a third
point $R$, and we define $P+Q$ to be the symmetrical point of $R$ with
respect to the $x$-axis. If $P=Q$, do the same with the tangent).
\item If the base field is $\C$ the group $E(\C)$ of complex points of an
elliptic curve $E$ is in canonical bijection with the quotient $\C/\Lambda$,
where $\Lambda$ is a \emph{lattice} of $\C$, thanks to the Weierstrass
$\wp$ function and its derivative.
\item If the base field is a finite field $\F_q$, we have the important
Hasse bound $|E(\F_q)-(q+1)|\le 2\sqrt{q}$.
\item If the base field is equal to $\Q_p$ (or more generally to a finite
extension of $\Q_p$), we have a good understanding of $E(\Q_p)$.
\end{itemize}

The reader will of course have noticed that I do not mention the most 
interesting case where the base field is equal to $\Q$, or more generally
a number field. Indeed, this deserves a theorem:

\begin{theorem}[Mordell--Weil] Let $K$ be a number field. The group $E(K)$
is a finitely generated abelian group, called the Mordell--Weil group
of $E$ \op over $K$\cps.\end{theorem}

Thus $E(K)\isom E(K)_{\text{tors}}\oplus\Z^r$, where $E(K)_{\text{tors}}$
is a finite group, and $r$ is of course called the rank of $E(K)$. The
finite group $E(K)_{\text{tors}}$ can be computed algorithmically (and
there are only a finite number of possibilities for it, which are known
for instance for $K=\Q$). On the other hand, one of the major unsolved
problems on elliptic curves is to compute algorithmically the rank $r$,
together with a system of generators. 

The goal of the next sections is to explain some methods which can be
used to compute the Mordell--Weil group over $\Q$, either rigorously
in certain cases, or heuristically. Keep in mind that there is no general
algorithm, but only partial ones, which luckily work in ``most'' cases.
We will mention the $2$ and $3$-descent techniques, the use of $L$-functions,
and finish with the beautiful Heegner point method, one of the most amazing
and useful tools in the theory, both for the theory and in practice.

\section{$2$-Descent with Rational $2$-Torsion}

See Section 8.2, pages 510--525.

\section{General $2$-Descent}

See Section 8.3, pages 526--534.

\section{$3$-Descent with Rational $3$-Torsion Subgroup}

See Section 8.4, pages 534--544.

Look in particular at the beautiful application to $ax^3+by^3+cz^3=0$ in
Section 8.4.5, pages 542--544.

\section{Use of $L(E,s)$}

See Section 8.5, pages 544--560.

\section{The Heegner Point Method}

See Section 8.6, pages 560--573.

\section{Computation of Integral Points}

See Section 8.7, pages 573--580.

\bigskip

\begin{thebibliography}{14}
\bibitem{Cre} J.~Cremona, {\it Computing the degree of the modular
parametrization of a modular elliptic curve\/}, Math.~Comp.~{\bf 64} (1995), 
1235--1250.
\bibitem{Dar} H.~Darmon, {\it Rational points on modular elliptic curves\/}, 
CBMS Regional Conference Series in Math.~{\bf 101} (2004), American Math.~Soc.,
also available on the author's web page.
\bibitem{Gro} B.~Gross, {\it Heegner points on $X_0(N)$\/}, in Modular forms,
edited by R.~Rankin (1984), 87--105.
\bibitem{Mis} M.~Mischler, {\it La conjecture de Catalan racont\'ee \`a
un ami qui a le temps\/}, preprint available on the web at the URL
{\tt http://arxiv.org/pdf/math.NT/0502350}.
\bibitem{Zag} D.~Zagier, {\it Modular parametrizations of elliptic curves\/},
Canad.~Math.~Bull. {\bf 28} (1985), 372--384.
\end{thebibliography}


\enddocument




