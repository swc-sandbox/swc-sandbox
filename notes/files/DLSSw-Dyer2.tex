\documentclass[12pt]{article}

\usepackage{amssymb}
\usepackage{amsmath}
\usepackage{amscd}

\def\bA{{\mathbf A}}
\def\bC{{\mathbf C}}
\def\bF{{\mathbf F}}
\def\bG{{\mathbf G}}
\def\bL{{\mathbf L}}
\def\bP{{\mathbf P}}
\def\bQ{{\mathbf Q}}
\def\bR{{\mathbf R}}
\def\bU{{\mathbf U}}
\def\bW{{\mathbf W}}
\def\bZ{{\mathbf Z}}

\def\fa{{\mathfrak a}}
\def\fA{{\mathfrak A}}
\def\fb{{\mathfrak b}}
\def\fB{{\mathfrak B}}
\def\fC{{\mathfrak C}}
\def\fd{{\mathfrak d}}
\def\fO{{\mathfrak O}}
\def\fo{{\mathfrak o}}
\def\fp{{\mathfrak p}}
\def\fP{{\mathfrak P}}
\def\fq{{\mathfrak q}}
\def\fQ{{\mathfrak Q}}
\def\fR{{\mathfrak R}}
\def\fS{{\mathfrak S}}

\def\ga{{\alpha}}
\def\gb{{\beta}}
\def\gG{{\Gamma}}
\def\gg{{\gamma}}
\def\gd{{\delta}}
\def\ge{{\epsilon}}
\def\gk{{\kappa}}
\def\gl{{\lambda}}
\def\gL{{\Lambda}}
\def\gs{{\sigma}}
\def\gt{{\theta}}
\def\gT{{\Theta}}

\def\sA{{\mathcal A}}
\def\sB{{\mathcal B}}
\def\sC{{\mathcal C}}
\def\sF{{\mathcal F}}
\def\sG{{\mathcal G}}
\def\sL{{\mathcal L}}
\def\sM{{\mathcal M}}
\def\sN{{\mathcal N}}
\def\sS{{\mathcal S}}
\def\sX{{\mathcal X}}

\newfont{\wncyr}{wncyr10 at 12pt}
\def\Sha{\mbox{\wncyr Sh}}

\def\qed{{\hfill$\square$}}

\def\beq{\begin{equation} \label}
\def\ble{\begin{lemma} \label}
\def\bpr{\begin{question} \label}
\def\bth{\begin{theorem} \label}
\def\ele{\end{lemma}}
\def\epr{\end{question}}
\def\eth{\end{theorem}}

\newtheorem{theorem}{Theorem}
\newtheorem{lemma}{Lemma}
\newtheorem{question}{Question}
\newtheorem{Cor}{Corollary}

\renewcommand{\theCor}{}

\def\half{{\textstyle{\frac{1}{2}}}}


\begin{document}
\begin{center}
\textbf{A NEW METHOD FOR CERTAIN \\ DIOPHANTINE EQUATIONS}
\end{center}


\vskip 0.1in


\noindent 1. \emph{Introduction}. \newline
The original purpose of the research described in this article was to obtain
results about Diophantine problems on rational surfaces --- that is, surfaces
defined over a field $k$ which are birationally equivalent to $\bP^2$ over
the algebraic closure $\bar{k}$. (Throughout this article, $K$ and $k$ will
always denote algebraic number fields, with respective rings of integers $\fO$
and $\fo$. Except in the phrase `rational surface' as defined above, `rational'
will always mean defined over $k$.) But as often happens, the research
turns out to be also applicable to other problems: in this case, to certain K3
surfaces. This is significant, because Diophantine problems on K3 surfaces have
hitherto been almost wholly intractable. Much of the research, which is still
ongoing, is joint with one or
both of Jean-Louis Colliot-Th\'{e}l\`{e}ne and Alexei Skorobogatov; and I am
grateful to both of them for their constructive comments.

This article is almost self-contained; but for some definitions and comments
the reader is advised to refer back to the previous article: `Diophantine
Equations: Progress and Problems.'
Most of the results depend on one or both of two major conjectures. The
first, which I shall refer to as Hypothesis \Sha, is as follows:
\begin{quote}
\emph{If $E$ is an elliptic curve defined over an algebraic number field $K$,
then the Tate-Shafarevich group $\Sha(E/K)$ is finite.}
\end{quote}
The second, which is Schinzel's Hypothesis, is described in \S2. But if one
is content to study rational 0-cycles of degree 1 instead of rational points,
then Schinzel's Hypothesis can almost always be replaced by 
Lemma \ref{L2}. This illustrates a general truth: that one
expects to have a more satisfactory theory
for 0-cycles of degree 1 than for points, because the former are a coset of the
group of 0-cycles of degree 0 whereas the set of rational
points has in general no apparent structure. Unfortunately, most families
$\sF$ of varieties do not have the property that if
$V$ in $\sF$ is defined over the algebraic number field $k$, and if $V$
contains a 0-cycle of degree 1 defined over $k$, then $V(k)$ is not empty.
One important family which does have this property consists of the Del Pezzo
surfaces of degree 4. For a proof of this, and some applications of the
associated ideas, see \S8. 

{From} the number-theoretic point of view, there are two kinds of rational
surface defined over an algebraic number field $k$:
\begin{itemize}
\item Pencils of conics, given by an equation of the form
\beq{E1} a_0(U,V)X_0^2+a_1(U,V)X_1^2+a_2(U,V)X_2^2=0 \end{equation}
where the $a_i(U,V)$ are homogeneous polynomials of the same degree with
coefficients in $k$. Pencils
of conics can be further classified according to the number of bad
fibres, but in this article we shall not need to do so. If one assumes
Schinzel's Hypothesis the obstructions to weak approximation and to the Hasse
principle on pencils of conics are given by Theorem \ref{T4} in \S4. The
corresponding results for 0-cycles, which do not depend on any unproved
hypotheses, can be found in and after Theorem \ref{T3}.
\item Del Pezzo surfaces of degree $d$, where $0<d<9$. Over $\bC$, such a
surface is obtained by blowing up $(9-d)$ points of $\bP^2$ in general
position. It is known that Del Pezzo surfaces of degree $d>4$
over $k$ satisfy the
Hasse principle and weak approximation; indeed those of degree 5 necessarily
contain rational points and are therefore birationally equivalent to
$\bP^2$ over $k$. Del Pezzo surfaces of degree 2 or 1
have no aesthetic merits and have attracted relatively little attention;
it seems sensible to ignore
them until the problems coming from those of degrees 4 and 3 have been solved.
The Del Pezzo surfaces of degree 3 are the nonsingular cubic
surfaces, which have an enormous but largely irrelevent literature;
and those of degree 4 are the nonsingular intersections of
two quadrics in $\bP^4$. For historical reasons, attention has been
concentrated on the Del Pezzo
surfaces of degree 3; but the problems presented by those
of degree 4 are simpler. That fact is illustrated in \S8.
\end{itemize}
In both these cases the main conjecture, due to Colliot-Th\'{e}l\`{e}ne and
Sansuc, is that the only obstruction to either the Hasse principle or weak
approximation is the Brauer-Manin obstruction. Unfortunately, in our present
state of knowledge it seems
very difficult to deduce anything from the absence of a Brauer-Manin
obstruction; indeed the only paper I know of in which the Brauer-Manin
obstruction plays a natural part in the argument is the proof by Salberger
and Skorobogatov [11] that it is the only obstruction to weak approximation on
Del Pezzo surfaces of degree 4. Usually, what one does is to
obtain a sufficient condition for the Hasse principle, or a subset of $V(\bA)$
contained in the closure of $V(k)$; and one then compares the obstruction
thus obtained with the Brauer-Manin obstruction. These notes are concerned
with the first half of this programme, so the Brauer-Manin obstruction will
not be defined and will only be peripherally mentioned. For a fuller account
of it, see [7] and [6].

Our proofs of results for pencils of conics depend in an essential way on the
fact that the Hasse principle holds for conics, and indeed for all curves of
genus 0. A Del Pezzo surface of degree 4 or 3 defined over $k$
does contain an infinity of curves of genus 0 defined over $k$, but it appears
that we can only find any of them explicitly if we already know at least one
rational point on the surface. This seems to block any
approach to the Hasse principle by the methods already
described; and for
rather deeper reasons it also appears to block any such approach to weak
approximation on Del Pezzo surfaces of degree 3. One can prove weak
approximation on Del Pezzo surfaces of degree 4 by these
methods (and indeed without using
Schinzel's Hypothesis), though the argument involves some additional
complications; for this, see Theorem \ref{T2} in \S8.

One is therefore led to study the existence of rational
points on pencils of curves
of genus 1. But here we run into a new complication, because the Hasse
principle notoriously does not hold for curves of genus 1. There is however a
weaker version which it is often possible to exploit. Most, but not all, of
the known applications of the following lemma are when $n=2$.
\ble{L9} Let $E$ be an elliptic curve defined over an algebraic number field
$k$, and suppose that the Tate-Shafarevich group of $E$ is finite and that
for some $n>1$ the image of the Mordell-Weil group of $E$ in the $n$-Selmer
group of $E$ has index strictly less than $n^2$ in the latter. Then every
curve which represents an element of order exactly $n$ in the $n$-Selmer group
contains a point defined over $k$.
\ele
\emph{Proof} 
By hypothesis the Tate-Shafarevich group is a torsion group, so by a theorem
of Cassels there is a non-singular alternating form on it
--- and in
particular on its $n$-torsion subgroup. Hence this subgroup must have an even
number of generators of order $n$. It is given that
this subgroup cannot have as many as two
generators of order $n$, so it must have none. 
But the elements of the $n$-Selmer group which lie in the image of the
Mordell-Weil group certainly contain points defined over $k$.  \qed

Now suppose that we are given a pencil of curves $C$ of genus 1, each
of which is a 2-covering of its Jacobian $J$. To be able to apply this lemma,
we need to be able to implement a 2-descent on every $J$. Because of this, the
natural pencils to examine are those for which all the 2-division points
of each $J$ are defined over its field of definition $k(J)$. To prove
solubility of such a pencil of curves, it is then enough to find some $C$
in the pencil such that the 2-Selmer group of the corresponding $J$ is
generated by $C$ and the 2-coverings corresponding to the 2-division points.
The general theory of 2-descents, in a form convenient for this application,
is given in \S5 and applied in \S6. A particularly interesting
example follows from the fact that the K3 surface
\[ a_0X_0^4+a_1X_1^4+a_2X_2^4+a_3X_3^4=0 \]
can be fibred by curves of genus 1 of this kind, provided that
\beq{E82} a_0a_1a_2a_3 \text{ is a square}. \end{equation}
This case is worked out in detail in \S9, where (subject to
the two major hypotheses stated above) necessary and
sufficient conditions for solubility are obtained in the
general case in which no $\pm a_ia_j$ is a square and
$a_0a_1a_2a_3$ is not a fourth power. These conditions turn
out to be just the Brauer-Manin conditions. To the best of my
knowledge, this is the first significant solubility theorem
for any family of K3 surfaces. Similar arguments could
undoubtedly be applied to the remaining special cases, but the
method makes essential use of (\ref{E82}). Numerical evidence
suggests that if we drop (\ref{E82}) then the Brauer-Manin
conditions cease to be sufficient.

By means of an additional trick, we can actually apply
Lemma \ref{L9} when we only know that $J$ has one 2-division
point defined over $k(J)$; this trick is given in \S7.2, and
its application to Del Pezzo surfaces of degree 4 is in \S8.
The ideas underlying Lemma \ref{L9} can also be applied to
diagonal cubic surfaces
\[ a_0X_0^3+a_1X_1^3=a_2X_2^3+a_3X_3^3; \]
but now there are considerable additional complications, some
but not all of which are due to the fact that we have to carry
out descent simultaneously on two unrelated elliptic curves.
A very brief summary of this work is given in \S7.1.

\newpage

\noindent 2. \emph{Schinzel's Hypothesis and Salberger's device}. \newline
Schinzel's Hypothesis gives a conjectural answer
to the following question: given finitely many polynomials $F_1(X),\ldots,
F_n(X)$ in $\bZ[X]$ with positive leading coefficients, is there an
arbitrarily large integer $x$ at which they all take prime values? There are
two obvious obstructions to this:
\begin{itemize}
\item One or more of the $F_i(X)$ may factorize in $\bZ[X]$.
\item There may be a prime $p$ such that for any value of $x$ mod $p$ at
least one of the $F_i(x)$ is divisible by $p$.
\end{itemize}
If the congruence $F_i(x)\equiv0\,$mod$\,p$ is non-trivial, it
has at most $\deg(F_i)$ solutions; so the second obstruction
can only happen for $p\leq\sum\deg(F_i)$ or
if $p$ divides every coefficient of some $F_i$.
Schinzel's Hypothesis is that these are the only obstructions: in other words,
if neither of them happens then we can choose an arbitrarily large $x$ so
that every $F_i(x)$ is a prime.

Serre deduced the corresponding
result over any algebraic number field; here we shall in
addition need to approximate to the arguments at finitely
many bad places. In most applications
there is a predetermined set $\fB$ of bad places, and we need
to impose local conditions on $x$ at some or all of them.
But these conditions constrain the values of the $F_i(x)$
at those places, and therefore we
cannot necessarily require these values to be units at the
bad primes; nor in the applications will we need to.
Because in this article we try to preserve homogeneity as far as possible,
we have stated Lemma \ref{L1} in a form which applies to
homogeneous polynomials $G_i$ in two variables; but
the reader who wishes to do so will have no difficulty in stating and proving
the corresponding (stronger) result for polynomials in one variable.
Just as with the original version of
Schinzel's Hypothesis, provided that the coefficients
of $G_i$ for each $i$ have no
common factor we need only verify the existence of the
$y_\fp,z_\fp$ in the statement of the lemma when the absolute
norm of $\fp$ does not exceed $\sum\deg(G_i)$.
\ble{L1} Let $k$ be an algebraic number field and $\fo$ the ring of
integers of $k$. Let $G_1(Y,Z),\ldots,G_n(Y,Z)$ be homogeneous
irreducible elements of $\fo[Y,Z]$ and $\fB$ a finite set of primes of $k$.
Suppose that for each $\fp$ not in $\fB$ there exist $y_\fp,z_\fp$ in $\fo$
such that none of the $G_i(y_\fp,z_\fp)$ is in $\fp$.
For each $\fp$ in $\fB$, let $V_\fp$ be a non-empty open subset of $k_\fp
\times k_\fp$;
and for each infinite place $v$ of $k$ let $V_v$ be a
non-empty open subset of $k^*_v$. Assume Schinzel's
Hypothesis; then there is a point
$\eta\times\zeta$ in $k^*\times k^*$, with $\eta,\zeta$ integral outside
$\fB$, such that
\begin{itemize}
\item $\eta\times\zeta$ lies in each $V_\fp$;
\item $\eta/\zeta$ lies in each $V_v$;
\item each ideal $(G_i(\eta,\zeta))$ is the product of a prime ideal not in
$\fB$ and possibly powers of primes in $\fB$.
\end{itemize} \ele
\emph{Proof} Choose $\ga,\gb$ in $\fo$ so that $\ga/\gb$ lies in $V_v$ for
each infinite place $v$ and no $G_i(\ga,\gb)$ vanishes.
We can repeatedly adjoin a further prime $\fp$ to $\fB$ provided we define
the corresponding $V_\fp$ to be the set of all $y\times z$ in $\fo_\fp
\times\fo_\fp$
such that each $G_i(y,z)$ is a unit at $\fp$. We can therefore assume that
$\fB$ contains all primes $\fp$ such that
\begin{itemize}
\item the absolute norm of $\fp$ is not greater than $[k:\bQ]\sum\deg(G_i)$; or
\item $\fp$ divides any $G_i(\ga,\gb)$.
\end{itemize}
Let $\sB$ be the set of primes in $\bQ$ which lie below some prime of
$\fB$, and further adjoin to $\fB$ all the primes of $k$ not already in $\fB$
which lie above some prime of $\sB$.
By the Chinese Remainder Theorem we can choose $\eta_0,\zeta_0$ in $k$,
integral outside $\fB$ and such that each $G_i(\eta_0,\zeta_0)$ is nonzero
and $\eta_0\times\zeta_0$ lies in $V_\fp$ for each $\fp$ in $\fB$. For
reasons which will become clear after (\ref{E5}), we also
need to ensure that $\gb\eta_0\neq\ga\zeta_0$; this can be
done by varying
$\eta_0$ or $\zeta_0$ by a suitable element of $\fo$ divisible by large powers
of each $\fp$ in $\fB$. As an ideal, write
\[ (G_i(\eta_0,\zeta_0))=\fa_i\fb_i \]
where the prime factors of each $\fa_i$ are outside $\fB$ and those of each
$\fb_i$ are in $\fB$; thus $\fa_i$ is integral. Let $N_i$ be the absolute norm
of $\fb_i$. Now choose $\gg\neq0$ in $\fo$ to be a unit at all the primes
outside $\fB$ which divide any $G_i(\eta_0,\zeta_0)$ and
to be divisible by such large powers of each $\fp$ in $\fB$
that
\[ \eta\times\zeta=(\ga\gg\xi+\eta_0)\times(\gb\gg\xi+\zeta_0) \]
is in $V_\fp$ for all $\xi\in\fo$ and all $\fp\in\fB$, and that
if we write
\beq{E5} g_i(X)=G_i(\ga\gg X+\eta_0,\gb\gg X+\zeta_0), \end{equation}
then every coefficient of $g_i(X)$ is divisible by at least as great a power
of $\fp$ as is $\fb_i$. We have arranged that the two arguments of $G_i$ in
(\ref{E5}), considered as linear forms in $X$, are not proportional; thus if
$g_i(X)$ factorizes in $k[X]$ then $G_i(\ga\gg U
+\eta_0V,\gb\gg U+\zeta_0V)$ would factorize in $k[U,V]$, contrary to the
irreducibility of $G_i(Y,Z)$. We shall also require for each $i$ that $g_i(X)$
is prime to all its conjugates as elements of $\bar{k}[X]$; since the zeros
of $g_i(X)$ have the form $\gg^{-1}\xi_{ij}$ for some $\xi_{ij}$ independent
of $\gg$, this merely requires the ratios of $\gg$ to its conjugates to avoid
finitely many values. Write
\[ R_i(X)={\mathrm{Norm}}_{k(X)/\bQ(X)}(g_i(X))/N_i; \] 
then $R_i(X)$ has all its coefficients integral, for at each prime
it is the norm of a polynomial with locally integral coefficients.
An irreducible factor of $R_i(X)$ in $\bQ[X]$ cannot be prime to 
$g_i(X)$, because then it would also be prime to all the conjugates
of $g_i(X)$ and therefore to their product --- which is absurd. If
$R_i(X)$ had two coprime factors in $\bQ[X]$, their highest common factors
with $g_i(X)$ would be nontrivial coprime factors of $g_i(X)$ in $k[X]$,
whence $g_i(X)$ would not be irreducible in $k[X]$. Finally, $R_i(X)$ cannot
have a repeated factor because the conjugates of $g_i(X)$ are pairwise
coprime. So $R_i(X)=A_iH_i(X)$ in $\bZ[X]$, with $H_i(X)$ irreducible.
Clearly we can require the leading coefficient of each
$H_i(X)$ to be positive.
But the only primes which divide the constant term in $R_i(X)$ are the primes
outside $\sB$ which divide $G_i(\eta_0,\zeta_0)$, and none of them divide
the leading coefficient of $R_i(X)$; hence $A_i=\pm1$.
Now apply Schinzel's Hypothesis to the $H_i(X)$, which we can do because no
$H_i(0)$ is divisible by any prime in $\sB$.
But if $H_i(x)$ is equal to a prime not in $\sB$ then the ideal $(g_i(x))$
must be equal to the product of $\fb_i$ and a prime ideal not in $\fB$.
\qed

\medskip

If we are content to obtain results about 0-cycles of degree 1 instead of
results about points, we can replace Schinzel's Hypothesis by an
argument which depends on the partial fraction formula (\ref{E2}); its use
in this context was pioneered by
Salberger. Of the various versions of the consequent algorithm, Lemma
\ref{L2} seems the simplest, both in its proof and in the way in which
it is used; in particular, it does not involve an auxiliary set of
primes and its proof does not depend on a deep result of Waldschmidt.
We need a preliminary lemma about approximation.
\ble{L3} Let $L$ be an algebraic number field, $\fB$ a finite set of places
of $L$ and $\fS$ a finite set of primes of $L$ not necessarily disjoint from
$\fB$. Let $b>1$ be in $\bZ$ and such that no prime of $L$ which divides $b$
is in $\fB$. Let $M>0$ be a rational integer and for each $v$ in $\fB$ let $\xi_v$
be in $L_v$. Then there exists $\xi$ in $L^*$ as close as we like to each
$\xi_v$ and such that $\xi=\ga\gg^M$, where $(\ga)$ is the product of a first
degree prime $\fp$ not in $\fB\cup\fS$ and primes in $\fB$, and
$\gg=\gg_1/\gg_2$ for coprime integers $\gg_1,\gg_2$ such that the prime
factorization of $\gg_1$ does not include any prime in $\fB\cup\fS\cup\{\fp\}$
and the only primes which divide $\gg_2$ also divide $b$.
\ele
\emph{Proof}. By Dirichlet's theorem on primes in arithmetic progression, we
can choose $\fp$ and
$\ga$ as in the statement of the lemma so that $\ga$ is as close
as we like to $\xi_v$ for each finite $v$ in $\fB$ and $\xi_v/\ga>0$ for each
real $v$ in $\fB$. For each
infinite $v$ in $\fB$ we choose $\gg_v$ in $L_v$ so that
$\gg_v^M=\xi_v/\ga$. Using weak approximation, choose $\gg'$ in $L$,
a unit at every finite prime in $\fB\cup\fS\cup\{\fp\}$, so that $\gg'$ is
arbitrarily close to 1 at every finite prime in $\fB$ and arbitrarily close
to $\gg_v$ at every infinite place $v$ in $\fB$. By writing $\gg'b^N$ for
large enough $N$ in terms of a base for $\fo_L/\bZ$ and
then changing the
coefficients by elements of $\bQ$ which are small at each finite prime in
$\sB\cup\sS$ and bounded at every infinite place in $\sB$, we can obtain an
integer $\gg_1$ which is prime to $\sS\cup\{\fp\}$ and to $b$
and close to $\gg'b^N$
at every place in $\sB$. Now take $\gg_2=b^N$; then $\xi=\ga\gg^M$ satisfies
all our requirements.  \qed

For the statement and proof of the following lemma, we shall call a place of
$k$ \emph{bad} if it lies in $\sB$ or divides $b$; and we shall call a place
in $\bQ$ or in a field containing $k$ \emph{bad} if it lies below or above a
bad place of $k$. For our purposes, the most important difference between
places in $\sB$ and primes dividing $b$ is that the latter have no
approximation conditions associated with them.
\ble{L2} Let $k$ be an algebraic number field and $P_1(X),\ldots,P_n(X)$
monic irreducible non-constant
polynomials in $k[X]$; and let $N\geq\sum\deg(P_i)$ be a
given integer. Let $\fB$ be a finite set of places of $k$ which contains the
infinite places, the primes which divide $2$,
the primes at which some coefficient of some $P_i$ is not integral and
any other primes $\fp$ at which $\prod P_i(X)$ does not remain separable
when reduced ${\mathrm{mod~}}\fp$. Let $b$ be as in Lemma $\ref{L3}$.
For each $v$ in $\fB$ let $U_v$ be a
non-empty open set of separable monic polynomials of degree $N$ in $k_v[X]$.
Let $M>0$ be a fixed rational integer.
Then we can find an irreducible monic polynomial
$G(X)$ in $k[X]$ of degree $N$ which lies in each $U_v$ and for which
$\gl$, the image of $X$ in $K=k[X]/G(X)$, satisfies
\beq{E12} (P_i(\gl))=\fP_i\fA_i\fC_i^M \end{equation}
for each $i$, where the $\fP_i$ are distinct first degree primes in $K$ not
lying above any prime in $\fB$, the $\fA_i$ are products of bad primes in $K$
and the $\fC_i$ are integral ideals in $K$.
Moreover we can arrange that
$\gl=\ga/\gb$ where $\ga$ is integral and $\gb$ is an integer all of whose 
prime factors are bad.
\ele
\emph{Proof} We shall need to apply Lemma \ref{L3} repeatedly with
the same value of $M$ as in Lemma \ref{L2}.
We can assume, after adding a constant to $X$ if necessary, that
none of the $P_i(X)$ is a multiple of $X$.
Write $R(X)=\prod P_i(X)$ and $R_i(X)=R(X)/P_i(X)$. Any polynomial
$G(X)$ in $k[X]$ can be written in just one way in the form
\beq{E2} G(X)=R(X)Q(X)+\sum R_i(X)\psi_i(X) \end{equation}
with $\deg\,\psi_i<\deg\,P_i$; for if $\gl_i$ is a zero of $P_i(X)$ this is
just the classical partial fractions formula
\[ \frac{G(X)}{\prod P_i(X)}=Q(X)+\sum\frac{\psi_i(X)}{P_i(X)} \]
with $\psi_i(\gl_i)=G(\gl_i)/R_i(\gl_i)$. This property
determines for each $i$ a unique
$\psi_i(X)$ in $k[X]$ of degree less than $\deg P_i$. The same result holds
over any $k_v$. If
the coefficients of $G$ are integral at $v$, for some $v$ not in $\fB$, then
so are those of $Q$ and each $\psi_i$ because $R$ and the $R_i$ are monic and
$R_i(\gl_i)$ is a unit outside $\fB$. For
each $v$ in $\fB$ let $G_v(X)$ be a polynomial of degree $N$ lying
in $U_v$, and write
\[ G_v(X)=R(X)Q_v(X)+\sum R_i(X)\psi_{iv}(X) \]
with $\deg\,\psi_{iv}<\deg\,P_i$. We adjoin to $\fB$ a further finite place $w$
at which $b$ is a unit, and associate with it a monic irreducible
polynomial $G_w(X)$ in $k_w[X]$ of 
degree $N$; the only purpose of $G_w$ is to ensure that the $G(X)$ which we
shall construct is irreducible over $k$. We build $G(X)$, close to
$G_v(X)$ for every $v\in\fB$ including $w$, in the following manner.

For the first step let $k_i=k[X]/P_i(X)$ and for each
$v\in\fB$ let $\phi_{iv}$ be the class of $\psi_{iv}$ in
$k_v[X]/P_i(X)=k_i\otimes_kk_v$. Take $\fS$ to consist of those primes in $k$
at which the constant terms of the $P_i(X)$ are not all
units. We apply Lemma \ref{L3} to each set of
$\phi_{iv}$ in turn, replacing $L$ by $k_i$ and $\fB$ and $\fS$ by the sets of
places of $k_i$ which lie above $\fB$ and $\fS$ respectively; let $\phi_i$ be
the element of $k_i$ thus obtained, and let $\fP_i$ be the associated prime
in $k_i$.
Let $\psi'_i(X)$ be the unique polynomial in $k[X]$ with $\deg\psi'_i<
\deg P_i$ whose class in $k_i$ is $\phi_i$.
Clearly $\psi'_i(X)$ is arbitrarily close to each $\psi_{iv}(X)$, and its
coefficients are integers outside $\fB$ because $\fB$ contains all the
primes which ramify in $k_i/k$. Now choose positive $c,T$ in $\bZ$ so that $c$
is a unit at all bad primes, divisible by all the primes outside $\fB\cup\{
\fP_i\}$ which divide the numerator of any $\phi_i$,
and close to $b^T$ at the real place and at all the primes below primes
in $\fB$. Let $\psi_i(X)=(c/b^T)^M\psi'_i(X)$.

We now choose $Q(X)$ to be close to $Q_v(X)$ for each $v$ in $\fB$, and to be
such that each coefficient other than the leading coefficient
(which is 1) is integral except perhaps at bad primes
and is divisible by $c$. We can do this by an argument like,
but very much simpler than, that in the proof of Lemma \ref{L3}.
This construction ensures that $G(X)$ is monic and arbitrarily
close to each $G_v(X)$ including $G_w(X)$.
The assumptions made about $G_w(X)$ ensure that $G(X)$
is irreducible in $k_w$ and therefore in $k$. Moreover, the coefficients of
$Q(X)$ are integers except perhaps at bad primes; and since $G(X)$ is monic
the denominator of any $P_i(\gl)$ only contains bad primes.
A consequence
of the choice of $\fS$ is that every $\gl_i$, and therefore every $Q(\gl_i)$,
is prime to $c$.

We have still to prove (\ref{E12}). Let $\fp_i$ be the prime in $k$ below
$\fP_i$.
By computing the resultant of $P_i(X)$ and $G(X)$ in two different ways, we
obtain
\beq{E10} {\mathrm{Norm}}_{K/k}P_i(\gl)=\pm{\mathrm{Norm}}_{k_i/k}G(\gl_i)
=\pm{\mathrm{Norm}}_{k_i/k}(\phi_iR_i(\gl_i)) \end{equation}
where $\gl_i$ is a zero of $P_i(X)$. By hypothesis $R_i(\gl_i)$ is
a unit at every place of $k(\gl_i)$ which does not lie above a place in
$\fB$; and we have arranged that the denominator of
Norm$_{k_i/k}\phi_i$ is only divisible by bad primes,
and its numerator is the product of the
first degree prime $\fp_i$, powers of primes in $\fB$ and $M$th powers of
norms of primes which come from the $\fC_i$
of Lemma \ref{L3}. Also $\gl$, and therefore $P_i(\gl)$, is integral
outside bad primes in $K$.
None of the latter lie above $\fp_i$. Hence
$P_i(\gl)$ is an integer at each prime of $K$ lying above $\fp_i$. It
follows that the ideal $(P_i(\gl))$ is divisible by just one prime of $K$
above $\fp_i$, and that to the first power. It only remains to show that,
apart from this prime and bad primes, what
we have is an $M$th power.

Let $L$ be a splitting field for all the $P_i(X)$ and let $\fP$ be a
prime in $L(\gl)$ which divides the numerator of $P_i(\gl)$. By (\ref{E10})
and the remarks on either side of it, $\fP$ must divide Norm$_{k_i/k}(\phi_i)$
and therefore must divide $c$. Hence
\beq{E54} \tilde{G}(X)=\tilde{R}(X)\tilde{Q}(X) \end{equation}
where the tilde denotes reduction mod $\fP$ of the coefficients. But the
construction of $Q(X)$ has ensured that the resultant of $Q(X)$ and $R(X)$,
which is $\pm\prod_i{\mathrm{Norm}}_{k_i/k}(Q(\gl_i))$, is prime to $c$; hence
$\tilde{R}(X)$ and $\tilde{Q}(X)$ are coprime. Moreover $\tilde{R}(X)$ is a
product of distinct linear factors over the residue field of $L$
at $\fP$. It follows that (\ref{E54}) can be lifted
to a factorization of $G(X)$ in the completion of $L(\gl)$ at $\fP$; and the
roots of $G(X)$ in this field consist of one near each root of each $P_i(X)$
together with roots which come (after a further field extension) from the lift
of $\tilde{Q}(X)$. The latter are not close to any root of any $P_i(X)$.

I now claim that the power of $\fP$ which divides $P_i(\gl)$ is $\fP^m$ where
$m$ is a multiple of $M$. For if $\gl$ is not close to a root of $P_i(X)$ then
$m=0$. On the other hand, (\ref{E2}) can be written
\[ G(X)=R_i(X)\psi_i(X)+f_i(X)P_i(X) \]
where
\[ f_i(X)=R_i(X)Q(X)+\sum_{j\neq i}\psi_j(X)R_j(X)/P_i(X). \]
By construction, if $\gl$ is close to a root of $P_i(X)$ then
$f_i(\gl)$ is a unit at $\fP$, as is $R_i(\gl)$. If
$\gl_i$ is that root of $P_i(X)$ which is close to $\gl$, then the
standard successive approximation process shows that $\gl-\gl_i$ has the
same valuation as $\psi_i(\gl_i)=\phi_i$; and by construction $\fP^m\|\phi_i$
where $M|m$. It follows that $\fP^m\|P_i(\gl)$ with $M|m$, as claimed, in both
cases.

Now let $\fp$ be a prime in $k$ which divides $c$, and let $\fq$ be any prime
of $k(\gl)$ above $\fp$. The factors of $P_i(\gl)$ coming from primes of
$L(\gl)$ above $\fq$ have the form
\beq{E55} \prod_{\fP|\fq}\fP^{m(\fP)} {\mathrm{~where~each~}}
m(\fP) {\mathrm{~is~divisible~by~}} M. \end{equation}
This is equal to the corestriction of $\fq^n$, where $\fq^n$ is the exact
power of $\fq$ which divides $P_i(\gl)$. But the 
extension $L(\gl)/k(\gl)$ is unramified at $\fq$, because it is only ramified
at places above places in $\fB$. Hence each $m(\fP)$ in (\ref{E55}) is equal to
$n$, and so $n$ is divisible by $M$. This holds for all primes in $k(\gl)$
which divide $c$.  \qed

\newpage

\noindent 3. \emph{The Legendre-Jacobi function}. \newline
If $\ga,\gb$ are elements of $k^*$ and $v$ is a
place of $k$, the \emph{Hilbert symbol} $(\ga,\gb)_v$ is defined by
\[ (\ga,\gb)_v= \begin{cases}
1 & \text{if $\ga X^2+\gb Y^2=Z^2$ is soluble in $k_v$}, \\
-1 & \text{otherwise}.
\end{cases} \]
The Hilbert symbol is symmetric in $\ga,\gb$. Its principal properties are
\begin{itemize}
\item $(\ga_1\ga_2,\gb)_v=(\ga_1,\gb)_v(\ga_2,\gb)_v$ and $(\ga,\gb_1\gb_2)_v
=(\ga,\gb_1)_v(\ga,\gb_2)_v$;
\item for fixed $\ga,\gb$, $(\ga,\gb)_v=1$ for almost all $v$, and
$\prod(\ga,\gb)_v=1$ where the product is taken over all places $v$ of $k$.
\end{itemize}
The first of these can be deduced from symmetry and
\[ (Z_1^2-\gb Y_1^2)(Z_2^2-\gb Y_2^2)=(Z_1Z_2+\gb Y_1Y_2)^2-\gb(Y_1Z_2+
Y_2Z_1)^2. \]
The second is one of the main results of class field theory.

Many of the proofs in these notes use the Legendre-Jacobi function $L$,
which is a mild modification of a function (also called $L$) which was
defined in rather crude form in [13] and more correctly in [14]. Let
$F(U,V),G(U,V)$ be homogeneous coprime square-free polynomials in $k[U,V]$.
Let $\sB$ be a finite set of places of $k$ containing the infinite places,
the primes dividing 2, those at which any coefficient of $F$ or $G$ is not
integral, and any other primes $\fp$ at which $FG$ does not remain separable
when reduced mod $\fp$. Note that we do not assume that $\sB$ contains a base
for the ideal class group of $k$.

Let $\sN^2=\sN^2(k)$ be the set of $\ga\times\gb$ with
$\ga,\gb$ integral and coprime outside $\sB$, and let $\sN^1=\sN^1(k)$ be
$k\cup\{\infty\}$. For $\ga\times\gb$ in $\bA^2(k)$ with $\ga,\gb$ not both
zero, we shall consistently
write $\gl=\ga/\gb$ with $\gl$ in $\sN^1(k)$. Provided
$F(\ga,\gb)$ and $G(\ga,\gb)$ are nonzero, we define the function
\beq{E18} L(\sB;F,G;\ga,\gb):\;\ga\times\gb\mapsto
\prod_\fp(F(\ga,\gb),G(\ga,\gb))_\fp  \end{equation}
on $\sN^2$, where the outer bracket on the right is the multiplicative Hilbert
symbol and the product is taken over all primes $\fp$ of $k$ outside
$\sB$ which
divide $G(\ga,\gb)$. By the definition of $\sB$, $F(\ga,\gb)$
is a unit at any such prime. Clearly we can restrict the product in
(\ref{E18}) to those $\fp$ which divide $G(\ga,\gb)$ to an odd power; thus we
can also write it as $\prod\chi_\fp(F(\ga,\gb))$ where $\chi_\fp$ is the
quadratic character mod $\fp$ and the product is taken over all $\fp$ outside
$\sB$ which divide $G(\ga,\gb)$ to an odd power. This relationship with the
quadratic residue symbol underlies the proof of Lemma \ref{L5}.
The function $L$ does depend on $\sB$, but the effect on the 
right hand side of (\ref{E18}) of increasing $\sB$ is obvious.
Some of the more interesting properties of $L$ only hold when
$\deg F$ is even, but this usually holds in applications;
this parity condition did not appear in [14], but it is already
needed if we are to make use of the results of [7].

In the course of the proofs, however, we need to
consider functions (\ref{E18}) with $\deg F$ odd; and for this reason it is
expedient to introduce
\[ M(\sB;F,G;\ga,\gb)=L(\sB;F,G;\ga,\gb)(L(\sB;U,V;\ga,\gb))^{(\deg F)(\deg G)}. \]
Here of course $L(\sB;U,V;\ga,\gb)=\prod(\ga,\gb)_\fp$ taken over all $\fp$
outside $\sB$ which divide $\gb$ and therefore do not divide $\ga$.
\ble{L5} The value of $M$ is continuous in the
topology induced on $\sN^2$ by $\sB$. For each $v$ in $\sB$ there
is a function $m(v;F,G;\ga,\gb)$ with values in $\{\pm1\}$ which is
continuous on $\sN^2$ in the $v$-adic topology, such that
\beq{E22} M(\sB;F,G;\ga,\gb)=\prod_{v\in\sB}m(v;F,G;\ga,\gb). \end{equation}.
\ele
\emph{Proof} If $\deg F$ is even, so that $M=L$, the neatest
proof of the lemma is by means of the evaluation formula in
[7], Lemma 7.2.4. When $\deg G$ is even but $\deg F$ may not
be, the result follows from (\ref{E8}), and (\ref{E47}) then
gives the general case. (The proof in [7] is for $k=\bQ$, but
there is not much
difficulty in modifying it to cover all $k$.)
However, the proof which we shall give, using the
ideas of [14], provides a more convenient method of evaluation.

For this proof we have to impose on $\sB$ the additional condition
that it contains all primes whose absolute norm does not exceed $\deg(FG)$. As
the proof in [7] shows, this condition is not needed for the truth of Lemma
\ref{L5} itself; but we use it in the proof of (\ref{E45})
below, and the latter is crucial to the subsequent argument. In any case, to
classify all small enough primes as bad is quite usual. We
repeatedly use the fact that $L(\sB;F,G)$ and $M(\sB;F,G)$ are multiplicative
in both $F$ and $G$; the effect of this is that we can reduce to the case when
both $F$ and $G$ are irreducible in $\fo_\sB[U,V]$, where $\fo_\sB$ is the
ring of elements of $k$ integral outside $\sB$. Introducing $M$ and
dropping the parity condition on $\deg F$ are not real generalizations
since if we increase $\sB$ so that
the leading coefficient of $F$ is a unit outside $\sB$ then
\beq{E47} M(\sB;F,G)=L(\sB;F,GV^{\deg G}) \end{equation}
by (\ref{E7}), and we can apply (\ref{E8}) to the right hand side.

It follows from the product formula for the Hilbert symbol that
\beq{E8} L(\sB;f,g;\ga,\gb)L(\sB;g,f;\ga,\gb)=
\prod_{v\in\sB}(f(\ga,\gb),g(\ga,\gb))_v, \end{equation}
provided that $f(\ga,\gb),\;g(\ga,\gb)$ are nonzero.
The right hand side of (\ref{E8}) is the product of continuous
terms each of which only depends on a single $v$ in $\sB$. This formula
enables us to interchange $F$ and $G$ when we want to, and in particular to
require that $\deg F\geq\deg G$ in the reduction process which follows. We
also have
\beq{E7} L(\sB;f,g;\ga,\gb)=L(\sB;f-gh,g;\ga,\gb) \end{equation}
for any homogeneous $h$ in $k[U,V]$ with $\deg h=\deg f-\deg g$ provided the
coefficients of $h$ are integral outside $\sB$, because
corresponding terms in the two products are equal. Both (\ref{E8}) and
(\ref{E7}) also hold for $M$.

We deal first with two special cases:
\begin{itemize}
\item $G$ is a constant. Now
$M(\sB;F,G)=1$ because all the prime factors of $G$ must be
in $\sB$, so that $M(\sB;F,G)=L(\sB;F,G)$ and
the product in the definition of $L(\sB;F,G)$ is empty.
\item $G=V$. Choose $H$ so that $F-GH=\gg U^{\deg F}$ for some
nonzero $\gg$. Now
$M(\sB;F,G)=1$ follows from the previous case and (\ref{E7}),
since all the prime factors of $\gg$ must be in $\sB$.
\end{itemize}
We now argue by induction on $\deg(FG)$.
Since we can assume that $F$ and $G$ are irreducible, we need only consider 
the case when
\[ \deg F\geq\deg G>0, \quad G=\gg U^{\deg G}+\ldots, \quad
F=\gd U^{\deg F}+\ldots \]
for some nonzero $\gg,\gd$. Let $\sB_1$ be obtained by adjoining to $\sB$
those primes of $k$ not in $\sB$ at which $\gg$
is not a unit. By (\ref{E7}) we have
\beq{E46} M(\sB_1;F,G)=M(\sB_1;F-\gg^{-1}\gd GU^{\deg F-\deg G},G).
\end{equation}
By taking a factor $V$ out of the middle argument on the right, and using
(\ref{E8}), the second special case above
and the induction hypothesis, we see that $M(\sB_1;F,G)$ is
continuous in the topology induced by $\sB_1$ and is a product taken over all
$v$ in $\sB_1$ of continuous terms each one of which depends on only one of
the $v$. Hence the same is true of
$M(\sB;F,G)$, because this differs from $M(\sB_1;F,G)$ by finitely many
continuous factors, each of which depends only on one prime in $\sB_1
\setminus\sB$.

But $\sB_1\setminus\sB$ only contains primes whose absolute norm is greater
than $\deg(FG)$. Thus by an integral unimodular transformation from $U,V$ to
$U,V_1$ we can arrange that $G=\gg_1U^{\deg G}+\ldots$ and $F=\gd_1U^{\deg F}
+\ldots$ where $\gg_1$ is a unit
at each prime in $\sB_1\setminus\sB$. Let $\sB_2$ be obtained from
$\sB$ by adjoining all the primes at which $\gg_1$
is not a unit; then $M(\sB;F,G)$ has the same properties with respect to
$\sB_2$ that we have already shown that it has with respect to $\sB_1$. Since
$\sB_1\cap\sB_2=\sB$, this implies that $M(\sB;F,G)$ already has these
properties with respect to $\sB$.  \qed

Of course there will be finitely many values of $\ga/\gb$ for which the right
hand side of (\ref{E8}) appears to be indeterminate; but by means of a
preliminary linear transformation on $U,V$ one can in fact ensure that the
formula (\ref{E22}) is meaningful except when $F(\ga,\gb)$ or
$G(\ga,\gb)$ vanishes.

When $\deg F$ is even, the value of $L(\sB;F,G;\ga,\gb)$ is already
determined by $\gl=\ga/\gb$ regardless of the values
of $\ga$ and $\gb$ separately; here $\gl$ lies in
$k\cup\{\infty\}$ with the roots of $F(\gl,1)$ and $G(\gl,1)$ deleted. We shall
therefore also write this function as $L(\sB;F,G;\gl)$. But note that
it is not necessarily a continuous function of $\gl$; see the discussions in
[13] and \S9 of [7], or Lemma \ref{L18} below. Moreover if $\sB$ does not
contain a base for the ideal class group of $k$ then not all
elements of $k\cup\{\infty\}$ can be written in the form $\ga/\gb$
with $\ga,\gb$ integers coprime outside $\sB$; so we have not yet defined
$L(\sB;F,G;\gl)$ for all $\gl$. To go further in the case when
$\deg F$ is even,
we modify the definition (\ref{E18}) so that it extends to all $\ga\times\gb$
in $k\times k$ such that $F(\ga,\gb)$ and $G(\ga,\gb)$ are nonzero. For any
such $\ga,\gb$ and any $\fp$ not in $\sB$, choose $\ga_\fp,\gb_\fp$ integral
at $\fp$,
not both divisible by $\fp$ and such that $\ga/\gb=\ga_\fp/\gb_\fp$. Write
\beq{E52} L(\sB;F,G;\ga,\gb)=
\prod(F(\ga_\fp,\gb_\fp),G(\ga_\fp,\gb_\fp))_\fp \end{equation}
where the product is taken over all $\fp$ not in $\sB$ such that
$\fp|G(\ga_\fp,\gb_\fp)$. This is a finite product whose value does not depend
on the choice of the $\ga_\fp$ and $\gb_\fp$; indeed it only depends on
$\gl=\ga/\gb$ and when $\ga,\gb$ are integers coprime outside $\sB$ it is the
same as the function given by (\ref{E18}). Thus we can again write it as
$L(\sB;F,G;\gl)$. This generalization is not really
needed until we come to (\ref{E13}); but at that stage we
cannot take account of the ideal class group of $K$ because
we need $\sB$
to be independent of $K$. Its disadvantage is that $L$ is no longer
necessarily a continuous function of $\ga\times\gb$; we investigate this
situation in more detail after the proof of Lemma \ref{L15}.

In discussing the continuity properties of $L$ as a function of $\gl$, we shall
need the following lemma.
\ble{L17} Let $\gl_0=\ga_0/\gb_0$ with $\ga_0,\gb_0$ non-zero and integral
outside	$\sB$; and let $\fa$ be an integral ideal in $k$ not divisible by
any prime in $\sB$. Then we can find $\ga,\gb$ in $k$, integral outside
$\sB$, with $(\ga,\gb)=\fa(\ga_0,\gb_0)$ and such that
$\ga\times\gb$ is arbitrarily close to $\ga_0\times\gb_0$ at each finite prime
in $\sB$, $\ga/\gb$ is arbitrarily close to $\ga_0/\gb_0$ at each infinite
place of $k$ and $\ga/\ga_0$ and $\gb/\gb_0$ are positive at each real infinite
place of $k$.
\ele
\emph{Proof} Let $\sS$ be the set of primes which divide $\ga_0$ or $\gb_0$.
We can write $\fa=(\gg_1,\gg_2)$ where $\gg_1$ and $\gg_2$ are units at every
prime in $\sB$ and both $\gg_1/\fa$ and $\gg_2/\fa$ are units at every prime in
$\sS$. Let $\gd$ in $\fo$, a
unit outside $\sB$, be such that $\ga_0\gd$ and $\gb_0\gd$ are in
$\fo$. Choose
positive coprime integers $a,b$ in $\bZ$ which are close to 1 at every finite
prime in $\sB$ and units at all the primes which divide $\gg_1$ or $\gg_2$;
and let $M,N$ be large positive integers. By writing $\ga_0\gd a^M/\gg_1$ in terms
of a base for $\fo/\bZ$ and changing the coefficients by elements of $\bQ$ which
are small at each finite prime in $\sB\cup\sS$ and $O(a)$ at the infinite place
of $\bQ$, we can obtain an integer $\ga_1$ in $\fo$ which is prime to $a$
and $\gg_2/\fa$ and such that $\ga_0\gd a^M/\ga_1\gg_1$ is close to 1
at each place in $\sB$ and $\ga_0$, $\ga_1$ are divisible by the same power
of $\fp$ for each $\fp$ in $\sS$. Similarly we can obtain $\gb_1$ in
$\fo$ which is prime to $b$ and $\gg_1/\fa$ and such that
$\gb_0\gd b^N/\gb_1\gg_2$ is close to 1 at each place of $\sB$ and $\gb_0$,
$\gb_1$ are divisible by the same power of $\fp$ for each $\fp$ in $\sS$.
We can further ensure that $\gb_1$ is
prime to $\ga_1$ outside $\sB\cup\sS$. Now $\ga=\ga_1 b^N\gg_1/\gd$ and
$\gb=\gb_1 a^M\gg_2/\gd$ satisfy all the requirements in the lemma. The
only difficult thing to verify is that $(\ga,\gb)=\fa(\ga_0,\gb_0)$. So far
as primes in $\sB$ are concerned, the two sides agree; and
\[ (\ga,\gb)=(\ga_1\gg_1,\gb_1\gg_2)=\fa(\ga_1(\gg_1/\fa),\gb_1(\gg_2
/\fa))=\fa(\ga_1,\gb_1) \]
up to such primes.  \qed

The proof of Lemma \ref{L5} constructs an evaluation formula all of whose terms
come from the right hand side of (\ref{E8}) for various pairs $f,g$. For
$\ga\times\gb$ in $\sN^2$, the
formula can therefore be described by an equation of the form
\beq{E45} m(v;F,G;\ga,\gb)=\prod_j(\phi_j(\ga,\gb),\psi_j(\ga,\gb))_v.
\end{equation}
Here the $\phi_j,\psi_j$ are homogeneous elements of $k[U,V]$ which depend
only on $F$ and $G$ and not on $v$ or $\sB$. The
decomposition (\ref{E45}) is not unique, and our next task is
to display an invariant aspect of it.

Let $\gt=\gg_1U+\gg_2V$ be a linear form with $\gg_1,\gg_2$ coprime integers
in $k$. By using $(\phi,\psi)_v=(\phi,\gt\psi)_v(\phi,\gt)_v$ and
$(-\gt,\gt)_v=1$, we can
ensure that all the $\phi_j,\psi_j$ in (\ref{E45}) have even degree except
that $\psi_0=\gt$. Denote by $\gT$ the group of elements of $k^*$ which are
not divisible to an odd power by any prime of $k$ outside $\sB$, and by
$\gT_0\subset\gT$ the subgroup consisting
of those $\xi$ which are quadratic residues mod $\fp$
for all $\fp$ outside $\sB$; thus we are free to multiply
$\phi_0$ by any element of $\gT_0$. (Actually
$\gT_0\subset k^{*2}$, but we shall not use this fact.)
\ble{L15} Suppose that $\deg F$ is even. With the convention for the
$\phi_j,\psi_j$ just adopted, we can take $\phi_0$ to be in $\gT$.
\ele
\emph{Proof} Let $\gg$ in $k^*$ be a unit outside
$\sB$, and apply (\ref{E45}) to the identity
\[ L(\sB;F,G;\gg\ga,\gg\gb)=L(\sB;F,G;\ga,\gb), \]
where $\ga\times\gb$ is in $\sN^2$. On cancelling common factors, we obtain
\beq{E48} \prod_{v\in\sB}(\phi_0(\ga,\gb),\gg)_v=1. \end{equation}
If we can choose $\ga\times\gb$ in $\sN^2$ so that
$\phi_0(\ga,\gb)$ is not in $\gT$, this gives a
contradiction. For let $\gd$
prime to $\phi_0(\ga,\gb)$ be such
that $\prod(\phi_0(\ga,\gb),\gd)_\fp=-1$ where the product is taken over all
primes $\fp$ outside $\sB$ at which $\phi_0(\ga,\gb)$ is not a unit. Let
$\sB_1$ be obtained by 
adjoining to $\sB$ all the primes at which $\gd$ is not a unit; then
$\prod(\phi_0(\ga,\gb),\gd)_v=-1$ by the Hilbert product formula, where
the product is taken over all places $v$ in $\sB_1$. Recalling that $\phi_0$
does not depend on $\sB$ and writing
$\sB_1,\gd$ for $\sB,\gg$ in (\ref{E48}), we obtain a contradiction.
It follows that $\phi_0(\ga,\gb)$ lies in $\gT$
for all $\ga,\gb$; this can
only happen if $\phi_0(U,V)$ is itself in $\gT$ modulo
squares of elements of $k[U,V]$.  \qed

Let $\sS$ be the set of primes $\fp$ outside $\sB$ for which $\fp|F(\ga_\fp,\gb_\fp)$
or $\fp|G(\ga_\fp,\gb_\fp)$ in the notation of (\ref{E52}). We can write
$\gl=\ga/\gb$ where $(\ga,\gb)$ is not divisible by any prime in $\sS$.
Let $\fa$ be an integral ideal in the class of $(\ga,\gb)$ not
divisible by any prime in $\sS$, and let $\gg$ be such that $(\gg)=\fa/(\ga,\gb)$;
then $\gl=\ga\gg/\gb\gg$ and $(\ga\gg,\gb\gg)=\fa$. If $\sB_1$ is obtained from $\sB$
by adjoining all the primes which divide $\fa$, then
\[ L(\sB;F,G;\gl)=L(\sB;F,G;\ga\gg,\gb\gg)=L(\sB_1;F,G;\ga\gg,\gb\gg), \]
where the second equality holds because the two products involved are term
by term the same. By (\ref{E45}) the right hand side is
equal to
\begin{align*}
\prod_{v\in\sB_1} & \prod_j(\phi_j(\ga\gg,\gb\gg),\psi_j(\ga\gg,\gb\gg))_v  \\
 & =\left\{\prod_{v\in\sB_1}\prod_j(\phi_j(\ga,\gb),\psi_j(\ga,\gb))_v\right\}
\prod_{v\in\sB_1}(\phi_0,\gg)_v
\end{align*}
because of the parity properties above. If we further require that no prime
which divides $\fa$ divides any of the $\phi_j(\ga,\gb)$ or $\psi_j(\ga,\gb)$,
then each of the terms in curly brackets with $v$ in $\sB_1\setminus\sB$ is
trivial;
so the outer product there reduces to a product over $v$ in $\sB$. By the
Hilbert product formula the product outside the curly brackets can be
replaced by a product over all $v$ not in $\sB_1$. In view of Lemma \ref{L15}
we can reduce this to a product over those $v$ outside $\sB_1$ which divide
$(\ga,\gb)$. If $\chi_\fp$ is again the quadratic residue symbol mod $\fp$, we
can write the result which we have just obtained in the form
\beq{E53} L(\sB;F,G;\gl)=\left\{\prod_{v\in\sB}\prod_j(\phi_j(\ga,\gb),
\psi_j(\ga,\gb))_v\right\}\prod\chi_\fp(\phi_0) \end{equation}
where the final product is taken over those $\fp$ outside $\sB$ which divide
$(\ga,\gb)$ to an odd power.
\ble{L18} Suppose that $\deg F$ is even and the conventions of
Lemma $\ref{L15}$ hold. Then $\phi_0$ is uniquely determined by $F$ and $G$
as an element of $\gT/\gT_0$; and $\phi_0$ is in $\gT_0$ if and only if
$L(\sB;F,G;\gl)$ is continuous in $\gl$ in the topology induced by $\sB$.
\ele
\emph{Proof} Suppose first that $\phi_0$ is in $\gT_0$. Thus the final
product in (\ref{E53}) is trivial. Now let $\gl=\ga/\gb$ and
let $\gl'$ be close to $\gl$ in the topology induced by $\sB$. Let $\gg$
in $\fo$ be such that $\gl'\gb\gg$ is integral. Applying (\ref{E53}) to the
representations
\[ \gl=\ga\gg/\gb\gg \quad {\mathrm{and}} \quad \gl'=\gl'\gb\gg/\gb\gg \]
we deduce that $L(\sB;F,G;\gl)=L(\sB;F,G;\gl')$.

Conversely suppose that $\phi_0$ is in $\gT$ but not in $\gT_0$. Choose
a prime $\fp$ outside $\sB$ at which $\phi_0$ is not a quadratic residue.
As before, let $\gl_0=\ga_0/\gb_0$, and let $\gl=\ga/\gb$ where $\ga,\gb$
have the properties stated in Lemma \ref{L17} with $\fa=\fp$.
Arguing as in the previous paragraph,
but taking account of the final product in (\ref{E53}), we obtain
\[ L(\sB;F,G;\gl)=L(\sB;F,G;\gl_0)\chi_\fp(\phi_0)=-L(\sB;F,G;\gl_0). \]
So $L(\sB;F,G;\gl)$ is not continuous at $\gl=\gl_0$ --- which means that it
is continuous nowhere.

Now suppose that $L(\sB;F,G;\ga,\gb)$ has two representations, say by
the $\phi'_i,\psi'_i$ and the $\phi''_j,\psi''_j$. Taking their quotient, we
obtain
\[ 1=\prod_{v\in\sB}\left\{(\phi'_0/\phi''_0,\gt(\ga,\gb))_v\prod_{i>0}
(\phi'_i(\ga,\gb)
,\psi'_i(\ga,\gb))_v\prod_{j>0}(\phi''_j(\ga,\gb),\psi''_j(\ga,\gb))_v
\right\}. \]
This is a representation of a function of $\gl$ which is
continuous; and it is of a kind
to which we can apply the results of the previous two paragraphs. Hence
$\phi'_0/\phi''_0$ is in $\gT_0$.

It remains only to show that $\phi_0$ is independent of the choice of $\gt$.
Using a notation like that of the previous paragraph, there is a representation
of 1 in which the terms with subscript 0 produce a quotient
\[ \prod_{v\in\sB}\{(\phi'_0/\phi''_0,\gt')_v(\phi''_0,\gt'\gt'')_v\}; \]
and since $\deg(\gt'\gt'')$ is even it follows as there
that $\phi'_0/\phi''_0$ is in $\gT_0$.
\qed

If deg $F$ or deg $G$ is 0 or 1, it is easy to obtain an
evaluation formula;
so the first case of interest is when $\deg F=\deg G=2$. Suppose that
\beq{E58} F=a_1U^2+b_1UV+c_1V^2, \quad G=a_2U^2+b_2UV+c_2V^2
\end{equation}
and that $\sB$ contains the infinite places and the primes which divide 2 or
\[ R=(a_1c_2-a_2c_1)^2-b_1b_2(a_1c_2+a_2c_1)+a_1c_1b_2^2+a_2c_2b_1^2, \]
the resultant of $F$ and $G$. Suppose also that $\eta\times\zeta$ and $\rho
\times\gs$ are in $\sN^2$. Then
\begin{align*} L(\sB; & F,G;\eta,\zeta)L(\sB;F,G;\rho,\gs)= \\
 & \prod_{v\in\sB}\{(f/(\gs\eta-\rho\zeta),R)_v(fG(\rho,\gs),
-fG(\eta,\zeta))_v\}
\end{align*}
where
\[ f=F(\eta,\zeta)G(\rho,\gs)-F(\rho,\gs)G(\eta,\zeta). \]
In accordance with Lemma \ref{L15}, the value of $R$ is in $\gT$. If we set
$\rho,\gs$ to convenient values, this gives the value of
$L(\sB;F,G;\eta,\zeta)$.

\medskip

In practice, what we usually need to study is the subspace of $\sN^2$ given by
$n$ conditions $L(\sB;F_\nu,G_\nu;\ga,\gb)=1$, or the subspace of
$\sN^1$ given by the $L(\sB;F_\nu,G_\nu;\gl)=1$, where the $\deg F_\nu$ are
all even. Let $\gL$ be the abelian group
of order $2^n$ whose elements are the $n$-tuples each component of which is
$\pm1$; then there is a natural identification, which we shall write $\tau$,
of each element of $\gL$ with a partial product of the $L(\sB;F_\nu,G_\nu)$.
Thus each element of $\gL$ can be interpreted as a condition, which we shall
write as $\sL=1$. If
$\phi_0$ is as in Lemma \ref{L15}, there is a homomorphism
\[ \phi_0\circ\tau: \gL\rightarrow\gT/\gT_0; \]
let $\gL_0$ denote its kernel. In view of Lemma \ref{L18}, the conditions
which are continuous in $\gl$ are just those which come from
$\gL_0$. The following lemma corresponds to Harari's Formal Lemma (Theorem
3.2.1 of [7]); it shows that for most purposes we need
only consider the conditions coming from the elements of
$\gL_0$. For obvious reasons, we call these the
\emph{continuous} conditions.
\ble{L16} Suppose that every $\deg F_\nu$ is even and all the
conditions corresponding to $\gL_0$ hold at some given
$\gl_0$. Then there exists $\gl$ arbitrarily
close to $\gl_0$ such that all the conditions
$L(\sB;F_\nu,G_\nu)=1$ hold at $\gl$.
\ele
\emph{Proof}
Let $\gl_0=\ga_0/\gb_0$. For a suitably chosen $\fa=(\gg)$ we show that
we can take $\gl=\ga/\gb$, where $\ga\times\gb$ is as in Lemma
\ref{L17}. For any $c$ in $\gL$, write $\phi_{0c}=\phi_0\circ\tau(c)$
for the corresponding element of $\gT/\gT_0$. If
$\theta$ is as defined just before Lemma \ref{L15}, the
corresponding partial product $\sL$ of the
$L(\sB;F_\nu,G_\nu;\gl)$ is equal to
\[ f_c(\gl)\prod_{v\in\sB}(\phi_{0c},\theta(\ga_0,\gb_0))_v
\prod_{v\in\sB}(\phi_{0c},\gg)_v \]
where $f_c$ comes from the $\phi_j,\psi_j$ with $j>0$ and is therefore
continuous. The map $c\mapsto f_c(\gl)$ is a homomorphism $\gL\rightarrow
\{\pm1\}$ for any fixed $\gl$; moreover if two distinct $c$ give rise to the
same $\phi_{0c}$ their quotient comes from an element of
$\gL_0$; so the quotient of the corresponding $f_c$
takes the value 1 at $\gl_0$. In other words, if $\gl$ is close
enough to $\gl_0$ then $f_c(\gl)$ only depends on the class of $c$ in
$\gL/\gL_0$. The map $c\mapsto\phi_{0c}$ is an embedding $\gL/\gL_0
\rightarrow\gT/\gT_0$, by Lemma \ref{L18}. The homomorphism
Image$(\gL/\gL_0)\rightarrow\{\pm1\}$ induced by $c\mapsto f_c(\gl)$ can be
extended to a homomorphism $\gT/\gT_0\rightarrow\{\pm1\}$ because
$\gT/\gT_0$ is killed by 2; and
any such homomorphism can be written in the form
\[ \theta\rightarrow\prod_{v\in\sB}(\theta,\gg)_v \]
for a suitably chosen $\gg$, because the Hilbert symbol induces a
nonsingular form on $\gT/\gT_0$. But 
given any such $\gg$ we can construct
$\gl=\ga/\gb$ having the properties listed in Lemma \ref{L17} with
$\fa=(\gg)$.  \qed

We shall need analogues of these last results for positive 0-cycles, and this
will require more notation. We continue to assume that $\deg F$ is even.
Let $K$ be the direct product of finitely many
fields $k_i$ each of finite degree over $k$, and let $\fB$ be the set of
places of $K$ lying over some place $v$ in $\sB$, and $\fB_i$ the
corresponding set of places of $k_i$. (The place $\prod v_i$,
where $v_i$ is a place of $k_i$, lies over $v$ if each $v_i$ does so.) For
$\gl$ in $\bP^1(K)$ write $\gl=\prod\gl_i$ with $\gl_i$ in $\bP^1(k_i)$; for
each place $w$ in $k_i$ write $\gl_i=\ga_{iw}/\gb_{iw}$ where $\ga_{iw},
\gb_{iw}$ are in $k_i$ and integral at $w$
and at least one of them is a unit at $w$.
For any $\gl$ in $K$ such that each $F(\gl_i,1)$
and $G(\gl_i,1)$ is nonzero, we define the function
\beq{E13} L^*(\sB;K;F,G;\gl):\,\gl\mapsto\prod_{\fP_i}(F(\ga_{iw},\gb_{iw}),
G(\ga_{iw},\gb_{iw}))_{\fP_i} \end{equation}
where $w$ is the place associated with the prime $\fP_i$ in $k_i$ and
the product is taken over all $i$ and all primes $\fP_i$ of $k_i$ not
lying in $\fB_i$ and such that $G(\ga_{iw},\gb_{iw})$ is divisible by $\fP_i$.
As with (\ref{E18}), we can restrict the product to those $\fP_i$ which divide
$G(\ga_{iw},\gb_{iw})$ to an odd power.
Note that the functions $\phi_j,\psi_j$ in the evaluation formula (\ref{E45})
are the same for $k_i\supset k$ as they are for $k$.
Now let $\fa$ be a positive 0-cycle on $\bP^1$
defined over $k$ and let $\fa=\cup\fa_i$
be its decomposition into irreducible components. Let $\gl_i$
be a point of $\fa_i$ and write $k_i=k(\gl_i)$. If $K=\prod k_i$ and $\gl=
\prod\gl_i$, write
\beq{E21} L^*(\sB;F,G;\fa)=L^*(\sB;K;F,G;\gl)={\prod}_iL(\fB_i;F,G;\gl_i).
\end{equation}
This is legitimate, because the right hand side does not depend on the choice
of the $\gl_i$.
If $K=k$ this $L^*$ is the same as the previous function $L$.
Moreover $L^*(\fa\cup\fb)=L^*(\fa)L^*(\fb)$.
We can define a topology on the set of positive 0-cycles $\fa$ of given degree
$N$ by means of the isomorphism between that set and the points on the $N$-fold
symmetric power of $\bP^1$. With this topology,
it is straightforward to extend to $L^*$ the
results already obtained for $L$.

The product in (\ref{E13}) is finite; so there is a finite set $\sS$ of
primes of $k$, disjoint from $\sB$ and such that every $\fP_i$ which appears
in this product lies above a prime in $\sS$. For each $i$ we can write
$\gl_i=\ga_i/\gb_i$ with $\ga_i,\gb_i$ integers in $k_i$. As in the
argument which follows the proof of Lemma \ref{L15},
let $(\ga_i,\gb_i)=\fa_i$ and choose an
integral ideal $\fb_i$ in $k_i$ which is prime to $\fa_i$, in the same ideal
class as $\fa_i$ and such that no prime of $k_i$ which divides $\fb_i$ also
divides $G(\ga_i,\gb_i)$ or any $\phi_j(\ga_i,\gb_i)$ or $\psi_j(\ga_i,\gb_i)$
or lies above any prime in $\sS$. Let $\gg_i$ be such that $(\gg_i)=\fb_i/\fa_i$
and let $\sB_1$ be obtained from $\sB$ by adjoining all the primes of $k$
which lie below any prime of $k_i$ which divides $\fb_i$. For most purposes
it costs us nothing to replace $\sB$ by $\sB_1$, and we then have
\[ \gl=\prod\gl_i=\prod(\ga_i\gg_i/\gb_i\gg_i) {\mathrm{~where~}}
\ga_i\gg_i\times\gb_i\gg_i {\mathrm{~is~in~}} \sN^2(k_i). \]

The following lemma is a trivial consequence of earlier results.
\ble{L4} Suppose that $\deg F$ is even, and let $\sL=1$ be a continuous
condition derived from the $L$ and $\sL^*=1$ the corresponding condition derived
from the $L^*$.
For each $v$ in $\sB$ there is a function $\ell^*(v;F,G;\fa)$ with
values in $\{\pm1\}$ which is a continuous function of $\fa$ in the $v$-adic
topology and is such that
\beq{E23} \sL^*(\sB;F,G;\fa)=\prod_{v\in\sB}\ell^*(v;F,G;\fa). \end{equation}
\ele

\newpage

\noindent 4. \emph{Pencils of conics}. \newline
Let $W$ be the surface fibred by the pencil of conics
\beq{E3} a_0(U,V)Y_0^2+a_1(U,V)Y_1^2+a_2(U,V)Y_2^2=0. \end{equation}
We normally expect this pencil to be presented in a form in which $a_0,a_1,a_2$
are homogeneous of the same degree. But this is not the most convenient form
for the arguments which follow. Instead we shall call the pencil
\emph{reduced} if $a_0,a_1,a_2$ are homogeneous elements of $k[U,V]$ coprime
in pairs and such that
\[ \deg a_0\equiv\deg a_1\equiv\deg a_2 {\mathrm{~mod}}\,2. \]
After a linear transformation on $U,V$ if necessary, we can also assume that
$a_0a_1a_2$ is not divisible by $V$.
Clearly any pencil of conics can be put into reduced form; for if $a_i$ has a
squared factor $f^2$ we write $f^{-1}Y_i$ for $Y_i$, and if for example $a_0$
and $a_1$ have a common factor $g$ we write $gY_2$ for $Y_2$ and divide
(\ref{E3}) by $g$. Suppose that
(\ref{E3}) is reduced and everywhere locally soluble. Let
$\gl=\ga/\gb$ be a point of $\bP^1(k)$; whether (\ref{E3})
is soluble at $\ga\times
\gb$ depends only on $\gl$ and not on the choice of $\ga,\gb$. Similar
statements hold for local solubility at a place $v$ and for solubility in the
adeles. Denote by $c(U,V)$ an irreducible factor of $a_0a_1a_2$ in
$k[U,V]$; we can assume that $c(U,V)$ has integer coefficients
whose highest common factor is not divisible by any prime
outside $\sB$. Let $\sB$
be a finite set of places of $k$ containing the infinite places, the primes
dividing 2, those whose absolute norm does not exceed $\deg(a_0a_1a_2)$,
those at which any coefficient of any $a_i$ or any $c$ is not integral,
and any other primes $\fp$ at which $a_0a_1a_2$ does not remain separable when
reduced mod $\fp$. One effect of this definition is that we need only check
local solubility at the places of $\sB$, because it is trivial at any other
prime. Local solubility of (\ref{E3}) at the place $v$ is
equivalent to $(-a_0a_1,-a_0a_2)_v=1$, which can be written in
the more symmetric form
\beq{E78} (a_0,-a_1)_v(a_1,-a_2)_v(a_2,-a_0)_v=(-1,-1)_v.
\end{equation}
For convenience, we also assume that $\sB$ contains a base
for the ideal class group of $k$.

The singular fibres of the pencil are given by the values of $\gl$ at which
$a_0a_1a_2$ vanishes. If there is a singular fibre defined over $k$, then
(\ref{E3}) is certainly soluble on it; but little if any of the argument
which follows makes sense there. We must therefore work not on $\bP^1$ but
on the subset $\bL^1$ obtained by deleting the zeros of $a_0a_1a_2$, and not
on $W$ but on $W_0$, the inverse image of $\bL^1$ in $W$.
Let $\gl\in k\cup\{\infty\}$ be a point of
$\bL^1(k)$, and write $\gl=\ga/\gb$ where $\ga,\gb$ are integers of $k$
coprime outside $\sB$; it will not matter which pair
$\ga,\gb$ we choose. Similar conventions will hold for other $\bL^1(\cdot)$.

There is a non-empty set $\sN\subset\bL^1(k)$, open in
the topology induced by $\sB$, such that the conic (\ref{E3}) is soluble at
every place of $\sB$ if and only if $\gl$ lies in $\sN$. Let $\fp$ be a prime
of $k$ not in $\sB$ and consider the solubility of (\ref{E3}) in $k_\fp$ at
the point $\gl$. If none of
the $a_i(\ga,\gb)$ is divisible by $\fp$, then local solubility of (\ref{E3})
is trivial. Otherwise there is just one $c$ such
that $c(\ga,\gb)$ is divisible by $\fp$; to fix ideas,
suppose that this $c$ divides $a_2$. The condition for local
solubility at $\fp$ is then
\beq{E4} (-a_0(\ga,\gb)a_1(\ga,\gb),c(\ga,\gb))_\fp=1 \end{equation}
where the outer bracket is the multiplicative Hilbert symbol. Hence necessary
conditions for the local solubility of (\ref{E3}) at $\gl$ for all $\fp$
outside $\sB$ are the conditions like
\beq{E9} L(\sB;-a_0a_1,c;\gl)=\prod(-a_0(\ga,\gb)a_1(\ga,\gb),c(\ga,\gb))_\fp
=1 \end{equation}
where the product is taken over all $\fp$ outside $\sB$ which divide $c(\ga,
\gb)$, and the function $L$ is well defined since $-a_0a_1$ has even
degree. There is one of these conditions for each $c$.

What makes the set of conditions (\ref{E9}) interesting is that they give not
merely a necessary but also a sufficient condition for solubility --- at least
if one assumes Schinzel's Hypothesis. The following theorem provides the exact
obstruction both to the Hasse principle and to weak approximation. In view of
Lemma \ref{L16}, it is enough to require the continuous conditions
derived from the conditions (\ref{E9}) to hold; and the
resulting $\sA$ is both open and closed.
\bth{T4} Assume Schinzel's Hypothesis. Let $\sA\subset\sN$ be the subset of
$\bL^1(k)$ at which all the continuous conditions derived from
$(\ref{E9})$ hold and $(\ref{E3})$
is locally soluble at each place in $\sB$. Then the $\gl$ in $\bL^1(k)$ at
which $(\ref{E3})$ is soluble form a dense subset of $\sA$ in the topology
induced by $\sB$.
\eth
\emph{Proof} Let $\ga_0\times\gb_0$ correspond to a point $\gl_0$ in $\sA$,
and let $\sN_0\subset\sA$ be an open neighbourhood of $\gl_0$. We have to show
that we can find $\gl_2$ in $\sN_0$ such that (\ref{E3}) is soluble at $\gl_2$;
for this it is enough to show that (\ref{E3}) is everywhere locally
soluble there. Let $c_i$ run through the factors $c$.
By Lemma \ref{L16} we can find $\ga_1,\gb_1$
in $k^*$, integral and coprime outside $\sB$ and such that
$\gl_1=\ga_1/\gb_1$ is in $\sN_0$ and all the conditions
(\ref{E9}) hold at $\ga_1\times\gb_1$. By Lemma \ref{L1} we
can now find $\ga_2\times\gb_2$ close to $\ga_1\times\gb_1$
and such that each ideal $(c_i(\ga_2,\gb_2))$ is the
product of a prime ideal
$\fp_i$ not in $\sB$ and prime ideals in $\sB$. We claim that
(\ref{E3}) is everywhere locally soluble at $\ga_2\times\gb_2$.
Since $\sN_0\subset\sA$, local solubility at each place of
$\sB$ is automatic. If $\fp$ is a prime outside $\sB$ which
does not divide any of the $a_j(\ga_2,\gb_2)$ then
(\ref{E3}) at $\ga_2\times\gb_2$ is certainly soluble at
$\fp$; so it only remains to consider the
$\fp_i$. To fix ideas, suppose that $c_i(U,V)$ is a factor of
$a_2(U,V)$. Taking $\ga=\ga_2, \gb=\gb_2$ and $c=c_i$, the
product in (\ref{E9}) reduces
to the single term with $\fp=\fp_i$. In other words, (\ref{E4}) holds in this
case, and this proves local solubility at $\fp_i$.  \qed

\medskip

The corresponding theorem for positive 0-cycles, or equivalently for
0-cycles of degree 1, does not require Schinzel's Hypothesis; instead we use
Lemma \ref{L2} and the notation introduced at (\ref{E13}).
We apply Lemma \ref{L2} to the surface $W_0$ fibred by the pencil (\ref{E3}),
again assuming that $\sB$ satisfies the conditions listed after (\ref{E3})
and that $\bL^1$ has the same meaning as there.
\bth{T3} With the notation above, let $N\geq\deg(a_0a_1a_2)$ be a fixed
integer. Let $\fa$ be a positive $0$-cycle of degree N on $\bL^1$
defined over $k$,
and for each place $v$ of $k$ suppose that $W_0$ contains a positive
$0$-cycle $\fb_v$ of degree $N$ defined over $k_v$; for $v$ in $\sB$ suppose
further that $\fb_v$ is so chosen that its
projection on $\bL^1$ is $\fa$. If all the continuous conditions derived from
the conditions
\beq{E17} L^*(\sB;-a_0a_1,c;\fa)=1 \end{equation}
hold, then there is a positive $0$-cycle of degree $N$ on $W_0$
defined over $k$ whose projection
is arbitrarily close to $\fa$ in the topology induced by $\sB$.
\eth
\emph{Proof} We must first show that for the purpose of proving this theorem
we are allowed to increase $\sB$. Suppose that $\sB_0$ satisfies the conditions
which were imposed on $\sB$ after (\ref{E3}), and let $\fp$ be a prime of $k$
not in $\sB_0$. Suppose also that the hypotheses of the theorem hold for
$\sB=\sB_0$ and $\fa=\fa_0$. Having chosen $\fb_\fp$ we can
find a positive 0-cycle $\fa'$ on $\bL^1$ of degree $N$ and defined over $k$
which is close at every $v$ in $\sB_0$ to $\fa$
and close at $\fp$ to the projection of
$\fb_\fp$. Now
\[ L^*(\sB_0\cup\{\fp\};-a_0a_1,c;\fa')=L^*(\sB_0;-a_0a_1,c;\fa'); \]
for writing both sides as products by means of (\ref{E13}), if there is a
factor on the right hand side which is not present on the
left, that factor must come from $\fp$ and is therefore equal to 1.
But a continuous condition for $\sB_0$ holds at $\fa'$ if and only if it
holds at $\fa$, which it does by hypothesis. Hence the continuous conditions
for $\sB_0\cup\{\fp\}$ hold at $\fa'$. Now suppose that the theorem
holds for $\sB_0\cup\{\fp\}$; then there
is a positive 0-cycle $\fb$ of degree $N$ on $W_0$ defined over $k$ whose
projection on $\bL^1$ is close to $\fa'$ in the topology induced by $\sB_0
\cup\{\fp\}$. The same projection is close to $\fa$ in the topology induced by $\sB_0$.
So the theorem also holds for $\sB_0$.

Note that if $\fa$ is actually the projection of a positive 0-cycle of degree $N$
in $W_0$, then the continuous conditions certainly hold in
view of (\ref{E21}); thus imposing the hypothesis that they all hold costs
us nothing. To simplify the notation, we assume henceforth
that $K$ is an algebraic number field;
this will be true for the application in this article
because $K$ will be constructed by means of Lemma \ref{L2}.
In view of the previous paragraph, we can assume that $\sB$
is so large that it satisfies the conditions imposed on $\fB$
in the statement of Lemma \ref{L2} and contains the additional
place $w$ which was adjoined to $\fB$ in the first
paragraph of the proof of Lemma \ref{L2}; and if
$b$ is as in Lemma \ref{L2} we also adjoin to $\sB$ all the primes in $k$
which divide $b$. By the analogue of Lemma \ref{L16}, we can now choose
$\fa''$ close to $\fa$ so that all the conditions like
$L^*(\sB;-a_0a_1,c;\fa'')=1$ hold. As was remarked in the
previous paragraph, we can now increase $\sB$ so that if
$\gl_0=\ga_0/\gb_0$ is a
point of $\bL^1(K)$ in $\fa''$ then $\ga_0,\gb_0$ are coprime and integral
except perhaps at primes of $K$ above a prime in $\sB$. Now apply
Lemma \ref{L2} with $M=2$, where we take the $c(X,1)$, normalized to be monic,
to be the $P_i(X)$ and each $U_v$ to be a small neighbourhood of the
monic polynomial
whose roots determine $\fa''$. Let $G(X)$ be given by Lemma \ref{L2}; let
$\fa'$ be the associated 0-cycle on $\bL^1(k)$ and $\gl$ a point of $\bL^1(K)$
in $\fa'$. For each $v$ in $\sB$, the
cycle $\fa'$ is close to $\fa''$ in the $v$-adic topology; so (\ref{E3}) at
$\gl$ is soluble in $K_w$ for each $w$
above $v$, by continuity. But $\gl=\ga/\gb$ with $\ga,\gb$ coprime except
at primes of $K$ above a prime of $\sB$. So
\[ \prod_\fP(-a_0(\ga,\gb)a_1(\ga,\gb),c(\ga,\gb))_\fP=
L^*(\sB;-a_0a_1,c;\ga,\gb)=1, \]
where the product
is taken over all primes $\fP$ not above a prime in $\sB$ and such that
$c(\ga,\gb)$ is divisible to an odd power by $\fP$.
Here the first equality holds
by definition and the second one follows from the evaluation formula
(\ref{E45}) by continuity. But
if $c(X,1)=P_i(X)$ then the product on the left reduces to the single term
for which $\fP$ is the prime of $K$ above $\fp_i$ whose existence was proved by
means of (\ref{E10}). Hence (\ref{E3}) at $\gl$ is locally soluble at
this prime; and
because these are the only primes not lying above a prime of $\sB$ which
divide any $c(\ga,\gb)$ or any $a_i(\ga,\gb)$ to an odd power, they are
the only primes not lying above a prime of $\sB$ at which
local solubility might present any difficulty. Thus $\gl$ can
be lifted to a point of the fibre above $\gl$, which is a
conic, and the theorem now follows
because weak approximation holds on conics.  \qed

Since (\ref{E3}) contains positive 0-cycles of degree 2 defined over $k$, it
is trivial to deduce from Theorem \ref{T3} the corresponding result for
0-cycles of degree 1; conversely, if we know the analogue of Theorem
\ref{T3} for 0-cycles of degree 1 we can deduce that (\ref{E3}) contains
positive 0-cycles of some odd degree defined over $k$. It is tempting to hope
that if a pencil of conics
contains 0-cycles of degree 1 then it contains points; indeed, the
corresponding result is true for Del Pezzo surfaces of degree 4, as is proved
in Theorem \ref{T5}. But this hope
is false. A simple example is given by the pencil
\beq{E35} Y_0^2+Y_1^2-7(U^2-UV-V^2)(U^2+UV-V^2)(U^2-2V^2)Y^2_2=0.
\end{equation}
This is insoluble in $\bQ$. For we can take $\sB=\{\infty,2,3,5,7\}$, and the
three possible $c(U,V)$ are $U^2-UV-V^2$, $U^2+UV-V^2$ and $U^2-2V^2$. By
(\ref{E8}) we have
\[ L(\sB;-1,c)=(-1,c)_\infty(-1,c)_2(-1,c)_7, \]
the factors at 3 and 5 being trivial. Local solubility of (\ref{E35}) holds
at each place; at $\ga\times\gb$ local solubility at 2 and
at 7 requires respectively that $4|\ga$ and
$\ga^2-2\gb^2$ is divisible by an odd power of 7. Hence
\[ (-1,\ga^2\pm\ga\gb-\gb^2)_2=-1, \quad (-1,\ga^2-2\gb^2)_2=-1 \]
and
\[ (-1,\ga^2\pm\ga\gb-\gb^2)_7=1, \quad (-1,\ga^2-2\gb^2)_7=-1. \]
To satisfy the conditions (\ref{E9}) we therefore need
\[ (-1,\ga^2\pm\ga\gb-\gb^2)_\infty=-1, \quad (-1,\ga^2-2\gb^2)_\infty=1; \]
but this is equivalent to $\ga^2\pm\ga\gb-\gb^2<0<\ga^2-2\gb^2$, which is
impossible. Now let $K=\bQ(\rho)$ where $\rho=2\cos(2\pi/7)$, so that
$\rho^3+\rho^2-2\rho-1=0$. If $U=\rho^2+2\rho-3$ and $V=\rho^2+\rho-2$ then
\[ Y_0=(\rho-2)^2(\rho^2-\rho+1),\;Y_1=(\rho-2)^2(\rho^2-1),\;Y_2=1 \]
gives a solution in $K$.

\medskip

It was asserted in the Introduction that on pencils of conics the appropriate
Brauer-Manin condition is a
necessary and sufficient condition for the Hasse principle and for weak
approximation (in each case subject to Schinzel's Hypothesis) and for the
existence of
positive 0-cycles of degree $N$ for all large enough $N$. This is the same
as saying that the appropriate Brauer-Manin condition is equivalent to the
necessary and sufficient conditions stated in Theorems \ref{T4} and \ref{T3}.
That is the content of the following lemma.
\ble{L7} Let $W_0$ be everywhere locally soluble. Then the continuous
conditions derived from $(\ref{E3})$
are collectively equivalent to the Brauer-Manin conditions for the existence
of points of $W_0$ defined over $k$. The continuous conditions similarly
derived from the $L^*(\fa)$
are collectively equivalent to the Brauer-Manin conditions for the existence of
positive $0$-cycles of degree $N$ on $W_0$ defined over $k$.
\ele
\emph{Proof} The first assertion is proved for $k=\bQ$ in [7], \S8; as with
Lemma \ref{L5}, the proof there can easily be extended
to our more general case. The second sentence follows trivially from the
first in the light of (\ref{E21}).  \qed

\newpage

\noindent 5. \emph{Descent on certain curves of genus $1$}.
\newline
Throughout this section, we shall be concerned with an elliptic curve $E$
which is defined over an algebraic number field $k$ and has all its 2-division
points rational. Such a curve can be written in the form
\beq{E27} E:\; Y^2=(X-c_1)(X-c_2)(X-c_3), \end{equation}
where without loss of generality we can assume that the $c_i$
are integers. We mainly discuss 2-descent on $E$, but there
is a brief mention of 4-descent at the end of the section.
The first usable exposition of 4-descent is due to Cassels
[2]. He showed, without any
assumption about the 2-division points, that a 4-descent
requires no bigger a field extension than a 2-descent. We
shall state his
algorithm, without proof, for the particular case (\ref{E27}).

The classical theory of
2-descent is expounded in the next few paragraphs. But there is also a well
concealed symmetry property, stated in Theorem \ref{T6}, and the
proof of this requires extra apparatus. This extra apparatus,
suitably modified, enables us to prove some results about the
effect of twisting on the 2-Selmer group of $E$.


The odd primes of bad reduction for $E$ are those which divide
\[ R=(c_1-c_2)(c_2-c_3)(c_3-c_1). \]
We shall need several distinct sets of bad places of $k$, the first two being
independent of $E$:
\begin{itemize}
\item $\sS_0$ consists of the infinite places and the primes which lie above
2.
\item $\sS_0^+$ consists of $\sS_0$ and a set of generators of the ideal
class group of $k$; for simplicity we require the latter to be
odd primes of good reduction for $E$.
\item $\sS_1$ is the union of $\sS_0$ and the odd primes of bad reduction for
$E$.
\item $\sS_1^+$ is the union of $\sS_1$ and $\sS_0^+$.
\end{itemize}
We shall usually denote by $\sB$ a finite set of places such that
$\sB\supset\sS^+_1$. In the Corollaries to Lemmas \ref{L10}
and \ref{L12} we
shall need to write $\sB$ as a disjoint union $\sB'\cup\sB''$
where $\sB'\supset\sS^+_0$.

To any triple $m=(m_1,m_2,m_3)$ of elements of $k^*$ with $m_1m_2m_3=1$ we
associate the 2-covering given by
\beq{E49} m_iY_i^2=X-c_i {\mathrm{~for~}} i=1,2,3 \quad {\mathrm{and}}
\quad Y=Y_1Y_2Y_3. \end{equation}
We shall frequently treat the $m_i$ as elements of $k^*/k^{*2}$; this involves
some abuse of notation. This system is equivalent to the three equations
\beq{E33} m_iY_i^2-m_jY_j^2=(c_j-c_i)Y_0^2, \end{equation}
of which only two are independent;
we denote by $\gG=\gG(m)$ the curve of genus 1 given by the three
equations (\ref{E33}) and by $C_{ij}=C_{ij}(m)$ the conic given by the single
equation (\ref{E33}).
These equations define an isomorphism between the $\bF_2$-vector space
of all 2-coverings of
$E$ and $(k^*/k^{*2})^2$, the addition of two 2-coverings corresponding to
componentwise multiplication of the triples $m$. 
The 2-covering corresponding to the 2-division point
$(c_1,0)$, for example, is given by the triple
\beq{E80} ((c_1-c_2)(c_1-c_3),c_1-c_2,c_1-c_3). \end{equation}
If $\sB$ is a finite set of places of $k$ containing
$\sS_1^+$, then the 2-coverings soluble in $k_v$ for every $v$ outside $\sB$
can be identified with the elements of $(\fo^*_\sB/\fo_\sB^{*2})^2$, where
$\fo_\sB^*$ consists of the elements of $k^*$ which are units outside $\sB$.
Moreover, if for example $\fp$ divides $c_2-c_3$ but not
$c_1-c_2$ or $c_1-c_3$, it is easy to check that $\fp\|m_1$
implies the local insolubility of $\gG$ at $\fp$.

For every finite set $\sB\supset\sS_0^+$ of places of $k$, of order $n$, write
\[ X_\sB=\fo^*_\sB/\fo_\sB^{*2}, \quad Y_v=k_v^*/k_v^{*2}, \quad
Y_\sB={\bigoplus}_{v\in\sB}Y_v. \]
More generally, if $\sS$ is any finite set of places we shall write
$Y_\sS=\oplus_{v\in\sS}Y_v$, and similarly for $V_\sS, T_\sS, W_\sS$ and
$K_\sS$; but note that the spaces $\fo^*_\sS, X_\sS$ and $U_\sS$ do not follow
this convention.
Here $X_\sB$ has dimension $n$ by Dirichlet's unit theorem, and $Y_\sB$ has
dimension $2n$ because $Y_v$ contains $4/|2|_v$ elements.
It is known from class field theory that $X_\sB\rightarrow
Y_\sB$ is injective. Now write
\[ V_v=Y_v\times Y_v, \quad V_\sB={\bigoplus}_{v\in\sB}V_v=Y_\sB\times Y_\sB \]
and let $U_\sB$ be the image of $X_\sB\times X_\sB$ in $V_\sB$. Thus
$\dim U_\sB=\half\dim V_\sB=2n$. Define the non-degenerate
alternating bilinear form $e_\sS$ on $V_\sS$ by
\beq{E50} e_\sS={\prod}_{v\in\sS}e_v \quad {\mathrm{where}} \quad
e_v((a,b),(c,d))=(a,d)_v(b,c)_v, \end{equation}
the factors on the right being Hilbert symbols. By the Hilbert
product formula
$U_\sB$ is isotropic with respect to $e_\sB$, and
comparison of dimensions shows that it is maximal isotropic
in $V_{\sB}$.

Let $T_v$ be the image of $(\fo^*_v/\fo_v^{*2})^2$ in $V_v$, where $\fo_v$ is
the ring of integers of $k_v$, and let $W_v$ be the image of $E(k_v)$ in $V_v$
under the Kummer map
\[ \partial:\;P=(X,Y)\mapsto(X-c_1,X-c_2) \]
in the notation of (\ref{E27}).
Tate has shown (see [9], p.56) that $W_v$ is
a maximal isotropic subspace of $V_v$ for the alternating form $e_v$, and
$W_v=T_v$ if $v$ is not in $\sS_1$. A 2-covering of $E$ is soluble in $k_v$ if
and only if the corresponding point of $V_v$ is in $W_v$.

The importance of isotropy in this context depends on the
following result.
\ble{L6} Let $\sS$ be a finite set of places and let $G$ be a
maximal isotropic subspace of $V_\sS$ with respect to $e_\sS$.
If $G$ contains $\gs_1\times\tau_1$ and $\gs_2
\times\tau_2$ then
\[ (\gs_1\times\tau_1)+(\gs_2\times\tau_2)=
\gs_1\gs_2\times\tau_1\tau_2. \]
\ele
\emph{Proof} Because $e_\sS$ is non-degenerate it is enough
to show that
\[ e_\sS(\gs_1\times\tau_1,\gs\times\tau)
e_\sS(\gs_2\times\tau_2,\gs\times\tau)
=e_\sS(\gs_1\gs_2\times\tau_1\tau_2,\gs\times\tau) \]
for every element $\gs\times\tau$ of $V_\sS$. This follows
immediately from 
the multiplicitivity of the Hilbert symbol.  \qed

Now suppose that $\sB\supset\sS_1^+$. A 2-covering of $E$ is soluble in $k_v$
for $v$ not in $\sB$ if and only if the corresponding point of $(k^*/k^{*2})^2$
is in $U_\sB$. Hence the 2-Selmer group of $E$ can be identified with
$U_\sB\cap W_\sB$. Because $U_\sB$ and $W_\sB$ are both
maximal isotropic, this group is both the left and
the right kernel of the bilinear map $U_\sB\times W_\sB\rightarrow\{\pm1\}$
induced by $e_\sB$.

\medskip

In \S6 we shall need to know in a particular case how the local solubility of
$\gG$ is related to the local solubility of its images $C_{ij}$.
\ble{L14} Let $\fp$ be a prime in $k$ not dividing $2$, and suppose that $\fp$
divides $(c_2-c_3)$ but not $(c_1-c_2)$ or $(c_1-c_3)$. Then local solubility
of all the $C_{ij}$ at $\fp$ implies local solubility of $\gG$ at $\fp$,
except in the case when $v_\fp(m_2)$, $v_\fp(m_3)$, $v_\fp(c_2-c_3)$ are all
even and $(c_3-c_1)$ is in $k_\fp^{*2}$. In this case local solubility of
$\gG$ further requires that $m_1$ is in $k_\fp^{*2}$.
\ele
\emph{Proof} We separate three cases. In each of them the
first step is to give necessary and sufficient conditions for
the image of the element $a\times b$ of $k_\fp\times k_\fp$
to lie in $W_\fp$. The verification becomes easier if one uses
the fact that $W_\fp$ has dimension 2. We have already noted
that $v_\fp(a)$ must always be even.
\begin{description}
\item[(i)] If $(c_3-c_1)$ is in $k_\fp^{*2}$ then $a$ is in
$k_\fp^{*2}$.
\item[(ii)] If $(c_3-c_1)$ is not in $k_\fp^{*2}$ and
$v_\fp(c_2-c_3)$ is odd, then either $a$ is in $k_\fp^{*2}$
and $v_\fp(b)$ is even or $a$ is in
$(c_3-c_1)k_\fp^{*2}$ and $v_\fp(b)$ is odd.
\item[(iii)] If $(c_3-c_1)$ is not in $k_\fp^{*2}$ and $v_\fp(c_2-c_3)$ is even,
then $v_\fp(a)$ and $v_\fp(b)$ are both even.
\end{description}

Suppose first that $v_\fp(m_2)$ and $v_\fp(m_3)$ are both odd.
Now local solubility of $C_{13}$ implies that $(c_3-c_1)m_1$ is a square;
and $\gG$ is locally soluble by (i) or (ii) above.
Next suppose that $v_\fp(m_2)$ and $v_\fp(m_3)$ are both even but
$v_\fp(c_2-c_3)$ is odd; then local solubility of $C_{23}$ implies that
$m_2m_3$ is a square and hence so is $m_1$. Now $\gG$ is locally soluble by
(i) or (ii) above.
Finally suppose that $v_\fp(m_2)$, $v_\fp(m_3)$ and $v_\fp(c_2-c_3)$ are all
even. Now the local solubility of the $C_{ij}$ provides no useful information.
If $(c_3-c_1)$ is not a square then $\gG$ is locally soluble by (iii); but if
$(c_3-c_1)$ is a square then (i) shows that $\gG$ is locally soluble if and
only if $a$ is a square.  \qed

\medskip

Except perhaps for Lemma \ref{L14}, this is traditional folklore, 
first systematically described by Tate. The next step, which is due to
Skorobogatov, is the construction inside each $V_v$ of a maximal isotropic
subspace $K_v$ such that $V_\sB=U_\sB\oplus K_\sB$. There is considerable
freedom in choosing the $K_v$, and the way in which
we do it in any particular case
depends on the additional properties which are needed for the intended
application. The
construction starts with two general vector space lemmas, the second of which
has a setting which generalizes the structure obtained above.
The only reason for stating these lemmas, and also Lemma
\ref{L12} below, in this more general
form is notational convenience.
\ble{L19} Let $V$ be a finite dimensional vector space over a field $k$,
equipped with a non-degenerate alternating bilinear form $\psi$; and let $W$
be a maximal isotropic subspace of $V$. Then $V$ can be expressed as a direct
sum $\oplus V_i$ of mutually orthogonal subspaces, each of dimension $2$, such
that the restriction of $\psi$ to any $V_i$ is non-degenerate and each
$V_i\cap W$ has dimension $1$.
\ele
\emph{Proof} The existence of $\psi$ shows that dim $V$ is even; so
let dim $V=2n$ with $n>1$, the case $n=1$ being trivial. It is enough to
show that if $w_1$ is a non-trivial element of $W$ then $w_1$ lies in a
subspace $V_1$ satisfying the conditions of the lemma, and that if $V'$
is the orthogonal complement of $V_1$ in $V$ then dim$(V'\cap W)=n-1$; for we
can then complete the proof by induction on $n$. For this, choose $x_1$ in
$V$ not orthogonal to $w_1$. Let $V_1$ be the vector space generated by $w_1$
and $x_1$ and let $V'$ be its orthogonal complement in $V$. Since $V_1$ is not
isotropic, the restriction of $\psi$ to $V_1$ is non-degenerate and
dim$(V_1\cap W)=1$. Now $V'\cap W$ is the subspace of $W$ orthogonal to $x_1$;
so dim$(V'\cap W)\geq n-1$. On the other hand, $w_1$ is not in $V'\cap W$; so
dim$(V'\cap W)\leq n-1$.  \qed
\ble{L10} Let the $V_i$ be $n$ vector spaces over a field $k$, each equipped
with a non-degenerate alternating bilinear form. Let $V=\bigoplus V_i$ be
equipped with the direct sum of these forms; let $U$ be maximal isotropic in
$V$ and for each $i$ let $W_i$ be maximal isotropic in $V_i$.
Then there exist maximal isotropic subspaces $K_i\subset V_i$ such that
$V=U\oplus(\bigoplus K_i)$. We can also ensure whichever we choose of
$V_i=K_i\oplus W_i$ for each $i$ or $W=(U\cap W)\oplus(K\cap W)$ where
$W=\oplus W_i$ and $K=\oplus K_i$.
\ele
\emph{Proof}
 If any $V_i$ has dimension greater than 2, we can by Lemma
\ref{L19} decompose it as a direct sum of mutually orthogonal subspaces of
dimension 2, on each of which the restriction of the
bilinear form is
non-degenerate and each of which meets $W_i$ in a subspace of
dimension 1. This only reduces our
freedom to choose the $K_i$; so
we can assume that every $V_i$ has dimension
$2$. We proceed by induction on $n$, the case $n=0$ being trivial. Suppose first that $W_n$ is contained in $U$; then since
$U$ is isotropic it cannot contain $V_n$, and we can choose
$\ga_n$ in $V_n$ but not in $U$. If instead $W_n$ is not
contained in $U$ then it meets $U$ only in the origin. Now
$V_n$ contains one element in $W_n\cap U$, at
most one element in $U$ but not in $W_n$, one element in
$W_n$ but not in $U$, and at least one element in neither
$W_n$ nor $U$. How we now choose $\ga_n$ depends on what we
are trying to achieve. For the first alternative in the lemma
we take $\ga_n$ to be in neither $W_n$ nor $U$; for the
second we take $\ga_n$ to be in $W_n$ but not in $U$.
In either case, let $K_n$ be the
vector space generated by $\ga_n$. Write
\beq{E32} V^-=V_1\oplus\ldots\oplus V_{n-1}, \quad U^-=V^-\cap(U\oplus K_n).
\end{equation}
If $u^-$ is in $U^-$ then $u^-=u+c\ga_n$ for some $c$ in $k$ and $u$ in $U$;
so the last component of $u$ as an element of $V$ is $-c\ga_n$ and $U^-$ is
isotropic. Since $\dim U^-\geq n-1$ by the second equation (\ref{E32}), $U^-$
is maximal isotropic in $V^-$. Applying the induction hypothesis to the pair
$U^-\subset V^-$, we can find $K_i$ maximal isotropic in $V_i$ for each $i<n$
such that $V^-=U^-\oplus(K_1\oplus\ldots\oplus K_{n-1})$. But now, using
(\ref{E32}) again,
\[ (U\oplus K_n)\cap(K_1\oplus\ldots\oplus K_{n-1})
\subset U^-\cap(K_1\oplus\ldots\oplus K_{n-1})=\{0\}. \]
By dimension count, this means that the sum of the two spaces
on the left is the whole space $V$, which is the first
assertion in the lemma. The second one is
obvious from the construction.  \qed

There is more freedom in the choice of the $K_i$ than
we have so far exploited. What we shall actually use in \S6
is the result below, which represents a compromise between
the two conclusions in Lemma
\ref{L10}. Let $\sB\supset\sS^+_1$ be the disjoint sum of
the sets $\sB'\supset\sS^+_0$ and $\sB''$.
\begin{Cor} In the notation introduced in the first part of
this section there exist maximal
isotropic subspaces $K_v\subset V_v$ such that $V_\sB=U_\sB\oplus K_\sB$,
\[ W_{\sB'}=(U_{\sB'}\cap W_{\sB'})\oplus(K_{\sB'}\cap W_{\sB'}), \]
and $K_v=T_v$ for all $v$ in $\sB''$.
\end{Cor}
\emph{Proof} For $\sB=\sB'$ this follows from the lemma. In the general
case, let the $K_v$ for $v$ in $\sB'$ be those constructed for $\sB=\sB'$
and let $K_v=T_v$ for $v$  in $\sB''$. Now we need only prove
that $V_\sB=U_\sB\oplus K_\sB$, and by dimension count it is
enough to prove that $K_\sB\cap U_\sB$ is trivial. By
Lemma \ref{L6} any element of $K_\sB$  has the
form $\gs\times\tau$. If $\gs\times\tau$ is an element
of $K_\sB\cap U_\sB$ then $\gs,\tau$ must be units at $v$ for any $v$ in
$\sB''$; so $\gs\times\tau$
belongs to the image of $U_{\sB'}$ in $V_\sB=V_{\sB'}\oplus V_{\sB''}$.
Hence the projection onto $V_{\sB'}$ of $\gs\times\tau$ lies in $K_{\sB'}
\cap U_{\sB'}$, which is trivial; so $\gs=\tau=1$.  \qed

Let $t_\sB:\;V_\sB\rightarrow U_\sB$ be the projection along $K_\sB$ and
write
\[ U'_\sB=U_\sB\cap(W_\sB+K_\sB), \quad
W'_\sB=W_\sB/(W_\sB\cap K_\sB)={\bigoplus}_{v\in\sB}W'_v \]
where $W'_v=W_v/(W_v\cap K_v)$. The map $t_\sB$ induces an isomorphism
\[ \tau_\sB:\;W'_\sB\rightarrow U'_\sB; \]
and the bilinear function $e_\sB$ induces a bilinear function
\[ e'_\sB:\;U'_\sB\times W'_\sB\rightarrow\{\pm1\}. \]
We have already identified the 2-Selmer group with $U_\sB\cap
W_\sB$, so it is contained in $U'_\sB$ and is therefore
the left kernel of $e'_\sB$. Dimension counting shows that it
is also isomorphic to the right kernel of $e'_\sB$. If we
choose the first alternative in Lemma \ref{L10} then
$U'_\sB=U_\sB$ and $W'_\sB=W_\sB$; so we have
constructed a natural (though not unique) isomorphism between
$U_\sB$ and $W_\sB$.

The following symmetry property is of central importance.
Later in this section we shall see that the $K_i$ can be
chosen so that the bilinear functions in Theorem \ref{T6} are
actually alternating --- that is, they take the value 1
when $u'_1=u'_2$ or $w'_1=w'_2$ respectively. But that result
lies deeper.
\bth{T6} The bilinear functions $U'_\sB\times U'_\sB\rightarrow\{\pm1\}$ and
$W'_\sB\times W'_\sB\rightarrow\{\pm1\}$ defined respectively by
\beq{E85} u'_1\times u'_2\mapsto e'_\sB(u'_1,
\tau^{-1}_\sB(u'_2)) \quad
{\mathit{and}} \quad w'_1\times w'_2\mapsto
e'_\sB(\tau_\sB w'_1,w'_2) \end{equation}
are symmetric and their kernels are isomorphic to the $2$-Selmer group of $E$.
\eth
\emph{Proof} We need only prove the symmetry, and it is enough to do so for
the first map. Given $u'_1,u'_2$ in $U'_\sB$ let $w_1,w_2$ in $W_\sB$ be such
that $t_\sB w_j=u'_j$. Since both the $(1-t_\sB)w_j$ are in $K_\sB$,
\begin{align*}
1= & e_\sB(w_1,w_2)=e_\sB(t_\sB w_1+(1-t_\sB)w_1,t_\sB w_2+(1-t_\sB)w_2) \\
 = & e_\sB(t_\sB w_1,(1-t_\sB)w_2)e_\sB((1-t_\sB)w_1,t_\sB w_2) \\
 = & e_\sB(t_\sB w_1,w_2)e_\sB(w_1,t_\sB w_2)=e'_\sB(u'_1,w'_2)e'_\sB(u'_2,
w'_1)
\end{align*}
where the $w'_j$ are the images in $W'_\sB$ of the $w_j$.  \qed

\medskip

These results raise two obvious questions:
\begin{itemize}
\item How small can we make $U'$ and $W'$?
\item Can we ensure that the functions (\ref{E85}) are not
merely symmetric but alternating?
\end{itemize}
Since $U'_\sB\supset U_\sB\cap W_\sB$, for the first question
the best that we can hope to achieve is
$U'_\sB=U_\sB\cap W_\sB$; and this follows from
\beq{E86} W_\sB=(U_\sB\cap W_\sB)\oplus(K_\sB\cap W_\sB)
\end{equation}
which is the second alternative in Lemma \ref{L10}. For
suppose that (\ref{E86}) holds; then
\[ W_\sB+K_\sB=(U_\sB\cap W_\sB)+K_\sB \]
and it follows immediately that $U'_\sB=U_\sB\cap(W_\sB+K_\sB)
=U_\sB\cap W_\sB$. Since this is the 2-Selmer group, and it
can be identified with the left and right kernels of each of
the functions (\ref{E85}), these functions are trivial and
therefore alternating. We can summarize what we have so far
achieved as follows.
\bth{T11} We can choose the $K_v$ so that $(\ref{E86})$ holds
and $U'_\sB=U_\sB\cap W_\sB$ is the $2$-Selmer group of $E$.
\eth

We also need to consider
other recipes for choosing the $K_v$, for which (\ref{E86})
does not hold but we still wish to prove that the functions
(\ref{E85}) are alternating.
One important reason for this is that it enables us to study
the effect of twisting on the 2-Selmer group. The
\emph{quadratic twist} of (\ref{E27}) by an element $b$
of $k^*$ is defined to be the elliptic curve
\[ Y^2=(X-bc_1)(X-bc_2)(X-bc_3) \]
where we can require the $bc_i$ as well as the $c_i$ to be
integers. It simplifies the exposition to restrict ourselves
to the case when $b=\gk c$ where $\gk$ is a unit outside
$\sS^+_1$ and $c$ is a unit at every prime in $\sS^+_1$; this
is not the most general case, though it is so when
$k$ has class number 1. Now if we replace every $c_i$ by $\gk
c_i$ and $E_\gk$ by $E$, we are reduced to studying twisting
by an integer $c$ which is a unit at every prime in $\sS^+_1$.
We now formulate a strengthened version of Lemma \ref{L10};
what we shall actually use is the Corollary to Lemma \ref{L12},
which is a strengthened version of the Corollary to Lemma
\ref{L10}.
\ble{L12} Suppose that the conditions of Lemma $\ref{L10}$
hold, and that there are functions
$\phi_i$ on $V_i$ with values in $\{\pm1\}$ which satisfy
\beq{E89} \phi_i(\xi+\eta)=\phi_i(\xi)\phi_i(\eta)\psi_i(\xi,\eta)
\end{equation}
for any $\xi,\eta$ in $V_i$. Let $\phi$ on $V$ be
the product of the $\phi_i$.
Assume that $\phi$ is trivial on $U$ and $\phi_i$ is trivial on
$W_i$. Then in addition to $V=U\oplus K$ and $(\ref{E86})$
we can ensure that $\phi_i$ is trivial on
$K_i$ and $\phi$ is trivial on $K$.
\ele
\emph{Proof} As in the proof of Lemma \ref{L10}, we can
assume that every $V_i$ has dimension $2$; for if we prove
that $\phi_i$ is trivial on $K_i$ in this special case, the
corresponding result for the general case will follow from
(\ref{E89}). As before we proceed by induction on $n$, the
case $n=0$ being trivial; and we split cases according as
$W_n$ is contained in $U$ or not. In the former case, since
$U$ is isotropic it cannot contain $V_n$; so there are two
elements of $V_n$ which do not lie in
$U$. Denote them by $\ga'_n$ and $\ga''_n$, and let $\gb_n$ be
the nontrivial element of $W_n$; thus $\ga''_n=\ga'_n+\gb_n$.
Now $\phi_n(\gb_n)=1$; so it follows from (\ref{E89}) that
\[ \phi_n(\ga'_n)\phi_n(\ga''_n)=\psi_n(\ga'_n,\gb_n)=-1 \]
and we can choose $\ga_n$ so that $\phi_n(\ga_n)=1$.
In the latter case $V_n$ meets $W_n\cap U$ only in the trivial
element; so it contains at
most one element in $U$ but not in $W_n$, exactly one element
in $W_n$ but not in $U$,
and at least one element in neither $W_n$ nor $U$; we
take $\ga_n$ to be in $W_n$ but not in $U$. In this case,
$\phi_n(\ga_n)=1$ by hypothesis.
Now that we have constructed $\ga_n$ the remainder of the
proof is identical with the last part of the proof of Lemma
\ref{L10}.
In contrast with the situation in Lemma \ref{L10}, any
freedom in the choice of the $K_i$ is restricted to the
way in which the $V_i$ of dimension greater than 2 are
decomposed.  \qed

When we apply Lemma \ref{L12} (with $v$ for
$i$) we replace $\psi_i$ by $e_v$ and take
\beq{E83} \phi_v(\gl\times\mu)=(\gl,(c_2-c_1)(c_2-c_3))_v
(\mu,(c_1-c_2)(c_1-c_3))_v(\gl,\mu)_v. \end{equation}
This formula can be put into a more convenient form, for
\[ ((c_1-c_2)(c_1-c_3),(c_1-c_2)(c_3-c_2))_v=
\left(\frac{c_1-c_3}{c_1-c_2},\frac{c_3-c_2}{c_1-c_2}\right)_v=1 \]
because the sum of the two arguments is 1. Hence
\[ \phi_v(\gl\times\mu)=
(\gl(c_1-c_2)(c_1-c_3),\mu(c_2-c_1)(c_2-c_3))_v. \]
The reason for this choice of $\phi_v$ will become evident
below; but we need to check that these $\phi_v$
satisfy the conditions of Lemma \ref{L12}. Here (\ref{E89}) is
obvious, as is the triviality of $\phi$ on $U$. To verify
that $\phi_v$ is trivial on $W_v$ we argue as follows. It
follows from (\ref{E33}) that
\beq{E79} (c_2-c_3)m_1Y_1^2+(c_3-c_1)m_2Y_2^2+
(c_1-c_2)m_3Y_3^2=0 \end{equation}
and therefore
\begin{align*} (c_2-c_1)(c_2-c_3) & m_2(m_1Y_1)^2+(c_1-c_2)(c_1-c_3)m_1(m_2Y_2)^2 \\
 & =m_1m_2m_3((c_1-c_2)Y_3)^2. \end{align*}
If the 2-covering (\ref{E33}) is soluble, then since
$m_1m_2m_3$ is in $k_v^{*2}$ this implies
\beq{E98} ((c_1-c_2)(c_1-c_3)m_1,(c_2-c_1)(c_2-c_3)m_2)_v=1, \end{equation}
which is just the result that we need.

We now combine the ideas of Lemma \ref{L12} and the Corollary
to Lemma \ref{L10}. The proof of the following result, apart
from $U'_{\sB'}=U_{\sB'}\cap W_{\sB'}$ which as we have
already seen follows from (\ref{E20}), is essentially the
same as that of the proof of the Corollary to Lemma \ref{L10}.
Let $\sB\supset\sS^+_1$ be the disjoint sum of the
sets $\sB'\supset\sS^+_0$ and $\sB''$, let $\psi_v=e_v$, and
let $\phi_v$ be as above.
\begin{Cor} In the notation above, there exist maximal
isotropic subspaces $K_v\subset V_v$ such that
$V_\sB=U_\sB\oplus K_\sB$. Moreover we can ensure that
the restriction of $\phi$ to $K_{\sB'}$ is trivial,
\beq{E20} W_{\sB'}=(U_{\sB'}\cap W_{\sB'})\oplus(K_{\sB'}\cap W_{\sB'}),
\end{equation}
$U'_{\sB'}=U_{\sB'}\cap W_{\sB'}$, and $K_v=T_v$ for all $v$
in $\sB''$.
\end{Cor}

\medskip

We use this machinery to study $d_c$, the dimension of the
2-Selmer group of $E_c$, under the constraint already
introduced, that $c$ is a unit at every
prime in $\sS^+_1$.
Let $(c)=\fp_1\ldots\fp_M$ where none of the $\fp_i$ are in
$\sS^+_1$. As was said above,
this restricts us to twisting by integers not divisible by any
bad prime; any other twisting must be achieved by
using $E_\gk$ instead of $E$.
In the notation of the last Corollary, we take $\sB'=\sS^+_1$
and write $\sB''=\{\fp_1,\ldots,\fp_M\}$. Since each
$W_{\fp_\nu}$ is generated by the 2-division points of $E_c$ and therefore
has trivial intersection with $K_{\fp_\nu}=T_{\fp_\nu}$,
\beq{E95} \dim U'_\sB=\dim W'_\sB=\dim W'_{\sB'}+\dim W'_{\sB''}=\dim U'_{\sB'}+2M. \end{equation}
Comparing dimensions now shows that the projection map
\[ U'_\sB\rightarrow V_\sB\rightarrow V_{\sB''}
\rightarrow\oplus^M_{\nu=1}
(k^*_{\fp_\nu}/\fo_{\fp_\nu}^*k^{*2}_{\fp_\nu})^2
\sim W_{\sB''}, \]
whose kernel is $U'_{\sB'}$, is onto; here the map $V_{\sB''}
\rightarrow W_{\sB''}$ is projection along $K_{\sB''}=
T_{\sB''}$. In particular, there are elements
$\ga_\nu\times\gb_\nu$ and $\gg_\nu\times\gd_\nu$ of
\beq{E28} U'_{\sB'\cup\{\fp_\nu\}}=U_{\sB'\cup\{\fp_\nu\}}\cap((W_{\sB'}+
K_{\sB'})\oplus V_{\fp_\nu}) \end{equation}
such that $\fp_\nu\|\ga_\nu$,
$\fp_\nu\|\gd_\nu$ and $\gb_\nu$, $\gg_\nu$ are units at $\fp_\nu$; and
$U'_{\sB'}$ and the $\ga_\nu\times\gb_\nu$ and $\gg_\nu\times\gd_\nu$ are
linearly independent and span $U'_\sB$. Here
$\ga_\nu\times\gb_\nu$ and $\gg_\nu\times\gd_\nu$ are only
determined up to elements of $U'_{\sB'}$; and their images in
$V_{\sB'}$ lie in $W_{\sB'}+K_{\sB'}$ by (\ref{E28}),
and therefore in $U'_{\sB'}\oplus K_{\sB'}$. We denote these
images by $\widehat{\ga_\nu}\times\widehat{\gb_\nu}$ and
$\widehat{\gg_\nu}\times\widehat{\gd_\nu}$; thus for example
$\widehat{\ga_\nu}\times\widehat{\gb_\nu}$ lies in $Y_{\sB'}
\times X_{\sB'}$. By subtracting suitable elements of $U'_{\sB'}$
from $\ga_\nu\times\gb_\nu$ and $\gg_\nu\times\gd_\nu$, we
normalize them so that their images actually lie in $K_{\sB'}$.

More generally, let $\gs\times\tau$ be any element of $U'_\sB$ and let
$\hat{\gs}\times\hat{\tau}$ be its projection onto $V_{\sB'}$;
then $\hat{\gs}\times\hat{\tau}$ lies in $U'_{\sB'}\oplus
K_{\sB'}$. It will be shown below that all the values which
interest us can be expressed in terms of such projections
and the places in $\sB'$; so we can largely confine
ourselves to these.

The recipe for calculating $\tau^{-1}_\sB u$ for any $u$ in
$U'_\sB$ is to project $u$ to an element $u_v$ of $V_v$ for each $v$ in
$\sB$ and then add whatever
element of $K_v$ is needed for the sum to lie in $W_v$; this sum is then
projected into $W'_v$. Thus for example, if $\gl\times\mu$ is
in $U'_{\sB'}$ then it is in $W_{\sB'}$
and the component of $\tau^{-1}_\sB(\gl\times
\mu)$ in $W'_v$ for $v$ in $\sB'$ is just the coset of $W_v\cap K_v$
containing $\gl\times\mu$; the component of $\tau^{-1}_\sB(\gl\times
\mu)$ in $W'_v$ for $v$ in $\sB''$ is trivial. Again the component of
$\tau^{-1}_\sB(\ga_\nu
\times\gb_\nu)$ in $W'_v$ for $v$ in $\sB'$ is trivial because
of the normalization above; since $W'_{\fp_\nu}=W_{\fp_\nu}$
is generated by the 2-division points, the component
of $\tau^{-1}_\sB(\ga_\nu\times\gb_\nu)$ in $W'_{\fp_\nu}$ is $c(c_2-c_1)\times
(c_2-c_1)(c_2-c_3)$ and its component in $W'_{\fp_\rho}$
for $\rho\neq\nu$ is again trivial. Similar statements hold
for $\gg_\nu\times\gd_\nu$. Denote by $U_\nu$ the subspace of
$V_\sB$ generated by
$\ga_\nu\times\gb_\nu$ and $\gg_\nu\times\gd_\nu$, and write $U_0=U'_{\sB'}$;
thus
\beq{E29} U'_\sB=U_0\oplus U_1\oplus\ldots\oplus U_M. \end{equation}
Our next task is to obtain formulae for
$\gt$ in terms of the decomposition (\ref{E29}), where
$\gt=\gt_\sB$ is the first of the bilinear functions in
(\ref{E85}).

By the recipe above, if $\gl\times\mu$ is in $U_0=U'_{\sB'}$
then it follows from (\ref{E50}) that
$\gt(\gl\times\mu,\gl\times\mu)=1$.
Let $\chi_\nu$ denote the quadratic character
mod $\fp_\nu$; if $\gl\times\mu$ is an element of $U_0$ we have
\[ \gt(\gl\times\mu,\ga_\nu\times\gb_\nu)=\chi_\nu(\mu),
\quad \gt(\gl\times\mu,\gg_\nu\times\gd_\nu)=\chi_\nu(\gl). \]
If $\rho$, $\nu$ are distinct and nonzero, the restriction of $\gt$
to $U_\rho\times U_\nu$ is given by
\begin{align*}
\gt(\ga_\rho\times\gb_\rho,\ga_\nu\times\gb_\nu)=\chi_\nu(\gb_\rho), & \quad
\gt(\gg_\rho\times\gd_\rho,\gg_\nu\times\gd_\nu)=\chi_\nu(\gg_\rho), \\
\gt(\ga_\rho\times\gb_\rho,\gg_\nu\times\gd_\nu)=\chi_\nu(\ga_\rho), & \quad
\gt(\gg_\rho\times\gd_\rho,\ga_\nu\times\gb_\nu)=\chi_\nu(\gd_\rho).
\end{align*}
Similarly the restriction of $\gt$ to $U_\nu\times U_\nu$,
where $\nu\neq0$, is given by
\begin{align*}
 & \gt(\ga_\nu\times\gb_\nu,\ga_\nu\times\gb_\nu)=\chi_\nu(\gb_\nu(c_2-c_1)(c_2-c_3)), \\
 & \gt(\gg_\nu\times\gd_\nu,\gg_\nu\times\gd_\nu)=\chi_\nu(\gg_\nu(c_1-c_2)(c_1-c_3)), \\
 & \gt(\ga_\nu\times\gb_\nu,\gg_\nu\times\gd_\nu)=(c(c_2-c_1),\gd_\nu)_{\fp_\nu}=\chi_\nu(c\gd_\nu(c_1-c_2)), \\
 & \gt(\gg_\nu\times\gd_\nu,\ga_\nu\times\gb_\nu)=(c(c_1-c_2),\ga_\nu)_{\fp_\nu}=\chi_\nu(c\ga_\nu(c_2-c_1)).
\end{align*}
Here symmetry gives us some non-trivial identities --- for example that
$\chi_\nu(-\ga_\nu\gd_\nu)=1$. We can also write all the right hand sides
above in terms of hatted letters and places in $\sB'$; for example
\begin{align*}
\gt(\ga_\nu & \times\gb_\nu,\ga_\nu\times\gb_\nu)=(\ga_\nu,\gb_\nu(c_2-c_1)
(c_2-c_3))_{\fp_\nu} \\
 & =\prod_{v\in\sB'}(\ga_\nu,\gb_\nu(c_2-c_1)(c_2-c_3))_v
=\prod_{v\in\sB'}(\widehat{\ga_\nu},\widehat{\gb_\nu}(c_2-c_1)(c_2-c_3))_v.
\end{align*}

Let $\gs\times\tau$ be any element of $U_1\oplus\ldots\oplus U_M$;
then $\hat{\gs}\times\hat{\tau}$ lies in $K_{\sB'}$ and an
argument similar to that above shows that
\begin{align*} \gt(\gs & \times\tau,\gs\times\tau)=\bigl\{\prod\chi_\nu(\tau(c_2-c_1)(c_2-c_3))\bigr\} \\
 & \bigl\{\prod\chi_\nu(\gs(c_1-c_2)(c_1-c_3))\bigr\}\bigl\{\prod\chi_\nu(\gs\tau(c_3-c_1)(c_3-c_2))\bigr\} \end{align*}
where the three products are taken over the following sets of
$\fp_\nu$ in $\sB''$:
\[ \begin{matrix}
{\mathrm{first~product}} & \fp_\nu {\mathrm{~divides~}}\gs {\mathrm{~but~not~}} \tau {\mathrm{~to~an~odd~power;}} \\
{\mathrm{second~product}} & \fp_\nu {\mathrm{~divides~}}\tau {\mathrm{~but~not~}} \gs {\mathrm{~to~an~odd~power;}} \\
{\mathrm{third~product}} & \fp_\nu {\mathrm{~divides~both~}}\gs {\mathrm{~and~}} \tau {\mathrm{~to~an~odd~power.}}
\end{matrix} \]
Replacing $\chi_\nu(\cdot)$ in the first product by $(\gs,\cdot)_{\fp_\nu}$ and in the
second by $(\tau,\cdot)_{\fp_\nu}$, using the fact that in the third product
\begin{align*} \chi_\nu & (\gs\tau(c_3-c_1)(c_3-c_2)) \\
 & =(\gs,(c_2-c_1)(c_2-c_3))_{\fp_\nu}(\tau,(c_1-c_2)(c_1-c_3))_{\fp_\nu}(\gs,-\gs\tau)_{\fp_\nu}
\end{align*} 
where the last factor is equal to $(\gs,\tau)_{\fp_\nu}$, and
inserting some extra trivial factors, we obtain
\[ \gt(\gs\times\tau,\gs\times\tau)=\prod_{\fp\in \sB''}\{(\gs,(c_2-c_1)
(c_2-c_3))_\fp(\tau,(c_1-c_2)(c_1-c_3))_\fp(\gs,\tau)_\fp\}. \]
By the Hilbert product formula this last expression is equal to
\beq{E51} \prod_{v\in \sB'}\{(\hat{\gs},(c_2-c_1)(c_2-c_3))_v
(\hat{\tau},(c_1-c_2)(c_1-c_3))_v(\hat{\gs},\hat{\tau})_v\},
\end{equation}
which is just $\prod_{v\in\sB'}\phi_v(\hat{\gs},\hat{\tau})$
in the notation of (\ref{E83}). By the Corollary to Lemma
\ref{L12}, this is equal to 1.
\bth{T20} If $d_c$ is the dimension of the $2$-Selmer group
of $E_c$ and
$c$ is a unit at every place in $\sB'=\sS^+_1$, then $d_c$ is
congruent mod $2$ to
the dimension of $U_{\sB'}\cap W_{\sB'}$.
\eth
\emph{Proof} We choose $K_{\sB'}$ in accordance with the
Corollary to Lemma \ref{L12}. What we have just shown is that
$\gt(\gs\times\tau,\gs\times\tau)$ is trivial on
$U_1\oplus\ldots\oplus U_M$, and we already know that it is
trivial on $U_0$. Since $\gt$ is a symmetric bilinear form
with values in $\{\pm1\}$,
this implies that $\gt(\gs\times\tau,\gs\times\tau)$ is
trivial on $U'_\sB$. Hence $\gt$ is alternating on $U'_\sB$,
so its rank is even and therefore its corank (which is $d_c$)
is congruent mod 2 to $\dim U'_\sB$ and therefore
to $\dim U'_{\sB'}=\dim(U_{\sB'}\cap W_{\sB'})$ by (\ref{E95}).
This last
expression is independent of the choice of $K_{\sB'}$.  \qed

We remind the reader that $W_{\sB'}$ does depend on $c$, or
more precisely on the image of $c$ in $Y_{\sB'}$.
If one wishes to compute the $d_b\bmod2$ for any particular
curve (\ref{E27}), where $b=\gk c$ as before, the best way
appears to be as follows. One considers all the elements of
$V_{\sB'}$ for which the associated equation (\ref{E79}) is
locally soluble at each place in $\sB'$, and for each of them
one determines the possible
images of $\gk c$ in $Y_{\sB'}$ from one of the three
relations like
\[ m_2Y_2^2-m_3Y_3^2=\gk c(c_3-c_2). \]
Because weak approximation holds for the conic (\ref{E79}),
these calculations can be carried out separately for each
place in $\sB'$. One then reverses the table thus
generated, so that for each element of $Y_{\sB'}$ one obtains a
list of the possible triples $m$. Each such list is an
$\bF_2$-vector space, of dimension $d_{\gk c}$.

\medskip


The algorithm of Cassels carries out a 4-descent; more
precisely he shows how to determine
which elements of the 2-Selmer group survive the second descent. For this
purpose he defines a skew-symmetric bilinear form $\langle\cdot,\cdot\rangle$
on the 2-Selmer group, whose kernel consists of those elements which survive
the second descent. Here we confine ourselves to curves of the
form (\ref{E27}). As before let $\sB\supset\sS^+_1$ and let $m^\sharp$ and
$m^\flat$ be two triples which correspond to elements of the 2-Selmer group of
(\ref{E27}). We can assume that the components of $m^\sharp$ and $m^\flat$
are integers all of whose prime factors lie in $\sB$.

Each conic $C_{ij}^\sharp$ is an image of the 2-covering associated with
$m^\sharp$; so it is soluble everywhere locally and therefore globally. Let
$P_{ij}^\sharp$ be a rational point on $C_{ij}^\sharp$ and let $f_{ij}^\sharp
=f_{ij}^\sharp(Y_0,Y_i,Y_j)$ be a homogeneous linear form such that 
$f_{ij}^\sharp=0$ is the tangent to $C_{ij}^\sharp$ at $P_{ij}^\sharp$.
For any
$v$ in $\sB$ let $Q_v^\sharp$ be a $v$-adic point on the affine 2-covering
induced by $m^\sharp$; we can clearly ensure that each $f_{ij}^\sharp
(Q_v^\sharp)$ is a nonzero element of $k_v$. Then Cassels's skew-symmetric
bilinear form is defined by
\beq{E34} \langle m^\sharp,m^\flat\rangle ={\prod}_{v\in\sB}\prod(f_{ij}^\sharp
(Q_v^\sharp),m_k^\flat)_v \end{equation}
where the inner product is taken over the three cyclic permutations $i,j,k$ of
$1,2,3$.


\newpage

\noindent 6. \emph{Pencils of curves of genus $1$}. \newline
In this section we shall be concerned with pencils of
2-coverings of elliptic curves, where the underlying pencil
of elliptic curves has the form
\beq{E24} E:\;Y^2=(X-c_1(U,V))(X-c_2(U,V))(X-c_3(U,V)). \end{equation}
Here the $c_i(U,V)$ are homogeneous polynomials in $\fo[U,V]$ all having the
same even degree. By means of a linear transformation on $U,V$ we can ensure
that the leading coefficients of the $c_i(U,V)$ are nonzero. Write
\[ R(U,V)=p_{12}(U,V)p_{23}(U,V)p_{31}(U,V) \]
where $p_{ij}=c_i-c_j$.

The 2-coverings of (\ref{E24}) are given by
\beq{E25} m_i(U,V)Y_i^2=X-c_i(U,V) {\mathrm{~for~}} i=1,2,3 \end{equation}
where the $m_i(U,V)$ are square-free homogeneous polynomials in $\fo[U,V]$ of
even degree such that $m_1m_2m_3$ is a square. We should really regard the
$m_i$ as homogeneous polynomials modulo squares, but this complicates the
notation. The equations (\ref{E25})
are equivalent to the three equations
\beq{E26} m_iY_i^2-m_jY_j^2=(c_j-c_i)Y_0^2 \end{equation}
of which only two are independent. The sum of two such 2-coverings is obtained
by multiplying the corresponding triples $(m_1,m_2,m_3)$ componentwise and then
removing squared factors. Denote by $V=V(m)$ the surface fibred by the curves
(\ref{E25}) or (\ref{E26}), by $\gG=\gG(m;U,V)$ the curve given by the three
equations (\ref{E25}) or the three equations (\ref{E26}) for particular
values of $m,U,V$, and by
$C_{ij}=C_{ij}(m;U,V)$ the conic given by a single equation (\ref{E26}).
There are natural maps $\gG\rightarrow C_{ij}$. The equations
(\ref{E26}) also imply
\beq{E76} m_1(c_2-c_3)Y_1^2+m_2(c_3-c_1)Y_2^2+m_3(c_1-c_2)Y_3^2=0,
\end{equation}
and for $\gG$ to be soluble so too must be this conic. This
additional condition does not appear in the statement of
Theorem \ref{T7}, and in fact it follows from the conditions
stated there; but it can be useful in some circumstances, as
we shall see in \S9.

Our objective is to provide sufficient conditions for the
solubility of a particular pencil of curves $\gG$. We shall
use a superscript 0 to denote this particular curve and other
objects connected with it. We need to distinguish
between $\sS$, the set of bad places for the
pencil of curves $\gG^0$, and the larger set $\sB$ of bad places for the
particular curve $\gG^0(\ga,\gb)$ on which we want to
prove that there are rational
points. The latter is the same $\sB\supset\sS_1^+$ as in \S5.
Thus $\sS$ is a finite set containing the infinite places, the primes
above 2, those which divide the resolvent of any two coprime factors of
$R(U,V)$ in $\fo[U,V]$ or have norm not
greater than $\deg(R(U,V))$, and those which are bad in the sense of \S4 for
any of the pencils of conics $C^0_{ij}$. In terms
of the definitions below, $\sB$ must contain $\sS$ and all
the $\fp_{k\tau}$. The additional prime $\fp$ which we
introduce at each step of the algorithm should be thought of
as being thereby adjoined to $\sS$.

We denote the irreducible factors of $p_{ij}(U,V)$ in $k[U,V]$
by $f_{k\tau}(U,V)$, and we assume that the coefficients of
any $f_{k\tau}$ are integers and that there is no prime
outside $\sS$ which divides all of them. When we apply the
results of \S5 it will be
with $U=\ga,V=\gb$ where $\ga\times\gb$ is so chosen that each
ideal $(f_{k\tau}(\ga,\gb))$ is the product of primes in $\sS$
and one prime outside $\sS$; to do this we appeal to Lemma
\ref{L1}. The arguments of \S5 show that we
can confine ourselves to those triples $m$ whose components
take values in $\fo^*_\sB$ when $U=\ga,V=\gb$. This means that
we can restrict the components of $m$ to be products of some
of the $f_{k\tau}(U,V)$ by elements of $\fo^*_\sS$. In view of
the description of 2-descents in \S5, we can further restrict
ourselves to the triples $m$ such that $m_1m_2m_3$
divides $R^2$ and $m_i$ is prime to $p_{jk}$ in $k[U,V]$,
where here and throughout this section $i,j,k$ is any
permutation of $1,2,3$.

We shall also assume that the $p_{ij}(U,V)$ are
coprime in $k[U,V]$. The case when
this condition fails is also of interest, but the methods used and the
conclusions are quite different; for a more detailed account see [4].
This assumption is weaker than that in [6], which
was that $R(U,V)$ is square-free in $k[U,V]$, and it enables us to bring the
example in \S9 within the scope of the general theory. But
because we need to use Lemma \ref{L14}, we do have to
assume that if an irreducible factor $f_{k\tau}(U,V)$ divides $p_{ij}$ to an
even power, it also divides $m^0_i$ and $m^0_j$.

The parity conditions on the degrees of the $c_i$ and $m_i$
are needed to ensure that the curves (\ref{E24})
and $\gG$ with $U=\ga,V=\gb$ only depend on
$\gl=\ga/\gb$ and not on $\ga,\gb$ separately; otherwise we
would not be dealing with pencils. But even if two of the
$m_i$ have odd degree, which can happen if $R$ has factors of
odd degree, the curve $\gG$ given by (\ref{E25})
or (\ref{E26}) is a 2-covering of $E$; and such 2-coverings
do play a part in our arguments. For given $E$, let $G$
be the group of all triples $(m_1,m_2,m_3)$ satisfying the
conditions above,
including that the degrees of the $m_i$ are even,
and define $G^*\supset G$ by dropping the condition that the
$m_i$ have even degree. Provided we take the $m_i$ modulo
squares, both $G$ and $G^*$ are finite; and either $G$ or
$G^*$ can be regarded as defining those pencils of
2-coverings of the pencil $E$ which are of
number-theoretic interest.

Now suppose that we are given a triple $m^0=(m^0_1,m^0_2,
m^0_3)$ in $G$. Denote by $\gG^0=\gG(m^0,U,V)$ the curve of
genus 1 given by the three equations
(\ref{E26}) with $m=m^0$, and similarly for the $C_{ij}^0$.
For the pencil of curves $\gG^0$ to contain
rational points it must be
everywhere locally soluble, and we shall always
assume this. For simplicity we also assume that the elliptic
curve (\ref{E24}) has no primitive 4-division points defined
over $k(U,V)$, and to avoid trivialities we also assume that
the 2-covering $\gG^0$ does not
correspond to a 2-division point.

An essential tool in proving solubility will be the special case $n=2$ of
Lemma \ref{L9}, which we restate for ease of reference.
\ble{L11} Suppose that the Tate-Shafarevich group of $E/k$ is finite and the
$2$-Selmer group of $E$ has order $8$. Then every curve representing an element
of the $2$-Selmer group contains rational points.
\ele
As this shows, everything in this section will depend on Hypothesis \Sha; and
almost everything will also depend on Schinzel's Hypothesis. We retain the
notation for 2-descents introduced in \S5, and in the notation of the Corollary
to Lemma \ref{L10} we take $\sB'=\sS_0^+$.

The only values of $U/V$ for which $\gG^0$ can be soluble are
ones for which $\gG^0$ is everywhere locally soluble; so for
any such value of $U/V$ the 2-Selmer group of $E$ must
contain the subgroup of order 8 generated by $\gG^0$ and
the 2-coverings coming from the 2-division points. We shall
call this the
\emph{inescapable} part of the 2-Selmer group.

In contrast to what happened in \S4, we cannot simply apply
Lemma \ref{L1} to choose $\ga,\gb$ so that all the
$f_{k\tau}(\ga,\gb)$ are prime up to possible factors in
$\sS$, because this might give rise to a
2-Selmer group too big for us to be able to apply Lemma
\ref{L11}. (Note that by (iii) below, the order of this
2-Selmer group will be independent of the choice of $\ga,\gb$,
provided that $\ga\times\gb$ is confined to a small enough
open set in the topology induced by $\sS$.)
What we do instead is most conveniently
described as an algorithm, which
consists of repeatedly introducing a further well-chosen
prime $\fp$ into $\sS$, with a corresponding extra
condition on the set $\sA$ of possible values of $U\times V$,
in such a way that if we then apply Lemma \ref{L1} the
dimension of the 2-Selmer group is one less than it would
have been before. If we can go on doing this as long as the
2-Selmer group remains too big, we shall eventually reach a
situation to which we can apply Lemma \ref{L11}. What we
thereby obtain is Theorem \ref{T7} below.

The process of introducing a new prime $\fp$ is as follows. We
choose an $f_{k\tau}$ and integers $\ga_\fp,\gb_\fp$ not both
divisible by $\fp$ and such that $\fp|f_{k\tau}(\ga_\fp,\gb_\fp)$.
Without loss of generality we can assume that $\ga_\fp,\gb_\fp$
are coprime and that $\fp\|f_{k\tau}(\ga_\fp,\gb_\fp)$. Choose
integers $\gg_\fp,\gd_\fp$ such that $\ga_\fp\gd_\fp-\gb_\fp
\gg_\fp=1$, make the change of variables
\[ U=\ga_\fp U_1+\gg_\fp V_1, \quad V=\gb_\fp U_1+\gd_\fp V_1 \]
and impose on $\sA$ the additional condition $\fp^2|V_1$. Thus
at any point of $\sA$ the value of $f_{k\tau}$ is exactly
divisible by $\fp$, and hence the values of all the other
functions $f_{\cdot\cdot}$ are prime to $\fp$.

As we noted just after (\ref{E80}), in the notation of \S5 all
the triples in $W_\fp$ have $v_\fp(m_k)$ even. Since
$K_\fp=T_\fp$, the set of triples all whose components
are units at $\fp$, it follows that $W_\fp\cap K_\fp$ has
dimension 1 and so has $W'_\fp$. A similar argument holds for
the primes $\fp_{k\tau}$ provided by Lemma \ref{L1}.

Which $\fp$ satisfy the condition that there exist
$\ga_\fp,\gb_\fp$ as above?
For any irreducible factor $f_{k\tau}(U,V)$ of $p_{ij}(U,V)$,
let $K_{k\tau}=k[X]/f_{k\tau}(X,1)$ be the field obtained by
adjoining to $k$ a root of $f_{k\tau}$, and let $\xi_{k\tau}$
be the class of $X$ in $K_{k\tau}$; thus $f_{k\tau}
(\xi_{k\tau},1)=0$. For the time being we suppose $m$ fixed,
and the field $L_{k\tau}$ which we shall
define will depend on $m$. The singular fibres of the pencil
of elliptic curves (\ref{E24}), as also those
of the pencil of 2-coverings (\ref{E26}), correspond to the
roots of the $f_{k\tau}$. The reason for being interested in
the singular fibres is as follows. Let $\fp$
be a prime of $k$ not in $\sS$, and let $\ga_\fp,\gb_\fp$
in $\fo$ be such that $\fp\|f_{k\tau}
(\ga_\fp,\gb_\fp)$; such $\ga_\fp,\gb_\fp$ exist if and only
if there is a prime $\fP$ in $K_{k\tau}$ whose relative norm
over $k$ is $\fp$. This last condition may appear tiresome.
But what one really does is to choose a first-degree prime
$\fP$ in $K_{k\tau}$ and define $\fp$ to be the prime below
it in $k$. Now norm$\fP=\fp$ is automatic.

The solubility of $\gG^0$ certainly requires that each pencil $C^0_{ij}$
be soluble; so Theorem \ref{T4} provides necessary conditions for the
solubility of $\gG^0$. But we have examples to show that
they are not sufficient to ensure that there is a point $\ga\times\gb$ at which
$\gG^0$ is soluble. In previous papers we have therefore introduced a further
condition, which we call Condition D; and we shall do this again in the last
part of the proof of Theorem \ref{T7}. The formulation of
Condition D here, which can be found near the end of this
section, is superficially rather different to that in
previous papers; but the new version
is easily seen to be essentially the same as the older one.
It is in fact a stronger condition than we need;
see the discussion at the end of this section.

As in \S4, we need to work not in $\bP^1$ but in the subset $\bL^1$
obtained by deleting the points $\gl=\ga/\gb$ at which $R(\ga,\gb)$ vanishes.
The topology on $\bL^1(k)$ will be that induced by $\sS$. There is an open
set $\sN\subset\bL^1(k)$ such that $\gG^0(\ga,\gb)$ is soluble at every place
of $\sS$ if and only
if $\gl$ lies in $\sN$. We have already imposed the
condition that $\sN$ is not empty.
\bth{T7} Assume Schinzel's Hypothesis and Hypothesis \Sha, and
suppose that the three $p_{ij}(U,V)$ are coprime in $k[U,V]$
and that any $f_{k\tau}$ which divides $p_{ij}$ to an even
power divides $m_i^0$ and $m_j^0$ to odd powers. Suppose that
Condition D holds, and let $\sA\subset\sN$ be the open subset
of $\bL^1(k)$ at which each pencil of $C^0_{ij}$ is soluble.
Then $\sA$ lies in the closure of the set of $\gl$ in
$\bL^1(k)$ at which $\gG^0(\ga,\gb)$ is soluble in $k$.
\eth
Technically Theorem \ref{T7} is not a weak
approximation theorem, in contrast with the situation for conics, because weak
approximation does not hold on curves of genus 1; but it can be regarded as a
theorem of `weak approximation type'.
Theorem \ref{T7} gives a sufficient condition for the Hasse
principle to hold, though the condition is not always
necessary. Indeed, we shall see at the end of this section
that we can replace Condition D by a potentially weaker
Condition E; but even the latter is not always necessary for
solubility. The relation between Condition D and the
Brauer-Manin obstructions is addressed in [6].

The arguments needed to validate each step of the algorithm
are lengthy, and for clarity we list them as (i) to (v) below.
We impose further conditions on the additional prime $\fp$,
which ensure (i); we then deduce
(ii), (iii) and (iv). Finally we show that unless the process
is complete, we can choose $\fp$ so that (v) holds. After all
this we choose $\ga\times\gb$ according to the recipe in
Lemma \ref{L1}
for the $f_{k\tau}$, and with the additional property that $L(\sS;U,V;\ga,
\gb)=1$ if there is any $f_{k\tau}$ of odd degree. One can satisfy this
additional requirement by a slight modification of the construction used to
prove Lemma \ref{L1}. Alternatively, one can render it unnecessary by replacing
$U,V$ by homogeneous quadratic forms in $U_1,V_1$; this does not alter the
values of the functions $L$.

Denote by $\fp_{k\tau}$ the additional prime in $k$ which
divides $f_{k\tau}(\ga,\gb)$ when $\ga\times\gb$ is chosen
according to the recipe in Lemma \ref{L1}; then every
$f_{k\tau}(\ga,\gb)$ is a unit outside the set $\sB$
obtained by adjoining $\fp$ and all the $\fp_{k\tau}$ to
$\sS$, and $f_{k\tau}(\ga,\gb)$ is divisible to precisely the
first power by $\fp_{k\tau}$ and by $\fp$.
The set $\sB$ thus defined will be the appropriate
set $\sB$ for applying the results of \S5. What is crucial is
that we have good local information about the $\fp_{k\tau}$
at each place in $\sS$ before $\ga$ and $\gb$ are chosen;
thus the descent process in \S5 can be carried out uniformly
in $\ga,\gb$ provided that
$\ga\times\gb$ lies in a small enough open set.
\begin{description}
\item[(i)] We determine necessary and sufficient conditions
for $\gG(\ga,\gb)$ to be locally soluble at $\fp$. We use
these immediately to ensure that $\gG^0(\ga,\gb)$ is locally
soluble at $\fp$; but in (v) we shall also need them to
ensure for a particular $m$ that the corresponding
$\gG(\ga,\gb)$ is not locally soluble at $\fp$.
\end{description}
By requiring that $\gl=\ga/\gb$ is in $\sA$ we have ensured
that $\gG^0(\ga,\gb)$ is soluble in $k_v$ for every $v$ in
$\sS$. From (i) we deduce:
\begin{description}
\item[(ii)] The curve $\gG^0(\ga,\gb)$ is locally soluble at
each $\fp_{k\tau}$.
\end{description}
Thus (i) and (ii) prove that the class of $\gG^0(\ga,\gb)$ is
in the 2-Selmer group of the curve $E(\ga,\gb)$ given by
(\ref{E24}) provided that $\ga,\gb$ are chosen according to
the recipe in Lemma \ref{L1}.
The $\fp_{k\tau}$ are not determined until we know $\ga$ and
$\gb$; but this is unimportant because of the next result:
\begin{description}
\item[(iii)] The bilinear form $e^*_\sB:\;W'_\sB\times W'_\sB\rightarrow
\{\pm1\}$ defined in Theorem \ref{T6} is effectively independent of the choice
of $\ga\times\gb$ and hence of the $\fp_{k\tau}$.
\end{description}
By this we mean that if we change $\ga,\gb$, thereby replacing the old $W'$ by
a new $W'$ canonically isomorphic to it and replacing the old $\fp_{k\tau}$ in
$\sB$ by the new ones, then this isomorphism preserves
$e^*_\sB$. The next result which we need,
which is only meaningful once we have proved (iii), is as follows:
\begin{description}
\item[(iv)] We determine the effect on the function $e^*_\sB$
of introducing a new prime $\fp$ in the way described above.
\end{description}
Finally, the condition which we need for our algorithm to
achieve what we want is as follows:
\begin{description}
\item[(v)] If $m$ is in the kernel of the old $e^*_\sB$ but not
in the inescapable part of it, then we can introduce a new
prime $\fp$ which removes $m$ from the kernel and does not put
anything new in.
\end{description}
It is in the proof of (v) that we need Condition D. Once we
have (v), we can after a sufficient number of steps satisfy
the conditions of Lemma \ref{L11}, and this implies that
$\gG^0(\ga,\gb)$ has rational solutions.

\medskip

\noindent \emph{Achieving} (i).
The condition that any particular
$\gG$ is soluble in $k_\fp$ throughout some neighbourhood of
$\ga_\fp\times\gb_\fp$ is that the
reduction of $\gG(\ga_\fp,\gb_\fp)$ mod $\fp$ should contain a
point defined over $\fo/\fp$ which is liftable to a point on
$\gG$ defined over $k_\fp$. Denote by $L_{k\tau}$ the least
extension of $K_{k\tau}$ over which some absolutely
irreducible component of the singular fibre at $\xi_{k\tau}
\times1$ is defined; conveniently, all these components are
defined over the same least extension, which is normal over
$K_{k\tau}$. The decomposition of $\gG(\ga_\fp,\gb_\fp)$
mod $\fp$ corresponds to the decomposition of the fibre
$\gG(\xi_{k\tau},1)$; so we can solve $\gG$ in
$k_\fp$ in a suitable neighbourhood of $\ga_\fp\times\gb_\fp$
if and only if $\fP$ splits completely in $L_{k\tau}$.

If $f_{k\tau}\|p_{ij}$,
each singular fibre given by $f_{k\tau}=0$
of the pencil of curves $\gG$ splits as a pair
of irreducible conics which meet in two points and are each
defined over the field $L_{k\tau}=K_{k\tau}(\sqrt
{g_{k\tau}(\xi_{k\tau},1)})$; here $g_{k\tau}=m_k$ if
$f_{k\tau}$ divides neither of $m_i$ and $m_j$ or
$g_{k\tau}=m_kp_{jk}$ if $f_{k\tau}$
divides both of them. The same holds if $f^2_{k\tau}|p_{ij}$ and $f_{k\tau}$
divides neither $m_i$ nor $m_j$, and again we have $g_{k\tau}=m_k$;
note that this is the case in which the supplementary condition in Lemma
\ref{L14} applies.
If $f^2_{k\tau}|p_{ij}$ and $f_{k\tau}$ divides both $m_i$ and $m_j$, then
each singular fibre given by $f_{k\tau}=0$ splits as a set of four lines which
form a skew quadrilateral, and each of these lines is defined over
\beq{E56} L_{k\tau}=K_{k\tau}\left(\sqrt{m_k(\xi_{k\tau},1)},
\sqrt{p_{jk}(\xi_{k\tau},1)}\right). \end{equation}
Write $L^0_{k\tau}$ for the field corresponding to
$\gG^0$ under this construction. To test for Condition D, we
need to list those $m$ for
which $L_{k\tau}$ is contained in $L^0_{k\tau}$. It is easy to
verify that they form a group, which contains $m^0$ and the
triples coming from the 2-division points.

\medskip

\noindent \emph{Proof of} (ii). It follows from Lemma \ref{L14} and the
hypotheses of Theorem \ref{T7} that $\gG^0(\ga,\gb)$ is locally soluble at
each $\fp_{k\tau}$ if and only if the same is true for each $C^0_{ij}
(\ga,\gb)$;
and similarly for $\fp$. We know by (i) that
$\gG^0(\ga,\gb)$ is locally soluble at $\fp$. We have
assumed that the
solubility conditions (\ref{E9}) for the $C^0_{ij}(\ga,\gb)$ hold. In the
product (\ref{E9}) if there is a  factor for $\fp$ its value
is 1, so the value of the factor for
$\fp_{k\tau}$ is also 1; hence each $C^0_{ij}(\ga,\gb)$ is locally soluble
at $\fp_{k\tau}$. Now Lemma \ref{L14} shows that $\gG^0(\ga,\gb)$ is
locally soluble at $\fp_{k\tau}$. It is the escape clause in Lemma \ref{L14}
which forces on us the additional condition in the first sentence of Theorem
\ref{T7}.

\medskip

\noindent \emph{Proof of} (iii). We are allowed to choose
$\ga\times\gb$ only
within a set which is small in the topology
induced by $\sS$. In particular, this means that the power of
any prime in $\sS$ which divides any $f_{k\tau}(\ga,\gb)$ is
independent of $\ga$ and $\gb$. Since the only other prime
which divides any particular $f_{k\tau}(\ga,\gb)$ is
$\fp_{k\tau}$,
which does so to the first power, the ideal class of
$\fp_{k\tau}$ is fixed. If the place $v$ is given by some
$\fp_{k\tau}$ then a generator of $W'_v$ can be lifted back to
$\gs\times\tau$ where each of $\gs$ and $\tau$ is either 1
or $f_{k\tau}(\ga,\gb)$; and if $v$ is in $\sS$ the elements of
a base for $W'_v$ can be lifted back to elements
$\gs\times\tau$ independent of $\ga,\gb$ with $\gs,\tau$ in
$\fo^*_\sS$. We choose a base for $W'_\sB$ composed of these
two kinds of elements; then the value of $e^*_\sB$ at any pair
of elements of this base is a product of expressions of the
form $(\gs'(\ga,\gb),\tau'(\ga,\gb))_v$ where $v$ is in $\sB$
and each of $\gs'$
and $\tau'$ is the product of an element of $\fo^*_\sS$ and
possibly an $f_{k\tau}$. If $v$ is in $\sS$ the value
of this expression is independent of $\ga,\gb$. If $v$ is
given by $\fp_{k\tau}$ then using symmetry and 
$(\xi,-\xi)_v=1$ if necessary we can reduce to the case when
$\gs'$ is not divisible by $f_{k\tau}$. If also $\tau'$ is not
divisible by $f_{k\tau}$ then $(\gs'(\ga,\gb),
\tau'(\ga,\gb))_v=1$; otherwise
$(\gs'(\ga,\gb),\tau'(\ga,\gb))_v=L(\sS;\gs',\tau';\ga,\gb)$
is continuous by Lemma \ref{L5} and the proviso shortly after
the statement of Theorem \ref{T7}.

\medskip

\noindent \emph{Achieving} (iv). Let $\ga',\gb'$ be such that
every ideal $(f_{i\gs}(\ga',\gb'))$ is the product of a prime
ideal $\fp'_{i\gs}$ and ideals in $\sS$; 
and let $\ga'',\gb''$ be such that $(f_{k\tau}(\ga'',\gb''))$
is the product of $\fp\fp''_{k\tau}$ and ideals in $\sS$,
while any other $(f_{i\gs}(\ga'',\gb''))$ is the product of
$\fp''_{i\gs}$ and ideals in $\sS$. In the obvious notation,
we take a base for $W'_{\sB'}$ as in the proof of (iii);
replacing $\ga'\times\gb'$ by $\ga''\times\gb''$
in this base, we obtain a linearly
independent set of elements of $W'_{\sB''}$ and we do not
change the values of $e^*$. By adjoining the non-trivial
element of $W'_\fp$ we can extend this linearly independent
set to a base for $W'_{\sB''}$. In other words, there is a
natural injection of $W'_{\sB'}$ into $W'_{\sB''}$ which
preserves $e^*$.

\medskip

\noindent \emph{Choice of} $\fp$. Let $w_\fp$ be a lift to
$W_\fp$ of the non-trivial element of $W'_\fp$, and let $m$ be
an element of $U_\sB\cap W_\sB$ which is not in the
inescapable part of the 2-Selmer group. Thus $\tau^{-1}_\sB m$
is in the kernel of $e^*_\sB$. Suppose that we can choose
$\fp$ so that the 2-covering corresponding to $m$ is locally
insoluble at $\fp$. On the one hand this is equivalent to
$e^*(\tau^{-1}_\sB m,w_\fp)=-1$, which in the notation of the
previous paragraph implies that the kernel of $e^*_{\sB''}$ is
contained in the image of the kernel of $e^*_{\sB'}$. On the
other hand it requires
$\fP$ to split completely in $L^0_{k\tau}$ but not in
$L_{k\tau}$. The condition
below, which in these notes we call Condition D,
ensures that such a choice is possible.
We shall see later that Condition D can be replaced by a
weaker condition, but one which is less natural and sometimes
less computationally convenient.
\begin{quote}
Condition D: \emph{If $m$ is not in the inescapable subgroup
of the $2$-Selmer group, then
there is a pair $k,\tau$ such that the field $L_{k\tau}$ is
not contained in $L^0_{k\tau}$.}
\end{quote}
We can incorporate the definitions of $L_{k\tau}$ and
$L^0_{k\tau}$ into this condition, thereby putting it into a
form closer to that of earlier papers, as follows:
\begin{quote}
\emph{The kernel of the composite map
\[ m\mapsto\oplus_{k,\tau}g_{k\tau}(m)\mapsto\oplus_{k,\tau}
K^*_{k\tau}/\langle K^{*2}_{k\tau},H_{k\tau}\rangle \]
is generated by the inescapable subgroup of the $2$-Selmer group,
where
\[ g_{k\tau}=\begin{cases}
m_k & \text{if $f_k$ divides neither of $m_i$ and $m_j$,} \\
m_kp_{jk} & \text {if $f_k$ divides both of $m_i$ and $m_j$,}
\end{cases} \]
and
\[ H_{k\tau}=\begin{cases}
m_k(\xi_{k\tau},1) & \text{if $f_{k\tau}$ divides neither of $m_i$ and $m_j$,} \\
m_k(\xi_{k\tau},1)p_{jk}(\xi_{k\tau},1) & \text{if $f_{k\tau}\|p_{ij}$ and $f_{k\tau}$ divides $m_i$ and $m_j$,} \\
\{m_k(\xi_{k\tau},1),p_{jk}(\xi_{k\tau},1)\} & \text{if $f_{k\tau}^2\|p_{ij}$ and $f_{k\tau}$ divides $m_i$ and $m_j$.}
\end{cases} \]
} \end{quote}
By a slight abuse of language, we shall say that the $m$ for
which $L_{k\tau}$ is contained in $L^0_{k\tau}$ are those
which do not satisfy Condition D.
If Condition D holds we can choose $k,\tau$ and a $\fP$ which
splits in $L^0_{k\tau}$ but not in $L_{k\tau}$. The underlying
$\fp$ has the properties we want. But the arguments in (iv)
show that the process has removed $m$ from the 2-Selmer group
without creating any new elements of that group. So we have
certainly decreased the dimension of the 2-Selmer group,
which is what we needed to show to justify the algorithm. In
fact it is easy to show that we have decreased it by exactly 1.

\medskip

It will be seen that we have not used the full force of
Condition D; indeed it is stated for all elements of $G^*$,
but we have only used it for those elements which lie in the
initial 2-Selmer group. These are the ones for which the
corresponding 2-covering is locally soluble at each place in
$\sB$. The proof of (ii) above shows that local solubility in
$\sS$ implies local solubility at each $\fp_{k\tau}$; and the
proof of (iii) shows that this 2-Selmer group, considered as
a subgroup of $G^*$, does not vary as $\ga\times\gb$ varies
within a small enough open set. We actually use Condition D
only for the $m$ which lie in this 2-Selmer group; and to
require merely that such $m$ satisfy Condition D is weaker
than the full Condition D. We call this weaker condition, an
equivalent form of which has already appeared in [14] and
[15], Condition E. Its disadvantage is that Condition D is
independent of $\ga$ and $\gb$, whereas this is only true of
Condition E when $\ga\times\gb$ is restricted to a
small enough open set. A particularly favourable case is when
the 2-Selmer group has order 8, so that Condition E is
trivial. I do not know whether Condition E, together with the
conditions imposed in Theorem \ref{T7}, is necessary as well
as sufficient for global solubility,
nor whether these conditions are
together equivalent to the Brauer-Manin conditions.

\medskip

Suppose however that even Condition E fails, and let $m^\sharp,m^\flat$ be
any two
elements of the kernel; to avoid trivialities we may assume that neither of
them is $(1,1,1)$ or comes from a 2-division point. We have examples in which
the Cassels form (\ref{E34}) for such triples can be evaluated as a function of
$\ga,\gb$; and whenever we can evaluate it, it turns out to be continuous in
the topology induced by $\sS$. It would be very interesting to know if this is
a general phenomenon.


\newpage

\noindent 7. \emph{Some variants}. \newline
In this section we outline two ways of extending the results
of \S6. What they have in common is that each of them involves
the simultaneous consideration of descent on two pencils of
elliptic curves, rather than just the one pencil which we were
concerned with in \S6. For each of these pencils we have a
pencil of coverings ($\sqrt{-3}$-coverings in \S7.1 and
2-coverings in \S7.2), both pencils being parametrized by the
same parameter. Our objective is to choose a value of the
parameter in such a way that both the coverings are
simultaneously soluble; as in \S6 we have to do this by
ensuring that both Selmer groups are small. There is however
one major difference between the two subsections; in \S7.1 the
two elliptic curves which we have to deal with are unrelated,
whereas in \S7.2 they are isogenous to each other. We
continue to need Hypothesis \Sha.

The arguments in \S7.1 enable us, without using Schinzel's
Hypothesis, to prove the solubility of
diagonal cubic surfaces
\beq{E39} a_0X_0^3+a_1X_1^3=a_2X_2^3+a_3X_3^3 \end{equation}
over certain algebraic number fields $k$, subject to a
condition on the $a_i$ which is stronger, but not much
stronger, than the Brauer-Manin condition. The
reason why we do not need Schinzel's Hypothesis in this case
is that we need the results of Lemma \ref{L1} only for a
single linear polynomial, so that
Lemma \ref{L1} can be replaced by Dirichlet's theorem on
primes in arithmetic progression.

In \S7.2 we mimic the ideas of \S6 in the
case when the Jacobian has only one rational 2-division point;
in particular this enables us to address in \S8 the question
of the existence of rational points on Del Pezzo surfaces of
degree 4. The approach in \S8 also does not require Schinzel's
Hypothesis, though for a totally different reason.
The price of each of these refinements is that the
corresponding argument becomes more
intricate; what we give here is only an outline sufficient
to show the ideas involved, and we refer any sufficiently
intrepid reader to the original papers
for the details. These papers are [16] for the first case,
and [1] or [3] for the second.

\bigskip

\noindent 7.1 \emph{Diagonal cubic surfaces}. \newline
Without loss of generality we can assume that the $a_i$ in
(\ref{E39}) are integers. To show that (\ref{E39}) has a
rational solution it is enough to show that
there exists $c$ in $k^*$ such that each of the two curves
\beq{E43} a_0X_0^3+a_1X_1^3=cX^3, \quad {\mathrm{and}}
\quad a_2X_2^3+a_3X_3^3=cX^3 \end{equation}
is soluble. The hypothesis that (\ref{E39}) is everywhere locally soluble
implies that for each place $v$ in $k$ there exists $c_v$ in $k^*_v$ such that
each of
\[ a_0X_0^3+a_1X_1^3=c_vX^3, \quad {\mathrm{and}}
\quad a_2X_2^3+a_3X_3^3=c_vX^3 \]
is soluble in $k_v$. The first step in the argument is to 
deduce the existence of $c$ in $k^*$ such that each of the two equations
(\ref{E43}) is everywhere locally soluble. Such a $c$ always exists; and indeed
if $\sS$ is any given finite set of places of $k$, we can choose $c$ integral
and such that $c/c_v$ is in $k^{*3}_v$ for each $v$ in $\sS$. Following the
methods of \S6, we denote by $\bL^1$ the affine line with the origin deleted.
Let $\sS$ be a set of bad places for the surface (\ref{E39}), which means that
$\sS$ must contain all the primes of $k$ dividing $3a_0a_1a_2a_3$; and let
$\sB\supset\sS$ be a set of bad places for the pair of curves (\ref{E43}), so
that $\sB$ must also contain all the primes dividing $c$. Under the topology
induced by $\sS$, let $\sA$ be the open subset of $\bL^1(k)$ on which each of
the two curves (\ref{E43}) is locally soluble at each place of $\sS$, let $c_0$
be a given point of $\sA$ and let $\sN_0\subset\sA$ be an open neighbourhood of
$c_0$. Because of the possible presence of Brauer-Manin obstructions, it is not
necessarily true that there exists $c$ in $\sN_0$ such that the two equations
(\ref{E43}) are both soluble. But one may still ask what additional assumptions
are needed in order to prove solubility by the methods of \S6 --- always of
course on the basis that Hypothesis {\Sha} holds.

The Jacobians of the two curves (\ref{E43}) are
\beq{E42} Y_0^3+Y_1^3=a_0a_1cY^3 \quad {\mathrm{and}} \quad
Y_2^3+Y_3^3=a_2a_3cY^3 \end{equation}
respectively. The obvious descent to apply to each of them is the
$\rho$-descent, where $\rho=\sqrt{-3}$. Applying this to the elliptic curve
\beq{E40} X^3+Y^3=AZ^3 \end{equation}
replaces it by the equations
\[ \rho X+\rho^2Y=m_1Z_1^3, \quad \rho^2X+\rho Y=m_2Z_2^3, \quad
X+Y=AZ_3^3/m_1m_2 \]
where $Z=Z_1Z_2Z_3$. Here $m_1,m_2,Z_1,Z_2$ are in $K=k(\rho)$ and $m_1,m_2$
are conjugate over $k$, as are $Z_1,Z_2$; but $Z_3$ is in
$k$. It would appear natural to work in
$K$ rather than $k$, since if (\ref{E39}) is soluble in $K$ it is soluble in
$k$. But actually our methods could not then be applied, for complex
multiplication by $\rho$ induces an isomorphism on (\ref{E40}), so that the
Mordell-Weil group of (\ref{E40}) over $K$ has an even number of generators
of infinite order and there is no possibility of applying Lemma \ref{L9}. In
other words, a prerequisite for applying the methods of \S6 is the following
unexpected constraint:
\beq{E41} \sqrt{-3} {\mathrm{~is~not~in~}} k. \end{equation}
This does however allow us to take $k=\bQ$, for example. But even if
(\ref{E41}) holds, there is considerable interplay between the descent theory
over $K$ and that over $k$; and it seems necessary to make use of this
interplay in the argument.

The basic idea is to write $c$ as a product of primes in $\sS$ (which are
forced on us by the choice of $\sN_0$) and some other well-chosen primes; the
latter make up the set $\sB\setminus\sS$. We need to choose the latter so that
the $\rho$-Selmer group of each of the curves (\ref{E42}) has order 9; and
following the precedent of \S6 we expect to do this by adjoining additional
primes one by one to $\sB$, always preserving the local solubility of the
curves (\ref{E43}) and keeping $c$ within $\sN_0$. The latter condition simply
means that each new prime $\fp$ should be close to 1 in our topology and
should be such that $a_0/a_1$ and $a_2/a_3$ are in $k_\fp^{*3}$. But here we
encounter the final pair of complications. To adjoin one more prime divides or
multiplies the order of each $\rho$-Selmer group by 3. If one of these orders
has already been reduced to 9 we cannot reduce it further; so adjoining one
more prime can no longer improve the situation. Instead we eventually reach
the stage when we have to adjoin two more primes simultaneously, in such a
way that the order of one of the $\rho$-Selmer
groups remains unchanged, while the order of the other is divided by 9. To be
able to reduce the orders of both $\rho$-Selmer groups to 9, we therefore need
the initial choice of $c$ to satisfy the following additional condition:
\begin{quote}
The product of the orders of the $\rho$-Selmer groups of the two curves
(\ref{E42}) is a power of 9.
\end{quote}
As should be clear from the preceding discussion, the truth or falsehood of
this statement depends only on $\sN_0$ (provided it is small enough) and not
on the value of $c$ within $\sN_0$. In other words, it depends only on the
choice of $c_0$; and we need to show that we can choose $c_0$ so that (in
addition to the previous requirements) this condition holds at $c_0$. Having
done all this, we still need the equivalent of Condition D or Condition E.

However, at the end of all these complications we do obtain a solubility
theorem for (\ref{E39}). It would be an act of supererogation to work hard to
obtain the best theorem which the method could yield, because the sufficient
condition for solubility which we obtain would almost certainly always be
stronger (though not much stronger) than the actual necessary condition.
The latter is conjectured to be that there is no Brauer-Manin
obstruction. The theorem which is obtained
in [16] is as follows.
\bth{T8} Let $k$ be an algebraic number field not containing the primitive cube
roots of unity. Assume that Hypothesis $\Sha$ holds. If $(\ref{E39})$ is
everywhere locally soluble and the $a_i$ are all cubefree, then each of the
following criteria is sufficient for the solubility of
$(\ref{E39})$ in $k$.

${\mathrm{(i)}}$ There exist primes $\fp_1,\fp_3$ of $k$ not dividing $3$ such
that $a_1$ is a non-unit at $\fp_1$ and $a_3$ is a non-unit at $\fp_3$, but for
$j=1$ or $3$ the three $a_i$ with $i\neq j$ are units at $\fp_j$.

${\mathrm{(ii)}}$ There is a prime $\fp$ of $k$ not dividing $3$ such
that $a_1$ is a non-unit at $\fp$ but the other $a_i$ are units there; and
$a_2,a_3,a_4$ are not all in the same coset of $k^{*3}_\fp$.

${\mathrm{(iii)}}$ There is a prime $\fp$ of $k$ not dividing $3$ such
that exactly two of the $a_i$ are units at $\fp$, and $(\ref{E39})$ is not
birationally equivalent to a plane over $k_\fp$.
\eth

\bigskip

\noindent 7.2 \emph{The case of one rational $2$-division
point}. \newline
In this subsection we shall be concerned with pencils of
2-coverings whose pencil of Jacobians has the form
\[ Y^2=(X-c(U,V))(X^2-d(U,V)) \]
where $c,d$ are homogeneous polynomials in $k[U,V]$ with
$\deg d=2\deg c$. We start by recalling the standard
machinery for 2-descent on
\[ E':\;Y^2=(X-c)(X^2-d) \]
for $c,d$ in $k$ and $d$ not in $k^2$.

If $O'$ is the point at infinity on $E'$ and $P'$ the
2-division point $(c,0)$ then there is an isogeny
$\phi':\;E'\rightarrow E''=E'/\{O',P'\}$ where $E''$ is
\[ E'':\;Y_1^2=(X_1+2c)(X_1^2+4(d-c^2)); \]
the places of bad reduction for $E''$ are the same
as those for $E'$. Explicitly, $\phi'$ is given by
\[ X_1=\frac{d-X^2}{c-X}-2c, \quad
Y_1=\frac{Y(X^2-2cX+d)}{(X-c)^2}. \]
There is also a dual isogeny $\phi'':\;E''\rightarrow E'$, and
$\phi''\circ\phi'$ and $\phi'\circ\phi''$ are the doubling
maps on $E'$ and $E''$ respectively. We are primarily
interested in the case when neither $d$ nor $c^2-d$ is a
square in $k$, so that $E'$ and $E''$ each contain only one
primitive 2-division point defined over $k$.

The elements of H$^1(k,\{O',P'\})\sim k^*/k^{*2}$ classify
the $\phi'$-coverings of $E''$; the covering corresponding
to the class of $m'$ is
\beq{E64} V_1^2=m'(X_1+2c), \quad V_2^2=m'(X_1^2+4(d-c^2))
\end{equation}
with the obvious two-to-one map to $E''$. The $\phi'$-covering
corresponding to $P''$ is given by $m'=d$. Similarly the
$\phi''$-coverings of $E'$ are classified by the elements of
H$^1(k,\{O'',P''\})\sim k^*/k^{*2}$, the covering
corresponding to the class of $m''$ being
\beq{E60} W_1^2=m''(X-c), \quad W_2^2=m''(X^2-d).
\end{equation}
The $\phi''$-covering corresponding to $P'$ is given by
$m''=c^2-d$.
We denote by $S'_2$ the 2-Selmer group of $E'$,
and by $S'_\phi,S''_\phi$ the $\phi'$-Selmer group of $E''$
and the $\phi''$-Selmer group of $E'$ respectively.

Write $K=k(d^{1/2})$; then the group of 2-coverings of $E'$
is naturally isomorphic to $K^*/K^{*2}$, where the 2-covering
corresponding to the class of $a+bd^{1/2}$ is given by
\[ Z_1^2=(a^2-db^2)(X-c), \quad
(Z_2\pm d^{1/2}Z_3)^2=(a\pm bd^{1/2})(X\pm d^{1/2}). \]
In homogeneous form, this can be written
\beq{E62} \left. \begin{array}{c}
Z_2^2+dZ_3^2=aZ_1^2/(a^2-db^2)+(ac+bd)Z_0^2, \\
2Z_2Z_3=bZ_1^2/(a^2-db^2)+(a+bc)Z_0^2.
\end{array} \right\} \end{equation}
Call this curve $\gG'$; then the map
$\gG'\rightarrow E'$ has degree 4 and is given by
\[ X=\frac{Z_1^2}{(a^2-db^2)Z_0^2}+c, \quad
Y=\frac{Z_1(Z_2^2-dZ_3^2)}{(a^2-db^2)Z_0^3}. \]
The map $\gG'\rightarrow E'$ can be factorized as
$\gG'\rightarrow C''\rightarrow E'$, where $C''$ is the
$\phi''$-covering of $E'$ given by (\ref{E60}) with
$m''=a^2-db^2$ and the map $\gG'\rightarrow C''$ is
\[ W_1=Z_1/Z_0, \quad W_2=(Z_2^2-dZ_3^2)/Z_0^2. \]

Conversely, suppose that we have a curve of genus 1 defined
over $k$ and given by the equations
\beq{E61} \left. \begin{array}{c}
\ga_0U_0^2+\ga_1U_1^2+\ga_2U_2^2+\ga_3U_3^2+2\ga_4U_2U_3=0, \\
\gb_0U_0^2+\gb_1U_1^2+\gb_2U_2^2+\gb_3U_3^2+2\gb_4U_2U_3=0,
\end{array} \right\} \end{equation}
where the $\ga_i,\gb_i$ are in $\fo$. We have just seen that
any 2-covering of an elliptic curve with one rational
2-division point can be put in this form, and we shall now
prove the converse. Write $d_{ij}=\ga_i\gb_j-\ga_j\gb_i$; then
the curve (\ref{E61}) takes the more convenient form
\beq{E63} \left. \begin{array}{c}
d_{10}U_0^2+d_{12}U_2^2+2d_{14}U_2U_3+d_{13}U_3^2=0, \\
d_{01}U_1^2+d_{02}U_2^2+2d_{04}U_2U_3+d_{03}U_3^2=0.
\end{array} \right\} \end{equation}
If we write $U_0=2Z_0(d_{14}^2-d_{12}d_{13})$ and
$U_1=Z_1/4d_{34}(d_{14}^2-d_{12}d_{13})$, this last pair of
equations can be identified with (\ref{E62}) provided that
\[ \begin{array}{c}
a=-2(2d_{14}d_{34}+d_{13}d_{23})(d_{14}^2-d_{12}d_{13}), \\
b=d_{01}^{-1}d_{13}(d_{14}^2-d_{12}d_{13}), \\
c=4d_{04}d_{14}-2d_{02}d_{13}-2d_{03}d_{12}, \\
d=4d_{01}^2(d_{23}^2+4d_{24}d_{34});
\end{array} \]
it also follows from these that
\[ \begin{array}{c}
c^2-d=16(d_{04}^2-d_{02}d_{03})(d_{14}^2-d_{12}d_{13}), \\
m''=a^2-db^2=16d_{34}^2(d_{14}^2-d_{12}d_{13})^3.
\end{array} \]
We assume that $d(c^2-d)\neq0$, so that (\ref{E61}) defines a
nonsingular curve of genus 1.

Now let $\sS$ be a finite set of places which contains the
infinite places, the primes which divide 2, the odd primes of
bad reduction for $E'$ (or $E''$) and a set of generators for
the ideal class group of $k$. For any $v$ in $\sS$ we write
\[ V'_v=\mathrm{H}^1(k_v,\{O',P'\})\sim k^*_v/k^{*2}_v \]
and similarly for $V''_v$; and we denote by $W'_v$ the image
of $E''(k_v)/\phi'E'(k_v)$ in $V'_v$ and similarly for
$W''_v$. Thus $m'$ lies in $W'_v$ if and only if $\gG'$ is
soluble over $k_v$, and similarly for $W''_v$. There is
a non-degenerate canonical pairing
\beq{E65} V'_v\times V''_v\rightarrow\{\pm1\} \end{equation}
induced by the Hilbert symbol,
under which the orthogonal complement of $W'_v$ is $W''_v$.
As in \S5, we write
\[ V'_\sS=\oplus_{v\in\sS}V'_v, \quad
W'_\sS=\oplus_{v\in\sS}W'_v \]
and similarly for $V''$ and $W''$. The machinery in the first
half of \S5 needs to be modified to take account of the
changed circumstances, but the proofs (which can be found in
[1] or [3]) involve no new ideas. The appropriate
replacement of the Corollary to Lemma \ref{L10} is as follows.
\ble{L25} Let $\sS_0^+,\sS_1^+$ be as in \S$5$ and let
$\sS\supset\sS_1^+$. For each $v$ in $\sS$ there exist
subspaces $K'_v\subset V'_v$ and $K''_v\subset V''_v$ such
that

$\mathrm{(i)}$ $K''_v$ is the orthogonal complement of $K'_v$
under the pairing $(\ref{E65})$;

$\mathrm{(ii)}$ $V'_\sS=U'_\sS\oplus K'_\sS$ and
$V''_\sS=U''_\sS\oplus K''_\sS$ where $U'_\sS,U''_\sS$ are the
images of $X_\sS\times X_\sS=(\fo^*_\sS/\fo^{*2}_\sS)^2$ in
$V'_\sS$ and $V''_\sS$ respectively;

$\mathrm{(iii)}$ if $v$ is not in $\sS^+_0$ we can take $K'_v$
and $K''_v$ to be the images of $(\fo^*_v/\fo^{*2}_v)^2$.
\ele
It follows from (\ref{E65}) that there is a non-degenerate
canonical pairing
\beq{E66} V'_\sS\times V''_\sS\rightarrow\{\pm1\}
\end{equation}
and from (i) that $K''_\sS=\oplus_{v\in\sS}K''_v$ is the
orthogonal complement of $K'_\sS$ under this pairing.
\ble{L26} If $\sS\supset\sS^+_1$ then $S'_\phi$ is
isomorphic to each of $U'_\sS\cap W'_\sS$, the left kernel of
the map $U'_\sS\times W''_\sS\rightarrow\{\pm1\}$ induced by
$(\ref{E66})$, and the left kernel of the map $W'_\sS\times
U''_\sS\rightarrow\{\pm1\}$ induced by $(\ref{E66})$. A
similar result holds for $S''_\phi$.
\ele

Let $t'_\sS:V'_\sS\rightarrow U'_\sS$ be the projection along
$K'_\sS$ and similarly for $t''_\sS$. We now diverge from the
notation of \S5, writing
\[ \bU'_\sS=U'_\sS\cap(W'_\sS+K'_\sS), \quad
\bW'_\sS=W'_\sS/(W'_\sS\cap K'_\sS) \]
and similarly for $\bU''_\sS$ and $\bW''_\sS$; as in \S5, the
map $t'_\sS$ induces an isomorphism $\tau'_\sS:\bW'_\sS
\rightarrow \bU'_\sS$, and there is an analogous isomorphism
$\tau''_\sS:\bW''_\sS\rightarrow \bU''_\sS$. The pairing
(\ref{E66}) induces pairings
\beq{E67} \bU'_\sS\times\bW''_\sS\rightarrow\{\pm1\}, \quad
\bW'_\sS\times\bU''_\sS\rightarrow\{\pm1\} \end{equation}
and the action of $\tau'_\sS\times(\tau''_\sS)^{-1}$ takes the
first pairing into the second. The left kernel of either of
these pairings is isomorphic to $S'_\phi$ and the right kernel
to $S''_\phi$. The action of $\tau'_\sS\times1$ takes the
first pairing into the pairing
\[ \bW'_\sS\times\bW''_\sS\rightarrow\{\pm1\}. \]

\medskip

Our objective is to prove the solubility in $k$ of pencils of
curves (\ref{E61}), where we assume that the $\ga_i,\gb_i$ are
homogeneous polynomials in $U,V$, that all the $\ga_i$ have
the same degree and that all the $\gb_i$ have the same degree.
Henceforth we denote by $\sS$ the set of bad places for the
pencil (\ref{E61}). As in \S6, we need to work not in
$\bP^1(k)$ but in the open
subset $\bL^1(k)$ in which $d(c^2-d)\neq0$. In order to make
the proof work, we shall have to impose additional conditions,
some but not all of which are necessary for solubility; these
will be introduced at the places in the proof where they are
first needed. Our eventual result will be as follows:
\bth{T10} Assume Schinzel's Hypothesis and Hypothesis \Sha.
Suppose that Conditions $1$ to $4$ below hold in an open subset
$\sA\subset\sN$ of $\bL^1(k)$ in which $\gG'(\ga,\gb)$ is
locally soluble at all places of $\sS$. Then $\sA$ lies in the
closure of the set of $\gl=\ga/\gb$ in $\bL^1(k)$ at which
$\gG'(\ga,\gb)$ is soluble in $k$. If instead of solubility
we merely require $\gG'(\ga,\gb)$ to contain a $0$-cycle of
odd degree, then we need no longer assume Schinzel's
Hypothesis.
\eth

The proof of the last sentence is derived from the proof of
the rest of the theorem by applying Lemma \ref{L2} instead of
Lemma \ref{L1}, and we shall make no further mention of it.
Now suppose that $\ga,\gb$ are such that $\gG'(\ga,\gb)$ is
everywhere locally soluble. In order to use Lemma \ref{L1} to
prove that $\gG'(\ga,\gb)$ is soluble in $k$, we have to show
that the 2-Selmer group of $E'$ has order 4. Unfortunately the
obvious way of computing the 2-Selmer group requires us to
know the ideal class group of $k(\sqrt{d(\ga,\gb)})$, and we
know very little about how this depends on $\ga\times\gb$.
However Lemma \ref{L22} below provides a way round this
difficulty. It is now
necessary to split cases according as $\gG'$ is in $S'_\phi$
or not; we confine ourselves in these notes to the latter
case, which is the more general and also the more complicated.
(A treatment of the other case can be found in [1] or
[3].) To ensure that we are in this latter case, we need
$C''$ above not to be $E''$; in other words, $m''$ must not
be a square. In view of the formulae above, the condition for
this is that $d_{14}^2-d_{12}d_{13}$ is not a square in
$k[U,V]$. This is included in Condition 1 below.
\ble{L22} Suppose that $P'$ is the only primitive
$2$-division point of $E'$ defined over $k$ and similarly for
$P''$ on $E''$. If the orders of $S'_\phi$ and $S''_\phi$ are
$2$ and $4$ respectively then the order of $S'_2$ is at most
$4$. \ele
\emph{Proof}. Let $\gG'$ be a 2-covering of $E'$ and denote
by $C''$ the quotient of $\gG'$ by the action of the group
$\{O',P'\}$; then $C''$ is a $\phi''$-covering of $E'$ and
we have a commutative diagram
\[ \begin{CD}
E' @>{\phi'}>> E'' @>{\phi''}>> E' \\
@|           @|              @| \\
\gG' @>>>      C'' @>>>         E'
\end{CD} \]
where the first two vertical double lines mean that $\gG'$
and $C''$ are principal homogeneous spaces for $E'$ and $E''$
respectively. If $\gG'$ is identified with the element $f$ of
H$^1(k,E'[2])$ then $C''$ is identified with $\phi'\circ f$
as an element of H$^1(k,E''[\phi''])$. If $\gG'$ is in $S'_2$
then $C''$ is in $S''_\phi$; so we can construct all the
elements of $S'_2$ by lifting back the elements of $S''_\phi$.
But by hypothesis $P''$ is not in $\phi'E'(k)$, so the two
elements of $S'_\phi$ must correspond to the points $O''$ and
$P''$ as members of $E''(k)/\phi'E'(k)$; hence regarded as
elements of $S'_2$ they are equivalent. In other words,
$E''$ regarded as an element of $S''_\phi$ lifts back to only
one element of $S'_2$; so the same is true of each
element of $S''_\phi$.  \qed

To make use of Lemma \ref{L22} we have to study simultaneously
the $\phi'$-descent on $E''$ and the $\phi''$-descent on $E'$.
We imitate
as far as possible the machinery of the proof of Theorem
\ref{T7}. Write
\[ R=d_{01}(d_{23}^2+4d_{24}d_{34})(d_{04}^2-d_{02}d_{03})
(d_{14}^2-d_{12}d_{13}), \]
so that the singular fibres of any of our pencils are given by
$R=0$. As in the proof of Theorem \ref{T7}, we let $f_\tau$
for $\tau=1,2,\ldots$ run through the distinct irreducible
factors of $R$; without loss of generality we can suppose that
the coefficients of any $f_\tau$ are integers and that there
is no prime outside $\sS$ which divides all of them. When we
choose $\ga,\gb$ by Lemma \ref{L1} so that all the
$f_\tau(\ga,\gb)$ are prime up to factors in $\sS$, we shall
denote this additional prime factor of $f_\tau(\ga,\gb)$ by
$\fp_\tau$. We shall see in (iii) below that $S'_\phi$ and
$S''_\phi$ are
effectively independent of the choice of $\ga,\gb$. Each step
of our algorithm will consist of introducing a new prime
$\fp$ in such a way that we diminish one of $S'_\phi$ and
$S''_\phi$ without increasing the other. Write
$K_\tau=k[X]/f_\tau(X,1)$ and let $\xi_\tau$ be the class of
$X$ in $K_\tau$; then those $\fp$ which we can introduce in
such a way that $\fp$ will divide the value of
$f_\tau(\ga,\gb)$ are just the ones such that
$\fp$ has a factor $\fP$ in $K_\tau$ whose relative norm for
$K_\tau/k$ is $\fp$. The arguments needed to validate each
step are again lengthy, and as in the proof of Theorem
\ref{T7} we list them as (i) to (v). Here (ii), (iii) and (iv)
are essentially identical with those in the earlier proof, (i)
is similar but considerably more complicated, and (v) is
substantially different.

\medskip

\noindent (i) \emph{Local solubility at $\fp$}. As before, for
the local solubility of $\gG'(\ga,\gb)$ at $\fp$ we need to
study the decomposition of $\gG'(\xi_\tau,1)$.
Condition 1 primarily restricts the multiplicity of the
singular fibres; but in order to simplify the rest of the
argument, it is rather stronger than is needed for this
purpose, and than the corresponding condition in \S6.
\begin{quote} Condition 1. \emph{$R$ has no repeated factor
in $k[U,V]$}.
\end{quote}
\ble{L23} If Condition $1$ holds then the absolutely
irreducible components of $\gG'(\xi_\tau,1)$ have
multiplicity one and there are at most two of them.
\ele
\emph{Proof} We shall write $L^0_\tau$ for the least field of
definition of any of these components. Condition 1 implies
that $f_\tau$ cannot divide both of $\ga_0$ and $\gb_0$, nor
both of $\ga_1$ and $\gb_1$. If $f_\tau\|d_{01}$ it follows
that $f_\tau$ cannot divide one of $\ga_0$ and $\ga_1$ and
also one of $\gb_0$ and $\gb_1$. Suppose that it divides
neither $\ga_0$ nor $\ga_1$. Since $f_\tau$ does not divide
$d_{04}^2-d_{02}d_{03}$ the second equation (\ref{E63})
splits over $L^0_\tau=K_\tau(\sqrt{d_{04}^2-d_{02}d_{03}})$
where the right hand side is to be evaluated at $\xi_\tau
\times1$. Now if the values of $\ga_2U_2^2+\ga_3U_3^2
+2\ga_4U_2U_3$ and $d_{02}U_2^2+d_{03}U_3^2+2d_{04}U_2U_3$
at $\xi_\tau\times1$, considered as quadratic forms in
$U_2,U_3$, have a common factor then $f_\tau|(d_{23}^2+
4d_{24}d_{34})$; but this contradicts Condition 1. Hence on
either of the planes given by the second equation (\ref{E63})
at $\xi_\tau\times1$ the first equation (\ref{E61}) determines
an irreducible conic. In other words, if $f_\tau|d_{01}$ the
singular fibre is the union of two irreducible conics each
defined over $L^0_\tau$.

If instead $f_\tau\|(d_{04}^2-d_{02}d_{03})$ then $f_\tau$
cannot divide both $d_{02}$ and $d_{03}$, so to fix ideas we
can assume that $f_\tau{\not|}d_{02}$. The second equation
(\ref{E63}) splits over $L^0_\tau=K_\tau(\sqrt{-d_{01}d_{02}})$
where the right hand side is again to be evaluated at
$\xi_\tau\times1$; and the first equation (\ref{E63}) is absolutely irreducible. Hence the singular fibre is again the union
of two irreducible conics each defined over $L^0_\tau$. A
similar argument works if $f_\tau\|(d_{14}^2-d_{12}d_{13})$.

If finally $f_\tau\|(d_{23}^2+4d_{24}d_{34})$ then at
$\xi_\tau\times1$ every linear combination of the two
equations (\ref{E63}) must be absolutely irreducible, because
otherwise $\ga_2U_2^2+\ga_3U_3^2+2\ga_4U_2U_3$ and
$\gb_2U_2^2+\gb_3U_3^2+2\gb_4U_2U_3$ would be proportional at
$\xi_\tau\times1$ and so $f_\tau|(d_{23}^2+4d_{24}d_{34})^2$.
Hence the singular fibre is absolutely irreducible (though it
will be singular) and we can take $L^0_\tau=K_\tau$.  \qed

In all these cases the condition that $\gG'$ is soluble in
$k_\fp$ at each point in some neighbourhood of $\ga_\fp\times
\gb_\fp$, where $\fp|f_\tau(\ga_\fp,\gb_\fp)$, is that $\fP$
splits in $L^0_\tau$. A more exhaustive list of possibilities,
under a hypothesis weaker than Condition 1, may be found in
Lemma 5 of [1].

We also need to know when the curves (\ref{E64}) and
(\ref{E60}) are not locally soluble at $\fp=\fp_\tau$, which
is the same as evaluating $W'_\fp$ and $W''_\fp$. Here
we adopt a somewhat weaker hypothesis than Condition 1.
\ble{L24} Suppose that $c$ and $d$ are in $\fo$ and let $\fp$
be an odd prime ideal of $k$ which divides one but not both
of $d$ and $c^2-d$. Then $W'_\fp$ and $W''_\fp$ are as in the
following table, in which $v$ denotes the normalized additive
valuation associated with $\fp$:
\[ \begin{matrix}
v(c^2-d) \text{ odd} & \quad & m'\in k_\fp^{*2}, \text{ any }m''; \\
v(c^2-d)>0 \text{ even}, 2c\in k^{*2}_\fp  & \quad & m'\in k_\fp^{*2}, \text{ any }m''; \\
v(c^2-d)>0 \text{ even}, 2c{\not\in} k^{*2}_\fp  & \quad & v(m')\text{ and $v(m'')$ even}; \\
v(d) \text{ odd} & \quad & m''\in k^{*2}_\fp,\text{ any }m'; \\
v(d)>0 \text{ even},-c\in k^{*2}_\fp & \quad & m''\in k^{*2}_\fp,\text{ any }m'; \\
v(d)>0 \text{ even},-c{\not\in} k^{*2}_\fp & \quad & v(m')\text{ and $v(m'')$ even}.
\end{matrix} \]
If $\fp$ divides $m''$ to an odd power but does not divide
$c^2-d$, then $(\ref{E60})$ is insoluble in $k_\fp$; similarly
if $\fp$ divides $m'$ to an odd power but does not divide
$d$, then $(\ref{E64})$ is insoluble in $k_\fp$.
\ele
\emph{Proof} The last sentence follows from the earlier ones.
For the first three lines, consider the equations (\ref{E64}).
In the first line of the table, if $v(X_1^2)<v(c^2-d)$ then
$m'$ is a square by the second equation (\ref{E64}); if not,
then $v(m')$ is odd and the first equation gives a
contradiction. Hence $m'$ is a square, and by the pairing
(\ref{E65}) there is no constraint on $m''$. In the second and
third lines, if $v(X_1^2)<v(c^2-d)$ then $m'$ must be a square;
if not, then $2cm'$ must be a square. Using (\ref{E65}) as
before, this gives the second line. We can choose $X_1$ so
that $v(X_1^2)=v(c^2-d)$ and $2c(X_1^2+4(d-c^2))$ is a square;
so $m'$ can be in the class of $2c$. Using (\ref{E65}) again,
this gives the third line. The remaining lines now follow by
duality.  \qed

Lines 2 and 3 in this table will not be needed, in view of
Condition 1 and the formula for $c^2-d$; but their duals
(which are lines 5 and 6) are needed, and lines 2 and 3 are
included for completeness.

\medskip

\noindent (ii) \emph{Local solubility of $\gG^1(\ga,\gb)$ at
$\fp_\tau$}. In \S5 we used Lemma \ref{L14} to relate local
solubility of $\gG'$ to local solubility of the $C_{ij}$;
similarly here we relate local solubility of $\gG'$ to local
solubility of each of the two equations (\ref{E63}). By the
results of \S4, the latter requires
\begin{quote} Condition 2. \emph{There is a non-empty open set
$\sA\subset\sN$ in which $\gG'$ is locally soluble at each
place in $\sS$ and the following conditions hold:
\[ \begin{align*}
& \text{if $f_\tau\|(d_{04}^2-d_{02}d_{03})$ then } L(\sS;-d_{01}d_{02},f_\tau;\ga,\gb)=1; \\
& \text{if $f_\tau\|(d_{14}^2-d_{12}d_{13})$ then } L(\sS;d_{01}d_{12},f_\tau;\ga,\gb)=1; \\
& \text{if $f_\tau\|d_{01}$ then } L(\sS;d_{04}^2-d_{02}d_{03},f_\tau;\ga,\gb)=1.
\end{align*} \]} \end{quote}
Thus Condition 2 is necessary for solubility. Note that
\[ L(\sS;d_{14}^2-d_{12}d_{13},f_\tau;\ga,\gb)
=L(\sS;d_{04}^2-d_{02}d_{03},f_\tau;\ga,\gb) \]
if $f_{\tau}|d_{01}$, so symmetry between $U_0$ and $U_1$ in
(\ref{E63}) is preserved.

If $\ga,\gb$ are chosen so as to meet the requirements of
Lemma \ref{L1}, then each Legendre-Jacobi function in
Condition 2 reduces to a single Hilbert symbol, taken at
$\fp_\tau$; so in each of the three cases listed, the
condition implies that $\fP$ splits in $L^0_\tau$ in the
notation of the proof of Lemma \ref{L23}. In other words,
$\gG'(\ga,\gb)$ is soluble in $k_\fp$. The argument in the
case $f_\tau\|(d_{23}^2+4d_{24}d_{34})$ is even simpler.

\medskip


\noindent (iii) \emph{Independence of} $\ga,\gb$. The argument
here
is exactly the same as in \S6. If $\sB$ is the union of $\sS$
and all the $\fp_\tau$, by abuse of language we can now
describe $\bW'_\sB$ as the union of $\bW'_\sS$ and spaces
$\bW'_\tau$ associated with $f_\tau$. The space $\bW'_\tau$ is
one-dimensional if (\ref{E64}) is soluble with $\fp_\tau\|m'$
and trivial otherwise; the former case corresponds to lines
4 and 5 of the table in Lemma \ref{L24}. A similar remark
holds for $\bW''_\sB$; and we can provide similar descriptions of $\bU'_\sB$ and $\bU''_\sB$.

\medskip

\noindent (iv) \emph{Effect of introducing} $\fp$. Here again
the
argument is essentially that of \S6. In the description given
in (iii), the effect of introducing $\fp$ into $f_\tau$ will
be as follows. Suppose first that $f_\tau|d$; then if the
conditions of lines 4 or 5 of the table in Lemma \ref{L24} are
satisfied, the new $\bW'$ will be the sum of the old $\bW'$
and a one-dimensional space $\bW'_\fp$, and $\bW''$ will be
unchanged. Moreover, if $m''$ represents an element of
$\bU''$ and $w'_\fp$ is the generator of $\bW'_\fp$ then the
image of $w'_\fp\times m''$ under the second pairing
(\ref{E67}) is 1 if $m''$ is in $k^{*2}_\fp$ and $-1$
otherwise. If instead the conditions of line 6 of the table
are satisfied, then both $\bW'$ and $\bW''$ are unchanged.
Secondly, suppose that $f_\tau|(c^2-d)$, so that by Condition
1 we must be in line 1 of the table; then the conclusions are
similar to those above for line 4.

\medskip

\noindent (v) \emph{Choice of} $\fp$. It remains to show that
if $S'_\phi,S''_\phi$ do not satisfy the hypotheses of
Lemma \ref{L22} then we can choose $\fp$ so as to decrease
one of $S'_\phi$ and $S''_\phi$ without increasing the other.
To achieve this we need to impose two further conditions,
which together serve the same purpose as Condition D in \S6.
It will be clear from (iv) that we cannot expect symmetry in
the treatment of $\bW'$ and $\bW''$.

Suppose first that $m'$ is in $S'_\phi$ and is not 1 or $d$.
To remove $m'$ from $S'_\phi$ we need to introduce $\fp$
satisfying line 1 of the table in Lemma \ref{L24}; when we do
so we shall not increase $S''_\phi$ because no element of the
new $\bW''$ not lying in the old $\bW''$ can be orthogonal to
$m'$. Thus we must find a first degree prime $\fP$ in $K_\tau$
which remains prime in $L_\tau=K_\tau(\sqrt{m'(\xi_\tau,1)})$;
and we already know that it must split in $L^0_\tau$, which is
the quadratic extension of $K_\tau$ defined in the proof of
Lemma \ref{L23}. Using the convention introduced in (iii),
what we need in order to ensure that this is possible is
\begin{quote}
Condition 3. \emph{If $m'$ is in $\bU'_\sB$ and not $1$ or $d$,
then there is an $f_\tau|(c^2-d)$ such that $L_\tau\neq
L^0_\tau$.}
\end{quote}

If instead $m''$ is in $S''_\phi$ and is not in the
subgroup generated by $(c^2-d)$ and $(a^2-db^2)$, then we
have $L_\tau=K_\tau(\sqrt{m''(\xi_\tau,1)})$. A similar
argument now shows that what we need is
\begin{quote}
Condition 4. \emph{If $m''$ is in $\bU''_\sB$ and not in the
subgroup generated by $d_{04}^2-d_{02}d_{03}$ and $d_{14}^2-
d_{12}d_{13}$ then there exists $f_\tau$ such that either
$f_\tau|(d_{23}^2+4d_{24}d_{34})$ with $m''(\xi_\tau,1)$ not
a square in $K_\tau$ or else $f_\tau|d_{01}$ with
$K_\tau(\sqrt{m''(\xi_\tau,1)})$ not contained in
$L^0_\tau(\sqrt{-c(\xi_\tau,1)})$.}
\end{quote}
Like Condition D, these can be rewritten in the language of
earlier papers; and they can be replaced by weaker but less
convenient conditions in just the same way that Condition D
can be replaced by Condition E.



\newpage

\noindent 8. \emph{Del Pezzo surfaces of degree $4$} \newline

Let $V$ be a Del Pezzo surface of degree 4 (that is, the smooth intersection
of two quadrics in $\bP^4$) defined over an algebraic number field $k$.
Salberger and Skorobogatov [11] have shown that the only obstruction to weak
approximation on $V$ is the Brauer-Manin obstruction. More precisely:
\bth{T1} Suppose that $V(k)$ is not empty. Let $\sA$ be the subset of the
adelic space $V(\bA)$ consisting of the points $\prod P_v$ such that
\[ \sum{\mathrm{inv}}_v(A(P_v))=0 {\mathrm{~in~}} \bQ/\bZ \]
for all $A$ in the Brauer group ${\mathrm{Br}}(V)$. Then the image of $V(k)$
is dense in $\sA$.
\eth
In the first part of this section I give a simpler proof of this theorem. What
I actually prove is Theorem \ref{T2} below, which is equivalent to Theorem
\ref{T1} because of Lemma \ref{L7}. Readers who are content with Theorem
\ref{T2} need not trouble themselves with the Brauer-Manin condition.
\bth{T2} Let $\sB$ be a finite set of places of $k$, satisfying the conditions
for $(\ref{E15})$ analogous to those stated above for $(\ref{E3})$.

${\mathrm{(i)}}$ For each
$v$ in $\sB$ let $S_v$ be a point of $V$ defined over $k_v$, and let $\gl_v$ be
its image under the projection to $\bP^1$. Suppose that all the conditions like
\beq{E19} \prod_{v\in\sB}\ell^*(v;-a_0a_1,c;\gl_v)=1
\end{equation}
hold. Then there is a point of $V(k)$ as close as we like to each $S_v$.

${\mathrm{(ii)}}$ Let $\gl$ be a point of $\bP^1(k)$ such that all the
conditions like
\[ L(\sB;-a_0a_1,c;\gl)=1 \]
hold. Then there is a point in $V(k)$ whose projection into $\bP^1(k)$ is as
close as we like to $\gl$ in the topology induced by $\sB$.
\eth

Since we can find $\gl$ arbitrarily close to each $\gl_v$, it
follows from the analogue of
(\ref{E22}) that the two parts of the theorem are equivalent. In view of the
first assertion of Lemma \ref{L7} and the fact that weak approximation holds
for conics, Theorem \ref{T2}(ii) is equivalent to Theorem \ref{T1}. The idea
of the proof is that we can use the existence of a point of $V(k)$ to fibre
$V$ by conics. Theorem \ref{T3} allows us to find a positive 0-cycle of degree
8 on $V$ defined over $k$ satisfying pre-assigned approximation conditions;
and the proof is then completed by a modification of an argument of Coray.
Later in this section, we give Coray's full result as Theorem \ref{T5}.

\medskip

Write the Del Pezzo surface $V$ as $Q_1\cap Q_2$ where $Q_1,Q_2$ are quadrics in $\bP^4$. Choose
coordinates so that the given point of $V(k)$ is $(1,0,0,0,0)$ and the tangents
to $Q_1,Q_2$ at this point are $X_1=0,X_2=0$ respectively. Thus the equations
of $Q_1$ and $Q_2$ can be written
\beq{E6} X_0X_1+f_1(X_1,\ldots,X_4)=0, \quad
X_0X_2+f_2(X_1,\ldots,X_4)=0 \end{equation}
where $f_1,f_2$ are homogeneous quadratic. The variety (\ref{E6}) is
birationally equivalent to the cubic surface $X_2f_1=X_1f_2$, which is indeed
obtained by blowing up the given point of $V(k)$; and this cubic surface is
birationally equivalent to the pencil of affine conics
\beq{E14} Vf_1(U,V,X_3,X_4)=Uf_2(U,V,X_3,X_4), \end{equation}
which with some abuse of language can be parametrized by the points $(U,V)$ of
$\bP^1$. Diagonalizing this equation and then making it homogeneous gives a
pencil of projective conics of the form
\[ Z_0^2g_1(U,V)+Z_1^2g_2(U,V)/g_1(U,V)+Z_3^2g_5(U,V)/g_2(U,V)=0, \]
where $g_r$ is homogeneous of degree $r$. Writing
\[ Z_0=g_2Y_0, \quad Z_1=g_1Y_1, \quad Z_2=g_1g_2Y_2 \]
and dividing by $g_1g_2$ we obtain
\beq{E15} g_2Y_0^2+Y_1^2+g_1g_5Y_2^2=0. \end{equation}
We shall assume that the $g_r$ are coprime in pairs in $k[U,V]$; if not, there
is a further simplification of (\ref{E15}) and of the subsequent argument which
is left to the reader.

In principle, the idea of the proof of Theorem \ref{T2} is to construct
a sequence
of positive 0-cycles defined over $k$ of decreasing degrees, each satisfying
the conditions like (\ref{E17}), until we obtain a
point $P_0$ in $V(k)$ satisfying the given local conditions; and
indeed this is what we shall do in the last part of the proof. But it is not
obvious how the local descriptions of successive elements of the sequence are
related. So although the application of Theorem \ref{T3} to (\ref{E15}) shows
that there is a positive 0-cycle of degree 8 satisfying any assigned
local conditions, we do not yet know what local conditions to impose on it for
$P_0$ to be close in the topology induced by $\sB$ to the adelic point which
is our target. To cope with this, we first run the process backwards.

{From} now on, any $\fb^r$ or $\fb^r_v$ will be a positive 0-cycle on $V$,
defined over $k$ or $k_v$ respectively, and $\fa^r$ or $\fa^r_v$ will be its
projection on $\bP^1$.
For each $v$ in $\sB$ we choose two distinct hyperplanes $H'_v$ and $H''_v$,
each defined over $k_v$ and passing through $S_v$. Choose $H'$, a hyperplane
defined over $k$ and close to each $H'_v$, and similarly for $H''$. The
intersection $H'\cap H''\cap V$ is a positive 0-cycle $\fb^1$ of degree 4
defined over
$k$; and though $\fb^1$ may be irreducible over $k$ it is reducible over $k_v$
for each $v$ in $\sB$ because it has one point close to $S_v$. Thus we can
write $\fb^1=\fb^2_v\cup\fb^3_v$ where $\fb^2_v,\fb^3_v$ are positive 0-cycles
of
degrees 1,\,3 respectively defined over $k_v$ and $\fb_v^2$ is close to $S_v$.
Hence
\begin{align*}
1= & L^*(\sB;-a_0a_1,c;\fa^1) = \prod\ell^*(v;-a_0a_1,c;\fa^2_v\cup\fa^3_v) \\
 = & \prod\ell^*(v;-a_0a_1,c;\fa^2_v)\prod\ell^*(v;-a_0a_1,c;\fa^3_v)
\end{align*}
where the products are each taken over all $v$ in $\sB$. But the first product
in the second line is 1, by continuity applied to (\ref{E19}); hence
\beq{E30} \prod\ell^*(v;-a_0a_1,c;\fa^3_v)=1. \end{equation}

Now let $P_1$ and $P_2$ be two points of $V(k)$; there are $\infty^6$
curves on $V$ which are the intersection of $V$ with a quadric and have double
points at $P_1$ and $P_2$. For each $v$ in $\sB$, let $C'_v$ and $C''_v$ be
two such curves defined over $k_v$ each of which also passes through the three
points of $\fb^3_v$, and let $Q'_v,Q''_v$ be quadrics defined over $k_v$ which
contain $C'_v,C''_v$ respectively but neither of which contains the whole of
$V$. Choose $Q'$, a quadric defined over $k$, close to each $Q'_v$ and touching
$V$ at $P_1$ and $P_2$, and similarly for $Q''$; since $Q'$ is given by a
single equation and the tangency conditions are linear in the coefficients,
this is just a matter of weak approximation. The intersection
\[ Q'\cap Q''\cap V=4\{P_1\}\cup4\{P_2\}\cup\fb^4. \]
(This fails if $Q'$ and $Q''$ have a common component; but we can ensure that
this does not happen by requiring $P_1,P_2$ and $\fb^1$ to be in sufficiently
general position. Similar remarks are needed at each stage of the proof.)

Much as before, $\fb^4=\fb^5_v\cup\fb^6_v$ over $k_v$ for each $v$ in $\sB$,
where each $\fb^5_v$ has degree 3 and is close to $\fb^3_v$, and each $\fb^6_v$
has degree 5; hence
\[ \prod\ell^*(v;-a_0a_1,c;\fa^5_v)=1 \]
follows from (\ref{E30}) by continuity. But
\[ L(\sB;-a_0a_1,c;\gl_1)=L(\sB;-a_0a_1,c;\gl_2)=1 \]
where $\gl_1,\gl_2$ are the projections of $P_1,P_2$ on $\bP^1$; so
\[ \prod\ell^*(v;-a_0a_1,c;\fa^6_v)=1. \]

Now let $P_3,P_4,P_5$ be three further points of $V(k)$; then there are
$\infty^9$ curves on $V$ which are the intersection of $V$ with a quadric and
pass through $P_3,P_4,P_5$. For each $v$ in $\sB$, let $D'_v$ and $D''_v$ be
two such curves defined over $k_v$ each of which also passes through the five
points of $\fb^6_v$, and let $R'_v,R''_v$ be quadrics defined over $k_v$
which contain $D'_v,D''_v$ respectively but neither of which contains the whole
of $V$. Choose $R'$, a quadric defined over $k$, close to each $R'_v$ and
passing through $P_3,P_4,P_5$, and similarly for $R''$. The intersection
\[ R'\cap R''\cap V=\{P_3\}\cup\{P_4\}\cup\{P_5\}\cup\fb^7, \]
where $\fb^7_v$ has degree 13.
Much as before, $\fb^7=\fb^8_v\cup\fb^9_v$ over $k_v$ for each $v$ in
$\sB$, where each $\fb^9_v$ is close to $\fb^6_v$, so that $\fb^8_v$ has degree
8 and
\[ \prod\ell^*(v;-a_0a_1,c;\fa^8_v)=1. \]

We now have the necessary map of how to go back.
By Theorem \ref{T3}, we can find a
positive 0-cycle $\fd^8$ of degree 8 on $V$, defined over $k$ and
arbitrarily near to each $\fb^8_v$. With the same $P_3,P_4,P_5$ as before,
there is a pencil of curves on $V$ which are the intersections of $V$ with a
quadric and pass through $P_3,P_4,P_5$ and the points of $\fd^8$. Let
$\fd^5$, of degree 5,
be the residual intersection of the curves of this pencil; since the
pencil contains a curve close to each $D'_v$ and another close to each
$D''_v$, it follows that $\fd^5$ is close to each $\fb^5_v$.
(This time, the curves in the pencil do not all have a common component,
because one of them is arbitrarily close to $R'\cap V$ and another to
$R''\cap V$.)

In the same way, we successively generate a 0-cycle $\fd^3$ on $V$ of degree 3
and arbitrarily close to each $\fb^3_v$, and then a point of $V(k)$ arbitrarily
close to each $S_v$. This last is the point which we want.\qed


\medskip

The following lemma and theorem are due to Coray [8].
Lemma \ref{L8} is weaker than Theorem \ref{T5}, but appears
to be a necessary step in the proof of the latter. Theorem
\ref{T5} is one of the two ingredients in the approach to the
solubility of Del Pezzo surfaces of degree 4 which forms the
last part of this section.
\ble{L8} Let $V$ be a Del Pezzo surface of degree $4$, defined over a field
$L$ of characteristic $0$. If $V$ contains a positive $0$-cycle of degree
$2$ and a positive $0$-cycle of odd degree $n$, both defined over $L$, then
$V(L)$ is not empty.
\ele
\emph{Proof} We can suppose $V$ embedded in $\bP^4$ as the intersection of two
quadrics. We proceed by induction on $n$. If the given 0-cycle of degree 2
consists of the two points $P'$ and $P''$ then we can suppose that they are
conjugate over $L$ and distinct, because otherwise the lemma would be trivial.
By a standard result, there are infinitely many points on $V$ defined over
$L(P')$ and hence infinitely many positive 0-cycles of degree 2 defined over
$L$. Choose $d$ so that
\[ 2d(d+1)>n>2d(d-1) \]
and let $\{P'_i,P''_i\}$ be $\half\{2d(d+1)-n-1\}$ distinct pairs of points
of $V$, the points of each pair being conjugate over $L$. The hypersurfaces
of degree $d$ cut out on $V$ a system of curves of dimension $2d(d+1)$; hence
there is at least a pencil of such curves passing through the $P'_i$ and
$P''_i$ and the points of the given 0-cycle of degree $n$, and this pencil is
defined over $L$. We have accounted for $2d(d+1)-1$ of the $4d^2$ base points
of the pencil; so the remaining ones form a positive 0-cycle of degree
$2d(d-1)+1$ defined over $L$. This completes the induction step unless
$n=2d(d-1)+1$.

In this latter case we must have $d>1$ because if $d=1$ then $n=1$ and the
lemma is already proved; hence $2d(d+1)-n-1=4d-2\geq6$. Instead of the previous
construction we now choose our pencil of curves to have double points at $P'_0$ and $P''_0$ and to pass through $\half\{2d(d+1)-n-7\}$ other pairs $P'_i,P''_i$
as well as through the points of the given 0-cycle of degree $n$. In this case
each of $P'_0$ and $P''_0$ is a base point of the pencil with multiplicity $4$;
so we have accounted for $2d(d+1)+1$ of the base points of the pencil, and the
remaining ones form a positive 0-cycle of degree $2d(d-1)-1$ defined over $L$.
This completes the induction step in this case.  \qed
\bth{T5} Let $V$ be a del Pezzo surface of degree $4$, defined over a field $L$
of characteristic $0$. If $V$ contains a $0$-cycle of odd degree defined over
$L$ then $V(L)$ is not empty.
\eth
\emph{Proof} By decomposing the 0-cycle into its irreducible components, we can
assume that $V$ contains a positive 0-cycle $\fa$ of odd degree defined over
$L$. We can write $V$ as the intersection of two quadrics, each defined over
$L$; let $W$ be one of them. We can find a field $L_1\supset L$ with $[L_1:L]
\leq2$ and a point $P$ on $W$ defined over $L_1$. The lines on $W$ through $P$
are parametrised by the points of a conic, so we can find a field $L_2\supset
L_1$ with $[L_2:L_1]\leq2$ and a line $\ell$ on $W$, passing through $P$ and
defined over $L_2$. The intersection of this line with another
quadric containing $V$ cuts out on $V$ a positive 0-cycle of degree 2 defined
over $L_2$. Applying Lemma \ref{L8} to $\fa$ and this 0-cycle, we obtain a
point $P_2$ on $V$ defined over $L_2$. Repeating this argument for $\fa$ and
the positive 0-cycle of degree 2 consisting of $P_2$ and its conjugate over
$L_1$, we obtain a point $P_1$ on $V$ defined over $L_1$; and one further
repetition of the argument gives us a point on $V$ defined
over $L$.  \qed

\medskip

The main theorem of \S7.2 provides a promising approach to the
problem of finding the obstruction to the Hasse principle for
Del Pezzo surfaces of degree 4. One such obstruction is that
of Brauer-Manin, and the classical conjecture (due to
Colliot-Th\'{e}l\`{e}ne and Sansuc) is that it is the only
one. But the reader is warned that I have not yet been able to
push this argument through to a successful conclusion.

The starting point is the following question. Let $V$ be a
nonsingular Del Pezzo surface of degree 4, defined over an
algebraic number field $k$ and everywhere locally soluble;
can we exhibit a family of hyperplane sections of $V$ which is
of the form considered in \S7.2? It turns out that, after a
field extension of odd degree, we can exhibit such a family
parametrised by the points of $\bP^3$ blown up along a certain
curve and at four other points. The construction is as follows.

The surface $V$ is the base locus of a pencil of quadrics;
because $V$ is nonsingular, the pencil contains exactly 5
cones defined over $\bar{k}$ and these are all distinct.
Hence one at least of them is defined over a field $k_1$
which is of odd degree over $k$; and by Theorem \ref{T5} it is
enough to ask whether $V$ contains points defined over $k_1$.
Henceforth we work over $k_1$. After a change of variables, we
can assume that the singular quadric just described has vertex
$(1,0,0,0,0)$ and therefore an equation of the form $f(X_1,
X_2,X_3,X_4)=0$. By absorbing multiples of the other $X_i$
into $X_0$, we can now assume that $V$ has the form
\beq{E68} f(X_1,X_2,X_3,X_4)=0, \quad
aX_0^2+g(X_1,X_2,X_3,X_4)=0 \end{equation}
with $a\neq0$.

Now let $P$ be any point on $X_0=0$, let $Q$ be the quadric
of the pencil (\ref{E68}) which passes through $P$, and let
$\Pi$ be the tangent hyperplane to $Q$ at $P$. I claim that
the curve of genus 1 in which $\Pi$ meets $V$ is of the type
considered in \S7.2. For this it is enough to show for general
$P$ that its equation can be put in the form (\ref{E63}). But
provided that $P$ does not lie on $f=0$, by a further change
of variables we can take $P$ to be $(0,1,0,0,0)$ and require
\[ f(X_1,X_2,X_3,X_4)=bX_1^2+f_1(X_2,X_3,X_4). \]
The equation of $Q$ has no term in $X_1^2$, so by a further
change of variables we can take it to have the form
\beq{E69} aX_0^2+cX_1X_4+h(X_2,X_3,X_4)=0 \end{equation}
with $c\neq0$; this is equivalent to requiring the equation of
$\Pi$ to be $X_4=0$. Since $V$ is given by $f=0$ and
(\ref{E69}), its intersection with $X_4=0$ has the required
form.

This construction breaks down if $P$ lies on $V$ or is the
vertex of one of the other singular quadrics of the pencil,
because then $\Pi$ is no longer well-defined. To remedy this,
what we do is to choose a point $P$ on $X_0=0$ together with
a hyperplane $\Pi$ which touches at $P$ some quadric of the
pencil (\ref{E68}). Thus $P$ should be considered as a point
of the variety $W$ obtained by blowing up $X_0=0$ (which can
be identified with $\bP^3$) along the curve $V\cap\{X_0=0\}$
and at the vertices of the other four singular quadrics of
the pencil.

Denote by $U$ the variety over $W$ whose fibres are the curves
$V\cap\Pi$ in the construction above; then what we have
obtained is a diagram
\[ W \longleftarrow U \longrightarrow V \]
in which the left hand map is a fibration. The right hand map
here is not a fibration, and it seems unlikely that there is
even a subvariety of $U$ on which the restriction of the map
is a fibration. But this is not important. What matters is
the existence of a section --- that is, a map $V\rightarrow
U$ such that the composite map $V\rightarrow U\rightarrow V$
is the identity; and for this we only need the map
$V\rightarrow U$ to be rational rather than everywhere
defined. In the notation of (\ref{E68}) let $P_0=(x_0,\ldots,
x_4)$ be a point of $V$ with $x_0\neq0$, and choose
$P=(0,x_1,x_2,x_3,x_4)$. The equation of $\Pi$ has no term in
$X_0$; hence since $P$ lies on $\Pi$ so does $P_0$. This
defines the rational map $V\rightarrow U$. Provided $V$ is
everywhere locally soluble, so is $U$. If we can find a field
extension $k_2/k_1$ of odd degree such that $U$ is soluble in
$k_2$, then $V$ will also be soluble in $k_2$ and two
applications of Theorem \ref{T5} will show that $V$ is
soluble in $k$.

We cannot apply the last sentence of Theorem \ref{T10} as it
stands, because $W$ is too big; but it is simple enough to
find a line $L$ defined over $k_1$ in the $\bP^3$ which
underlies $W$ such that
\begin{itemize}
\item $L$ is in sufficiently general position, and
\item the inverse image of $L$ in $U$ is everywhere locally
soluble.
\end{itemize}
To do this, we choose any $P_1$ on $X_0=0$ and defined over
$k_1$. The fibre above $P_1$ is locally soluble except at a
finite set $\sS$ of places. For each of these places there is
a point of $U$ in the corresponding local field, and this maps
down to a point of $\bP^3$. Using weak approximation on
$\bP^3$ we can therefore find a point $P_2$ in $\bP^3$ such
that the fibre above $P_2$ is locally soluble at each place in
$\sS$. We can now take $L$ to be the line $P_1P_2$ and apply
Theorem \ref{T10} to the inverse image of $L$ in $U$.

To obtain a satisfactory theorem for $V$, we have to translate
Conditions 1 to 4 of \S7.2 into conditions on $V$. A tedious
calculation, which can be found in [1], shows that
Conditions 1, 3 and 4 are satisfied provided $L$ is in
sufficiently general position. The difficulty is with
Condition 2, or more precisely with the continuous conditions
in the sense of \S3 which are generated by Condition 2. It
ought to be true that these come from the continuous
conditions on the two pencils of conics each of which is given
by one of the two equations (\ref{E63}) --- and which are
known to be Brauer-Manin. It ought also to be true that they
correspond to the Brauer-Manin conditions on $V$. But as yet
I have been unable to prove either of these assertions.


\newpage

\noindent 9. \emph{Diagonal quartic surfaces}. \newline
We now apply the ideas of \S6 to K3 surfaces defined over $\bQ$ whose
equation has the form
\beq{E36} a_0X_0^4+a_1X_1^4+a_2X_2^4+a_3X_3^4=0. \end{equation}
We shall always assume that (\ref{E36}) is everywhere locally
soluble and the $a_i$ are integral. The surfaces (\ref{E36})
are very special within the
family of nonsingular quartic surfaces for at least two
reasons: they are Kummer surfaces, and their N\'{e}ron-Severi
groups over $\bC$ have maximal rank, which is 20. But this is
probably the simplest family of K3 surfaces that can be
written down explicitly.

We can take $\sB$, the set of bad places for (\ref{E36}), to
consist of $\infty,2$ and the odd primes which divide
$a_0a_1a_2a_3$.
It is known that the N\'{e}ron-Severi group of (\ref{E36}) over $\bC$ is
generated by the 48 lines on the surface. However, what is equally important
for our purposes is the N\'{e}ron-Severi group over $\bQ$. There are now 282
possibilities for the Galois group over $\bQ$ of the least field of definition
of the 48 lines; these have been tabulated by Martin Bright in his Cambridge
Ph.D. thesis, which can be found at
\newline \begin{center} http://www.boojum.org.uk/maths/quartic-surfaces/
\end{center} 
together with a good deal of other relevent material. We shall be interested
in the special case when
\beq{E44} a_0a_1a_2a_3 {\mathrm{~is~a~square}}, \end{equation}
because then the surface contains a pencil of curves of genus 1 of the kind
considered in \S6. There are some other special cases in which the surface
(\ref{E36}) contains such a pencil; but this is not true in
general and it seems unlikely that one can apply the methods
expounded in these notes to the general surface (\ref{E36}).

There is an obvious map from (\ref{E36}) to the quadric surface
\beq{E70} a_0Y_0^2+a_1Y_1^2+a_2Y_2^2+a_3Y_3^2=0. \end{equation}
We have assumed that (\ref{E36}), and therefore (\ref{E70}),
is everywhere locally soluble; so (\ref{E70}) is soluble in
$\bQ$.
The reason why the case (\ref{E44}) is more tractable than
the general case is that if (\ref{E44}) holds then each of
the two pencils of lines on (\ref{E70}) is defined over $\bQ$,
and a general line of either pencil pulls back to a curve of
genus 1 on (\ref{E36}) which is a 2-covering of its Jacobian.
It turns out that these curves are of the kind considered in
\S\S5 and 6. More generally,
consider a quadric of the form
\beq{E71} A(Y)D(Y)=B(Y)C(Y) \end{equation}
where $A(Y)=\ga_0Y_0+\ga_1Y_1+\ga_2Y_2+\ga_3Y_3$ and so on.
This quadric is the image of the K3 surface
\beq{E75} A(X^2)D(X^2)=B(X^2)C(X^2) \end{equation}
in an obvious notation, and the pull-backs of the two pencils
of lines on (\ref{E71}) can be written
\beq{E72} yA(X^2)=zB(X^2), \quad yC(X^2)=zD(X^2) \end{equation}
and
\beq{E73} yA(X^2)=zC(X^2), \quad yB(X^2)=zD(X^2). \end{equation}
For the time being, we work with (\ref{E72}). Eliminating each
of the four
variables $X_\nu$ in turn, we obtain four equations of the form
\beq{E74} d_{i\ell}X_i^2+d_{j\ell}X_j^2+d_{k\ell}X_k^2=0, \end{equation}
only two of which are linearly independent. Here $i,j,k,\ell$
is any permutation of $1,2,3,4$ and $d_{\mu\nu}$ is the value
of the determinant
formed by columns $\mu$ and $\nu$ of the matrix
\[ \begin{pmatrix}
\ga_0y-\gb_0z & \ga_1y-\gb_1z & \ga_2y-\gb_2z & \ga_3y-\gb_3z \\
\gg_0y-\gd_0z & \gg_1y-\gd_1z & \gg_2y-\gd_2z & \gg_3y-\gd_3z
\end{pmatrix}. \]
We note the identity
\[ d_{01}d_{23}+d_{02}d_{31}+d_{03}d_{12}=0, \]
which is frequently useful. The Jacobian of the curve
(\ref{E72}) has the form
\[ E:Y^2=(X-c_1)(X-c_2)(X-c_3) \]
where
\[ c_1-c_2=d_{03}d_{21}, \quad c_2-c_3=d_{01}d_{32},
\quad c_3-c_1=d_{02}d_{13}, \]
and the map from the curve (\ref{E72}) to its Jacobian is
given by
\[ Y=d_{12}d_{23}d_{31}X_1X_2X_3/X_0^3, \quad X-c_i=d_{ij}d_{ki}X_i^2/X_0^2 \]
where $i,j,k$ is any permutation of $1,2,3$. Although
everything so far is homogeneous in $y,z$, we have to work in
$\bQ(y,z)$ rather than $\bQ(y/z)$, for reasons which are
already implicit in \S3.

Up to this point, the formulae hold for any nonsingular
quartic surface which can be written in the form (\ref{E75}).
For the diagonal quartic surface (\ref{E36}) we have the
unexpected result that each $d_{k\ell}$ is a constant multiple
of $d_{ij}$, where $i,j,k,\ell$ is any permutation of
$0,1,2,3$. For it follows from the solubility of (\ref{E70})
and the fact that $a_0a_1a_2a_3$ is a square that $-a_1$ is
represented by $a_2Y_2^2+a_3Y_3^2$ over $\bQ$. In other words,
there exist integers $r_1,r_2,r_3$ and $h$ such that
\[ a_1r_1^2+a_2r_2^2+a_3r_3^2=0, \quad h^2=a_0a_1a_2a_3. \]
After rescaling the equation (\ref{E36}) if necessary, we
can take
\begin{align*}
A(X^2)= & h r_2X_0^2+a_1a_3(r_3X_1^2-r_1X_3^2), \\
B(X^2)= & h r_3X_0^2-a_1a_2(r_2X_1^2+r_1X_2^2), \\
C(X^2)= & a_3hr_3X_0^2-a_1a_2a_3(r_2X_1^2-r_1X_2^2), \\
D(X^2)= & -a_2hr_2X_0^2-a_1a_2a_3(r_3X_1^2+r_1X_3^2);
\end{align*}
and the $d_{ij}$ are given by
\begin{align*}
d_{23}=a_1^2a_2a_3r_1^2(a_3y^2+a_2z^2), & \quad
d_{01}=(h/a_2a_3)d_{23}, \\
d_{31}=a_1^2a_2a_3r_1(a_3r_2y^2-2a_3r_3yz-a_2r_2z^2), & \quad
d_{02}=(h/a_3a_1)d_{31}, \\
d_{12}=a_1^2a_2a_3r_1(a_3r_3y^2+2a_2r_2yz-a_2r_3z^2), & \quad
d_{03}=(h/a_1a_2)d_{12}.
\end{align*}
These choices do not preserve the symmetry, but that loss
appears to be unavoidable.
Changing the $r_i$ corresponds to a linear
transformation on $y,z$; changing the sign of $h$ gives the
pencil (\ref{E73}) instead of (\ref{E72}).

The 2-covering of $E$ given by the triple $(m_1,m_2,m_3)$
with $m_1m_2m_3=1$ is
\[ m_iZ_i^2=X-c_i \mathrm{~for~} i=1,2,3 \quad \mathrm{and}
\quad Y^2=Z_1Z_2Z_3. \]
As in \S6, values associated with the particular
2-covering given by (\ref{E72}) will be denoted by a superfix
0; the 2-covering itself is given by
\[ m_1^0=-d_{21}d_{31}, \quad m_2^0=-d_{12}d_{32}, \quad
m_3^0=-d_{13}d_{23}. \]
We shall also need to know the 2-coverings corresponding to
the 2-division points. That corresponding to
$(c_1,0)$, for example, is given by
\beq{E81} m_1=-a_0a_1, \quad m_2=d_{03}d_{21},
\quad m_3=d_{02}d_{31}, \end{equation}
which can alternatively be written
\[ m_1=-a_0a_1, \quad m_2=-h/a_1a_2, \quad
m_3=h/a_3a_1. \]

It follows from the expressions for the $d_{ij}$ that, up to
a squared factor, the
discriminant of $d_{ij}$ is equal to $-a_ia_j$; thus in
particular $d_{ij}$ has no repeated linear factor and it is
a product of two linear factors over $\bQ$ if and only if
$-a_ia_j$ is in $\bQ^{*2}$. If
$i,j,k$ is a cyclic permutation of $1,2,3$ then
\[ d_{0i}/d_{jk}=a_0a_i/h=h/a_ja_k. \]
Moreover the resultant of $d_{ij}$ and $d_{ik}$ is
$-4a_i^2a_ja_k$, so that $d_{ij}$ and $d_{ik}$ cannot have a
common root. The pencil (\ref{E72}) has six singular fibres,
given by the roots of $d_{01}d_{02}d_{03}=0$, and each
singular fibre consists of four lines which form a skew
quadrilateral. Thus each of the 48 lines on (\ref{E36}) is
part of a singular fibre of either (\ref{E72}) or (\ref{E73}).

Martin Bright's thesis contains a dictionary which gives the
N\'{e}ron-Severi group of (\ref{E36}) over any field $k$. This
group has rank at least 2 whenever (\ref{E44}) holds; subject
to (\ref{E44}), it has rank greater than 2 if and only if up
to
fourth powers there is a relation of the form $a_j=4a_i$ or
$a_j=-a_i$ or $a_ia_j=a_ka_\ell$.

In order to apply the results in \S6, we must know when
Condition D
holds, and we must evaluate the relevent Legendre-Jacobi
functions. This is where a splitting of cases becomes
necessary. In what follows, we confine ourselves to the cases
when none of the $-a_ia_j$ is in $\bQ^{*2}$, which is
equivalent to requiring that all the $d_{ij}$ are irreducible
over $\bQ$.
\ble{L21} Suppose that no $-a_ia_j$ is in $\bQ^{*2}$. Then for
any $m$ which does not satisfy Condition D, one of $m$ and
$mm^0$ can be chosen to be independent of $y$ and $z$.
Moreover the group of such $m$ has order exactly $8$ (and
consists of the inescapable part of the $2$-Selmer group) if
and only if $a_0a_1a_2a_3$ is not a fourth power and no
$a_ia_j$ is a square.
\ele
\emph{Proof} A boring calculation shows that the primitive
4-division points satisfy $(X-c_1)^2=-a_0d^2_{12}d^2_{13}/a_1$
and so on; so under our hypothesis none of them are rational.
Now suppose that the triple $m$ does not satisfy Condition D.
As was pointed out at the beginning of \S6, we can confine
ourselves to those triples $m$ for which the
value of $m_3$ lies in the group generated by
\[ -1,2,d_{23},d_{31} \mathrm{~and~the~odd~primes~in~} \sB; \]
and similarly for $m_1$ and $m_2$. In the notation of the
first part of \S6 each of the $p_{ij}$ has only a single
irreducible factor $f_{k\tau}$ in $\bQ[y,z]$ and $f_{k\tau}^2
\|p_{ij}$; so we can drop the subscript
$\tau$ which appears there. Let $\xi_3$ satisfy
$d_{03}(\xi_3,1)=0$; then
\[ \xi_3=(-a_2r_2\pm r_1\sqrt{-a_1a_2})/a_3r_3 \]
and therefore
\begin{align*}
d_{23}(\xi_3,1)= & -2a_1^2a_2^2r_1^3r_3^{-2}(a_1r_1\pm r_2\sqrt{-a_1a_2}), \\
d_{31}(\xi_3,1)= & -2a_1^3a_2r_1^3r_3^{-2}(a_2r_2\mp r_1\sqrt{-a_1a_2})=\mp d_{23}(\xi_3,1)\sqrt{-a_1/a_2}.
\end{align*}
Here $K_3=\bQ(\xi_3)=\bQ(\sqrt{-a_2/a_1})$ and
\[ L_3^0=K_3(\sqrt{d_{31}(\xi_3,1)d_{23}(\xi_3,1)},
\sqrt{-h/a_2a_3})=\bQ(\sqrt[4]{-a_1a_2},\sqrt{-h/a_2a_3}) \]
in the notation of \S6. Now suppose for example
that $m_3$ is divisible to
the first power by $d_{23}$ but not by $d_{31}$. Because
$L_3\supset K_3(\sqrt{m_3})$ and we have assumed $L_3\subset
L^0_3$, it follows that $\sqrt{m_3}$ is in $L^0_3$, which is
a biquadratic extension of $K_3$. Hence one of
\[ m_3,\;m_3\sqrt{-a_1/a_2},\;(-h/a_2a_3)m_3,\;(-h/a_2a_3)m_3\sqrt{-a_1/a_2} \]
is in $K_3^{*2}$. But the norm of $m_3$ for $K_3/\bQ$ is
$-a_1a_3$ times an element of $\bQ^{*2}$, so this would
require $-a_1a_3$ or $-a_2a_3$ to be in $\bQ^{*2}$, contrary
to hypothesis. A similar argument works if $m_3$ is divisible
to the first power by $d_{31}$ but not by $d_{23}$. It
follows that $m_3$ must contain both or neither of
$d_{23}$ and $d_{31}$ as factors. Applying a similar argument
to $m_1$ and $m_2$, and remembering that $m_1m_2m_3$ must be
a square, we find that either $m$ or $mm^0$ must be
independent of $y/z$. It is enough to consider the former
case; now as elements of $\bQ^*/\bQ^{*2}$,
\begin{align*}
 & m_1 \mathrm{~is~in~} \{1,-a_2a_3,ha_0a_3,-ha_0a_2\}, \\
 & m_2 \mathrm{~is~in~} \{1,-a_3a_1,ha_0a_1,-ha_0a_3\}, \\
 & m_3 \mathrm{~is~in~} \{1,-a_1a_2,ha_0a_2,-ha_0a_1\}.
\end{align*}
In general this allows us only four choices for the $m_i$ such
that $m_1m_2m_3$ is a square; these correspond to the origin
and the three 2-division points on $E$. Additional
possibilities happen only when at least one of
\[ h,\;ha_0a_1,\;ha_0a_2,\;ha_0a_3,\;a_1a_2,\;a_2a_3,\;a_3a_1 \]
is in $\pm\bQ^{*2}$. But if $ha_0a_1$ is in $\pm\bQ^{*2}$ for
example, then all we obtain is a new way of describing triples
$m$ which are already known to lie in the inescapable part
of the 2-Selmer group; so these cases can be ignored. The
others give the exceptions listed.

If for example $a_2a_3$ is a square then $(1,-a_1a_2,-a_1a_2)$
does not satisfy Condition D. Again, if $h$ is in $-\bQ^{*2}$
then $(a_1a_3,a_1a_2,a_2a_3)$ does not satisfy Condition D,
whereas if $h$ is in $\bQ^{*2}$ then $(a_1a_2,a_2a_3,a_3a_1)$
does not satisfy Condition D. In each of these cases, the
group of inescapable elements of the 2-Selmer group acquires
one extra generator which is the $m$ just listed. If both
$a_2a_3$ and one of $\pm h$ is a square, then we acquire two
extra generators in this way.  \qed

We can now state the main result of this section, which is
simply the specialization of Theorem \ref{T7} to our case, and
which therefore requires no further proof. If $\sN^2$ is as
at the beginning of \S3, we shall denote by $\sA$ the closure
of the set of points $\ga\times\gb$ in $\sN^2$ at which
(\ref{E72}) is locally soluble for $y=\ga,z=\gb$ at each
place of $\sB$
and all the Legendre-Jacobi conditions associated with
any pencil of conics (\ref{E74}) hold.
\bth{T9} Assume Schinzel's Hypothesis and Hypothesis \Sha. Let
$(\ref{E36})$ be everywhere locally soluble and such that
$a_0a_1a_2a_3$ is a square. Suppose also that no $-a_ia_j$ is
in $\bQ^{*2}$. If $\sA$ is not empty and
Condition D holds, then $(\ref{E36})$ contains rational points.
\eth

As was remarked at the end of \S6, we can here replace
Condition D by the weaker Condition E. This will hold unless
a relevent one of the $m$ listed at the end of the proof of
Lemma \ref{L21} corresponds to an everywhere locally soluble
2-covering. It turns out that solubility in $\bR$ is automatic.
Lemma \ref{L14} provides a simple test, which is not always
satisfied, for solubility at odd primes. But to test for
solubility in $\bQ_2$ is more tedious.

By the results of \S4, the solubility of the pencil of conics
(\ref{E74})
is equivalent to three Legendre-Jacobi conditions, of which a
typical one is
\beq{E77} L(\sB;-d_{i\ell}d_{j\ell},d_{k\ell})=1.
\end{equation}
There are twelve conditions of this kind, but they are not all
independent. Indeed in the notation of Lemma \ref{L16} the
continuous conditions, which form a subgroup there called
$\gL_0$, are all Brauer-Manin; and Bright's table shows that
in the most general case satisfying (\ref{E44}) there is only
one Brauer-Manin condition. This gives us advance assurance
that the algebra on which we now embark will be fruitful.
Since as an element of
$\bQ^*/\bQ^{*2}$ the discriminant of $d_{k\ell}$ is equal to
$-a_ka_\ell$,
\[ (-a_ka_\ell,d_{k\ell}(\ga,\gb))_p=1 \mathrm{~for~}
\ga\times\gb \mathrm{~in~} \sA \mathrm{~and~} p
\mathrm{~not~in~} \sB. \]
Since $d_{i\ell}d_{j\ell}/d_{ik}d_{jk}$ is also equal to
$-a_ka_\ell\bmod\bQ^{*2}$,
\[ L(\sB;-d_{i\ell}d_{j\ell},d_{k\ell};\ga,\gb)=
L(\sB;-d_{ik}d_{jk},d_{\ell k};\ga,\gb). \]
We shall denote either of these last two expressions by
$F_{ij}$. Again,
\[ L(d_{ij},d_{ki})L(d_{ki},d_{ij})=\prod_{p \mathrm{~not~in~}
\sB}(d_{ij},d_{ki})_p=\prod_{v\in\sB}(d_{ij},d_{ki})_v \]
from which it follows that
\[ F_{ij}F_{jk}F_{ki}=\prod_{v\in\sB}\{(d_{ij},d_{jk})_v
(d_{jk},d_{ki})_v(d_{ki},d_{ij})_v\}. \]
We know that (\ref{E74}) is locally soluble at each place $v$
in $\sB$. The local solubility condition for (\ref{E3}) is
(\ref{E78}); applying this to (\ref{E74}) we obtain
\[ (d_{i\ell},-d_{j\ell})_v(d_{j\ell},-d_{k\ell})_v
(d_{k\ell},-d_{i\ell})_v=(-1,-1)_v. \]
Taking the product of this equation over all $v$ in $\sB$ and
using the Hilbert product formula, we obtain
\begin{align*}
F_{ij}F_{jk}F_{ki} & =\prod_{v\in\sB}\{(d_{jk},-a_ja_k)_v
(d_{ki},-a_ka_i)_v(d_{ij},-a_ia_j)_v\} \\
 & =\prod_{p\mathrm{~not~in~}\sB}\{(d_{jk},-a_ja_k)_p
(d_{ki},-a_ka_i)_p(d_{ij},-a_ia_j)_p\}=1,
\end{align*}
because for example $-a_ja_k$ is up to a squared factor the
discriminant of $d_{jk}$ and is therefore a square mod $p$
for any prime $p$ outside $\sB$ which divides
$d_{jk}(\ga,\gb)$. One now deduces that
\[ F_{i\ell}F_{j\ell}F_{k\ell}=F_{i\ell}F_{jk}=F_{j\ell}
F_{ki}=F_{k\ell}F_{ij} \]
on $\sA$, whence all the $F_{i\ell}F_{j\ell}F_{k\ell}$ are
equal on $\sA$.

The explicit formulae which follow (\ref{E58}) show that,
in the notation of \S3, the value of $\gt$ associated with
$L(\sB;\pm d_{i\ell},d_{k\ell})$ is $-a_ia_k$; so the value
of $\gt$ associated with $F_{ij}$ is $a_ia_j$. It follows
from the calculations in the previous paragraph that in
general there is only one non-trivial continuous condition,
which can be written $F_{12}F_{23}F_{31}=1$. If however one
of the $a_ia_j$ is a square then the corresponding condition
$F_{ij}=1$ is also in $\gL_0$. The remarks at the end of the
proof of Lemma \ref{L21} show that Condition D cannot then
hold, but Condition E may still hold in some part of $\sA$.

The easiest way to evaluate the one condition which is
non-trivial and continuous even in the general case involves
dropping the symmetry; we have for example
\begin{align*}
F_{01} & F_{23}=L(-d_{03}d_{13},d_{23})L(-d_{02}d_{03},d_{01})=L(d_{02}d_{13},d_{23}) \\
 & =L(-ha_1a_3,d_{23})=\prod_{p \mathrm{~not~in~} \sB}(-ha_1a_3,d_{23})_p=\prod_{v\in \sB}(-ha_1a_3,d_{23}(\ga,\gb))_v.
\end{align*}
Of the surfaces (\ref{E36}) satisfying (\ref{E44}) and with
each $|a_i|<16$, there are just two
which are everywhere locally soluble but are not known to have
a solution in $\bQ$. They are
\[ 2X_0^4+9X_1^4=6X_2^4+12X_3^4 \quad \mathrm{and}
\quad 4X_0^4+9X_1^4=8X_2^4+8X_3^4. \]
It turns out that both of them are insoluble in $\bQ$: the
first fails the condition $F_{01}F_{23}=1$ and the
second has $a_0a_1$ square and fails the condition $F_{01}=1$.
We give the details for the first one. The calculations for
the second one are more tedious, since to evaluate $F_{01}$
one needs to use the formulae which follow (\ref{E58}).

For the first surface we have $\sB=\{2,3,\infty\}$, and the
surface can be written in the form
\begin{align*}
2(X_0^2-X_2^2 & -2X_3^2)(X_0^2+X_2^2+2X_3^2) \\
 & =-(3X_1^2-2X_2^2+2X_3^2)(3X_1^2+2X_2^2-2X_3^2);
\end{align*}
thus the pencil (\ref{E72}) can be taken to be 
\begin{align*}
 & 2y(X_0^2-X_2^2-2X_3^2)+z(3X_1^2-2X_2^2+2X_3^2)=0, \\
 & y(3X_1^2+2X_2^2-2X_3^2)-z(X_0^2+X_2^2+2X_3^2)=0,
\end{align*}
and the $d_{ij}$ are given by
\[ \begin{array}{ll}
d_{01}=3(2y^2+z^2), & d_{23}=6(2y^2+z^2), \\
d_{02}=2(2y^2-2yz-z^2), & d_{31}=-6(2y^2-2yz-z^2), \\
d_{03}=-2(2y^2+4yz-z^2), & d_{12}=3(2y^2+4yz-z^2).
\end{array} \]
It follows that $h=36$; recall that it is only $h^2$ that was
determined earlier, and the choice of sign was
equivalent to the choice between the pencils (\ref{E72}) and
(\ref{E73}).

We do not need information about $\bR$, but in fact
we have $c_2>c_1>c_3$, so the curve (\ref{E72}) is
soluble in $\bR$ if and only if $m_2^0>0$. All primitive
solutions must have $X_0,X_2,X_3$ odd and $2\|X_1$, whence
$y+z\equiv0\bmod4$; a full analysis shows that this condition
is sufficient for solubility in $\bQ_2$ as well as necessary,
but we do not need this. The analysis of solubility in $\bQ_3$
is more tedious. We must have $3|X_0$ and $X_2,X_3$ prime to
3. The three triples like (\ref{E81}) lie in
$W_3$ and therefore generate it; so $m_1^0$ must be in
$\bQ^{*2}_3$ or $3\bQ^{*2}_3$ and $m_3^0$ must be in
$\bQ^{*2}_3$ or $6\bQ^{*2}_3$. But
\[ m_1^0=-d_{21}d_{31}=-18\{3y^2-(y+z)^2\}\{2(y+z)^2-3z^2\}, \]
so $3{\not|}(y+z)$. Thus $2y^2-2yz-z^2$ is in $2\bQ_3^{*2}$,
whence consideration of $m_3^0$ shows that
\[ 2y^2+z^2 \mathrm{~is~in~} \bQ^{*2}_3 \mathrm{~or~}
6\bQ^{*2}_3. \]
It now follows easily that
$F_{01}F_{23}=-1$ throughout $\sA$.

\medskip

Subject to our two major hypotheses, Theorem \ref{T9} asserts
that in general the only obstructions to the Hasse principle
for surfaces (\ref{E36}) subject to (\ref{E44}) are the
continuous Legendre-Jacobi obstructions or equivalently the
Brauer-Manin obstructions. `In general' here means that no
$\pm a_ia_j$ is a square and $a_0a_1a_2a_3$ is not a fourth
power. I have not attempted to investigate the exceptional
cases, but for them there are additional Brauer-Manin
obstructions and the assertion above may well remain true.
However, without (\ref{E44}) there is strong numerical
evidence that the situation is quite different. First, the
special surface
\[ X_0^4+2X_1^4=X_2^4+4X_3^4 \]
appears to contain no rational points other than the two
obvious ones. This surface has non-trivial Brauer group, but
Brauer-Manin obstructions seem to be incapable of showing
that a non-singular surface contains a finite non-zero
number of rational points. Second, Bright has investigated
surfaces of the form
\[ X_0^4+cX_1^4=4X_2^4+2cd^2X_3^4 \]
where $c$ is not a square. This is one of the simplest
families which have no Brauer-Manin obstruction arising from
the arithmetic part of the Brauer group --- that is, the part
of the Brauer group which is killed by replacing the ground
field by its algebraic closure. Surfaces of this form have
N\`{e}ron-Severi group over $\bQ$ of rank 2, but they do not
admit pencils of curves of genus 1. There are many surfaces of
this form which are everywhere locally soluble but do not
appear to have any rational solutions. Presumably therefore,
there are further obstructions to the Hasse principle as yet
undiscovered.








\newpage

\begin{center} REFERENCES \end{center}

\noindent [1] A.O.Bender and Sir Peter Swinnerton-Dyer, Solubility of certain
pencils of curves of genus 1, and of the intersection of two quadrics in
$\bP^4$, Proc. London Math.Soc. (3)83(2001), 299-329.

\noindent [2] J.W.S.Cassels, Second descent for elliptic curves, J. reine
angew. Math. 494(1998), 101-127.

\noindent [3] J-L.Colliot-Th\'{e}l\`{e}ne, Hasse principle for pencils of
curves of genus one whose Jacobians have a rational 2-division point, (close
variation on a paper of Bender and Swinnerton-Dyer), Progr. Math. 199(2001),
117-161. 

\noindent [4] J-L.Colliot-Th\'{e}l\`{e}ne, A.N.Skorobogatov and Sir Peter
Swinnerton-Dyer, Double fibres and double covers: paucity of rational points,
Acta Arith. 79(1997), 113-135.

\noindent [5] J-L.Colliot-Th\'{e}l\`{e}ne, A.N.Skorobogatov and Sir Peter
Swinnerton-Dyer, Rational points and zero-cycles on fibred varieties:
Schinzel's Hypothesis and Salberger's device, J. reine angew. Math. 495(1998),
1-28.

\noindent [6] J-L.Colliot-Th\'{e}l\`{e}ne, A.N.Skorobogatov and Sir Peter
Swinnerton-Dyer, Hasse principle for pencils of curves of genus one whose
Jacobians have rational 2-division points, Invent. Math. 134(1998), 579-650.

\noindent [7] J-L.Colliot-Th\'{e}l\`{e}ne and Sir Peter Swinnerton-Dyer,
Hasse principle and weak approximation for pencils of Severi-Brauer and
similar varieties, J. Reine Angew. Math. 453(1994), 49-112.

\noindent [8] D.Coray, Points alg\'{e}briques sur les surfaces de Del Pezzo,
C. R. Acad. Sci. Paris 284(1977), 1531-4.

\noindent [9] J.S.Milne, Arithmetic Duality Theorems (Boston, 1986).

\noindent [10] P.Salberger, Zero-cycles on rational surfaces over number
fields, Invent. Math. 91(1988), 505-524.

\noindent [11] P.Salberger and A.N.Skorobogatov, Weak approximation for
surfaces defined by two quadratic forms, Duke J. Math. 63(1991), 517-536.

\noindent [12] J-J.Sansuc, Descente et principe de Hasse pour certains
vari\'{e}t\'{e}s rationnelles, in \textit{S\'{e}minaire de Th\'{e}orie des
Nombres, Paris $1980-81$} (ed. M-J.Bertin), 253-272 (Progr. Math. 22).

\noindent [13] Sir Peter Swinnerton-Dyer, Rational points on pencils of conics
and on pencils of quadrics, J. London Math. Soc. (2)50(1994), 231-242.

\noindent [14] Sir Peter Swinnerton-Dyer, Some applications of Schinzel's
hypothesis to diophantine equations, in Number theory in progress (ed.
K.Gy\"{o}ry, H.Iwaniec and J.Urbanowicz), 503-530 (Berlin, 1999).

\noindent [15] Sir Peter Swinnerton-Dyer, Arithmetic of diagonal quartic
surfaces, II, Proc. London Math. Soc. (3)80(2000), 513-544.

\noindent [16] Sir Peter Swinnerton-Dyer, The solubility of diagonal cubic
surfaces, Ann. Scient. \'{E}c. Norm. Sup. (4)34(2001), 891-912.





\end{document}


