\documentclass[12pt,leqno]{article}
\usepackage{amsthm,amsbsy,amsfonts,amssymb,amsmath}
\usepackage{latexsym}
\usepackage{fontenc}
%\usepackage[mathscr]{euscript}

\setlength{\textwidth}{6.5in}
\setlength{\textheight}{8.5in}
\setlength{\topmargin}{0pt}
\setlength{\oddsidemargin}{0pt}
\setlength{\evensidemargin}{0pt}
\setlength{\headheight}{0pt}
\setlength{\headsep}{0pt}


\newcommand{\dbC}{{\mathbb{C}}} %Blackboard Bold
\newcommand{\dbG}{{\mathbb{G}}}
\newcommand{\dbP}{{\mathbb{P}}}
\newcommand{\dbQ}{{\mathbb{Q}}}
\newcommand{\dbR}{{\mathbb{R}}}

%\newcommand{\grA}{{\mathfrak{A}}}  %Gothic or German
%\newcommand{\grB}{{\mathfrak{B}}}
%\newcommand{\grC}{{\mathfrak{C}}} 
%\newcommand{\grD}{{\mathfrak{D}}}
%\newcommand{\grG}{{\mathfrak{G}}} 

%\newcommand{\ScrA}{{\mathscr{A}}} %Script 
%\newcommand{\ScrB}{{\mathscr{B}}}
%\newcommand{\ScrC}{{\mathscr{C}}} 
%\newcommand{\ScrL}{{\mathscr{L}}} 
%\newcommand{\ScrS}{{\mathscr{S}}}

%\renewcommand{\baselinestretch}{1.5}


%\newcommand{\Le}{{{\mathchoice{\,{\scriptstyle\le}\,}
%  {\,{\scriptstyle\le}\,}
%  {\,{\scriptscriptstyle\le}\,}{\,{\scriptscriptstyle\le}\,}}}}
\newcommand{\Ge}{{{\mathchoice{\,{\scriptstyle\ge}\,}
  {\,{\scriptstyle\ge}\,}
  {\,{\scriptscriptstyle\ge}\,}{\,{\scriptscriptstyle\ge}\,}}}}

\font\titlefont=cmss10 scaled\magstep2
\font\sectionfont=cmss10 scaled\magstep1

\newcommand{\dspace}{\lineskip=2pt
     \baselineskip=18pt\lineskiplimit=0pt}
%\newcommand{\w}{{\mathchoice{\,{\scriptstyle\wedge}\,}
%  {{\scriptstyle\wedge}}
%  {{\scriptscriptstyle\wedge}}{{\scriptscriptstyle\wedge}}}}
\newcommand{\SubSet}{\raise2pt\hbox{$\,\scriptstyle
     \subset\,$}}
\newcommand{\otimesop}{\operatornamewithlimits{\otimes}\limits}

\title{\titlefont Periods for the Fundamental Group\\
Pierre Deligne\\
Arizona Winter School 2002}
\author{}
\date{}

\overfullrule=5pt
\begin{document}

\maketitle


\bigskip
\dspace
\noindent
{\sectionfont Course description}

Let $X$ be a non singular complex algebraic
variety.
By Grothendieck, its complex
cohomology can be described as (a) singular
cohomology, with complex coefficients;
(b) hypercohomology of the algebraic de Rham
complex $\Omega_X^*$; for $X$ affine: the
cohomology of the complex of algebraic
differential forms on $X$.
The description (a) gives a rational
structure: use rational coefficients.
If $X$ is defined over $k\SubSet\dbC$,
the description (b) gives a $k$-structure: use
forms defined over $k$.
The period matrix is the change of basis
matrix, from a rational basis for the
$\dbQ$-structure (a), to a $k$-basis for the
$k$-structgure (b).
Basic example: $k=\dbQ$, $X$ the
multiplicative group $\dbG_m$.
Here, singular $H_1$ (dual to $H^1$) is
generated by a loop around $0$, while de Rham
$H^1$ is generated by $\frac{dz}{z}$.
The period matrix is one by one; it is $2\pi i$.

Fix a base point $o$.
Algebraic geometry has few tools to understand
$\pi_1(X,o)$ itself.
If we make $\pi_1$ abelian, we obtain $H_1$,
which has a de Rham description, periods,
$\ldots\,\,$.
The story is almost as good for the group
algebra $\dbQ[\pi_1(X,o)]$, divided by a power
of the augmentation ideal $I$, for instance
because this quotient has a description as
some relative homology group in $X^N$.
While periods in cohomology have mainly been
considered for projective $X$,
$\dbQ[\pi_1]/I^{N+1}$ is interesting for $X$
as simple as $\dbP^1-\{0,1,\infty\}$.
For any $X$, $I/I^2$ is $H_1$, hence
$I^N/I^{N+1}$ is a quotient of $\otimesop^{N}H$.
For $X=\dbP^1-\{0,1,\infty\}$, the interest
lies in the extensions.

The course will explain how for
$X=\dbP^1-\{0,1,\infty\}$, the periods of
$\dbQ[\pi_1]/I^{N+1}$ are encoded in the
multi-zeta values: the values of
$$
\zeta (s_1,\dotsc,s_r)=
\sum\limits_{n_1>\ldots>n_r>0}
\frac{1}{n_1^{s_1}\ldots n_r^{s_r}}
$$
for $s_i$ integers $\Ge 1$.
This for a suitable (tangential) base point.

If we consider $X$ defined over $\dbQ$, with
``good reduction'' $\mod\,p$, the de Rham
analog of $\dbQ[\pi_1]/I^{N+1}$, tensored with
$\dbQ_p$, depends only on the reduction
$\mod\,p$ of $X$, and is acted upon by its
Frobenius endomorphism.
For $X=\dbP^1-\{0,1,\infty\}$, this de Rham
analog is the quotient of the algebra of non
commutative formal power series $\dbQ\ll
e_0,e_1\gg$ by the part of degree $\Ge N+1$.
For the same base point as previously, the
Frobenius action on $\dbQ_p\ll e_0,e_1\gg$
is of the form
\begin{align*}
e_1 &\to p\,e_0\\
e_1 &\to g^{-1}.p\,e_1.g,
\end{align*}
where in $g$ the coefficient of $1$ is $1$ and
that of $e_1^n$ ($n>0$) is $0$.

Define $\zeta^{(p)}(s_1,\dotsc,s_r)$ to be the
coefficient in $g$ of
$e_0^{s_1-1}e_1e_0^{s_2-1}e_1\ldots
e_0^{s_r-1}e_1$.
We will explain why the
$\zeta^{(p)}(s_1,\dotsc,s_r)\in\dbQ_p$ should
satisfy the same polynomial identities (with
rational coefficients) as the
$\zeta(s_1,\dotsc,s_r)\in\dbR$, plus the
analog of ``$2\pi i=0$'' (vanishing of the
$\zeta^{(p)}(2n)$).

\end{document}

