\documentclass[12pt,titlepage]{article}
\usepackage{epsfig}
\usepackage{amsmath}
\usepackage{amssymb}
\usepackage{amscd}

\def\noi{\noindent}
\def\O{\Omega}


\textwidth=5.5in
\hoffset=-.5in
\textheight=7in
\newenvironment{proof}{\medskip {\it Proof.\ }}{\ \rule[-0.5mm]{1mm}
          {3.5mm}\medskip\noindent}

\def\dbar{\overline{\partial}}

\def\noi{\noindent}
\def\O{\Omega}
\def\deg{\mbox{\rm deg}\,}
\def\Res{\mbox{\rm Res}\,}
\def\ind{\mbox{\rm ind}\,}
\def\rank{\mbox{\rm rank}\,}
\def\Tr{\mbox{\rm Tr}\,}
\def\tr{\mbox{\rm tr}\,}
\def\gcd{\mbox{\rm gcd}\,}
\def\we{\wedge}
\def\et{e^{it}}
\def\ett{e^{i\theta}}
\def\G{\Gamma}
\def\ii{\sqrt{-1}}
\def\l{\lambda}
\def\a{\alpha}

\def\qed{\hbox{\hskip.2in\vrule width.4pt height6.65pt depth.15pt\vrule
width2.5pt height6.65pt depth-6.25pt\hskip-2.5pt\vrule width2.5pt
height.25pt depth.15pt\vrule width.4pt height6.65pt depth.15pt}}



\newcommand{\bea}{\begin{eqnarray*}}
\newcommand{\eea}{\end{eqnarray*}}

\newcommand{\holo}[1]{\ensuremath{{\cal O}(#1)}}
\newcommand{\sholo}[1]{\ensuremath{{\cal H}(#1)}}
\newcommand{\Sholo}[2]{\ensuremath{{\cal O}_{#2}(#1)}}
\newcommand{\hhull}[2]{\ensuremath{\hat{#1}_{\holo{#2}}}}
\newcommand{\germ}[2]{\ensuremath{\mathbf{#1}_{#2}}}
\newcommand{\pr}{\ensuremath{\pi}}
\newcommand{\envholo}[2]{\ensuremath{\widetilde{#1}_{#2}}}
\newcommand{\define}[1]{\textbf{#1}}
\newcommand{\id}{\ensuremath{Id}}
%\newcommand{\re}{\ensuremath{\Re}}
%\newcommand{\im}{\ensuremath{\Im}}
\newcommand{\tangent}[2]{\ensuremath{T({#1})_{{#2}}}}
\newcommand{\Ctangent}[2]{\ensuremath{T_{\CC}({#1})_{{#2}}}}
\newcommand{\Pl}{{\mathbb P}}

\newtheorem{lem}{LEMMA}[section]
\newtheorem{theo}[lem]{Theorem}
\newtheorem{cla}[lem]{Claim}
\newtheorem{coro}[lem]{Corollary}
\newtheorem{prop}[lem]{Proposition}
\newtheorem{problem}[lem]{Problem}
\newtheorem{proposition}[lem]{Proposition}
\newtheorem{definition}[lem]{Definition}
\newtheorem{remark}[lem]{Remark}
\def\CQFD{\hfill \vrule width 7pt height 7pt depth 1pt}
\newtheorem{question}[lem]{Question}
\newtheorem{example}[lem]{Example}
\newtheorem{conjecture}[lem]{Conjecture}
\newtheorem{obs}[lem]{Observation}
\newtheorem{summary}[lem]{Summary}


\begin{document}

\section{Introduction}
The aim of this work is to investigate when the variation of Hodge
structures associated to a family of Calabi-Yaus is decomposable.
We show that assuming the conjecture "Hodge implies Absolutely
Hodge", the Hodge structure being decomposable can be translated
into a condition in Galois theory.  Finally we look at some
numerical data to try and ascertain whether there is sufficient
evidence for a conjecture that the variation of Hodge structures
is "rarely" decomposable.

\subsection{Hodge Theory and Galois Theory}

\begin{definition}
    The \emph{total cohomology} of a variety $X$ over an algebraically closed field $k$ is
   $H^{i}_{tot}(X) = H^{i}_{DR}(X) \times \prod_{l} H^{i}_{\Acute{e}t}(X,\mathbb{Q}_{l})$ where $H^{i}_{DR}$ is the \\ de Rham
    cohomology and $H^{i}_{\Acute{e}t}$ is the etale cohomology.
\end{definition}

Let $\sigma : k \hookrightarrow \mathbb{C}$ and set $\sigma X = X \times_{k} $Spec$ \mathbb{C}$.  Recall that we have comparision isomorphisms:
\[H^{i}_{B}(\sigma X) \otimes_{\mathbb{Q}} k \overset{\cong}{\rightarrow} H^{i}_{DR}(X)\]
\[H^{i}_{B}(\sigma X) \otimes_{\mathbb{Q}} \mathbb{Q}_{l} \overset{\cong}{\rightarrow} H^{i}_{\Acute{e}t}(X,\mathbb{Q}_{l})\]
where $H^{i}_{B}$ is the Betti cohomology.

\begin{definition}
  Let $\underline{t} = (t_{DR} , (t_{l})_{l}) \in H^{2i}_{tot}(X)$.  Then $t$ is a \emph{Hodge cycle} with respect to
   $\sigma : k \hookrightarrow \mathbb{C}$ if there exists $t_{B,\sigma} \in H^{2i}_{B}(X)$ such that:\\
     \indent 1. (rationality) $t_{DR} = $comp$_{B,DR}(t_{B,\sigma} \otimes 1)$ and $t_{l} =
        $comp$_{B,l}(t_{B,\sigma}\otimes 1)$\\
     \indent 2. (Hodge condition) $(t_{B,\sigma}) \otimes 1 \in H^{i,i}(\sigma X) \subset H^{2i}_{B}(\sigma X) \otimes
        \mathbb{C}$
\end{definition}

\begin{definition}
   We call a Hodge cycle an \emph{absolute Hodge cycle} if it is a Hodge cycle relative to every embedding
   $\sigma : k \hookrightarrow \mathbb{C}$.
\end{definition}

We will assume the following:

\begin{conjecture}\label{AH}
   Every Hodge cycle is an absolute Hodge cycle.
\end{conjecture}

Let $k_0$ be a field (not necessarily algebraically closed) contained in $k$.  Then we have the following commutative diagram:

\[
\begin{CD}
  X_{\mathbb{C}}            @>>> X              @>>> X_0\\
    @VVV                      @VVV                @VVV\\
  \text{Spec}(\mathbb{C})   @>>>\text{Spec}(k)  @>>>\text{Spec}(k_0)\\
\end{CD}
\]

There is a natural action of $\text{Gal}(k/k_0)$ on
$H^{i}_{\Acute{e}t}(X,\mathbb{Q}_{l})$ and the comparision isomorphisms
can be used to get an action on $V=H^{i}_{B}(X_{\mathbb{C}})$.

\begin{definition}
   One says that $W$ is a \emph{tensor construction} from $V$ if it satisfies the following two properties:\\
    \indent 1.  $W = V^{\otimes n_1} \otimes \Check{V}^{\otimes n_2} \otimes \mathbb{Q}(1)^{\otimes n_3}$\\
    \indent 2.  $W_{Hdg} = W \bigcap (W \otimes \mathbb{C})^{0,0}$
\end{definition}

Using Deligne's article (Art 1, LNM 900, Prop 2.9), the above conjecture implies that
the image $W_{Hdg} \hookrightarrow W\otimes \mathbb{Q}_{l}$ is stabilized by $\text{Gal}(k/k_0)$.

\begin{definition}
  The \emph{Mumford-Tate group MT($V$)} is the largest $\mathbb{Q}$-algebraic subgroup of
  $\text{GL}(V) \times \text{GL}(\mathbb{Q}(1))$ whose rational points fix all the elements of $W_{Hdg}$. $\text{MT }'(V)$
  is the projection onto $\text{GL}(V)$.
\end{definition}

When the associated Hodge structure is polarizable, it can be
shown (loc. cit.) that this implies that the Mumford Tate group is
reductive.  In particular this means that a $\mathbb{Q}$-linear
transformation is in MT($V$) iff it stabilizes all Hodge cycles;
i.e. this property uniquely characterizes MT($V$).

\begin{proposition}
  The action of $\text{Gal}(k/k_0)$ normalizes \\
  $\text{MT }'(V) \subset \text{GL}(V \otimes \mathbb{Q}_{l})$.
\end{proposition}

\noindent
\begin{proof}
  Let $\sigma \in \text{Gal}(k/k_0)$, $m \in \text{MT }'$, and $w \in W_{Hdg}$.  Then $\sigma w = \Acute{w}$ for some
  $\Acute{w} \in W_{Hdg}$.  We also observe that $m\Acute{w} = \Acute{w}$.  Then $(\sigma^{-1}m\sigma)w = w$
  By the
fact mentioned above (i.e. stabilizing all Hodge cycles implies
membership of MT($V$)) we have that $(\sigma^{-1}m\sigma) \in
\text{MT}'(V)$.
\end{proof}

\subsection{The Mumford-Tate Group of a certain 3-dimensional Hodge
structure}

In this section we consider a Hodge structure $V$ of weight 3 and
rank 4 such that $V \otimes \mathbb{C} = V^{0,3} \oplus V^{1,2}
\oplus V^{2,1} \oplus V^{3,0}$ with Hodge numbers all equal to 1.
We will assume $V = V_1 \oplus V_2$, $V_1 \otimes \mathbb{C} =
V^{0,3} \oplus V^{3,0}$
,and $V_2 \otimes \mathbb{C} = V^{1,2} \oplus V^{2,1}$.\\
Working over $\mathbb{C}$, we can pick a basis for $V_1 \otimes \mathbb{C}$ of the form
$\left( \begin{smallmatrix} 1 \\ \tau_1 \end{smallmatrix} \right),
\left( \begin{smallmatrix} 1 \\ \Bar{\tau_1} \end{smallmatrix} \right)$
and similarly
$\left( \begin{smallmatrix} 1 \\ \tau_2 \end{smallmatrix} \right),
\left( \begin{smallmatrix} 1 \\ \Bar{\tau_2} \end{smallmatrix} \right)$
for $V_2 \otimes \mathbb{C}$.
Using the polarization one gets that $Im(\tau_1)<0$ and $Im(\tau_2)>0$.\\
For a Hodge structure $W$, we define a map $\mu: \mathbb{G}_{m} \rightarrow \text{GL}(W)$ by
$\mu (\lambda)(w^{p,q}) = \lambda^{-p} w^{p,q}$ for $w^{p,q} \in W^{p,q}$.

\begin{prop}
   The Mumford-Tate group of a Hodge structure $W$ is the smallest algebraic subgroup $G$  of GL($V$)$\times \mathbb{G}_{m}$
   such that $\mu (\mathbb{G}_{m}) \subset G_{\mathbb{C}}$.
\end{prop}

\noindent
\begin{proof}
  See the article by Deligne referenced above.
\end{proof}

For $W=V_1$ with respect to the basis given above we have that $\mu_1(\lambda) =
\left( \begin{smallmatrix} 1&0\\ 0&\lambda^{-3} \end{smallmatrix} \right)$
,and on $V_2$ with respect to the given basis $\mu_2(\lambda)=
\left( \begin{smallmatrix} \lambda^{-1}&0 \\ 0&\lambda^{-2} \end{smallmatrix} \right).$\\
Now with respect to the standard basis $\left( \begin{smallmatrix} 0\\1 \end{smallmatrix} \right)
\left( \begin{smallmatrix} 1\\0 \end{smallmatrix} \right)$, we have

\[\mu_1(\lambda) = \frac{1}{\Bar{\tau} - \tau} \left(
     \begin{smallmatrix} \Bar{\tau}\lambda^{-3} - \tau & -\lambda^{-3} + 1 \\
                         \Bar{\tau}\lambda^{-3}\tau - \tau\Bar{\tau} & -\lambda^{-3}\tau + \Bar{\tau}
     \end{smallmatrix} \right) \]
\[\mu_2(\lambda) =  \frac{1}{\Bar{\tau} - \tau} \left(
     \begin{smallmatrix} \Bar{\tau}\lambda^{-2} - \tau\lambda^{-1} & -\lambda^{-2} + \lambda^{-1}\\
                         \Bar{\tau}\lambda^{-2}\tau - \lambda^{-1}\tau\Bar{\tau} & \lambda^{-2}\tau + \lambda^{-1}\Bar{\tau}
     \end{smallmatrix} \right) \]

\noindent
and that det$\mu_1(\lambda)=$det$\mu_2(\lambda) = \lambda^{-3}$.\\

Now, using the calculations for $\lambda$, we can remove most of the possibilities for MT$'(V_{i})$.
The only possibilities are $\text{MT}'(V_{i}) = \text{GL}_2$ or $T^2$.\\
Now we figure out what the possibilities for MT$'(V)$ are for
dimension 4.  One can show that $\text{MT}'(V) \subset
\text{MT}'(V_1) \oplus \text{MT}'(V_2)$.  Hence, MT$'(V) \subset
G$ where $G$ is one of the following:\\
  \indent a. GL$_2 \times$GL$_2$\\
  \indent b. GL$_2 \times T^2$\\
  \indent c. $T^2 \times$ GL$_2$\\
  \indent d. $T^2 \times T^2$

\begin{prop}
  $\text{MT }'(V)$ is one of the following groups:\\
    \indent a. $\{ \left( \begin{smallmatrix} g_1&0\\0&g_2 \end{smallmatrix} \right) \arrowvert g_{i} \in \text{GL}_2,
               \text{det}g_1 = \text{det}g_2 \}$\\
  \indent b. $\{ \left( \begin{smallmatrix} g_1&0\\0&g_2 \end{smallmatrix} \right) \arrowvert g_{1} \in \text{GL}_2,
               g_2 \in T^2, \text{det}g_1 = \text{det}g_2 \}^{o}$\\
  \indent c. $\{ \left( \begin{smallmatrix} g_1&0\\0&g_2 \end{smallmatrix} \right) \arrowvert g_{1} \in T^2,
               g_2 \in \text{GL}_2, \text{det}g_1 = \text{det}g_2 \}^{o}$\\
  \indent d. $\{ \left( \begin{smallmatrix} g_1&0\\0&g_2 \end{smallmatrix} \right) \arrowvert g_{1} \in T^2,
               g_2 \in T^2, \text{det}g_1 = \text{det}g_2 \}$\\
  \indent e. $\{ \left( \begin{smallmatrix} t_1&0&0&0\\0&t_2&0&0\\0&0&t_3&0\\0&0&0&t_4 \end{smallmatrix} \right)
                  \arrowvert t_1t_2=t_3t_4 $ and $ \frac{t_1}{t_2} = (\frac{t_3}{t_4})^3 \} = \text{MT }'(V)_\mathbb{C}$

\end{prop}

\begin{proof} First note that the equality of determinants is an
immediate consequence of our calculations following Proposition 1.8.  
In case a one can see that the Mumford-Tate group
is either the one indicated in the statement of the theorem or it's a
subgroup thereof with an additional defining condition $g_1=g_2$ or
$g_2=g_0g_1g_0^{-1}$ for some $2\times 2$ invertible matrix $g_0$. Suppose
the first possibility occurs, i.e. $g_1=g_2=g$. Then the matrix
$\left(\begin{smallmatrix} 0&I\\I&0 \end{smallmatrix}\right)$ commutes
with all the
elements of MT$'(V)$. Hence the corresponding element in $(\Check{V} 
\otimes
V)_{\mathbb{C}}$ is fixed by the complexification of the Mumford-Tate
group. It follows that $\left(\begin{smallmatrix} 0&I \\ I&0
\end{smallmatrix}\right)$
corresponds to an element of bidegree (0,0) in the weight 0 Hodge
structure $(\Check{V} \otimes V)_\mathbb{Q}$. Thus it gives an
endomorphism of Hodge structures of $V$, which maps $V_1$ isomorphically
onto $V_2$, which is absurd as those Hodge structures are not
isomorphic. This rules out the case $g_1=g_2$. The argument that allows one to
exclude the
second case is similar.

For case b note that MT$'(V)$ contains both $SL_2$ and $T$ which intersect
trivially, hence the claim follows from a dimension argument. 

For d and e we have $2\leq \text{dim MT}'(V) \leq 3$. If dim
MT$'(V)=3$ we obtain case d. If dim MT$'(V)=2$, the two tori are isogenous
and the associated quadratic imaginary fields are the same. Consideration
of the action of the Galois group forces the additional relation occuring
in case e. 
\end{proof}


\begin{obs} \label{almost_never flips}
  In cases b-e in the above proposition, an element of the normalizer of $\text{MT }'(V)$ in GL$(V)$ doesn't interchange the pieces
  of $V$, but in case a it can happen. However, the matrix that flips the pieces has trace 0.
\end{obs}

\subsection{A Criterion for Irreducibility}

Now we consider the family $Y_s$ of hypersurfaces in $\Pl^4$ given
by:
$$Y_s: s(y_0+y_1+y_2+y_3+y_4)^5 = y_0y_1y_2y_3y_4.$$ We want to apply
 the results of the previous sections to determine the $s \in
 \mathbb{C}$ for which  $H^3(Y_{s},\mathbb{Q})$ is reducible. We will give a
 sufficient condition for the irreducibility of the Hodge
 structure. \\ Assume the Hodge structure is reducible for some $s$.
 Then from de Jong's lecture we know that $s\in \Bar{\mathbb{Q}}$ if
 we assume Conjecture 1.4.  Let $k_0 = \mathbb{Q}[s]$, and let
 $\mathcal{O}_{k_0}$ be the ring of integers of $k_0$. Pick a prime
 $\wp \subset \mathcal{O}_{k_0}$ and let $Frob_{\wp}$ be a Frobenius
 element in Gal$(\Bar{\mathbb{Q}}/k_0)$ for $\wp$. For simplicity,
 assume that $\wp$ is a degree one prime. Also assume that $\wp$ lies
 above $(p) \subset \mathbb{Z}$. We can then compute the characteristic
 polynomial $P_{\wp}$ of $Frob_{\wp}$ acting on $H^3_{et}(Y_{s}
 \otimes \Bar{\mathbb{Q}}, \mathbb{Q}_{l})
 \overset{\cong}{\rightarrow} H^3_{B}(Y_{s} \otimes \mathbb{C},
 \mathbb{Q})\otimes \mathbb{Q}_{l}$ for $l \neq p$. (Note that
 $P_{\wp}$ depends on $s$, but for simplicity we omit the $s$ here.)
 We can compute $P_{\wp}$ by reducing the equation for $Y_s \mod \wp$ and
 by counting points over $\mathbb{F}_p, \mathbb{F}_{p^2}$. (We know the
 contributions of the pieces $H^i_{et}(Y_s\otimes \Bar{\mathbb{Q}},
 \mathbb{Q}_{l})$ for $i \neq 3$, because the Betti numbers for those
 pieces are all 0 or 1.)  Then there are two cases to consider:\\
 \textbf{Case 1:} $Frob_{\wp}$ flips the two factors $V_1$ and $V_2$.
 Then, as observed above (See Observation 1.10), we know that
 trace$(Frob_{\wp}) = 0$. This occured only once in our numerical
 data, so we will ignore this case.\\ \textbf{Case 2:} $Frob_{\wp}$
 does not flip the pieces.  In this case, the characteristic
 polynomial $P_{\wp}$ which has degree $4$ must factor into two
 quadratic factors over $\mathbb{Q}_{l}$ for every $l\neq p$, because
 $Frob_{\wp}$ preserves the two two-dimensional spaces $V_1$ and
 $V_2$. We know the shape of the polynomial $P_{\wp}$:
  \[P_{\wp} = x^4 + a_1 x^3 + a_2 x^2 + p^3 a_1 x + p^6 \]
  and the roots $\lambda_1, \lambda_2, \lambda_3, \lambda_4$ can be
  numbered so that they satisfy the equations $\lambda_1 \lambda_2 =
  p^3$, $\lambda_3 \lambda_4 = p^3$. \\ Now we can ask: When is there
  an $l$ such that $P_{\wp}$ is irreducible over $\mathbb{Q}_{l}$? It
  is certainly sufficient that \\ \indent 1. $P_{\wp}$ is irreducible
  over $\mathbb{Q}$ and \\ \indent 2. If $K$ is the splitting field of
  $P_{\wp}$, then Gal$(K/\mathbb{Q})$ contains a 4-cycle.\\ However,
  because of the shape of the polynomial and relations on the roots,
  the Galois group Gal$(K/\mathbb{Q})$ can only be the Klein 4-group
  $V_4$ or it contains a 4-cycle.  Therefore, the group we wish to
  eliminate is $V_4$.  We can check if Gal$(K/\mathbb{Q})$ is $V_4$ by
  checking whether the discriminant of $P_{\wp}$ is a square in $\mathbb{Q}^*$.

With  notation as above, we can summarize our results as follows:
\begin{summary}
  Given an $s$ in $\Bar{\mathbb{Q}}$, if there exists a prime $\wp$ of
  $\mathcal{O}_{k_0}$ such that $P_{\wp}$ satisfies:\\ \indent
  1. trace$(P_{\wp}) \neq 0$\\ \indent 2. $P_{\wp}$ is irreducible
  over $\mathbb{Q}$ \\ \indent 3. disc$(P_{\wp})$ is not a square in
  $\mathbb{Q}^*$ \\ then $H^3(Y_{s}, \mathbb{Q})$ is
  irreducible.
\end{summary}
\subsection{Some Numerical Data}
Now we present some numerical data to see if the $P_{\wp}$ are ever
reducible for any $s$. There certainly is not enough evidence to even
conjecture whether the Hodge structure is ever decomposable or not
because we were only able to count the points of $Y_s$ over
$\mathbb{F}_p$ and $\mathbb{F}_{p^2}$ for very small primes $p$.  We
are excluding the case where $p=5$, and for each $p$ we look at $s \in
\{1, \cdots, p-1\}$, $s \neq 1/5^5 \mod p$. In the following we
compute the characteristic polynomial $P_{\wp}=P_{\wp,s}$ for a given
prime $p$ and all allowed values of $s$.
\begin{center}
\begin{tabular}{|c|c|c|c|}
\hline  $s$ & $P_{\wp}$ for $p=3$ & irredcible ?& disc a square ?
\\ \hline $1$ & $x^4  + 5 x^3 + 45 x^2  + 135 x + 729$ & yes & no

 \\ 

 \hline
\end{tabular}
\end{center}

\begin{center}
\begin{tabular}{|c|c|c|c|}
\hline  $s$ & $P_{\wp}$ for $p=7$ & irreducible ? & disc a square ?
\\ \hline $1$ & $x^4  +5 x^3  + 385 x^2  + 7^3 \cdot 5 x + 7^6 $ & yes &no

 \\ \hline $2$ & $x^4 + 25 x^3 + 350 x^2 + 7^3 \cdot 25 x + 7^6 $ &
 yes & no\\


  \hline $3$ & $x^4  + 10 x^3  + 420 x^2  + 7^3 \cdot 10 x + 7^6$ & yes & no\\

\hline $4$ & $x^4  -5  x^3  - 210 x^2  - 7^3 \cdot 5 x + 7^6$ & yes & no \\
\hline $6$ & $x^4 - 35 x^3 + 805 x^2 - 35 \cdot 7^3 x + 7^6$ & yes & no\\
\hline
\end{tabular}
\end{center}
For $p=11$, again, for each allowed value of $s$, the corresponding
characteristic polynomial has nonzero trace, is irreducible and its
discriminant is not a square in $\mathbb{Q}^*$.
\begin{center}
\begin{tabular}{|c|c|c|c|}
\hline $s$ & $P_{\wp}$ for $p=13$ & irreducible ? & disc a square ?
\\ \hline $4$ & $x^4 + 10 x^3 -910 x^2 + 13^3 \cdot 10 x + 13^6 $ &
{\bf{no}} ! &no

 \\ \hline $6$ & $x^4 -120 x^3 + 7670 x^2 + 13^3 \cdot (-120) x + 13^6
 $ & {\bf{no}} !& no\\

\hline
\end{tabular}
\end{center}
All other allowed values for $s$ are ok when $p$ is $13$.\\
 For $p=17$ all allowed values of $s$ are ok.
\begin{center}
\begin{tabular}{|c|c|c|c|}
\hline  $s$ & $P_{\wp}$ for $p=19$ & irreducible ? & disc a square ?
\\ \hline $13$ & $(x^2-95x+19^3)(x^2+100x+19^3) $ & {\bf{no}} ! &no

 \\ \hline

\hline
\end{tabular}
\end{center}
All other allowed values for $s$ are ok when $s$ is $19$.\\ When $p$
is $23$ and $s=18$, the corresponding characteristic polynomial is
$x^4 + 8050 x^2 + 23^6$ which has trace $0$ and whose discriminant is
a square.\\ When $p=31$ and $s=5$, the corresponding charcateristic
polynomial factors as $(x^2-217x+31^3)\cdot(x^2+ 108x+31^3)$.\\ When
$p=41$ and $s=18$, the corresponding characteristic polynomial
factors as $(x^2-372x+41^3)\cdot(x^2+328x+41^3)$.



\end{document}
