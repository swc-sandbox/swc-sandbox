\documentstyle[twoside,amssymb,amsmath,12pt]{article}
\setlength{\textheight}{9in}
\addtolength{\textwidth}{0.5in}
\setlength{\oddsidemargin}{0in}
\setlength{\evensidemargin}{0in}
\addtolength{\topmargin}{-.75in}
\renewcommand{\thefootnote}{}
\renewcommand{\baselinestretch}{1.0}

\newcommand{\no}{\noindent}
\def\C{{\Bbb C}}
\def\H{{\Bbb H}}
\def\N{{\Bbb N}}
\def\Q{{\Bbb Q}}
\def\R{{\Bbb R}}
\def\P{{\Bbb P}}
\def\Z{{\Bbb Z}}
\def\A{{\Bbb A}}
\def\G{{\Bbb G}}
\def\e{{\varepsilon}}

\newtheorem{prop}{Proposition}[section]
\newtheorem{dfn}[prop]{Definition}
\newtheorem{theo}[prop]{Theorem}
\newtheorem{conj}[prop]{Conjecture}
\newtheorem{rem}[prop]{Remark}
\newtheorem{coro}[prop]{Corollary}
\newtheorem{lem}[prop]{Lemma}
\newtheorem{exam}[prop]{Example}
\newtheorem{ques}[prop]{Question}
\newtheorem{prob}[prop]{Problem}

\title{\sc Cohomology, periods and the Hodge structure of 
toric hypersurfaces \\
{\small  (Course description) }} 

\author{{\sc Victor V. Batyrev} \\
\small  {\em Mathematisches Institut, Universit\"at T\"ubingen}   \\
\small  {\em Auf der Morgenstelle 10,  72076  T\"ubingen, Germany}  \\
\small  {\em e-mail: victor.batyrev@uni-tuebingen.de} \\
 }


\begin{document}

\date{}

\maketitle

\begin{abstract}
The aim of this course is to study the cohomology groups $H^*(Z_f)$
of nondegenerate affine toric hypersurfaces $Z_f \subset ({\bf C}^*)^d$.
Some properties of the cohomology  groups  
can be described in terms of the 
Newton polytope of the equation $f$. 
We relate the  periods of $Z_f $ to the  GKZ-hypergeometric functions and 
give applications in physics and number theory. 
\end{abstract}



%\thispagestyle{empty}
\section{Introduction} 

Let $M \cong {\bf Z}^d$ be a free abelian group of rank $d$. We 
identify $M$ with the group of characters of the $d$-dimensional
torus ${\bf T}_d \cong  ({\bf C}^*)^d$ 
$${\bf T}_d = {\rm Spec}\, {\bf C}[M] \cong  
{\rm Spec}\, {\bf C}[X_1^{\pm 1}, \ldots, X_d^{\pm 1}].$$
Let $\Delta \subset M 
\otimes {\bf R}$  be a $d$-dimensional convex polytope  such that 
all vertices of $\Delta$ belong to 
the lattice $M$. We choose a finite subset $A \subset  \Delta \cap M$
which contains all vertices of $\Delta$ and   
consider a Laurent polynomial
\[ f= f(X_1, \ldots, X_d) = \sum_{ m \in A} a_m X^m, \]
where $a_m$ $(m \in A)$ are sufficiently general complex
numbers.  

We will be interested in cohomology groups $H^i(Z_f, {\bf Z})$ 
of the affine hypersurface $Z_f$ in 
${\bf T}_d$ defined by the equation $f =0$. Since $Z_f$ is affine, 
one has   $H^i(Z_f, {\bf Z}) =0$ for $i \geq d$. By the Lefschetz-type
theorem, one obtains the isomorphisms
\[ H^i(Z_f, {\bf Z}) \cong H^i({\bf T}_d, {\bf Z}) =  \Lambda^{i} M, 
\;\; i < d-1. \]
Therefore the groups  $H^{d-1}(Z_f, {\bf Z})$ and  $H^{d-1}(Z_f, {\bf C})$
are the only interesting objects for our study. 

The group  $H^{d-1}(Z_f, {\bf C})$ has a mixed Hodge structure which can 
be characterized by Hodge-Deligne numbers \cite{DH}. On the other hand, 
the periods, i.e., integrals of $(d-1)$-differential forms 
on $Z_f$ over (d-1)-dimensional 
cycles satisfy a system of differential equations of Picard-Fuchs type. 
These differential equations are important for applications in physics 
\cite{BS} and they  
have $p$-adic analogs \cite{Dwork} which are related to the Zeta-function 
of $Z_f$ over a finite field.



\section{Course content} 

$\;$

1. The toric compactification of ${\bf C}^d$  with respect to a lattice 
polytope $\Delta$. The nondegeneracy condition for hypersurfaces 
$Z_f \subset {\bf C}^d$. The Euler number of $Z_f$. The number of critical
points of $f$ in $ {\bf C}^d$. The Lefschetz-type theorem for $Z_f$. 


2. De Rham cohomology of a nondegenerate hypersurface 
$Z_f \subset {\bf C}^d$. Logarithmic de Rham complex. Principal 
$A$-determinant
of $f$ in the sense of Gelfand-Kapranov-Zelevin\-sky 
\cite{GKZ2}. Jacobian ring $R_f$ and 
its canonical  module \cite{B-duke}. 
Cohomology with compact supports. Duality and 
toric residues. Hodge-Deligne numbers of $Z_f$ \cite{BB1}. 


3. Generalized hypergeometric differential system of  
Gelfand-Kapranov-Zelevin\-sky \cite{GKZ1}. 
The dimension of the solution space of 
GKZ-system. Coherent triangulations of the Newton polytope and a basis 
of the solution space. Generalized GKZ-hypergeometric functions as periods
of hypersurfaces $Z_f$. 


4. The secondary polytope ${\rm Sec}(\Delta)$ as the Newton polytope of the 
pricipal $A$-determinant of $f$.  The asymptotics  of complex and real 
hypersurfaces corresponding to vertices of  ${\rm Sec}(\Delta)$. The monodromy 
of $1$-parameter familites. The 
method of Viro and methods of tropical geometry \cite{Mikh}. 


5. Applications in physics and number theory. 
The toric mirror symmetry \cite{BA}. 
Monomial-divisor mirror corespondence \cite{AGM}.
The Seiberg duality. $P$-adic versions of $GKZ$-hypergeometric functions 
and period. Affine toric Fermat-type hypersurfaces. 




\section{Student project} 

First interesting examples for study are families of affine 
algebraic curves $Z_f \subset {\bf T}_2$ defined by a 
$2$-dimensional polytope (polygone) $\Delta \subset M \otimes {\bf R}$. 
If $n$ is the number of lattice points on the boundary of $\Delta$ 
and $g$ is the number of interior lattice points in  $\Delta$, then 
$Z_f$ can be seen as a Riemann surface $\overline{Z_f}$ 
of genus $g$ minus $n$ points.
Periods of $\overline{Z_f}$ are classical objects of algebraic geometry 
\cite{clem}.



\section{Prerequisites}

It is recommended to have some background on algebraic geometry 
(see e.g. the book of Griffiths and Harris \cite{GH}) and toric
geometry (see e.g. the book of Fulton \cite{F}). 




\begin{thebibliography}{99}
\bibitem{AGM} P.S. Aspinwall,  B.R. Greene, and D.R. Morrison, .
{\em The monomial-divisor mirror map}, 
Int. Math. Res. Not., No.12, (1993),  319-337. 

 \bibitem{B-duke} {V.V. Batyrev} {\em Variations of the mixed Hodge 
structure of affine hypersurfaces in algebraic tori}.  
Duke Math. J.  69  (1993),  no. 2, 349--409.

\bibitem{BA} {V.V. Batyrev}, {\em Dual polyhedra and mirror symmetry for 
Calabi-Yau hypersurfaces in toric varieties}, J. Algebraic Geom., 
{\bf 3} (1994), 493-535. 



\bibitem{BS}  {V.V. Batyrev} and D. van Straten, {\em 
Generalized Hypergeometric Functions and Rational Curves on Calabi-Yau 
complete intersections in Toric Varieties}, Commun. Math. Phys. 
{\bf 168} (1995), 493-533. 



\bibitem{BB1} {V.V. Batyrev, L.A. Borisov}, 
{\em Mirror duality and string-theoretic Hodge numbers},  
Invent. Math., {\bf 126} (1996), p. 183-203.


\bibitem{clem} H. Clemens, {\em A scrapbook of complex curve theory}. 
Second edition. Graduate Studies in Mathematics, 55. 
AMS, Providence, RI, 2003.


\bibitem{DH} 
Danilov, V. I.; Khovanski, A. G. {\em 
Newton polyhedra and an algorithm for calculating Hodge-Deligne numbers}, 
Izv. Akad. Nauk SSSR Ser. Mat.  50  (1986),  no. 5, 925--945.

\bibitem{Dwork} B. Dwork, {\em 
Lectures on $p$-adic differential equations} 
Grundlehren der Mathematischen Wissenschaften,  253. Springer-Verlag, 
New York-Berlin, 1982


\bibitem{F} W. Fulton, {\em Introduction to toric varieties}, 
Annals of Mathematics Studies, 131. 
The William H. Roever Lectures in Geometry. Princeton University Press, 
Princeton, NJ, 1993.


\bibitem{GKZ1} Gelfand, I. M.; Zelevinski, A. V.; Kapranov, M. M. 
{\em Hypergeometric functions and toric varieties} 
Funct. Anal. Appl.  23  (1989),  no. 2, 94--106

\bibitem{GKZ2} Gelfand, I. M.; Zelevinski, A. V.; Kapranov, M. M. {\em 
Discriminants, resultants, and multidimensional determinants}, 
Mathematics: Theory \& Applications. Birkh\"auser Boston, Inc., 
Boston, MA, 1994


\bibitem{GH} Ph. Griffiths, J. Harris, {\em 
Principles of algebraic geometry} John Wiley \& Sons, Inc., New York, 1994.

\bibitem{Mikh} G. Mikhalkin, 
{\em Decomposition into pairs-of-pants for complex algebraic 
hypersurfaces}, math.GT/0205011, to appear in Topology.


\bibitem{oka} M. Oka, {\em On the topology of full nondegenerate complete 
intersection variety},   Nagoya Math. J.  121  (1991), 137--148.




\end{thebibliography}

\end{document}


