\documentclass[12pt]{article}
\usepackage{amsfonts, amsmath, amsthm}
\usepackage[all]{xy}

\setlength{\textwidth}{6.5in}
\setlength{\oddsidemargin}{0in}
\setlength{\textheight}{8.5in}
\setlength{\topmargin}{0in}
\setlength{\headheight}{0in}
\setlength{\headsep}{0in}
\setlength{\parskip}{0pt}
\setlength{\parindent}{20pt}

\newtheorem{theorem}{Theorem}[section]
\newtheorem{lemma}[theorem]{Lemma}
\newtheorem{conj}[theorem]{Conjecture}
\newtheorem{cor}[theorem]{Corollary}
\newtheorem{prop}[theorem]{Proposition}

\def\AAA{\mathbb{A}}
\def\CC{\mathbb{C}}
\def\GG{\mathbb{G}}
\def\HH{\mathbb{H}}
\def\PP{\mathbb{P}}
\def\QQ{\mathbb{Q}}
\def\RR{\mathbb{R}}
\def\ZZ{\mathbb{Z}}
\def\calL{\mathcal{L}}
\def\calO{\mathcal{O}}
\def\dual{\vee}

\DeclareMathOperator{\ab}{ab}
\DeclareMathOperator{\Assoc}{Assoc}
\DeclareMathOperator{\B}{B}
\DeclareMathOperator{\comp}{comp}
\DeclareMathOperator{\dR}{dR}
\DeclareMathOperator{\End}{End}
\DeclareMathOperator{\Fil}{Fil}
\DeclareMathOperator{\Gal}{Gal}
\DeclareMathOperator{\GL}{GL}
\DeclareMathOperator{\Gr}{Gr}
\DeclareMathOperator{\FreeLie}{FreeLie}
\DeclareMathOperator{\Lie}{Lie}
\DeclareMathOperator{\Res}{Res}
\DeclareMathOperator{\Spec}{Spec}
\DeclareMathOperator{\Sym}{Sym}

\newcounter{fixmectr}
\def\fixme#1{\addtocounter{fixmectr}{1} \textbf{FIXME \#\thefixmectr.} (#1)}

\begin{document}

\begin{center}
\bf
Periods for the Fundamental Group \\
Lectures by Pierre Deligne; notes by Kiran Kedlaya \\
Arizona Winter School 2002
\end{center}

\section*{About these notes}

These notes are an attempt to transcribe/translate my notes from Deligne's
lectures at the 2002 AWS. It was difficult to make much of an accurate record
of what was going on; I have attempted to ``add value'' by filling in things
that were not said explicitly. Any resulting errors are my fault alone.
For all (or at least most) of the details, the \"uberreference is
Deligne's article \cite{bib:del}; suggestions for additional references
would be welcome.

The five sections correspond approximately to the five lectures given by
Deligne; as the ``cuts'' between topics did not quite coincide with the
breaks between lectures, I've followed the former instead of the latter
in placing the section breaks.

These notes are copyright 2002-2003 by Kiran S. Kedlaya. You may freely
distribute unmodified 
copies of these notes; I fully intend to grant permission to distribute
modified copies also, especially if errors have been corrected!

This version was last revised 20 Mar 2003, and covers all of the first
4 lectures and part of the 5th lecture.
I plan to ultimately distribute a version that includes all 5
lectures, but I have many other things to do in the interim, so please
be patient.

\section{Introduction}

Throughout these lectures, let $X/K$ be a smooth affine variety over a
number field $K$, and choose an embedding $K \hookrightarrow \CC$.
The goal is to relate two cohomology theories associated to $X$: the
(torsion-free part of the)
singular, or ``Betti'' cohomology of the topological space $X(\CC)$,
and the algebraic ``de Rham'' cohomology, i.e., the (hyper)cohomology of
the complex $\Omega_X^*$ of algebraic differential forms on $X$. (The ``hyper''
would be relevant if $X$ were not required to be affine.) Specifically,
there is a natural (``motivic'') comparison isomorphism
\[
\comp_{\B,\dR}: H_{\dR}(X) \otimes_K \CC \stackrel{\sim}{\to}
H_{\B}(X) \otimes_{\QQ} \CC
\]
and we want to know what it is. Explicitly, a basis for $H_{\dR}$ is given
by a set of algebraic differentials, a dual basis for $H_B(X)$ is given by
a set of topological cycles (by Poincar\'e duality),
and the comparison isomorphism, viewed as a
perfect pairing on $H_{\dR}(X) \otimes_K \CC$ with
$(H_{\B}(X) \otimes_{\QQ} \CC)^\dual$, is integration of the differential along
the cycle (the so-called \emph{period pairing}).

\subsection*{Example: $\GG_m$}
For example, if $X = \GG_m = \PP^1 - \{0, \infty\}$, then $H_{1,\B} =
(H^1_{\B})^\dual$ is generated by a counterclockwise loop $\sigma$ around
0, and $H^1_{\dR}$ is the cokernel of $d$ as a map from $\QQ[z, z^{-1}]$
to $\QQ[z, z^{-1}] \,dz$. That cokernel is generated by $dz/z$, and in this
case the pairing simply pairs $\sigma$ and $dz/z$ to $\int_\sigma dz/z
= 2 \pi i$.

\subsection*{Aside: a bit of functoriality}

Suppose the embedding $K \hookrightarrow \CC$ factors through $\RR$.
Then $X(\CC)$ comes with an involution, namely complex conjugation,
which thus induces an involution $F_\infty$ on $H_{\B}(X)$. For example,
in $\GG_m$, $F_\infty$ turns the counterclockwise loop $\sigma$ into the
clockwise loop $\sigma^{-1}$. In other words, if we view $\CC$ as
``an'' algebraic closure of $\RR$, then $H_{\B, k \hookrightarrow \CC}$
is ``functorial in $\CC$''.

\subsection*{The Hodge filtration}

Besides the usual analytic construction, the Hodge filtration on
$H_{\dR}(X)$ can be constructed algebraically in at least two ways.
In the process, it will be clear that the Hodge filtration is respected
by functorial maps on cohomology.

First way: embed $X$ into $\overline{X}$, a smooth proper variety in which the
complement $D = \overline{X} \setminus X$ is a normal crossings divisor. If you
are paranoid, take $D$ to be a strict normal crossings divisor, that is,
each component of $D$ is itself a normal variety. (That
this can always 
be done follows from the resolution of singularities theorem
of Hironaka.) Now let $\Omega_{\overline{X}}^*(\log D)$ be the complex of 
differential forms which are permitted to have logarithmic (i.e, $dz/z$)
poles along components of $D$. The hypercohomology of this complex (and
this time we really need the ``hyper'', since we're on $\overline{X}$
which is not affine) is precisely the de Rham cohomology of $X$. A bit
more precisely, there is a spectral sequence
$E^1_{pq}= H^q(\overline{X}, \Omega^p_{\overline{X}}(\log D))
\Rightarrow \HH(\Omega_{\overline{X}}^{p+q}(\log D))$,
but it degenerates at $E^1$.

In this notation, the steps of the Hodge filtration are, for each $p$,
precisely the cohomology of the subcomplex of 
$\Omega^*_{\overline{X}}(\log D)$ where you look at $p$-forms or higher.

Second way (better suited for finite characteristic): 
for each $n$, one usually denotes by $\calO_{\overline{X}}(nD)$ the sheaf of
rational functions on $\overline{X}$ with poles of order at worst $n$ along the
components of $D$ (and no other poles). Let $\calO_{\overline{X}}(\infty D)$
be the union of these; then the cohomology of the complex
\[
\calO_{\overline{X}}(\infty D) \to \Omega^1_{\overline{X}}(\infty D) \to \cdots
\]
is again the de Rham cohomology of $X$, and the Hodge filtration
can be given by taking $\Fil^i = \calO_{\overline{X}}(i D)$.

\subsection*{Random stuff about motives}

These seem to be a few universal comments about cohomology of varieties,
so I'll denote the cohomology without specifying whether I mean Betti,
de~Rham, or anything else; I mean all of them. Of course, what we're really
doing is talking about motives, but we're not going to define what a motive is.
Think of it as a piece of the ``universal'' cohomology of a variety; that is,
the only operations allowed are ones that come from geometry. For
example, a morphism $X \to Y$ induces maps $H^{\cdot}(Y) \to H^{\cdot}(X)$,
and there is a natural isomorphism of $H^{i}(X \times Y)$ with
$\sum_{j+k=i} H^j(X) \otimes H^k(Y)$.

Denote by $\ZZ(-1)$ the Tate motive $H^1(\GG_m) = H^2(\PP^1)
= H^2_c(\AAA^1) = H^2_{\{0\}}(\AAA^1)$. The point is that these are all
canonically isomorphic. Let $\ZZ(-n)$ denote the $n$-th tensor power of 
$\ZZ(-1)$ (for $n$ positive, negative or zero).
Note that for $X$ projective, smooth, and irreducible
of dimension $d$, $H^{2d}(X)$ is canonically isomorphic to $\ZZ(-d)$.

Also, if $Z \subset X$ is an algebraic cycle of codimension $d$, then
one gets a canonical class in $H^{2d}_Z(X) \otimes \ZZ(d)$.

One says that a cohomological structure is ``motivic'' if it can be defined
purely in terms of algebraic geometry, without reference to a specific
cohomology theory. The Hodge filtration above is not motivic; its construction
is specific to the de~Rham theory. There is another filtration, the
\emph{weight filtration}, which is motivic, but unfortunately I couldn't
follow the description given in the lecture.

There was also a bit more about periods that didn't make much sense. One
comment cribbed from Lecture~2 (which starts immediately below);
periods can be viewed as ``coordinates
on Hodge structures'' (a la de~Jong's lectures).

\subsection*{A little more ``motive''-ation: realizations}

(Caution: notihng we're about to say is going to be rigorous, or even
particularly sensical. Nonetheless it ought to provide some helpful context.
Also, this actually happened at the start of Lecture 2, but it belongs more
naturally with the first lecture.)

Let $M$ be a ``motive'', i.e., a piece of cohomology of a variety cut out
by purely geometric means. For example, $M$ might be the full $i$-th cohomology
of a variety. Each cohomology theory for algebraic varieties is
what we call a ``realization'' of $M$. In the cases at hand, we
have the ``Betti realization'' $M_B$ which comes from the topology of the
complex points of a variety, and the ``de Rham'' realization $M_{dR}$ which
comes from differential forms. There may be additional motivic structures
on $M$, e.g., if $M$ is the middle cohomology of a projective variety,
then the intersection pairing gives a ``polarization'' $M \otimes M
\to \ZZ(-d)$. 
In any case, there should exist a comparison isomorphism
\[
\comp: M_{dR} \to M_B
\]
that respects any additional structures.

Now in each realization of $M$, there should be a group of automorphisms
$G_B$ or $G_{dR}$, respectively, that respect all the motivic structures.
These in turn should be ``realizations'' of a ``motivic Galois group''
of $M$. A little bit more precisely, there should be a scheme
$ _{dR}P_B$ of isomorphisms from $G_{dR}$ and $G_B$, which has points over
$\CC$ such as $\comp$, but typically not any points over $\QQ$.

It is conjectured that $\comp$ is a \emph{generic} point of $P$, at least
if our varieties are over a number field. (Already over $\QQ(t)$, it is
possible to have ``accidental'' dependences.)

(some stuff omitted here because I didn't get it written down.)

In passing, we note that there are additional realizations besides the
Betti and de Rham realizations (i.e., additional cohomology theories on
algebraic varieties over a number field) that are important and useful.
For example, say $M$ is a motive over a number field $k$. Then for each
prime $\ell$, there is an $\ell$-adic realization $M_{\ell}$, which is
a $\QQ_\ell$-vector space carrying an action of $\Gal(\overline{k}/k)$.
If $M$ is the full cohomology of a variety $X$, then $M_{\ell}$ is the
full $\ell$-adic (\'etale) cohomology of $X \times_k \overline{k}$.

Another example is the ``crystalline'' realization.
If $M$ is a motive over $\ZZ_p$ (i.e., the cohomology of a smooth scheme
$X/\ZZ_p$, i.e., a smooth variety over $\QQ_p$ with good reduction), then
the de Rham cohomology (of the generic fibre)
$H_{\dR}(X)$ is a $\QQ_p$-vector space, which
depends functorially just on the reduction of $X$ modulo $p$. That means
the absolute Frobenius on $X$ (which acts on the structure sheaf by
sending $x$ to $x^p$ for all $x$) induces an action on $H_{\dR}(X)$.
This construction is important, for example, in the project carried out
by Deligne's students at the AWS.

\section{The unipotent fundamental group: Betti realization}

\subsection*{Coproducts}

Let $\QQ[\pi_1]$ be the group algebra of $\pi_1$ with coefficients in
$\QQ$. (That is, it's the $\QQ$-vector space generated by the elements of
$\pi_1$, with the multiplication in the algebra given by
$(\sum_i q_i [\gamma_i])(\sum_j r_j [\eta_j]) =
\sum_{i,j} q_i r_j [\gamma_i \eta_j]$.)
Then $\QQ[\pi_1]$ admits a natural coproduct structure:
$\Delta: \QQ[\pi_1] \mapsto \QQ[\pi_1] \otimes \QQ[\pi_1]$ sending
$[\sigma] \mapsto [\sigma] \otimes [\sigma]$. (A coproduct on an $R$-algebra
$A$ is an algebra homomorphism $\Delta: R \to R \otimes_A R$ which
is coassociative, i.e., the two natural maps $R \to R \otimes_A R \otimes_A R$
you can make out of $\Delta$ are the same.)

If we let $I$ be the augmentation ideal of $\QQ[\pi_1]$,
generated by $[\sigma] - 1$ for all $\sigma \in \pi_i$, then
the coproduct induces a coproduct $\Delta$ on $\QQ[\pi_1]/I^n$.
This coproduct happens to be cocommutative (you get the same thing if you
postcompose with switching the two factors of $\QQ[\pi_1]$ in the tensor
product), so taking $\Spec$ of
the $\QQ$-dual $(\QQ[\pi_1]/I^n)^\dual$ gives an algebraic group. If
we take the inverse limit of these (i.e., take the direct limit of the duals
and then take $\Spec$ of that), we get a pro-algebraic group.
This is the \emph{unipotent fundamental group} over $\QQ$.

\subsection*{A filtration on the fundamental group}

Here is another, more group-theoretic interpretation of the pro-algebraic group
we constructed in the previous section.

Given $\pi_1$, we first construct the commutator subgroup $(\pi_1, \pi_1)$,
so that the quotient $\pi_1/(\pi_1,\pi_1)$ is the maximal abelian
quotient of $\pi_1$. This cannot sit inside the set of $\QQ$-valued points of
$(\QQ[\pi_1]/I^2)^\dual$ because the latter is torsion-free. So we
define $Z^1(\pi_1)$ as the subgroup of $\pi_1$ consisting of all $g
\in \pi_1$ some power of which is in $(\pi_1, \pi_1)$. Now define
$Z^{n+1}(\pi_1)$ as the subgroup of $\pi_1$ consisting of all
$g \in \pi_1$ some power of which is in $(\pi_1, Z^n(\pi_1))$. Then
each quotient $Z^n(\pi_1)/Z^{n+1}(\pi_1)$ is a free abelian group;
since $\pi_1$ is finitely generated, so are these quotients.

The inverse limit $\lim_{\leftarrow} \pi_1/Z^n(\pi_1)$ sits inside
the inverse limit $\lim_{\leftarrow} \Spec (\QQ[\pi_1]/I^n)^\dual$;
in fact, we can write it as the $\ZZ$-valued points of a certain
pro-algebraic group. Namely, pick free generators $e_1, \dots, e_n$
of $\pi_1/Z^1(\pi_1)$, and lift them to $\tilde{e}_1, \dots,
\tilde{e}_n$ in $\pi_1$. Then pick free generators $e_{n+1}, \dots, e_{n+m}$
of $Z^1(\pi_1)/Z^2(\pi_1)$ and lift them to $\tilde{e}_{n+1},
\dots, \tilde{e}_{n+m}$ and so on. We can write each element of the
inverse limit as an infinite product $\tilde{e}_1^{z_1} \tilde{e}_2^{z_2}
\cdots$; think of the $z_i$ as ``coordinates'' on the group.
In these coordinates, the group law is unipotent in the $z_i$: given
$g = \prod \tilde{e}_i^{y_i}$ and $h = \prod \tilde{e}_i^{z_i}$, the
coordinate of $\tilde{e}_i$ in $gh$ is $y_i+z_i$ plus a polynomial in
the prior $y_j$ and $z_j$. That polynomial is of course integer-valued
on integer arguments, but need not have integer coefficients.

In this formulation, you recover 
$\lim_{\leftarrow} \Spec (\QQ[\pi_1]/I^n)^\dual$
by formally allowing the $z_i$ to be rational numbers, using the
aforementioned polynomials to give the group law. In group theory, I think
this construction is called the Mal'cev completion of $\pi_1$.

An interesting point of view: giving the affine algebra 
$\lim_{\rightarrow} (\QQ[\pi_1]/I^n)^\dual$ amounts to specifying
which functions on $\pi_1$ are ``algebraic''. The condition of algebraicity
is as follows: for each $\tau \in \pi_1$, define the ``difference operator''
$\Delta_\tau$
on the space of $\QQ$-valued functions on $\pi_1$ as follows:
\[
(\Delta_\tau f)(\sigma) = f(\sigma \tau) - f(\sigma).
\]
Then the algebraicity condition on a function $f$ is that there exists $N$
such that $\Delta_{\tau_1} \Delta_{\tau_2}\cdots \Delta_{\tau_N} f = 0$
for all $\tau_1, \dots, \tau_N$. For example, if $\pi_1 = \ZZ$, then this
condition says precisely that $f: \ZZ \to \QQ$ is a polynomial
(of degree at most $N-1$).

Geometric aside that I don't really understand:
it is a familiar fact that $H_1(X, \CC)$ is canonically isomorphic to
the torsion-free quotient of the abelianization of $\pi_1$, that is,
$\pi_1/Z^1(\pi_1)$. Apparently (and I didn't follow this remark) the 
successive steps $Z^{n-1}(\pi_1)/Z^n(\pi_1)$ in the filtration 
correspond (maybe are canonically isomorphic?) to $H^n(X, \CC)$.

\section{The unipotent fundamental group: de Rham realization}

\section*{Unipotent groups and their Lie algebras}

Recall that for any field $K$ of characteristic 0, there is a
correspondence
\[
\{ \mbox{unipotent groups over $K$} \}
\leftrightarrow \{ \mbox{nilpotent Lie algebras over $K$} \}
\]
by taking logarithms/exponentials. Going from right to left,
one can multiply two exponentials by using the Campbell-Hausdorff formula.
Or from the Lie algebra, make its universal enveloping algebra with
the coproduct $x \mapsto x \otimes 1 + 1 \otimes x$ for $x$ in the Lie
algebra, then take the group to be the ``grouplike'' elements,
those $y$ such that $y \mapsto y \otimes y$.

That correspondence works either for honest algebraic groups or pro-algebraic
groups. Thus in trying to construct the de Rham realization of the
unipotent fundamental group, we can (and will) first construct a nilpotent
Lie algebra. That in turn we will do by first constructing 
the finite-dimensional representations of the group/algebra, i.e.,
local systems.

\subsection*{Local systems and their monodromy}

Given an algebraic variety $X$ over a number field $k$, let $\pi_1
= \pi_1(X(\CC), P)$ be the topological fundamental group of the set
of complex points $X(\CC)$ with some chosen (algebraic) base point $P$.
Then there is a bijection between finite dimensional linear representations
$\rho: \pi_1 \to \GL_n(\CC)$ and rank $n$ local systems on $X(\CC)$,
i.e., complex-analytic vector bundles of rank $n$ over $X(\CC)$ equipped
with an integrable connection; we will describe one direction of
this bijection below. (Note: this
description is not in the original notes.)

Quick refresher on local systems: if
$V$ is the vector bundle (i.e., locally free module over the structure sheaf),
then a connection is a bundle map
$\nabla: V \to V \otimes \Omega^1_{X(\CC)}$ which satisfies the
Leibniz rule: $\nabla(fv) = f \nabla(v) + v \otimes df$. This induces
maps $\nabla_i: V \otimes \Omega^i_{X(\CC)} \to
V \otimes \Omega^{i+1}_{X(\CC)}$; the connection is said to be integrable
if $\nabla_{i+1} \circ \nabla_i = 0$ for all $i$, or equivalently just for
$i=0$. (Integrability is a vacuous condition if $X$ is a curve, which will
be the case in our principal example, the projective line minus three points.)

Example: on $\GG_m$ with coordinate $z$,
consider the rank one local system where the
vector bundle is trivial (so sections can be identified with functions
on $\GG_m$), and the connection maps $f$ to $df - \alpha
\frac{dz}{z}$. Locally $z^\alpha$ is a horizontal section, but it is only
defined globally if $\alpha$ is an integer. Otherwise, if one attempts to
analytically continue $z^\alpha$ around the origin, when one gets back to
the starting point the function has been multiplied by $\exp(2 \pi i \alpha)$.
More generally, given a local system on $X(\CC)$ and a loop $\gamma \in \pi_1$,
analytically continuing a basis of horizontal sections along $\gamma$ results
in a new basis which is related to the old one by some matrix
$M \in \GL_n(\CC)$. That matrix is called the \emph{monodromy} of
$\gamma$; the map that associates to each loop its monodromy gives a
representation $\rho: \pi \to \GL_n(\CC)$, and this is one direction
of the bijection given above.
(We will not describe the reverse direction here.)

For various reasons, we are interested in representations/local systems
in which the monodromy is \emph{unipotent}. Recall that a matrix $M$
is called unipotent if $I-M$ is nilpotent. A representation is called
unipotent if its image consists of unipotent matrices; this implies that
the matrices $I - \rho(\gamma)$ are simultaneously nilpotent, i.e., they
have nontrivial common
kernel, modulo that kernel they again have a common kernel, and
so on. The example on $\GG_m$ above is of course unipotent if and only
if $\alpha \in \ZZ$. (More generally, one might consider representations
which are \emph{quasi-unipotent}, i.e., such that the restriction of the
representation to some subgroup of finite index is unipotent. We won't here.)

\subsection*{Local systems and a Lie algebra}

We continue to suppose $X$ is a smooth variety over a number field $K$. But
now we also assume that $X$ can be embedded into a smooth proper variety
$\overline{X}$ such that the complement $D = \overline{X} \setminus X$
is a normal crossings divisor (you can even assume it's a strict normal
crossings divisor, that is, every component is itself normal),
and such that $H^1(\overline{X}, \calO_{\overline{X}}) = 0$. Note that
only the last condition imposes any restriction: the others can always be
satisfied by Hironaka's resolution of singularities.

The assumption on $H^1$ means (I believe) that every vector bundle
on $X$ extends to a vector bundle on $\overline{X}$. It definitely
means that every local system $(V, \nabla)$ on $X$ extends uniquely
to $\overline{X}$, where it can be written as $(V_0,
d - \omega$ for some $\omega \in \Omega^1(\log D) \otimes \End V_0$
which is integrable, i.e., $d\omega = \omega \wedge \omega$.
(The $\log D$ reflects the fact that
the connection must be allowed to have simple, or ``logarithmic'',
poles along $D$. Recall the examples on $\GG_m$, where $\omega = dz/z$.)

By Hodge theory (?), every global $p$-form on a complete variety with
at most logarithmic poles (maybe along a normal crossings divisor) is
closed. Thus integrability becomes just $\omega \wedge \omega = 0$.
(That looks like an empty condition, but remember that $\omega$ is
a \emph{matrix} of 1-forms. Already when that matrix is $2 \times 2$,
the condition is nontrivial. Try it!)

We now proceed to reformulate the integrability condition $\omega \wedge
\omega = 0$ in a more convenient form. Before imposing it, we simply
have
\[
\omega: (\Omega^1(\log D))^\dual \to \End(V_0).
\]
Now the sections of $\Omega^1(\log D)$ are just $H^1_{\dR}(X)$, and the
sections of the dual form the homology group
$H_1^{\dR}(X)$. On cohomology, we have the cup
product
\[
\cup: \wedge^2 \Gamma(\Omega^1(\log D)) \to \Gamma(\Omega^2(\log D)).
\]
Again, $\Gamma(\Omega^1(\log D)) = H^1_{\dR}(X)$, and
$\Gamma(\Omega^2(\log D))$ is contained in $H^2_{\dR}(X)$.
Composing the cup product with that containment, then transposing, gives
\[
\cup^T: H_2^{\dR}(X) \to \wedge^2 H_1^{\dR}(X).
\]
Now $\omega$ can be viewed as a map $\rho: H_1^{\dR} \to \End(V)$,
and integrability of $\omega$ becomes the condition that the composition
\[
H_2^{\dR} \stackrel{\cup^T}{\to} \wedge^2 H_1^{\dR} \to \End(V) \qquad
\]
is zero, where the second map sends $u \wedge v$ to the Lie bracket
$[\rho(u), \rho(v)] = \rho(u) \rho(v) - \rho(v) \rho(u)$. 

To sum up, the data of a local system is the data of a vector bundle $V$
equipped with a representation of the Lie algebra
\[
\FreeLie(H_1^{\dR}(X)) / \Im(\cup^T).
\]
where $\FreeLie$ denotes the free Lie algebra on $H_1^{\dR}(X)$. 

Reminder of what that means: given a vector space $V$, the free Lie algebra
of $V$ is the smallest vector subspace
of the symmetric algebra $\Sym^{\cdot} V$
containing $V$ and closed under Lie brackets. (We'll give another
characterization shortly.) When we mod out by the image of $\cup^T$,
we are actually quotienting out by the ideal in the free Lie algebra
generated by that image (i.e., the smallest subspace of the free Lie
algebra containing the image of $\cup^T$, and
closed under taking the Lie bracket of any of its elements with
anything in the entire Lie algebra).

As mentioned earlier, we don't actually want to consider all local 
systems, just the unipotent ones. That is, we don't want to allow arbitrary
representations of the Lie algebra we just constructed, just unipotent ones.
We can accomplish this by modifying the Lie algebra to only allow 
unipotent representations.

Note that the free Lie algebra on a vector space admits a grading by
what we will call \emph{degree}. Namely, the free generators have degree 1,
and the Lie bracket of something in degree $i$ with something in degree $j$
has degree $i+j$. Quotienting by the image of $\cup^T$ kills off some
elements which are homogeneous of degree 2, so the result still has a grading.

In terms of degree, a unipotent representation of our Lie algebra is one
in which anything of degree at least $N$ acts trivially, for some
sufficiently large $N$. Let $Z_N$ be the stuff of degree at least $N$;
we will replace the Lie algebra with
\[
\Lie \pi_1^{\dR} = \lim_{\leftarrow} \FreeLie(H_1^{\dR}(X))/(\Im(\cup^T)+Z_N).
\]
Of course, this is not yet honest, because we don't have a group $\pi_1^{\dR}$
of which this can be the Lie algebra!

\subsection*{Recovering the group}

Having constructed what is supposed to be the Lie algebra of the de Rham
realization of the unipotent fundamental group, essentially by declaring that
its representations are the unipotent local systems, it is time to recover
the group itself. There is a highly abstract way of doing this kind of thing
in general (more on this later), but this task is pretty straightforward.

For a nilpotent Lie algebra, or in our case a pro-nilpotent Lie algebra,
one can produce the corresponding group by exponentiation, which is really
to say using the Campbell-Hausdorff formula. This formula is written down as
follows: take two noncommuting indeterminates $x$ and $y$. Then
\[
\exp(x) \exp(y) = \exp\left(x + y + \frac{1}{2}[x,y] + \cdots \right)
\]
where everything on the right is in the free Lie algebra generated by $x$ 
and $y$. Moreover, the right side has only finitely many terms of any
given degree. Thus if $x$ and $y$ are actually taken in a nilpotent Lie
algebra, the sum on the right becomes finite; if in a pro-nilpotent Lie
algebra, it becomes a convergent series. In any case, you can use it
to define a group structure on the symbols $\exp(x)$.

That's how you compute in practice, but for conceptual purposes (and for
proving that the above recipe works!) there is a simpler description.
Given a Lie algebra $\calL$, let $U\calL$ denote its universal enveloping
algebra (the associative algebra $\Sym^{\cdot} \calL$ modulo relations
$xy -yx - [x,y]$; note that $U\FreeLie(V) = \Sym^{\cdot} V$). Let $I$
be the augmentation ideal of $U\calL$, i.e., the ideal generated by
the elements of $\calL$. Let $U\calL^\wedge$ denote the $I$-adic completion
of $U\calL$.

The universal enveloping algebra of a Lie algebra $\calL$ comes with a
canonical coproduct $\Delta: U\calL \to U\calL \otimes U\calL$, defined
by the relation
\[
\Delta(v) = v \otimes 1 + 1 \otimes v \qquad v \in \calL.
\]
Observe that the set of $v \in U\calL$ such that
$\Delta(v) = v \otimes 1 + 1 \otimes v$ is itself a Lie algebra,
containing $\calL$; unless I'm mistaken, it is actually $\calL$ itself,
at least in characteristic 0. (In characteristic $p$, it is the closure
of $\calL$ under the operation $v \mapsto v^p$.)

Given a coproduct on $U\calL$, we say $x \in U\calL$ is \emph{grouplike}
if 
\[
\Delta(x) = x \otimes x.
\]
Notice that for any grouplike element $x$, $x-1 \in I$.
In fact, the set of grouplike elements indeed forms a group, and
if $\calL$ is (pro-)nilpotent, it is the same as the group we constructed
earlier.

\subsection*{Filtrations}

(Warning: this section is completely garbled. Make sense of it at your
own risk.)

Given that $\pi_1^{\dR}$ comes from the de Rham realization, it should
carry weight and Hodge filtrations. What might those be?

First note that a function $f: X \to \GG_m$ gives rise to a logarithmic
differential $df/f$ in $H^1_{\dR}(X)$. Also note that $H^1_{\dR}(\GG_m)$
is canonically $\ZZ(-1)$, generated by $dz/z$ for $z$ a coordinate
on $\GG_m$. In general, $H^1_{\dR}(X)$ will be a direct sum of copies
of $\ZZ(-1)$.

We now have a projection $\Lie \pi_1^{\dR} \to \Lie \pi_1^{\ab} = H_1^{\dR}$.
This should be compatible with any structures we define, like filtrations.

The $Z$ filtration (the modified central descending series) we used to
construct the Betti realization of the unipoten fundamental group is
motivic, so the corresponding graded ring $\Gr_Z(\Lie \pi_1)$ is too.
(Something happens here that I couldn't follow.)

The weight filtration on $\Lie \pi_1^{\dR}$ will turn out to be precisely
the descending degree filtration. That is, each step will be the set of
terms of degree at least $i$ for some $i$. The Hodge filtration will go
the other way: each step will be the set of terms of degree at most $i$.

(Some throwaway comment about what a mixed Hodge structure is follows,
apparently irrelevant to the sequel.)

\section{The Betti-de Rham comparison isomorphism}

\subsection*{Parallel transport and the Betti-de Rham comparison}

We now describe the construction of the Betti-de Rham isomorphism,
from a ``motivic'' point of view. Instead of the fundamental group,
it will be helpful to consider more generally the (Betti) fundamental
groupoid ${}_bP_a$ of paths from $b$ to $a$. For any fixed $a$ and $b$,
this object is a principal homogeneous space for $\pi_1$, and we
can form the unipotent version as $\Spec (\lim_{\rightarrow} 
\QQ[{}_bP_a]/I^n)^\dual$.

In the de Rham realization, this construction isn't necessary, because
there is a \emph{canonical} path from one point to another! The point
is that because of our condition $H^1(\overline{X}, \calO) = 0$,
vector bundles with flat connection canonically trivialize, so it doesn't 
matter what path you use for integration.

The way the comparison isomorphism should work is this: given a
Betti path (i.e., an honest path on the topological space) from $a$ to $b$
and a vector bundle with integrable connection (i.e.,
a representation of $\pi_1^{\dR}$), write it as the connection $d - \omega$
on a trivial vector bundle (using the canonical trivialization from above);
then parallel transport along the Betti path gives an isomorphism of 
the fibres of the bundle at $a$ and $b$, i.e., of $V$ with itself.
That should give a map from ${}_bP_a$ to $\pi_1^{\dR}$.

Our task now is to give an ``algebraic'' version of parallel transport;
the result ends up involving iterated integrals. Say the Betti path
$\gamma$ is parametrized in terms of $t$, from $t=0$ to $t=1$.
Given a vector $v \in V$ at position $t$, its image under parallel transport
to position $t + \Delta t$ is, to a first-order approximation,
\[
v + \langle \omega, \gamma'(t) \rangle (\Delta t),
\]
where the angle brackets denote contraction of a 1-form with a tangent
vector. (Note that $\omega$ is a 1-form with values in $\End(V)$.)
If $t_0 = 0 <  t_1 < \dots < t_n = 1$,
then the parallel transport morphism from $a$ to $b$ is approximately
the composition of the linear transformations
\[
\prod_{i=n-1}^0 (I + \langle \omega, \gamma'(t_i) \rangle (t_{i+1}-t_i)).
\]
Since $\omega$ is nilpotent, when we multiply this out all of the terms
beyond a certain length vanish. Thus when we take the limit as
$n \to \infty$ and $\max_i \{t_{i+1}-t_i\} \to 0$, this product
turns into the expansion
\[
I + \int_0^1 \omega + \int_0^1 \omega \int_0^t \omega + \cdots
\]

\subsection*{Tangential base points: Betti realization}

We are considering $\PP^1 - \{0, 1, \infty\}$ because it has
``good reduction''; modulo any prime $p$, the morphisms $\Spec \ZZ
\to \PP^1_\ZZ$ given by each of $0, 1, \infty$ have disjoint images.
But this stops being true as soon as we add an additional point,
since over $p=2$ there are no more points!

The upshot: we need to choose a base point from which to draw paths to other
points, in the Betti and de~Rham realizations. However, we should not choose
an actual $\ZZ$-valued point for this purpose or else we will encounter
bad reduction. So instead we want to use a tangent vector at $0$
in place of the base point.

In the Betti picture, this amounts to taking a base point ``infinitesimally
close'' to 0 in some direction. More precisely, that means we take a
point on the blowup of $X(\CC)$ at 0. The blowup is obtained by
cutting out a small disc around 0, cutting a small disc out of
a plane, then gluing along the edges of the discs. That plane we
used can be canonically identified with the tangent space to the curve
at 0.
In the bargain, one gets a canonical generator
of $\pi_1^B$, namely the counterclockwise loop around 0.

\subsection*{Tangential base points: de Rham realization}

How do we make sense of the notion of a tangential base point in the 
de Rham realization? That is, given a local system $(V, \nabla)$ on
our curve, how do we pull it back to the blowup?

We first extend $(V, \nabla)$ canonically to $(\overline{V}, \nabla)$,
where $\overline{V}$ is a trivial vector bundle and $\nabla = d - \omega$
where $\omega$ has a simple pole at 0 and $\Res(\omega)$ is a nilpotent
endomorphism of the fibre $\overline{V}_0$. We then extend
to the tangent space by taking the constant vector bundle with fibre
$\overline{V}_0$ and connection $d - \Res(\omega) du/u$, where $u$
is a linear coordinate on the tangent space.

\subsection*{The comparison isomorphism}

Note: we will shamelessly exploit the fact that $\pi_1^{\dR}$ in our case
is canonically independent of the base point, in order to simplify the
description.

The comparison isomorphism should be a map, from the set of paths
from the basepoint $x$ to itself, to $\pi_1^{\dR}$, at least after 
tensoring with $\CC$. That is, we are looking for
\[
\comp: \pi_1^B \otimes \CC \stackrel{\sim}{\to} \pi_1^{\dR} \otimes \CC.
\]

Given a representation of $\pi_1^{\dR}$, that is, a unipotent local
system $V$, we also have a representation of $\Lie \pi_1^{\dR}$;
recall that the latter was constructed as the free Lie algebra on
$H^1(X)$ modulo relations obtained from $H^2(X)$.

For $X = \PP^1 - \{0, 1, \infty\}$, the cohomology $H^1_{\dR}$
admits the basis $\frac{dz}{z}, \frac{dz}{1-z}$. In homology
$H_1^{\dR}$ (i.e., the dual of cohomology), we have elements
$e_0 = \Res_0$, $e_1 = \Res_1$, $e_\infty = \Res_{\infty}$
such that $e_0 + e_1 + e_\infty = 0$.

We now have a tautological 1-form with values in $H_1^{\dR}$,
namely
\[
\omega = \frac{dz}{z} e_0 + \frac{dz}{1-z} e_1
\in \Omega^1(\log D) \otimes H^1 \subset \Omega^1(\log D)
\otimes \Lie \pi_1^{\dR}.
\]
Given a representation
$\rho: \Lie \pi_1^{\dR} \to \End(V)$ (necessarily with nilpotent
image), $\rho(\omega)$ gives a 1-form with values in $\End(V)$, that is,
$(V, d - \rho(\omega))$ is a unipotent local system. (This is the
canonical trivialization we keep mentioning.)

To describe $\comp$, we must for starters
give for each Betti path $\gamma$ and
each representation $\rho: \Lie \pi_1^{\dR} \to \End(V)$ an
element of $\End(V)$ in a natural way. 
 Namely, we compute the parallel
transport along $\gamma$ of $\rho$ as described above.
That gives us a map 
from $\pi_1^B$ to the universal enveloping algebra of
$\Lie \pi_1^{\dR}$, which is the completion (with respect to the
degree grading) of the free associative algebra on $e_0$ and $e_1$,
also notated $\Assoc(e_0, e_1)^\wedge$.

For this map to actually have image in $\pi_1^{\dR}$, its image
must consist of grouplike elements. Recall what this means: there is
a coproduct $\Delta:
\Assoc(e_0, e_1)^\wedge \to \Assoc(e_0, e_1)^\wedge \otimes
\Assoc(e_0, e_1)^\wedge$ with $\Delta(e_0)=e_0 \otimes 1 + 1 \otimes e_0$
and $\Delta(e_1) = e_1 \otimes 1 + 1 \otimes e_1$. (In fact, it sends
$x$ to $x \otimes 1 + 1 \otimes x$ for any $x \in \Lie \pi_1^{\dR}$.)
Then the grouplike elements are the set
\[
\{g \in \Assoc(e_0, e_1)^\wedge | \Delta(g) = g \otimes g\};
\]
these do indeed form a group under multiplication, and that is what
we are calling $\pi_1^{\dR}$.

In fact, parallel transport always produces grouplike elements. (There should be
a ``pure thought'' reason for this, but I wasn't able to see it.)

\subsection*{A computation}

We now compute the action of parallel transport along
the path $\gamma$ I described
earlier. First we must explain a bit more precisely what $\gamma$ is doing.

As noted earlier, $\gamma$ is actually defined on the topological blowup
of $\PP^1(\CC)$ at 0 and 1. This blowup is obtained by removing a small disk at
0 and 1, and glueing along this disk a copy of the tangent space at the
point with a disk removed around the point. Our path starts at the
point 1 on the tangent space at 0, runs along the line towards 0 to the
glueing disk, then along the real line in $\PP^1(\CC)$ to the glueing disk
around 1, then runs back to 0 on the tangent space at 1.

Remember that given a representation $\rho: \Lie \pi_1^{\dR} \to
\End(V)$, the canonical 1-form on $\PP^1(\CC)$ is given by 
$d - \rho(e_0) \frac{dz}{z} - \rho(e_1) \frac{dz}{1-z}$.
On the tangent spaces at 0 and 1, each with parameter $u$ vanishing at
the center, the canonical 1-form is given by
$d - \rho(e_0) \frac{du}{u}$ and $d + \rho(e_1) \frac{du}{u}$,
respectively.

We now compute the parallel transport along $\gamma$ in three steps.
Say we are using the radius $\epsilon$ for the glueing disks.
First, we integrate in the tangent space at 0 from 1 to $\epsilon$
to get $\exp((\log \epsilon) \rho(e_0))$. Last, we integrate in the tangent
space at 1 from $1-\epsilon$ to 0 to get $\exp((\log \epsilon) \rho(e_1))$.
So we really have
\[
\exp((\log \epsilon) \rho(e_1)) h \exp((\log \epsilon) \rho(e_0))
\]
where $h$ is the parallel transport from $\epsilon$ to $1-\epsilon$.
%\footnote{It looks to me like left and right got switched somewhere in
%my computations in the associative algebra, but I haven't taken the trouble
%to trace down where this happened. My apologies.}

We know $h$ is the image under $\rho$ of a
grouplike element of $\Assoc(e_0, e_1)^\wedge$;
let us try to read off its coefficients and see how they behave as
$\epsilon \to 0$. Since $1/(1-z) = \sum_{n=0}^\infty
z^n$, we can rewrite the parallel transport as
\[
1 + \sum_{\epsilon}^{1-\epsilon} \omega(t_1)\,dt_1 + \sum_\epsilon^{1-\epsilon}
\omega(t_1) \int_{\epsilon}^{t_1} \omega(t_2) \,dt_2\,dt_1 + 
\sum_\epsilon^{1-\epsilon} \omega(t_1) \int_{\epsilon}^{t_1}
\omega(t_2) \int_{\epsilon}^{t_2} \omega(t_3)\,dt_3\,dt_2\,dt_1 + \cdots,
\]
where each term is in $\Assoc(e_0, e_1)^\wedge$.

We would like to extract the coefficient in this sum of $e_{i_1} \cdots
e_{i_k}$, where $i_j \in \{0,1\}$ for each $j$. First suppose
$i_1 \neq 1$ and $i_k \neq 0$. Then in any term of the sum which contributes
a multiple of $e_{i_1}\cdots e_{i_k}$, we pick up an $e_1$ on the right
from the innermost integral, i.e., from
\[
\int_\epsilon^t \frac{dz}{1-z} = \int_\epsilon^t \sum_{n=0}^\infty z^n\,dz;
\]
as $\epsilon \to 0$, this tends to
\[
\int_0^t \sum_{n=0}^\infty z^n\,dz = \sum_{n=1}^\infty \frac{t^n}{n}.
\]
If the next term to the left is $e_0$, this becomes
\[
\int_0^t \frac{dz}{z} \sum_{n=1}^\infty \frac{t^n}{n} = \sum_{n=1}^\infty
\frac{t^n}{n^2};
\]
likewise, if there are $m$ copies of $e_0$ immediately to the left of
the rightmost $e_1$, the corresponding $m+1$-fold integral is
$\sum_{n=1}^\infty t^n/n^{m+1}$.

Going back to our expression $e_{i_1}\cdots e_{i_k}$, suppose again that
$i_1 \neq 1$ and $i_k \neq 0$. Let $s_1-1, s_2-1, \dots, s_l-1$ 
be the lengths of the
runs of $e_0$'s that separate consecutive occurrences of $e_1$. (Yes, these
could be zero, but at least we have $s_1-1 \neq 0$.) Then the integral
we get after the second $e_1$ is
\[
\int_0^t \frac{dz}{1-z} \sum_{n=1}^\infty \frac{z^n}{n^{s_l}} =
\int_0^t \sum_{m,n: m \geq 0} \frac{z^{n+m}}{n^{s_l}}
= \sum_{m,n: m \geq 1} \frac{t^{n+m}}{n^{s_l}(n+m)}.
\]
Likewise, the full multiple integral comes out to
\[
\sum_{m_1, \dots, m_l \geq 1} \frac{t^{m_l + \cdots + m_1}}{m_l^{s_l}
(m_l+m_{l-1})^{s_{l-1}} \cdots (m_l + \cdots + m_1)^{s_l + \cdots + s_1}}.
\]
To lighten notation, put $n_i = m_l + \cdots + m_i$. Then we conclude that
the coefficient of $e_0^{m_1} e_1 e_0^{m_2}e_1\cdots e_0^{m_l}e_1$ is
\[
\lim_{t \to 1} \sum_{n_1> \cdots > n_l} \frac{t^{n_1}}{n_1^{s_1}\cdots
n_l^{s_l}} = \sum_{n_1 > \cdots > n_l} \frac{1}{n_1^{s_1} \cdots + 
n_l^{s_l}},
\]
a/k/a the multiple zeta value $\zeta(s_1, \dots, s_l)$.
(Note this converges because $s_1 \geq 2$ by our assumption that
$e_{i_0} \neq 1$.)

\subsection*{What about the bad terms?}

Warning: there are a number of missing details here, especially on the
subject of whether certain limits converge.

To recap: we have computed that on $\PP^1 \setminus \{0,1, \infty\}$, the
parallel transport from $\epsilon$ to $1 - \epsilon$ yields an element
of $\pi_1^{\dR} \subset \Assoc(e_0, e_1)^\wedge$ in which the
coefficient
of $e_0^{s_1-1} e_1 e_0^{s_2-1}e_1\cdots e_0^{s_k-1}e_1$ for
$s_1 \geq 2$ and $s_2, \dots, s_k \geq 1$ is equal to the multiple
zeta value $\zeta(s_1, \dots, s_k)$. So what does this tell us about
$\comp(\gamma)$?

It can be shown that
\[
\exp((\log \epsilon) \rho(e_1))) h \exp((\log \epsilon) \rho(e_0))
\]
converges as $\epsilon \to 0$ to the image under $\rho$ of some
element $T$ of $\pi_1^{\dR} \otimes \CC$; 
this is what we mean by ``parallel transport
along $\gamma$''. (I haven't
verified the convergence completely, except in the case below.)
The coefficient
of $e_0$ in this is
\[
(\log \epsilon) + \sum_{\epsilon}^{1-\epsilon} \frac{dz}{z}
= log (1 - \epsilon),
\]
which tends to 0 as $\epsilon \to 0$. Likewise the coefficient of $e_1$
tends to 0 as $\epsilon \to 0$.

For any term of the form
$e_0^{s_1-1} e_1 e_0^{s_2-1}e_1\cdots e_0^{s_k-1}e_1$
with $s_1 > 1$, i.e., a term starting with an $e_0$ and ending with an $e_1$,
any corresponding term in
\[
\exp((\log \epsilon) \rho(e_1))) h \exp((\log \epsilon) \rho(e_0))
\]
must come directly from $h$, since the factor on the left either does nothing
or puts an $e_1$ on the right, and ditto on the right. So the coefficient
of such a term is precisely the multiple zeta value $\zeta(s_1, \dots, s_k)$.

For other terms, one now uses the following algebra fact (left to
the reader): since the parallel transport of $\gamma$
is grouplike and its coefficients of $e_0$ and $e_1$ are both zero, all of
its coefficients are determined by the coefficients of the terms that
start with $e_0$ and end with $e_1$. Each coefficient comes out being
some polynomial in the multiple zeta values (possibly a rational linear
combination, but I wasn't clear on this).

Now we have computed the parallel transport along essentially a path from 0
to 1 (or rather, from one tangential base point to another). To relate this
to $\pi_1^B$, we need to write loops from a single base point to itself in
terms of this path. We will use the tangential base point $\vec{0}_1$
at 0 in the direction of the positive real line. Our generators of 
$\pi_1^B$ will be a counterclockwise loop around 0, and $\gamma$ followed
by a counterclockwise loop around 1 followed by $\gamma^{-1}$. We map these to
\[
2 \pi i e_0 \qquad \mbox{and} \qquad T^{-1} (2 \pi i e_1) T,
\]
respectively. This yields the desired map
\[
\comp: \pi_1^B \otimes \CC \stackrel{\sim}{\to} \pi_1^{\dR} \otimes \CC.
\]

\section{Complements}

\subsection{The infinite Frobenius}

If $X$ is a variety over $\RR$, then any realization of $X$
is equipped with a canonical involution $F_{\infty}$
given by complex conjugation. How
does it look in the Betti and de~Rham realizations?

In the Betti realization, $F_\infty$ acts as conjugation of paths
on $X(\CC)$; call this $\sigma_B$. In the de~Rham realization,
$F_\infty$ acts as conjugation of complex-valued differential forms;
call this $\sigma_{\dR}$.

The main point here is that if $\omega$ is an algebraic differential
over $\CC$ and $Z$ is a path in $X(\CC)$, then
\[
\int_Z \overline{\omega} = \int_{\overline{Z}} \omega,
\]
where $\overline{Z}$ is the path obtained from $Z$ by pointwise conjugation.
That shows that the two operations we described are compatible with
the Betti-de~Rham comparison.
For a general realization, we have the factorization
$F_{\infty} = \sigma_B \sigma_{\dR}$.

For a variety $X$ over $\QQ$, the involution $F_\infty$ plays the role
of a Frobenius automorphism at the infinite place. There is also a
Frobenius automorphism at each finite prime $p$, which act naturally on
$X$ over $\QQ_p$. This is the so-called ``crystalline Frobenius'', which
was discussed further in the student lecture.

\subsection{Motivic considerations}

Very, very loosely speaking,
a cohomological construction for algebraic varieties is said to be
``motivic'' if it depends purely on geometric considerations,
so that it can be canonically defined in all realizations. This begs
the question of what a ``motive'' is, a question we will make no attempt
to answer. Suffice to say for the moment that motives are supposed to be
objects in an abelian category that behave like geometrically defined
pieces of cohomology spaces, and which admit ``realizations'' such as the
Betti and de~Rham realization. For example, there should be a motive attached
to each scheme $X$, whose realizations are the total cohomologies of $X$;
this motive should also decompose as a sum of motives whose realizations
are the individual cohomology spaces of $X$.

A bit more precisely, there exists
a category of \emph{motives of mixed Tate type} over $\Spec \ZZ$
(or $\Spec \QQ$), coming from a triangulated construction by a construction
\`a la Voevodsky. It also makes sense to look at motives of mixed
Tate type over a scheme $X$.

The category of mixed Tate motives over $X$ is a Tannakian category,
so we can construct its automorphism group and call it the
``motivic Galois group'' of a scheme $X$. Then realizations of $X$
induce fibre functors on the category of mixed Tate motives,
giving specializations of the motivic Galois group. By Tannaka duality
(\`a la Saavedra),
the category of finite dimensional representations of one of
these specializations is equivalent to the category of realizations
of motives.

For example, in the de Rham realization, we get a group $G_{\dR}$ whose
category of finite dimensional representations is equivalent to
the category of vector bundles on $X$ with integrable
connection and unipotent monodromy. And in fact, one can read off this
group by simply taking the group of automorphisms of the latter category,
i.e., functors to itself that commute with direct sum and tensor product.
The result is a pro-algebraic group (the ``pro'' needed because it
is an inverse limit of algebraic groups but possibly not algebraic).


\begin{thebibliography}{999}
\bibitem[D]{bib:del}
P. Deligne, Le groupe fondamental de la droite projective moins trois
points, in \textit{Galois Groups over $\QQ$} (Y. Ihara, K. Ribet, and
J.-P. Serre, eds.), Springer-Verlag (1989), 79--298.
\end{thebibliography}

\end{document}
